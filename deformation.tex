\documentclass[a4paper,11pt]{article}

\author{Sebastian Pancratz, Jan Tuitman}
\title{Improvements to the deformation method}

% Geometry and page layout %%%%%%%%%%%%%%%%%%%%%%%%%%%%%%%%%%%%%%%%%%%%%%%%%%%%

\usepackage[hmargin=3.2cm,vmargin=3.2cm,a4paper,centering,twoside]{geometry}

% Other packages %%%%%%%%%%%%%%%%%%%%%%%%%%%%%%%%%%%%%%%%%%%%%%%%%%%%%%%%%%%%%%

\usepackage[T1]{fontenc}
\usepackage{ae,aecompl}

\usepackage{ifpdf}

% hyperref %%%%%%%%%%%%%%%%%%%%%%%%%%%%%%%%%%%%%%%%%%%%%%%%%%%%%%%%%%%%%%%%%%%%

\usepackage{hyperref}
\hypersetup{
    colorlinks=false,   % false: boxed links; true: colored links
    citecolor=green,    % color of links to bibliography
    filecolor=magenta,  % color of file links
    linkcolor=red,      % color of internal links
    urlcolor=blue       % color of external links
}

\makeatletter
\newcommand\org@hypertarget{}
\let\org@hypertarget\hypertarget
\renewcommand\hypertarget[2]{%
    \Hy@raisedlink{\org@hypertarget{#1}{}}#2%
} 
\makeatother

\ifpdf
    \hypersetup{
        pdftitle={Deformation method},
        pdfauthor={Sebastian Pancratz, Jan Tuitman},
        pdfsubject={Computational Number Theory},
        bookmarks=true,
        bookmarksnumbered=true,
        unicode=true,
        pdfstartview={FitH},
        pdfpagemode={UseOutlines}
    }
\fi

% algorithmic %%%%%%%%%%%%%%%%%%%%%%%%%%%%%%%%%%%%%%%%%%%%%%%%%%%%%%%%%%%%%%%%%

\usepackage[section]{algorithm}
\usepackage[noend]{algpseudocode}

\renewcommand{\algorithmicrequire}{\textbf{Input:}}
\renewcommand{\algorithmicensure}{\textbf{Output:}}

% natbib %%%%%%%%%%%%%%%%%%%%%%%%%%%%%%%%%%%%%%%%%%%%%%%%%%%%%%%%%%%%%%%%%%%%%%

\usepackage{natbib}

\bibpunct{[}{]}{,}{n}{}{}

% url %%%%%%%%%%%%%%%%%%%%%%%%%%%%%%%%%%%%%%%%%%%%%%%%%%%%%%%%%%%%%%%%%%%%%%%%%

\usepackage{url}

\makeatletter
\def\url@leostyle{%
  \@ifundefined{selectfont}{\def\UrlFont{\sf}}{\def\UrlFont{\small\ttfamily}}}
\makeatother
\urlstyle{leostyle}

% Enumeration %%%%%%%%%%%%%%%%%%%%%%%%%%%%%%%%%%%%%%%%%%%%%%%%%%%%%%%%%%%%%%%%%

\usepackage{paralist}

\setlength{\pltopsep}{0.24em}
\setlength{\plpartopsep}{0em}
\setlength{\plitemsep}{0.24em}

% This should do what we want
%   \setdefaultenum{(i)}{(a)}{1.}{A}
% but it does not work for references, dropping the parentheses.  The following
% hack does work.

\renewcommand{\theenumi}{(\roman{enumi})}
\renewcommand{\theenumii}{(\alph{enumii})}
\renewcommand{\theenumiii}{\arabic{enumiii}.}
\renewcommand{\theenumiv}{\Alph{enumiv}}

\renewcommand{\labelenumi}{\theenumi}
\renewcommand{\labelenumii}{\theenumii}
\renewcommand{\labelenumiii}{\theenumiii}
\renewcommand{\labelenumiv}{\theenumiv}

%%%%%%%%%%%%%%%%%%%%%%%%%%%%%%%%%%%%%%%%%%%%%%%%%%%%%%%%%%%%%%%%%%%%%%%%%%%%%%%
% Mathematics

% Packages %%%%%%%%%%%%%%%%%%%%%%%%%%%%%%%%%%%%%%%%%%%%%%%%%%%%%%%%%%%%%%%%%%%%

\usepackage{amsmath,amsthm,amscd,amsfonts,amssymb}
\usepackage{cases}
\usepackage[all]{xy}

\allowdisplaybreaks[4]
\numberwithin{equation}{section}

% Customised notation %%%%%%%%%%%%%%%%%%%%%%%%%%%%%%%%%%%%%%%%%%%%%%%%%%%%%%%%%

\providecommand{\abs}[1]{\lvert#1\rvert}                 % Absolute value
\providecommand{\absbig}[1]{\bigl\lvert#1\bigr\rvert}    % Absolute value
\providecommand{\absBig}[1]{\Bigl\lvert#1\Bigr\rvert}    % Absolute value
\providecommand{\absbigg}[1]{\biggl\lvert#1\biggr\rvert} % Absolute value

\providecommand{\norm}[1]{\lVert#1\rVert}              % Norm
\providecommand{\normbig}[1]{\bigl\lVert#1\bigr\rVert} % Norm
\providecommand{\normBig}[1]{\Bigl\lVert#1\Bigr\rVert} % Norm

\providecommand{\floor}[1]{\left\lfloor#1\right\rfloor}   % Floor
\providecommand{\floorts}[1]{\lfloor#1\rfloor}            % Floor
\providecommand{\floorbig}[1]{\bigl\lfloor#1\bigr\rfloor} % Floor
\providecommand{\floorBig}[1]{\Bigl\lfloor#1\Bigr\rfloor} % Floor

\providecommand{\ceil}[1]{\left\lceil#1\right\rceil}   % Ceiling
\providecommand{\ceilts}[1]{\lceil#1\rceil}            % Ceiling
\providecommand{\ceilbig}[1]{\bigl\lceil#1\bigr\rceil} % Ceiling
\providecommand{\ceilBig}[1]{\Bigl\lceil#1\Bigr\rceil} % Ceiling

\newcommand{\NN}{\mathbf{N}} % Natural numbers
\newcommand{\ZZ}{\mathbf{Z}} % Integers
\newcommand{\QQ}{\mathbf{Q}} % Rationals
\newcommand{\RR}{\mathbf{R}} % Real numbers
\newcommand{\CC}{\mathbf{C}} % Complex numbers
\newcommand{\FF}{\mathbf{F}} % Finite field

\renewcommand{\to}{\rightarrow}        % Right arrow
\newcommand{\into}{\hookrightarrow}    % Injection arrow
\newcommand{\onto}{\twoheadrightarrow} % Surjection arrow

\DeclareMathOperator{\fCoKer}{coker} % Cokernel
\DeclareMathOperator{\fKer}{ker}     % Kernel
\DeclareMathOperator{\fIm}{im}       % Image

\DeclareMathOperator{\Res}{Res}   % Resultant
\DeclareMathOperator{\Tr}{Tr}     % Trace
\DeclareMathOperator{\Trace}{Tr}  % Trace
\DeclareMathOperator{\Norm}{N}    % Norm
\DeclareMathOperator{\Disc}{Disc} % Discriminant

\DeclareMathOperator{\Gal}{Gal}          % Galois group
\DeclareMathOperator{\ord}{ord}          % Order
\DeclareMathOperator{\sgn}{sgn}          % Sign, signature
\DeclareMathOperator{\Frob}{\mathcal{F}} % Frobenius
\DeclareMathOperator{\Hom}{Hom}          % Space of homomorphisms
\DeclareMathOperator{\Spec}{Spec}        % Spectrum

\providecommand{\HdR}{H_{\text{dR}}}    % de Rham cohomology
\providecommand{\Het}{H_{\text{\'et}}}  % etale cohomology
\providecommand{\Hrig}{H_{\text{rig}}}  % rigid cohomology

\providecommand{\cB}{\mathcal{B}} % Basis
\providecommand{\cR}{\mathcal{R}} % Row index set
\providecommand{\cC}{\mathcal{C}} % Column index set

\providecommand{\BigOh}{\mathcal{O}} % Big-oh notation

% Theorems etc %%%%%%%%%%%%%%%%%%%%%%%%%%%%%%%%%%%%%%%%%%%%%%%%%%%%%%%%%%%%%%%%

\theoremstyle{definition}

\newtheorem{thm}{Theorem}[section]
\newtheorem{lem}[thm]{Lemma}
\newtheorem{prop}[thm]{Proposition}
\newtheorem{cor}[thm]{Corollary}
\newtheorem{defn}[thm]{Definition}
\newtheorem{exmp}[thm]{Example}
\newtheorem{rem}[thm]{Remark}
\newtheorem{prob}[thm]{Problem}

%%%%%%%%%%%%%%%%%%%%%%%%%%%%%%%%%%%%%%%%%%%%%%%%%%%%%%%%%%%%%%%%%%%%%%%%%%%%%%%
% DOCUMENT                                                                    %
%%%%%%%%%%%%%%%%%%%%%%%%%%%%%%%%%%%%%%%%%%%%%%%%%%%%%%%%%%%%%%%%%%%%%%%%%%%%%%%

\begin{document}

\maketitle

[[TEST:  This is just a stupid sentence to test the references~\citep{Gerkmann2007}.]]

\section{Introduction}

\section{Theoretical background}

\section{Computing in de~Rham cohomology}

In this section we present a concrete representation of the algebraic 
de~Rham cohomology spaces $\HdR^{n}(\mathfrak{U}/\mathfrak{S})$, 
restricted to the case of the complement $\mathfrak{U}/\mathfrak{S}$ 
of a smooth projective family $\mathfrak{X} / \mathfrak{S}$.  We first 
follow Abbott, Kedlaya and Roe~\citep[Remark~3.2.5]{AbbottKedlayaRoe2006} 
in the description of the so-called \emph{reduction of poles} procedure for 
the de~Rham cohomology $L$-vector space $\HdR^{n}(\mathfrak{U} / L)$, 
considering $\mathfrak{U}$ as a scheme over $L = \QQ_{q_1}(t)$.

\begin{thm}
Let $\Omega$ denote the $n$-form $\Omega$ in $\HdR^{n}(\mathfrak{U}/L)$ 
defined by 
\begin{equation}
\Omega = \sum_{i=0}^n (-1)^i x_i d x_0 \wedge \dotsb \wedge \widehat{d x_i} \wedge \dotsb \wedge d x_n.
\end{equation}
The algebraic de~Rham cohomology space $\HdR^{n}(\mathfrak{U}/L)$ is 
isomorphic as a $L$-vector space to the quotient of the group of $n$-forms 
$Q \Omega / P^k$ with $k \in \NN$ and $Q \in L[x_0, x_1, \dotsc, x_n]$ 
homogeneous of degree $k d - (n + 1)$ by the subgroup generated by the 
relation 
\begin{equation} \label{eq:deRhamRel}
\frac{(\partial_i Q) \Omega}{P^k} - k \frac{Q (\partial_i P) \Omega}{P^{k+1}}
\end{equation}
for all $0 \leq i \leq n$, where here and in the following $\partial_i$ 
denotes the partial derivative operator with respect to~$x_i$.
\end{thm}

\begin{proof}
This result can be obtained following Griffiths~\citep[\S 4]{Griffiths1969}.
\end{proof}

It is clear that $\HdR^{n}(\mathfrak{U}/L)$ can be equipped with a 
filtration whose $i$th parts consists of all elements which can be 
represented by $n$-forms with degree $\deg Q = kd - (n + 1)$ for 
$1 \leq k \leq i + 1$. We can obtain a basis for $\HdR^{n}(\mathfrak{U}/L)$ 
respecting this filtration as follows.  For every $k \in \NN$, we find 
an independent set $B_k$ of polynomials of degree $kd-(n+1)$ generating 
the quotient of the space of all such polynomials by the Jacobian ideal 
$(\partial_0 P, \dotsc, \partial_n P)$.  This yields a generating set 
$\bigcup_{k \in \NN} \cB_k$ for $\HdR^n(\mathfrak{U}/L)$ where 
$\cB_k = \{Q \Omega / P^k : Q \in B_k\}$.  It follows from a theorem of 
Macaulay~\citep[\S 4, (4.11)]{Griffiths1969} that in fact the set 
$\cB = \cB_1 \cup \dotsb \cup \cB_n$ already forms a generating set.  
Later we shall exhibit an explicit basis of monomials in the case where 
the family of projective hypersurfaces contains a diagonal fibre.

Now, to obtain a unique representative for the class of $Q \Omega / P^k$ 
in terms of the basis elements in $\cB$, we express $Q$ in terms of 
$\partial_0 P, \dotsc, \partial_n P$ as well as elements of $B_k$, and 
then iteratively reduce the pole order of the first part according to 
the relations given by the expressions from Equation~\eqref{eq:deRhamRel}. 
Assume that we have at our disposal a routine {\sc Decompose} which, 
given a polynomial $Q$ of degree $kd - (n+1)$ in the ideal generated by 
$(\partial_0 P, \dotsc, \partial_n P)$ and $B_k$, returns an expression 
$Q = Q_0 \partial_0 P + \dotsb + Q_n \partial_n P + \gamma_k$ with 
$Q_0, \dotsc, Q_n$ homogeneous polynomials in $L[x_0, \dotsc, x_n]$ and 
$\gamma_k$ in the $L$-span of $B_k$.  The reduction of poles procedure can 
then be formalised as in Algorithm~\ref{alg:PoleRed} below, which we will 
refer to as {\sc Reduce}.  In this generality, the correctness of the 
algorithm depends on a theorem of Macaulay~\citep[\S 4, (4.11)]{Griffiths1969}.

\begin{algorithm}
\caption{Reduce $Q \Omega / P^k$ in $\HdR^n(\mathfrak{U}/L)$}
\label{alg:PoleRed}
\begin{algorithmic}
\vspace{1mm}
\Require $Q \in L[x_0, \dotsc, x_n]$ homogeneous of degree $kd - (n+1)$.
\Ensure  $\gamma_i$ in the $L$-span of $B_i$, for $1 \leq i \leq n$, with  
         $Q \Omega / P^k \equiv \gamma_{1} \Omega / P^{1} + \dotsb + \gamma_n \Omega / P^n$.
\Procedure{Reduce}{$P, B_1, \dotsc, B_n, Q$}
\While{$k \geq n+1$}
\State $Q_0, \dotsc, Q_n, \bullet \gets \Call{Decompose}{Q, \partial_0 P, \dotsc, \partial_n P, B_k}$
\State $k \gets k-1$
\State $Q \gets k^{-1} \sum_{i=0}^n \partial_i Q_i$
\EndWhile
\State $\gamma_{k+1}, \dotsc, \gamma_{n} \gets 0$
\While{$Q \not \in B_k$}
\State $Q_0, \dotsc, Q_n, \gamma_k \gets \Call{Decompose}{Q, \partial_0 P, \dotsc, \partial_n P, B_k}$
\State $k \gets k-1$
\State $Q \gets k^{-1} \sum_{i=0}^n \partial_i Q_i$
\EndWhile
\If{$Q \neq 0$}
\State $\gamma_{k} \gets Q$
\State $k \gets k-1$
\EndIf
\State $\gamma_{1}, \dotsc, \gamma_{k} \gets 0$
\State \textbf{return} $\gamma_{1}, \dotsc, \gamma_n$
\EndProcedure
\end{algorithmic}
\end{algorithm}

We now specialise to the case of a smooth family of projective 
hypersurfaces containing a diagonal fibre.  We exhibit a basis 
of monomials~$B = B_1 \cup \dotsb \cup B_n$ such that the 
corresponding set~$\cB$ forms a basis for $\HdR^n(\mathfrak{U}/L)$ 
and re-express the problem of decomposing a homogeneous polynomial 
in the language of linear algebra.  From this description, we furnish 
an explicit reduction procedure in terms of matrices.  The approach 
is based on a generalisation of Sylvester matrices from two 
polynomials to $n+1$~polynomials, following Macaulay~\citep{Macaulay1994}.

\begin{defn} \label{defn:MonBasis}
For $k \in \NN$, we define the following sets of monomials, 
\begin{align*}
F_k & = \{ x^i : i \in \mathbf{N}_{0}^{n+1}, \abs{i} = k d - (n+1) \}, \\
B_k & = \{ x^i : i \in \mathbf{N}_{0}^{n+1}, \abs{i} = k d - (n+1) \text{ and $i_j < d-1$ for $0 \leq j \leq n$}\},
\end{align*}
where $x^i = x_0^{i_0} \dotsm x_n^{i_n}$ and 
$\abs{i} = i_0 + \dotsb + i_n$.
\end{defn}

In order to ensure that our computations are not vacuous, we assume 
that $\HdR^n(\mathfrak{U}/L)$ is non-zero.  After setting 
\begin{equation}
\ell = \ceil{\frac{n+1}{d}}, \quad u = \floor{\frac{(n+1)(d-1)}{d}} = n+1 - \ell,
\end{equation}
we note that $B_k$ is non-empty if and only if $\ell \leq k \leq u$.  The 
assumption is thus that $\ell \leq u$, which is equivalent to the statement 
that $d \geq 2$ whenever $n$ is odd and $d \geq 3$ whenever $n$ is even.

In the case when the family $\mathfrak{X}/\mathfrak{S}$ contains a 
diagonal fibre, it turns out that we can add additional constraints 
to our decomposition problem:

\begin{prob} \label{prob:Decomposition}
Given a homogeneous multivariate polynomial $Q \in L[x_0, \dotsc, x_n]$ 
of degree \mbox{$k d - (n + 1)$}, for some $k \in \NN$, we try to find 
homogeneous polynomials $Q_0, \dotsc, Q_n$ in $L[x_0, \dotsc, x_n]$ such 
that 
\begin{equation} \label{eq:Decomposition}
Q \equiv Q_0 \partial_0 P + \dotsb + Q_n \partial_n P
\end{equation}
modulo the $L$-span of $B_k$.  Moreover, for each $1 \leq j \leq n$,  the 
polynomial $Q_j$ may only contain non-zero coefficients for monomials of 
degree $(k-1)d-n$ that are not divisible by any of the monomials 
$x_0^{d-1}, \dotsc, x_{j-1}^{d-1}$.
\end{prob}

\begin{rem}
Immediately, we see that, for each $0 \leq i \leq n$, either $Q_i$ is 
identically zero or has degree $(k - 1) d - n$ since $\partial_i P$ is 
homogeneous of degree $d - 1$ and the elements of $B_k$ have degree 
$kd - (n+1)$.
\end{rem}

\begin{defn} \label{defn:IndexSets}
For $k \in \NN$, we define the following sets of monomials in 
$L[x_0, \dotsc, x_n]$.  Let $R_k = F_k - B_k$ be the set of monomials of 
total degree $kd-(n+1)$ and partial degree at least $d-1$ with respect to some 
of the $n+1$ variables.  Let $C_k^{(0)}$ be the set of monomials of total 
degree $(k-1)d - n$, and then inductively, for $j = 1, \dotsc, n$, define 
$C_k^{(j)}$ to be the set of monomials in $C_k^{(j-1)}$ except for those 
divisible by $x_{j-1}^{d-1}$.  Moreover, we define the multi-set $C_k$ as 
the disjoint union of $C_k^{(0)}, \dotsc, C_k^{(n)}$.  We shall write an 
element of this multi-set as $(j, g)$, referring to a monomial~$g$ 
in~$C_k^{(j)}$.
\end{defn}

With this set-up, the following theorem provides a solution to 
Problem~\ref{prob:Decomposition} in the cases that we are interested in:

\begin{thm} \label{thm:Isomorphism}
Suppose that the family of projective hypersurfaces given by the 
polynomial~$P$ in $L[x_0, \dotsc, x_n]$ contains a diagonal fibre.  
Let $k \in \NN$ and suppose that $R_k$ and $C_k$ are non-empty.  For 
$0 \leq j \leq n$, let $V_k^{(j)}$ be the $L$-vector space of 
polynomials with basis $C_k^{(j)}$, and let $V_k$ denote their cartesian 
product $V_k = V_k^{(0)} \times \dotsb \times V_k^{(n)}$.  Let $W_k$ be 
the $L$-vector space of polynomials with basis~$R_k$.  Then the $L$-linear 
map 
\begin{equation}
\phi_k \colon V_k \to W_k, 
(Q_0, \dotsc, Q_n) \mapsto Q_0 \partial_0 P + \dotsb + Q_n \partial_n P
\end{equation}
is an isomorphism of $L$-vector spaces.
\end{thm}

\begin{proof}
We first show that, for all $k \in \NN$, the multi-sets $R_k$ and $C_k$ 
have the same cardinality:

We construct the following bijection $R_k \to C_k$, representing the 
monomials by their exponent tuple.  Let $i = (i_0, \dotsc, i_n)$ be in 
$R_k$.  If $i_0 \geq d-1$, we define the image as
 $(i_0-d-1, i_1, \dotsc, i_n) \in C_k^{(0)}$.  More generally, if 
$i_0 < d-1, \dotsc, i_{j-1} < d-1$ and $i_j \geq d-1$, the image is 
$(i_0, \dotsc, i_{j-1}, i_j-d-1, i_{j+1}, \dotsc, i_n) \in C_k^{(j)}$.  
It is easy to verify that this map is indeed a bijection.

In order to establish that the map $\phi_k \colon V_k \to W_k$ is an 
isomorphism of $L$-vector spaces, we now exhibit its matrix with respect to 
the given basis:

Let $k \in \NN$ and suppose that $R_k$ and $C_k$ are non-empty, that is 
to say, $k \geq n/d + 1$.  We define the auxiliary matrix $\Delta_k$ with 
row and column index sets $R_k$ and $C_k$, respectively, as follows.  
Given $f \in R_k$ and $(j,g) \in C_k$, we set the corresponding entry in 
$\Delta_k$ to be the monomial coefficient of $f/g$ in $\partial_j P$ if 
$g$ divides $f$ and $0$ otherwise.  It is immediate that $\Delta_k$ is the 
matrix representing $\phi_k$ with respect to the bases $C_k$ and $R_k$ of 
$V_k$ and $W_k$, respectively.

The assumption that the family~$\mathfrak{X}$ of projective hypersurfaces 
given by~$P$ contains a diagonal hypersurface means that for some~$t_0$ 
the fibre $\mathfrak{X}_{t_0}$ is given by an equation of the form 
\begin{equation}
P_{t_0}(x_0, \dotsc, x_n) = a_0 x_0^d + \dotsb a_n x_n^d = 0
\end{equation}
with $a_0, \dotsc, a_n \in L^{\times}$.

We show that the determinant of $\Delta_k$ is non-zero in~$L$.  Since 
the specialisation to the diagonal fibre viz.\ evaluation of the matrix at 
$t = t_0$ commutes with computing the determinant, it suffices to show that 
the determinant of $(\Delta_k) \big |_{t=t_0}$ is non-zero in~$L$.
Since, for $0 \leq j \leq n$, 
$\partial_j P_{t_0} (x_0, \dotsc, x_n) = d a_j x_j^{d-1}$, there is 
precisely one non-zero entry in each column of $\Delta_k$.  Namely, in column 
$(j, g) \in C_k$ and row $g x_j^{d-1} \in R_k$ there is the non-zero entry 
$d \alpha_j$, concluding the proof.
\end{proof}

In principle, by including any of the methods available for solving linear 
systems, we are in a position to furnish a routine {\sc Decompose}, which 
we formalise in Algorithm~\ref{alg:Decompose}.

\begin{algorithm}[ht]
\caption{Obtain co-ordinates for $Q$ in the Jacobian ideal modulo basis elements}
\label{alg:Decompose}
\begin{algorithmic}
\Require $Q$ is a homogeneous polynomial of degree $kd - (n+1)$.
\Ensure  Homogeneous polynomials $Q_0, \dotsc, Q_n$ such that 
         $Q \equiv Q_0 \partial P_0 + \dotsb + Q_n \partial_n P$ modulo the 
         $L$-span of $B_k$.
\Procedure{Decompose}{$Q, \partial_0 P, \dotsc, \partial_n P, B_k$}
\State \begin{compactenum}[\it {Step} I.] \vspace{-1.24em}
\item Let $b$ be the vector of length $\abs{R_k}$ such that the entry 
      corresponding to the monomial $x^i \in R_k$ is the coefficient of 
      $x^i$ in $Q$.
\item Let $v$ be the unique vector of length $\abs{C_k}$ satisfying 
      $\Delta_k v = b$.  From the description of $C_k$ as a disjoint union, 
      we can write $v$ accordingly as $\bigl(v^{(0)}, \dotsc, v^{(n)}\bigr)$ 
      where, for $0 \leq j \leq n$, $v^{(j)}$ is a vector of length 
      $\abs{C_k^{(j)}}$.
\item For $j = 0, \dotsc, n$, set $Q_j = \sum_{g \in C_k^{(j)}} v_g^{(j)} g$,
      where $v_g^{(j)}$ is the entry in $v^{(j)}$ corresponding to the 
      monomial $g \in C_k^{(j)}$.
\item \textbf{return} $Q_0, \dotsc, Q_n$
\end{compactenum}
\EndProcedure
\end{algorithmic}
\end{algorithm}

We now establish that the set $\cB$ is indeed a basis for 
$\HdR^n(\mathfrak{U}/L)$, as claimed earlier, using the 
reduction of poles procedure.

\begin{thm} \label{thm:Basis}
Suppose that the family of projective hypersurfaces $\mathfrak{X}$ contains 
a diagonal fibre.  Then the set $\cB$ as in Definition~\ref{defn:MonBasis} 
is a basis for the $L$-vector space $\HdR^n(\mathfrak{U}/L)$.
\end{thm}

\begin{proof}
We know that $\HdR^n(\mathfrak{U}/L)$ is spanned by the classes of the 
$n$-forms $Q \Omega / P^k$ for all homogeneous polynomials~$Q$ of degree 
$kd-(n+1)$ and $k \in \NN$.  
By a theorem of Macaulay~\citep[\S 4, (4.11)]{Griffiths1969}, 
we may assume that $1 \leq k \leq n$, that is to say, any class in 
$\HdR^n(\mathfrak{U}/L)$ can be represented by an $n$-form with a pole 
of order at most~$n$.

Without loss of generality, we may thus start the reduction of poles 
procedure with a homogeneous polynomial~$Q$ of degree $(n+1)d-(n+1)$.  Then, 
since $B_{n+1} = \emptyset$ and $R_{n+1} = F_{n+1}$, 
Theorem~\ref{thm:Isomorphism} shows that there exist homogeneous polynomials 
$Q_0, \dotsc, Q_n$ either zero or homogeneous of degree $nd-(n+1)$ such that 
$Q = Q_0 \partial_0 P + \dotsb + Q_n \partial_n P$.  Continuing with the 
reduction of poles procedure as described in Algorithm~\ref{alg:PoleRed}, 
we obtain an expression for $Q \Omega / P^{n+1}$ as an $L$-linear combination 
of elements in $\cB$.  This shows that this set spans the vector space 
$\HdR^n(\mathfrak{U}/L)$.

To see that this set is linearly independent, note that it contains only 
monomials whose partial degrees are strictly less than $d-1$.  However, since 
$P$ is a homogeneous polynomial of degree $d$ and the family of hypersurfaces 
contains a diagonal fibre, it follows that, for each $0 \leq i \leq n$, the 
partial derivative $\partial_i P$ is a homogeneous polynomial of degree $d-1$ 
and contains precisely one monomial term with partial degree equal to $d-1$, 
namely that of the monomial $x_i^{d-1}$.  It follows that the elements of~$B$ 
cannot be reduced further modulo the Jacobian ideal and hence that the 
set~$\cB$ is linearly independent.
\end{proof}

\section{Frobenius on diagonal hypersurfaces}

\section{Solving Picard--Fuchs differential systems}

\section{Analytic continuation and evaluation}

\section{Recovering zeta functions}

\section{Complexity analysis}

\section{Examples}

\phantomsection

\bibliographystyle{plainnat}
\bibliography{deformation}

\end{document}

