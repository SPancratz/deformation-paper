\documentclass[a4paper,11pt]{article}

\author{Sebastian Pancratz, Jan Tuitman}
\title{Improvements to the deformation method for counting points 
on smooth projective hypersurfaces}

% Geometry and page layout %%%%%%%%%%%%%%%%%%%%%%%%%%%%%%%%%%%%%%%%%%%%%%%%%%%%

\usepackage[hmargin=3.2cm,vmargin=3.2cm,a4paper,centering,twoside]{geometry}

% Other packages %%%%%%%%%%%%%%%%%%%%%%%%%%%%%%%%%%%%%%%%%%%%%%%%%%%%%%%%%%%%%%

\usepackage[T1]{fontenc}
\usepackage{ae,aecompl}
\usepackage{verbatim}
\usepackage{ifpdf}
\usepackage{booktabs}

% hyperref %%%%%%%%%%%%%%%%%%%%%%%%%%%%%%%%%%%%%%%%%%%%%%%%%%%%%%%%%%%%%%%%%%%%

\usepackage{hyperref}
\hypersetup{
    colorlinks=false,   % false: boxed links; true: colored links
    citecolor=green,    % color of links to bibliography
    filecolor=magenta,  % color of file links
    linkcolor=red,      % color of internal links
    urlcolor=blue       % color of external links
}

\makeatletter
\newcommand\org@hypertarget{}
\let\org@hypertarget\hypertarget
\renewcommand\hypertarget[2]{%
    \Hy@raisedlink{\org@hypertarget{#1}{}}#2%
} 
\makeatother

\ifpdf
    \hypersetup{
        pdftitle={Deformation method},
        pdfauthor={Sebastian Pancratz, Jan Tuitman},
        pdfsubject={Computational Number Theory},
        bookmarks=true,
        bookmarksnumbered=true,
        unicode=true,
        pdfstartview={FitH},
        pdfpagemode={UseOutlines}
    }
\fi

% algorithmic %%%%%%%%%%%%%%%%%%%%%%%%%%%%%%%%%%%%%%%%%%%%%%%%%%%%%%%%%%%%%%%%%

\usepackage[section]{algorithm}
\usepackage[noend]{algpseudocode}

\renewcommand{\algorithmicrequire}{\textbf{Input:}}
\renewcommand{\algorithmicensure}{\textbf{Output:}}

% natbib %%%%%%%%%%%%%%%%%%%%%%%%%%%%%%%%%%%%%%%%%%%%%%%%%%%%%%%%%%%%%%%%%%%%%%

\usepackage{natbib}

\bibpunct{[}{]}{,}{n}{}{}

% url %%%%%%%%%%%%%%%%%%%%%%%%%%%%%%%%%%%%%%%%%%%%%%%%%%%%%%%%%%%%%%%%%%%%%%%%%

\usepackage{url}

\makeatletter
\def\url@leostyle{%
  \@ifundefined{selectfont}{\def\UrlFont{\sf}}{\def\UrlFont{\small\ttfamily}}}
\makeatother
\urlstyle{leostyle}

% Enumeration %%%%%%%%%%%%%%%%%%%%%%%%%%%%%%%%%%%%%%%%%%%%%%%%%%%%%%%%%%%%%%%%%

\usepackage{paralist}

% TODO Seb
%(Why) are these better?
%\setlength{\pltopsep}{0.24em}
%\setlength{\plpartopsep}{0em}
%\setlength{\plitemsep}{0.24em}

% This should do what we want
%   \setdefaultenum{(i)}{(a)}{1.}{A}
% but it does not work for references, dropping the parentheses.  The following
% hack does work.

\renewcommand{\theenumi}{(\roman{enumi})}
\renewcommand{\theenumii}{(\alph{enumii})}
\renewcommand{\theenumiii}{\arabic{enumiii}.}
\renewcommand{\theenumiv}{\Alph{enumiv}}

\renewcommand{\labelenumi}{\theenumi}
\renewcommand{\labelenumii}{\theenumii}
\renewcommand{\labelenumiii}{\theenumiii}
\renewcommand{\labelenumiv}{\theenumiv}

%%%%%%%%%%%%%%%%%%%%%%%%%%%%%%%%%%%%%%%%%%%%%%%%%%%%%%%%%%%%%%%%%%%%%%%%%%%%%%%
% Mathematics

% Packages %%%%%%%%%%%%%%%%%%%%%%%%%%%%%%%%%%%%%%%%%%%%%%%%%%%%%%%%%%%%%%%%%%%%

\usepackage{amsmath,amsthm,amscd,amsfonts,amssymb}
\usepackage{cases}
\usepackage[all]{xy}

\allowdisplaybreaks[4]
\numberwithin{equation}{section}

% Customised notation %%%%%%%%%%%%%%%%%%%%%%%%%%%%%%%%%%%%%%%%%%%%%%%%%%%%%%%%%

\providecommand{\card}[1]{\lvert#1\rvert}                % Cardinality

\providecommand{\abs}[1]{\lvert#1\rvert}                 % Absolute value
\providecommand{\absbig}[1]{\bigl\lvert#1\bigr\rvert}    % Absolute value
\providecommand{\absBig}[1]{\Bigl\lvert#1\Bigr\rvert}    % Absolute value
\providecommand{\absbigg}[1]{\biggl\lvert#1\biggr\rvert} % Absolute value

\providecommand{\norm}[1]{\lVert#1\rVert}              % Norm
\providecommand{\normbig}[1]{\bigl\lVert#1\bigr\rVert} % Norm
\providecommand{\normBig}[1]{\Bigl\lVert#1\Bigr\rVert} % Norm

\providecommand{\floor}[1]{\left\lfloor#1\right\rfloor}   % Floor
\providecommand{\floorts}[1]{\lfloor#1\rfloor}            % Floor
\providecommand{\floorbig}[1]{\bigl\lfloor#1\bigr\rfloor} % Floor
\providecommand{\floorBig}[1]{\Bigl\lfloor#1\Bigr\rfloor} % Floor

\providecommand{\ceil}[1]{\left\lceil#1\right\rceil}   % Ceiling
\providecommand{\ceilts}[1]{\lceil#1\rceil}            % Ceiling
\providecommand{\ceilbig}[1]{\bigl\lceil#1\bigr\rceil} % Ceiling
\providecommand{\ceilBig}[1]{\Bigl\lceil#1\Bigr\rceil} % Ceiling

\newcommand{\NN}{\mathbf{N}} % Natural numbers
\newcommand{\ZZ}{\mathbf{Z}} % Integers
\newcommand{\QQ}{\mathbf{Q}} % Rationals
\newcommand{\RR}{\mathbf{R}} % Real numbers
\newcommand{\CC}{\mathbf{C}} % Complex numbers
\newcommand{\FF}{\mathbf{F}} % Finite field

\renewcommand{\to}{\rightarrow}        % Right arrow
\newcommand{\into}{\hookrightarrow}    % Injection arrow
\newcommand{\onto}{\twoheadrightarrow} % Surjection arrow

\DeclareMathOperator{\fCoKer}{coker} % Cokernel
\DeclareMathOperator{\fKer}{ker}     % Kernel
\DeclareMathOperator{\fIm}{im}       % Image

\DeclareMathOperator{\Res}{Res}   % Resultant
\DeclareMathOperator{\Tr}{Tr}     % Trace
\DeclareMathOperator{\Trace}{Tr}  % Trace
\DeclareMathOperator{\Norm}{N}    % Norm
\DeclareMathOperator{\Disc}{Disc} % Discriminant

\DeclareMathOperator{\Gal}{Gal}          % Galois group
\DeclareMathOperator{\ord}{ord}          % Order
\DeclareMathOperator{\sgn}{sgn}          % Sign, signature
\DeclareMathOperator{\Frob}{F}           % Frobenius
\DeclareMathOperator{\Hom}{Hom}          % Space of homomorphisms
\DeclareMathOperator{\Spec}{Spec}        % Spectrum

\providecommand{\HdR}{H_{\text{dR}}}    % de Rham cohomology
\providecommand{\Het}{H_{\text{\'et}}}  % etale cohomology
\providecommand{\Hrig}{H_{\text{rig}}}  % rigid cohomology

\providecommand{\cB}{\mathcal{B}} % Basis
\providecommand{\cR}{\mathcal{R}} % Row index set
\providecommand{\cC}{\mathcal{C}} % Column index set
\providecommand{\cM}{\mathcal{M}} % Complexity of multiplication

\providecommand{\BigOh}{\mathcal{O}}          % Big-oh notation
\providecommand{\SoftOh}{\tilde{\mathcal{O}}} % Soft-oh notation

% Theorems etc %%%%%%%%%%%%%%%%%%%%%%%%%%%%%%%%%%%%%%%%%%%%%%%%%%%%%%%%%%%%%%%%

\theoremstyle{definition}

\newtheorem{thm}{Theorem}[section]
\newtheorem{lem}[thm]{Lemma}
\newtheorem{prop}[thm]{Proposition}
\newtheorem{cor}[thm]{Corollary}
\newtheorem{defn}[thm]{Definition}
\newtheorem{exmp}[thm]{Example}
\newtheorem{rem}[thm]{Remark}
\newtheorem{prob}[thm]{Problem}
\newtheorem{assump}[thm]{Assumption}

% Roman numerals %%%%%%%%%%%%%%%%%%%%%%%%%%%%%%%%%%%%%%%%%%%%%%%%%%%%%%%%%%%%%%

\makeatletter
\newcommand{\rmnum}[1]{\romannumeral #1}
\newcommand{\Rmnum}[1]{\expandafter\@slowromancap\romannumeral #1@}
\makeatother

%%%%%%%%%%%%%%%%%%%%%%%%%%%%%%%%%%%%%%%%%%%%%%%%%%%%%%%%%%%%%%%%%%%%%%%%%%%%%%%
% DOCUMENT                                                                    %
%%%%%%%%%%%%%%%%%%%%%%%%%%%%%%%%%%%%%%%%%%%%%%%%%%%%%%%%%%%%%%%%%%%%%%%%%%%%%%%

\begin{document}

\maketitle

\setcounter{tocdepth}{2}
\tableofcontents

%%%%%%%%%%%%%%%%%%%%%%%%%%%%%%%%%%%%%%%%%%%%%%%%%%%%%%%%%%%%%%%%%%%%%%%%%%%%%%%

\section{Introduction}
\label{sec:Introduction}

Let $\FF_q$ denote a finite field with $q$ elements, where $q$ is a power of 
the prime number~$p$, and let $X$ denote an algebraic variety over~$\FF_q$. 

\begin{defn}
The zeta function of $X$ is the formal power series
\[
Z(X,T) = \exp \Bigl(\sum_{i=1}^{\infty} \card{X(\FF_{q^i})} \frac{T^i}{i} \Bigr).
\]
\end{defn}

As we will see in the next section, $Z(X,T)$ 
is a rational function, i.e.\ it is contained in $\QQ(T)$, and 
hence can be given by a finite amount of data. Therefore, it is natural 
to ask whether it can be computed effectively and, in fact, it is not 
hard to provide an algorithm as follows. Using well known bounds 
by Bombieri~\citep{Bombieri1966} for the degrees of the numerator and 
denominator of $Z(X,T)$, one can reduce the computation of $Z(X,T)$ to that 
of a finite number of the $\card{X(\FF_{q^i})}$, which can be determined 
by naive counting.

A more interesting problem is whether, and if so how, $Z(X,T)$ can be 
computed efficiently, where `efficiently' can mean with low time complexity 
or just fast in practice. When $X$ is a (hyper-)elliptic curve, this problem
is important in cryptographic applications and has been the subject of 
much attention, resulting in very efficient algorithms.  For example, when 
$X$ is an elliptic curve, Schoof's algorithm~\citep{Schoof1995}, which uses 
$\ell$-adic \'etale cohomology, has runtime polynomial in~$\log(q)$, and is 
also very fast in practice using improvements due to Atkins and Elkies.

For more general algebraic varieties~$X$, the only available option is 
usually to compute the rigid cohomology spaces of~$X$, with their natural 
action of the Frobenius map, and then use a Lefschetz formula to deduce the 
zeta function. This method was introduced by Kedlaya in the case of 
hyperelliptic curves in odd characteristic~\citep{Kedlaya2001}.  The same 
idea has been shown to work in much greater generality, for example for 
smooth projective hypersurfaces \citep{AbbottKedlayaRoe2006}. 

Lauder \citep{Lauder2004a,Lauder2004b} showed that instead of computing the 
action of the Frobenius map on the rigid cohomology spaces of a smooth 
projective hypersurface~$X$ directly, it is better, at least in terms of 
time complexity, to embed $X$ in a family of smooth projective hypersurfaces 
containing a diagonal hypersurface.  Following his deformation method, 
one first computes the action of the Frobenius map on the rigid cohomology 
of the diagonal hypersurface, and then solves a $p$-adic differential equation 
to obtain the Frobenius map on the rigid cohomology of the original 
hypersurface. 

To be more precise, in~\citep{Lauder2004a,Lauder2004b} Lauder did not directly 
work with rigid cohomology but with Dwork cohomology. While the two 
cohomology theories are equivalent, various comparison and finiteness results 
are more easily stated and proved in the context of rigid cohomology.  In~\citep{Gerkmann2007} 
Gerkmann reformulated Lauder's deformation method in terms of rigid 
cohomology.  Moreover, he improved various precision bounds, making 
the algorithm more practical, which he demonstrated with many examples. Kedlaya 
introduced a lot of new ideas and results to further lower precision bounds for 
the deformation method in \citep{Kedlaya2012}.

The aim of this paper is to continue where Lauder, Gerkmann and Kedlaya left 
off. We make improvements to about every step of the algorithm. This results 
in an algorithm with both lower time and space complexity than Lauder's 
original algorithm, but perhaps more importantly, which is a lot more efficient
in practice. The first author has written a (publicly available) 
implementation of our algorithm using the library FLINT~\citep{FLINT}. 
This implementation lowers the runtimes of the examples in~\citep{Gerkmann2007} 
by a factor of $50$ to $5000$. Moreover, it can be used to compute the zeta 
function in many cases where this was not possible before, e.g. for generic 
quartic surfaces.

We now briefly describe the contents of the remaining sections,
necessarily relying on some terminology that is introduced only 
in Section~\ref{sec:Background}.  The reader who is not familiar 
with this terminology might prefer to start reading there.

In Section~\ref{sec:Background}, we recall the main theoretical results that 
underpin the remaining sections of the paper.  We limit ourselves to the bare 
minimum as there are already good references available for the relevant theory, 
see e.g.~\citep{Kedlaya2012}. We also introduce the required terminology and 
notation. We conclude the section with an overview of the different steps of 
the deformation method, which are then treated individually in the next four 
sections.

In Section~\ref{sec:Connection}, we explain how to compute the Gauss--Manin 
connection on the cohomology of a family of smooth projective hypersurfaces. 
We define an explicit monomial basis for the cohomology that we will use 
throughout the paper. Our most important result in this section
is Theorem~\ref{thm:Isomorphism}, which allows us to compute very efficiently 
in the cohomology. We formalise the computation of the Gauss--Manin connection 
matrix in Algorithm~\ref{alg:Connection}. We also prove some lower bounds for 
the valuation of the matrix of Frobenius and its inverse that are essential 
for controlling the $p$-adic precision loss in the algorithm.

In Section~\ref{sec:Diagonal}, we show how to compute the Frobenius matrix of
a diagonal hypersurface over a prime field. Our method is based on a formula 
of Dwork, but by rewriting this formula we obtain Algorithm~\ref{alg:Diagfrob}, 
which is a significant improvement to the corresponding algorithms of Lauder 
and Gerkmann, both in terms of time complexity and in practice.

In Section~\ref{sec:DifferentialSystem}, we explain how to solve the 
differential equation for the Frobenius matrix. We use the same method as 
Lauder but incorporate improved convergence bounds for $p$-adic differential 
equations by Kedlaya. We collect the precision bounds that follow from our 
analysis in Theorem~\ref{thm:Ni} and formalise the computation of the power 
series expansion of the Frobenius matrix in Algorithm~\ref{alg:expansion}.

In Section~\ref{sec:ZetaFunctions}, we describe how to evaluate the Frobenius 
matrix at some fibre and compute its zeta function. We combine various bounds 
from different sources to lower the required $p$-adic and $t$-adic precisions. 
Finally, this results in Algorithm~\ref{alg:complete}, which combines all of 
our previous algorithms, and is the main result of the paper.

In Section~\ref{sec:Complexity}, we analyse the time and space complexity of 
our algorithm and compare these to Lauder's work~\citep{Lauder2004a}. 
In Section~\ref{sec:Examples}, we compute various numerical examples, and 
compare our runtimes to those provided by Gerkmann~\citep{Gerkmann2007}. 

%%%%%%%%%%%%%%%%%%%%%%%%%%%%%%%%%%%%%%%%%%%%%%%%%%%%%%%%%%%%%%%%%%%%%%%%%%%%%%
\section{Theoretical background}
\label{sec:Background}

We start by recalling the main result about the zeta function of algebraic
varieties over finite fields.

\begin{thm}[Weil conjectures] If $X/\FF_q$ is a smooth projective variety of 
dimension~$m$, then \label{thm:weildeligne}
\[
Z(X,T)=\frac{p_1 p_3 \dotsm p_{2m-1}}{p_0 p_2 p_4 \dotsm p_{2m}},
\]
where for all $i$:
\begin{enumerate}
\item $p_i = \prod_j (1-\alpha_{i,j}T) \in \ZZ[T]$, 
\item the transformation $t \rightarrow q^m/t$ maps the $\alpha_{i,j}$ 
      bijectively to the $\alpha_{2m-i,k}$, preserving multiplicities,
\item $\abs{\alpha_{i,j}} = q^{i/2}$ for all $j$, and every embedding 
      $\bar{\QQ} \hookrightarrow \CC$. 
\end{enumerate}
\end{thm}

\begin{proof}
The proof of this theorem was completed by Deligne in \citep{Deligne1974}.
\end{proof}

We let $\QQ_q$ denote the unique unramified extension of $\QQ_p$ with 
residue field $\FF_q$ and $\ZZ_q$ its ring of integers. We denote
the $p$-adic valuation on $\QQ_q$ by $\ord_p(-)$. 

\begin{defn}
Let $\Hrig^{i}(X)$ denote the rigid
cohomology spaces of $X$. These are finite dimensional vector spaces 
over $\QQ_q$ that are contravariantly functorial in $X$, and they are 
equipped with an action of the $p$-th and $q$-th power Frobenius map 
on $X$ that we denote by $\Frob_p$ and $\Frob_q$, respectively. For the 
construction and basic properties of these spaces we refer 
to~\citep{Berthelot1986}.
\end{defn}

The relation between the zeta function and the rigid cohomology spaces 
is given by the so called Lefschetz formula.

\begin{thm}[Lefschetz formula] \label{thm:Lefschetz}
If $X$ is a smooth proper algebraic variety over $\FF_q$ of dimension~$m$, 
then 
\[
Z(X,T) = \prod_{i=0}^{2m} \det(1- T \Frob_q | \Hrig^i(X))^{(-1)^{i+1}}.
\]
\end{thm}

\begin{proof}
See for example \citep[Theorem 6.3]{EtesseLeStum1993}.
\end{proof}

Let $\pi:\mathfrak{X} \rightarrow \mathfrak{S}$ be a smooth family of algebraic varieties 
defined over $\QQ_q$.

\begin{defn}
Let $\HdR^i(\mathfrak{X}/\mathfrak{S})$ denote the relative algebraic 
de Rham cohomology sheaf on $\mathfrak{S}$. 
If $\mathfrak{X}/\mathfrak{S}$ admits a relative normal crossing 
compactification, then this is a vector bundle.
\end{defn}

The $\HdR^i(\mathfrak{X}/\mathfrak{S})$ come equipped with an integrable 
connection, which is called the Gauss--Manin connection. Let us first recall 
the notion of a connection on a vector bundle.

\begin{defn}
Let $\mathfrak{E}$ be a vector bundle on $\mathfrak{S}$. A connection on 
$\mathfrak{E}$ is a map of vector bundles 
$\nabla: \mathfrak{E} \rightarrow \Omega^1_{\mathfrak{S}} \otimes \mathfrak{E}$
which satisfies the Leibniz rule
\begin{align*}
\nabla(f e)=f\nabla(e)+df \otimes e
\end{align*} 
for all local sections~$f$ of $\mathcal{O}_{\mathfrak{S}}$ and $e$ 
of $\mathfrak{E}$.
\end{defn}

The Gauss--Manin connection on $\HdR^i(\mathfrak{X}/\mathfrak{S})$ can 
be defined as follows.

\begin{defn}
The de Rham complex $\Omega^{\bullet}_{\mathfrak{X}}$ can be equipped 
with the decreasing filtration
\[
F^i=\fIm(\Omega^{\bullet-i}_{\mathfrak{X}} \otimes \pi^* \Omega^i_{\mathfrak{S}} \rightarrow \Omega^{\bullet}_{\mathfrak{X}}). 
\]
The spectral sequence associated to this filtration has as its first sheet 
\[
E_1^{p,q}=\Omega^p_{\mathfrak{S}} \otimes \HdR^q(\mathfrak{X}/\mathfrak{S}).
\]
The Gauss--Manin connection 
$\nabla:H^i(\mathfrak{X}/\mathfrak{S}) \rightarrow \Omega^1_{\mathfrak{S}} \otimes H^i(\mathfrak{X}/\mathfrak{S})$ 
is now defined as the differential $d_1: E_1^{0,i} \rightarrow E_1^{1,i}$ 
in this spectral sequence.
\end{defn}

\begin{rem}
We can give a more explicit description of $\nabla$ when 
$\mathfrak{X}/\mathfrak{S}$ is affine. If we lift a relative $i$-cocycle 
$\omega \in \Omega^i_{\mathfrak{X}/\mathfrak{S}}$ to an absolute $i$-form 
$\omega' \in \Omega^i_{\mathfrak{X}}$ and apply the absolute differential~$d$, 
we get an element of 
$\Omega^1_{\mathfrak{S}} \wedge \Omega^i_{\mathfrak{X}/\mathfrak{S}}$. 
Projecting onto 
$\Omega^1_{\mathfrak{S}} \otimes \HdR^i(\mathfrak{X}/\mathfrak{S})$, 
we obtain $\nabla(\omega)$. 
\end{rem}

\begin{defn} \label{defn:sigma}
We write $\sigma$ for the standard $p$-th power Frobenius lift on 
$\mathbf{P}^1_{\QQ_q}$, i.e.\ the semilinear map that lifts the $p$-th power 
Frobenius map on $\mathbf{P}^1_{\FF_q}$ and satisfies $\sigma(t)=t^p$. 
\end{defn}

Now suppose that $\mathfrak{E}$ is a vector bundle with connection on 
some Zariski open subset~$\mathfrak{S}$ of $\mathbf{P}^1_{\QQ_q}$ with 
complement~$\mathfrak{Z}$. Let $V$ denote the rigid analytic subspace 
of $\mathbf{P}^1_{\QQ_q}$ which is the complement of the union of the 
open disks of radius~$1$ around the points of $\mathfrak{Z}$.

\begin{defn}
A Frobenius structure on $\mathfrak{E}$ is an isomorphism of vector bundles 
with connection $F \colon \sigma^* \mathcal{E} \rightarrow \mathcal{E}$ 
defined on some strict neighbourhood of $V$. 
\end{defn}

\begin{thm} \label{thm:frobstruc}
Let $\mathcal{S}$ be a Zariski open subset of $\mathbf{P}^1_{\ZZ_q}$ and 
suppose that $\mathcal{X}/\mathcal{S}$ is a smooth family of algebraic 
varieties that admits a relative normal crossing compactification. Denote 
the generic fibres of $\mathcal{S}$, $\mathcal{X}$ by 
$\mathfrak{S}=\mathcal{S} \otimes \QQ_q$, $\mathfrak{X}=\mathcal{X} \otimes \QQ_q$ 
and the special fibres by $S=\mathcal{S} \otimes \FF_q$, 
$X=\mathcal{X} \otimes \FF_q$, respectively. The vector bundle 
$\HdR^i(\mathfrak{X}/\mathfrak{S})$ with the Gauss--Manin connection $\nabla$ 
admits a Frobenius structure $F$ with the following property. 
For any finite field extension $\FF_{\mathfrak{q}}/\FF_q$ and
all $\tau \in S(\FF_{\mathfrak{q}})$,
\[
(\Hrig^i(X_{\tau}),\Frob_p) \cong (\HdR^i(\mathfrak{X}/\mathfrak{S}),F)_{\hat{\tau}}
\] 
as $\QQ_{\mathfrak{q}}$ vector spaces with a $\sigma$-semilinear 
endomorphism, where $\hat{\tau} \in \mathcal{S}(\ZZ_{\mathfrak{q}})$ denotes 
the Teichm\"uller lift of $\tau$. We will therefore denote this Frobenius 
structure on $\HdR^i(\mathfrak{X}/\mathfrak{S})$ by $\Frob_p$ as well.
\end{thm}

\begin{proof}
This theorem is well known, and can be obtained by combining results of 
Berthelot from various papers, most notably \citep{Berthelot1986}.   
\end{proof}

\begin{defn}
Let $\Hrig^i(X/S)$ denote the vector bundle $\HdR^i(\mathfrak{X}/\mathfrak{S})$ 
with its Frobenius structure $\Frob_p$ from Theorem~\ref{thm:frobstruc}.
\end{defn}

\begin{rem}
One can show that $\Hrig^i(X/S)$ is again functorial in $X/S$ and so does 
not depend on the lift $\mathcal{X}/\mathcal{S}$. Moreover, one can still 
define $\Hrig^i(X/S)$ when $X/S$ cannot be lifted to characteristic zero, 
see~\citep{Berthelot1986}.  However, for our purposes the above definition 
will be sufficient.
\end{rem}

In this paper we restrict our attention to one-parameter families of smooth 
projective hypersurfaces. So we let $P \in \ZZ_q[t][x_0,\dotsc,x_n]$ denote 
a homogeneous polynomial of degree $d$ and let 
$\mathcal{S} \subset \mathbf{P}^1_{\ZZ_q}$ be a Zariski open subset such that 
$P$ defines a family $\mathcal{X}/\mathcal{S}$ of smooth hypersurfaces contained 
in $\mathbf{P}^n_{\mathcal{S}}$. 
%TODO I was thinking about including this, but only makes things messier 
%\begin{rem} 
%We let $\mathfrak{r} \in \ZZ_q[t]$ denote
%the Macaulay resultant  \citep[Chapter 1]{Macaulay1994} 
%of the sequence of homogeneous polynomials
%$(P,x_0 \frac{\partial P}{\partial x_0},\dotsc,x_n\frac{ \partial P}{\partial x_n})$.
%The zero locus of $\mathfrak{r}$ contains the points
%$t \in \mathbf{P}^1_{\ZZ_q} \setminus \{\infty\}$
%for which $X_t$ is not smooth. Hence we can take $\mathcal{S}$ 
%to be the complement of the closed set $\{\mathfrak{r}=0 \} \cup \{\infty\}$
%in $\mathbf{P}^1_{\ZZ_q}$.   
%\end{rem}
We let $\mathcal{U}/\mathcal{S}$ 
denote the complement of $\mathcal{X}/\mathcal{S}$ in $\mathbf{P}^n_{\mathcal{S}}$,
and write $\mathfrak{X}=\mathcal{X} \otimes \QQ_q$, 
$\mathfrak{U}=\mathcal{U} \otimes \QQ_q$, $\mathfrak{S}=\mathcal{S} \otimes \QQ_q$
for the generic fibres, and $X=\mathcal{X} \otimes \FF_q$, 
$U=\mathcal{U} \otimes \QQ_q$, $S=\mathcal{S} \otimes \FF_q$ for the special 
fibres of $\mathcal{X},\mathcal{U}$, $\mathcal{S}$, respectively. Moreover, 
we let $\FF_{\mathfrak{q}}/\FF_q$ denote a finite field extension and 
denote $a=\log_p(\mathfrak{q})$.

\begin{thm} \label{thm:hypersurface} 
For all $\tau \in S(\FF_{\mathfrak{q}})$, we have
\begin{equation} \label{eq:formulazeta}
Z(X_{\tau},T) = \frac{\chi(T)^{(-1)^n}}{(1 - T) (1 - \mathfrak{q}T) \dotsm (1 - \mathfrak{q}^{n-1}T)},
\end{equation}
where 
$\chi(T) = \det \bigl( 1 - T \mathfrak{q}^{-1} \Frob_{\mathfrak{q}} | \Hrig^n(U_{\tau}) \bigr) \in \ZZ[T]$
denotes the reverse characteristic polynomial of the action of 
$\mathfrak{q}^{-1} \Frob_{\mathfrak{q}}$ on $\Hrig^n(U_{\tau})$. 
Moreover, the polynomial $\chi(T)$ has degree 
\begin{equation} \label{eq:formulab}
\frac{1}{d} \bigl((d-1)^{n+1} + (-1)^{n+1}(d-1) \bigr).
\end{equation}
\end{thm}

\begin{proof}
This theorem is well known, see for example~\citep{AbbottKedlayaRoe2006}. 
Since we need some intermediate results from the proof later on, we will 
give a brief sketch here. First, by Theorem~\ref{thm:Lefschetz}, we have 
\[
Z(X_{\tau},T) = \prod_{i=0}^{2(n-1)} \det(1- T \Frob_{\mathfrak{q}} | \Hrig^i(X_{\tau}))^{(-1)^{i+1}}.
\]
Then, by the Lefschetz hyperplane theorem and Poincar\'e duality, we see 
that $\Hrig^i(X_{\tau}) \cong \Hrig^i(\mathbf{P}^n_{\FF_{\mathfrak{q}}})$ 
for all $i \neq (n-1)$. Next, one shows by a computation that
\begin{equation} \label{eqn:projcoho}
\Hrig^i(\mathbf{P}^n_{\FF_{\mathfrak{q}}}) 
\cong 
\begin{cases}
\QQ_{\mathfrak{q}}(i) &\mbox{if $i$ even,} \\
$0$ &\mbox{if $i$ odd,} 
\end{cases} 
\end{equation}
where $(i)$ denotes the $i$-th Tate twist, for which $\Frob_p$ is multiplied 
by $p^{-i}$. It remains to determine $\Hrig^{n-1}(X_{\tau})$. One uses the 
excision short exact sequence
\begin{equation} \label{eqn:excision}
\begin{CD}
0 @>>> \Hrig^{n}(U_{\tau}) @>>> \Hrig^{n-1}(X_{\tau})(-1) @>>> \Hrig^{n+1}(\mathbf{P}^n_{\FF_{\mathfrak{q}}}) @>>> 0
\end{CD} 
\end{equation}
to relate $\Hrig^{n-1}(X_{\tau})$ to $\Hrig^{n}(U_{\tau})$ and complete 
the proof of~\eqref{eq:formulazeta}. We will show in Proposition~\ref{prop:rankcoho}
that the dimension of $\Hrig^{n-1}(U_{\tau})$ is given by~\eqref{eq:formulab}.
\end{proof}

Let $[e_1, \dotsc, e_b]$ be some basis of sections of 
$\HdR^n(\mathfrak{U}/\mathfrak{S})$, and let $M \in M_{b \times b}(\QQ_q(t))$ 
denote the matrix of the Gauss--Manin connection $\nabla$ with respect 
to this basis, i.e.
\[
\nabla (e_j) = \sum_{i=1}^b M_{i,j} e_i.
\]
Let $r \in \ZZ_q[t]$ with $\ord_p(r)=0$ be a denominator for $M$, i.e.\ such that we can write 
$M = G/r$ with $G \in M_{b \times b}(\QQ_q[t])$, and let $\Phi$ denote the 
matrix of $p^{-1}\Frob_p$ with respect to the basis $[e_1, \dotsc, e_b]$, i.e.\
\[
p^{-1} \Frob_p (e_j) = \sum_{i=1}^b \Phi_{i,j} e_i.
\]

\begin{defn}
We define the ring of overconvergent functions
\[
\QQ_q \langle t, \frac{1}{r} \rangle^{\dag} = 
\biggl\{\sum_{i,j=0}^{\infty} a_{i,j} \frac{t^i}{r^j} \; : \; 
a_{i,j} \in \QQ_q, \; \exists c > 0 \text{ s.t.}  
\lim_{i+j \rightarrow \infty} \bigl(\ord_p(a_{i,j}) - c(i+j)\bigr) \geq 0
\biggr\},
\]
as the $p$-adically meromorphic functions on $\mathbf{P}^1_{\QQ_q}$ that are 
analytic outside of the open disks of radius $\rho$ around the zeros of $r$ 
and the point at infinity for some $\rho<1$. 
\end{defn}

\begin{defn}
We extend the $p$-adic valuation 
to $\QQ_q \langle t, 1/r \rangle^{\dag}$ in the standard way, i.e.\ 
$\ord_p(f)$ is defined as the maximum, over all ways of writing 
$f = \sum_{i,j=0}^{\infty} a_{i,j} t^i / r^j$,
of the minimum, over $i,j \geq 0$, of $\ord_p(a_{i,j})$. Note that the  
norm on $\QQ_q \langle t, 1/r \rangle^{\dag}$ corresponding to this valuation
is the Gauss norm. We also extend $\ord_p(-)$ to polynomials and matrices over 
$\QQ_q \langle t, 1/r \rangle^{\dag}$, by taking the 
minimum over the coefficients and entries, respectively.
\end{defn}

\begin{assump} \label{assump:S}
From now on we will always assume that $0 \in \mathcal{S}$. 
Note that if this is not the case, then it can be achieved by 
applying a translation.
\end{assump}

\begin{thm} \label{thm:eqphi} 
The matrix $\Phi$ is an element of 
$M_{b \times b}(\QQ_q \langle t, 1/r \rangle^{\dag})$ 
and satisfies the differential equation
\begin{align*}
\Bigl(\frac{d}{dt} + M\Bigr) \Phi &= p t^{p-1} \Phi \sigma(M), &\Phi(0)& = \Phi_0,
\end{align*}
where $\Phi_0$ is the 
matrix of $p^{-1}\Frob_p$ on $\Hrig^n(U_0)$ with respect to the 
basis $[e_0,\dotsc,e_b]$.
\end{thm}

\begin{proof}
That the differential equation is satisfied is an immediate consequence of the 
fact that $\Frob_p$ is a horizontal map of vector bundles with connection, and 
that $\Phi(0)=\Phi_0$ is also clear from Theorem~\ref{thm:frobstruc}. Now, if 
$M$ does not have any poles in a given residue disk, then $\Phi$ cannot have 
any poles in that residue disk either, by Theorem~\ref{thm:KedlayaTuitman} 
below. Hence the entries of $\Phi$ are contained in 
$\QQ_q \langle t, 1/r \rangle^{\dag}$.
\end{proof}

The deformation method can now be sketched as follows:
\begin{enumerate}[\it Step 1.]
\item Compute the matrix $M$ of the Gauss--Manin connection $\nabla$.
\item Compute the matrix $\Phi_0$ of the action of $p^{-1}\Frob_p$ on 
      $\Hrig^n(U_0)$. If the family is chosen such that $X_0$ is a diagonal 
      hypersurface, this can be achieved using a formula of Dwork.
\item Solve the differential equation from Theorem~\ref{thm:eqphi} for $\Phi$.
\item Substitute the Teichm\"uller lift $\hat{\tau}$ of an element 
      $\tau \in S(\FF_{\mathfrak{q}})$ into $\Phi$ to obtain the 
      matrix $\Phi_{\tau}$ of the action of $p^{-1}\Frob_p$ on 
      $\Hrig^n(U_{\tau})$. Compute the matrix $\Phi_{\tau}^{(a)}$ 
      of the action of $\mathfrak{q}^{-1} \Frob_{\mathfrak{q}}$ 
      on~$\Hrig^n(U_{\tau})$, which is also equal to $(p^{-1}\Frob_p)^a$.
      Use Theorem~\ref{thm:hypersurface} to compute the zeta function 
      $Z(X_{\tau},T)$ of the fibre $X_{\tau}$.
\end{enumerate}

%This way one can `deform' the computation for some complicated fibre $X_{\tau}$ to that 
%for some easier diagonal fibre, by solving the $p$-adic differential equation from 
%Theorem~\ref{thm:eqphi}.

Note that we can only carry out these computations to finite $p$-adic 
precision. Therefore, we need to recall some bounds on the loss of $p$-adic 
precision when multiplying $p$-adic numbers and matrices. 

\begin{prop} \label{prop:productval}
Let $v_1,\dotsc,v_{\ell} \in \ZZ$ and $x_1, \dotsc, x_{\ell} \in \mathbf{Q}_q$,  
$\ell \geq 2$, be such that $\ord_p(x_i) \geq v_i$ for all $i$. Suppose that 
$N \in \ZZ$ satisfies $N \geq \sum_{j=1}^{\ell} v_j$. Let 
$\tilde{x}_1, \dotsc, \tilde{x}_{\ell}$ denote $p$-adic approximations to 
$x_1, \dotsc, x_{\ell}$ such that 
\[
\ord_p(x_i - \tilde{x}_i) \geq N - \sum_{j \neq i} v_j
\] 
for all $i$.  Then 
\begin{equation*}
\ord_p(x_1 \dotsm x_{\ell} - \tilde{x}_1 \dotsm \tilde{x}_{\ell}) \geq N.
\end{equation*}
\end{prop}

\begin{proof}
For all $i$,
\begin{align*}
\ord_p(\tilde{x}_i) &\geq \min \{ \ord_p(x_i-\tilde{x}_i), \ord_p(x_i) \} \\
                    &\geq \min \{ N- \sum_{j \neq i} v_j, \ord_p(x_i)\} \geq v_i.
\end{align*}
Therefore, we also have that
\begin{equation*}
\ord_p \bigl( (x_{i}-\tilde{x}_{i})
    (\tilde{x}_1 \dotsm \tilde{x}_{i-1} x_{i+1} \dotsm x_{\ell} \bigr) \geq N,
\end{equation*}
for all $i$. The result now follows by combining these inequalities.
\end{proof}

\begin{prop} \label{prop:matrixproductval}
Let $v_1,\dotsc,v_{\ell} \in \ZZ$ and 
$A_1, \dotsc, A_{\ell} \in M_{b \times b}(\QQ_q)$, $\ell \geq 2$, be 
such that $\ord_p(A_i) \geq v_i$ for all $i$. Suppose that $N \in \ZZ$ 
satisfies $N \geq \sum_{j=1}^{\ell} v_j$. 
Let $\tilde{A}_1, \dotsc, \tilde{A}_{\ell}$ denote $p$-adic approximations 
to $A_1, \dotsc A_{\ell}$ such that
\[
\ord_p(A_i - \tilde{A}_i) \geq N - \sum_{j \neq i} v_j
\]
for all $i$.  Then 
\begin{equation}
\ord_p(A_1 \dotsm A_{\ell} - \tilde{A}_1 \dotsm \tilde{A}_{\ell}) \geq N.
\end{equation}
\end{prop}

\begin{proof}
We can follow the proof of Proposition~\ref{prop:productval}, 
observing that for matrices $A,B \in M_{b \times b}(\QQ_q)$, we still 
have that $\ord_p(A + B) \geq \min \{\ord_p(A), \ord_p(B)\}$ and 
$\ord_p(AB) \geq \ord_p(A)+\ord_p(B)$.
\end{proof}

%%%%%%%%%%%%%%%%%%%%%%%%%%%%%%%%%%%%%%%%%%%%%%%%%%%%%%%%%%%%%%%%%%%%%%%%%%%%%%%
\section{Computing the connection matrix}
\label{sec:Connection}

In this section we compute the action of the Gauss--Manin connection $\nabla$ 
on the algebraic de~Rham cohomology space $\HdR^{n}(\mathfrak{U}/\mathfrak{S})$ 
of the generic fiber 
$\mathfrak{U}/\mathfrak{S}=\mathcal{U}/\mathcal{S} \otimes \QQ_q$ of the 
complement $\mathcal{U}/\mathcal{S}$ of a family of smooth hypersurfaces 
$\mathcal{X}/\mathcal{S}$ contained in $\mathbf{P}^n_{\mathcal{S}}$ over
some Zariski open subset $\mathcal{S} \subset \mathbf{P}^1_{\ZZ_q}$. Let 
$\mathcal{X}/\mathcal{S}$ be defined by a homogeneous polynomial 
$P \in \ZZ_q[t][x_0,\dotsc,x_n]$ of degree $d$. First we recall how to 
compute in $\HdR^{n}(\mathfrak{U}/\mathfrak{S})$ following the method 
of Griffiths and Dwork.  

\begin{prop}
Let $\Omega$ denote the $n$-form on $\mathfrak{U}/\mathfrak{S}$ defined by 
\begin{equation*}
\Omega = \sum_{i=0}^n (-1)^i x_i d x_0 \wedge \dotsb \wedge \widehat{d x_i} \wedge \dotsb \wedge d x_n.
\end{equation*}
The algebraic de~Rham cohomology space $\HdR^{n}(\mathfrak{U}/\mathfrak{S})$ 
is isomorphic to the quotient of the space of closed $n$-forms 
$Q \Omega / P^k$ with $k \in \NN$ and 
$Q \in H^0(\mathfrak{S},\mathcal{O}_{\mathfrak{S}})[x_0, x_1, \dotsc, x_n]$ 
homogeneous of degree $k d - (n + 1)$, by the subspace of exact $n$-forms 
generated by
\begin{equation*} 
\frac{(\partial_i Q) \Omega}{P^k} - k \frac{Q (\partial_i P) \Omega}{P^{k+1}},
\end{equation*}
for all $0 \leq i \leq n$ with $k \in \NN$ and 
$Q \in H^0(\mathfrak{S}, \mathcal{O}_{\mathfrak{S}})[x_0, x_1, \dotsc, x_n]$ 
homogeneous of degree $kd-n$, where $\partial_i$ denotes the partial 
derivative operator with respect to~$x_i$.
\end{prop}

\begin{proof}
The proof is straightforward, for details see \citep{Griffiths1969}.
\end{proof}

The cohomology space $\HdR^{n}(\mathfrak{U}/\mathfrak{S})$ is 
equipped with an increasing filtration by the pole order, for which 
$\mbox{Fil}^k \HdR^{n}(\mathfrak{U}/\mathfrak{S})$ consists of all elements 
that can be represented by $n$-forms $Q \Omega / P^k$ with 
$Q \in H^0(\mathfrak{S},\mathcal{O}_{\mathfrak{S}})[x_0, x_1, \dotsc, x_n]$ 
homogeneous of degree $kd - (n + 1)$.  It follows from a theorem of 
Macaulay~\citep[\S 4, (4.11)]{Griffiths1969} that 
$\mbox{Fil}^n \HdR^{n}(\mathfrak{U}/\mathfrak{S}) = \HdR^{n}(\mathfrak{U}/\mathfrak{S})$. 
Actually, the reverse filtration $H_i=\mbox{Fil}^{n-i} \HdR^{n}(\mathfrak{U}/\mathfrak{S})$
corresponds to the restriction of the Hodge filtration on 
$\HdR^{n-1}(\mathfrak{X}/\mathfrak{S})$ to $\HdR^{n}(\mathfrak{U}/\mathfrak{S})$ 
by~\citep{Griffiths1969}.

As we prefer to perform linear 
algebra operations over a field, we will actually work with the de~Rham 
cohomology vector space $\HdR^{n}\bigl(\mathfrak{U}_{\QQ_q(t)}\bigr)$ of 
the generic fibre 
\[
\mathfrak{U}_{\QQ_q(t)} = \mathfrak{U}/\mathfrak{S} \times_{\mathfrak{S}} \Spec \QQ_q(t).
\] 
We now define an explicit basis of a simple form for 
$\HdR^{n}\bigl(\mathfrak{U}_{\QQ_q(t)}\bigr)$ 
for the families that we are interested in.

\begin{defn} \label{defn:MonBasis}
For $k \in \NN$, we define the following sets of monomials: 
\begin{align*}
F_k & = \{ x^u : u \in \mathbf{N}_{0}^{n+1}, \abs{u} = k d - (n+1) \}, \\
B_k & = \{ x^u : u \in \mathbf{N}_{0}^{n+1}, \abs{u} = k d - (n+1) \text{ and $u_i < d-1$ for all $i$}\}, \\
R_k & = F_k - B_k,
\end{align*}
where $x^u = x_0^{u_0} \dotsm x_n^{u_n}$ and $\abs{u}=\sum_{i=0}^n u_i$. 
We also define 
\begin{equation*}
\cB_k = \{Q \Omega / P^k : Q \in B_k\}, 
\end{equation*}
and write $B = B_1 \cup \dotsb \cup B_n$ and $\cB = \cB_1 \cup \dotsb \cup \cB_n$.
\end{defn}

We will show below that if the family $\mathcal{X}/{\mathcal{S}}$ contains 
a diagonal fibre, then the set $\cB$ forms a basis for 
$\HdR^n\bigl(\mathfrak{U}_{\QQ_q(t)}\bigr)$.

\begin{defn} \label{defn:IndexSets}
For $k \in \NN$, let $C_k^{(0)}$ be the set of monomials of total 
degree $(k-1)d - n$ and then inductively, for $1 \leq j \leq n$, define 
$C_k^{(j)}$ to be the set of monomials in $C_k^{(j-1)}$ except for those 
divisible by $x_{j-1}^{d-1}$.  Moreover, we define the multi-set $C_k$ as 
the disjoint union of $C_k^{(0)}, \dotsc, C_k^{(n)}$.  We shall write an 
element of this multi-set as $(j, g)$, when referring to a monomial~$g$ 
in~$C_k^{(j)}$.
\end{defn}

\begin{lem} \label{lem:bijection}
For all $k \in \NN$, the multi-sets $R_k$ and $C_k$ 
have the same cardinality.
\end{lem}

\begin{proof}
We construct a bijection $R_k \to C_k$, representing the 
monomials by their exponent tuples.  Let $u = (u_0, \dotsc, u_n)$ be an
element of $R_k$.  If $u_0 \geq d-1$, we define the image as
$(u_0-d-1, u_1, \dotsc, u_n) \in C_k^{(0)}$.  More generally, if 
$u_0 < d-1, \dotsc, u_{j-1} < d-1$ and $u_j \geq d-1$, we define the image as 
$(u_0, \dotsc, u_{j-1}, u_j-(d-1), u_{j+1}, \dotsc, u_n) \in C_k^{(j)}$.  
It is easy to verify that this map is indeed a bijection.
\end{proof}

\begin{defn} \label{defn:Deltak}
We define a square matrix $\Delta_k$ with 
row and column index sets $R_k$ and $C_k$ as follows.  
Given $f \in R_k$ and $(j,g) \in C_k$, we set the corresponding entry in 
$\Delta_k$ to be the coefficient of the monomial $f/g$ in $\partial_j P$ if 
$g$ divides $f$ and $0$ otherwise.
\end{defn}

\begin{thm} \label{thm:Isomorphism}
Suppose that the family $\mathcal{X}/\mathcal{S}$ of smooth projective
hypersurfaces given by the polynomial~$P$ in $\ZZ_q[t][x_0, \dotsc, x_n]$ contains 
a diagonal fibre.  For $k \in \NN$ and $0 \leq j \leq n$, let $U_k^{(j)}$ be 
the $\QQ_q(t)$-vector space of polynomials with basis $C_k^{(j)}$, and let $U_k$ 
denote the cartesian product $U_k = U_k^{(0)} \times \dotsb \times U_k^{(n)}$. 
Moreover, let $V_k$ and $W_k$ be the $\QQ_q(t)$-vector spaces of polynomials with 
bases $F_k$ and $R_k$, respectively, and let $\pi \colon V_k \rightarrow W_k$ 
denote the linear map that sends the elements of $B_k$ to zero and the 
elements of $R_k$ to themselves. %safer formulation, projection not unique
Then the map 
\begin{align}
\phi_k \colon U_k &\to W_k,
&(Q_0, \dotsc, Q_n) &\mapsto \pi \bigl( Q_0 \partial_0 P + \dotsb + Q_n \partial_n P \bigr)
\end{align}
is an isomorphism of $\QQ_q(t)$-vector spaces.
\end{thm}

\begin{proof}
Recall that $R_k$ and $C_k$ have the same cardinality by Lemma~\ref{lem:bijection}.
If $R_k$ and $C_k$ are empty, then $U_k$ and $W_k$ are the zero vector spaces, 
and the theorem holds trivially. So suppose that $R_k$ and $C_k$ are nonempty. 
It is immediate that $\Delta_k$ is the matrix 
representing $\phi_k$ with respect to the bases $C_k$ and $R_k$ of $U_k$ and $W_k$, 
respectively.

The assumption that the family~$\mathcal{X}/\mathcal{S}$ contains a diagonal 
hypersurface means that for some~$t_0 \in \mathcal{S}(\ZZ_q)$, 
the fibre $\mathcal{X}_{t_0}$ is defined by a polynomial of the form 
\begin{equation*}
P_{t_0}(x_0, \dotsc, x_n) = a_0 x_0^d + \dotsb + a_n x_n^d
\end{equation*}
with $a_0, \dotsc, a_n \in \ZZ_q^{\times}$.

We now show that the determinant of $\Delta_k$ is nonzero.  Since 
evaluation of the matrix at \mbox{$t = t_0$} commutes with computing the 
determinant, it suffices to show that the determinant of 
$(\Delta_k) \big |_{t=t_0}$ is nonzero. Since, for $0 \leq j \leq n$, 
we have $\partial_j P_{t_0} (x_0, \dotsc, x_n) = d a_j x_j^{d-1}$, there is 
precisely one nonzero entry in each column and each row of~$\Delta_k$.  
Namely, in column $(j, g) \in C_k$ and row $g x_j^{d-1} \in R_k$ there is 
the nonzero entry $d a_j$. Note that this also implies that 
$(\Delta_k) \big |_{t=t_0} \in \ZZ_{q}^{\times}$.
\end{proof}

We can use Theorem~\ref{thm:Isomorphism} to give a routine {\sc Decompose}, 
formalised in Algorithm~\ref{alg:Decompose}, which given 
$Q \in \QQ_q(t)[x_0, \dotsc, x_n]$ homogeneous of degree $kd - (n+1)$
returns an expression 
\begin{equation*}
Q = Q_0 \partial_0 P + \dotsb + Q_n \partial_n P + \gamma_k
\end{equation*} 
with $Q_0, \dotsc, Q_n \in \QQ_q(t)[x_0, \dotsc, x_n]$ homogeneous of 
degree $kd-n$ and $\gamma_k$ in the $\QQ_q(t)$-span of~$B_k$. We can in turn 
use {\sc Decompose} to furnish another routine {\sc Reduce}, formalised 
in Algorithm~\ref{alg:PoleRed}, which given a closed $n$-form $Q\Omega/P^k$ 
with $Q \in \QQ_q(t)[x_0, \dotsc, x_n]$ homogeneous of degree $kd - (n+1)$ returns 
an expression
\begin{equation*}
\frac{Q \Omega}{P^k} \equiv \frac{\gamma_{1} \Omega}{P^{1}} 
                            + \dotsb + \frac{\gamma_n \Omega}{P^n},
\end{equation*}
with $\gamma_i$ in the $\QQ_q(t)$-span of $B_i$ for $1 \leq i \leq n$ and 
where $\equiv$ denotes equality in cohomology.


\begin{algorithm}
\caption{Obtain coordinates in the Jacobian ideal modulo basis elements}
\label{alg:Decompose}
\begin{algorithmic}
\Require $P$ in $\ZZ_q[t][x_0, \dotsc, x_n]$ homogeneous of degree~$d$, 
         defining a family $\mathcal{X}/\mathcal{S}$ of smooth projective 
         hypersurfaces that contains a diagonal fibre, 
         $Q \in \QQ_q(t)[x_0, \dotsc, x_n]$ homogeneous of degree $kd - (n+1)$.
\Ensure  $Q_0, \dotsc, Q_n \in \QQ_q(t)[x_0, \dotsc, x_n]$ homogeneous of degree 
         $k(d-1)-n$, and $\gamma_k$ in the $\QQ_q(t)$-span of $B_k$, such that 
         $Q = Q_0 \partial P_0 + \dotsb + Q_n \partial_n P +\gamma_k$.
\Procedure{Decompose}{$P,Q$}
\State \begin{compactenum}[{\hspace{1em}} 1.] \vspace{-1.24em}
\item Let $w$ be the vector of length $\abs{R_k}$ such that the entry 
      corresponding to $x^u \in R_k$ is the coefficient of 
      $x^u$ in $Q$.
\item Solve for the unique vector $v$ of length $\abs{C_k}$ satisfying 
      $\Delta_k v = w$.  We write $v$ 
      as $\bigl(v^{(0)}, \dotsc, v^{(n)}\bigr)$, where $v^{(j)}$ is 
      a vector of length $\abs{C_k^{(j)}}$ for $0 \leq j \leq n$,
      and let $v_g^{(j)}$ be the entry in $v^{(j)}$ corresponding 
      to $g \in C_k^{(j)}$.
\item For $0 \leq j \leq n$, compute $Q_j \gets \sum_{g \in C_k^{(j)}} v_g^{(j)} g$.
\item Set $\gamma_k \gets Q-(Q_0 \partial P_0 + \dotsb + Q_n \partial_n P)$.
\item \textbf{return} $Q_0, \dotsc, Q_n,\gamma_k$      
\EndProcedure
\end{compactenum}
\end{algorithmic}
\end{algorithm}


\begin{algorithm}
\caption{Reduce $Q \Omega / P^k$ in $\HdR^n\bigl(\mathfrak{U}_{\QQ_q(t)}\bigr)$}
\label{alg:PoleRed}
\begin{algorithmic}
\vspace{1mm}
\Require $P$ in $\ZZ_q[t][x_0, \dotsc, x_n]$ homogeneous of degree~$d$, 
         defining a family $\mathcal{X}/\mathcal{S}$ of smooth projective 
         hypersurfaces that contains a diagonal fibre, $Q \in \QQ_q(t)[x_0, \dotsc, x_n]$ 
         homogeneous of degree $kd - (n+1)$.
\Ensure  $\gamma_i$ in the $\QQ_q(t)$-span of $B_i$ for $1 \leq i \leq n$, with  
         $Q \Omega / P^k \equiv \gamma_{1} \Omega / P^{1} + \dotsb + \gamma_n \Omega / P^n$.
\Procedure{Reduce}{$P,Q$}
\While{$k \geq n+1$}
\State $Q_0, \dotsc, Q_n, \bullet \gets \Call{Decompose}{Q}$
\State $k \gets k-1$
\State $Q \gets k^{-1} \sum_{i=0}^n \partial_i Q_i$
\EndWhile
\While{$Q \not \in \QQ_q(t)$-span of $B_k$}
\State $Q_0, \dotsc, Q_n, \gamma_k \gets \Call{Decompose}{Q}$
\State $k \gets k-1$
\State $Q \gets k^{-1} \sum_{i=0}^n \partial_i Q_i$
\EndWhile
\If{$Q \neq 0$}
\State $\gamma_{k} \gets Q$
\State $k \gets k-1$
\EndIf
\State $\gamma_{1}, \dotsc, \gamma_{k} \gets 0$
\State \textbf{return} $\gamma_{1}, \dotsc, \gamma_n$
\EndProcedure
\end{algorithmic}
\end{algorithm}

We now establish that the set~$\cB$ indeed forms a basis for 
$\HdR^n\bigl(\mathfrak{U}_{\QQ_q(t)}\bigr)$, as announced before.  
We start with an auxiliary result describing the cardinality of 
the set~$\cB$.

\begin{prop} \label{prop:BasisSize}
The set $\cB$ has cardinality
\begin{equation*}
\frac{1}{d} \bigl((d-1)^{n+1} + (-1)^{n+1}(d-1) \bigr).
\end{equation*}
\end{prop}

\begin{proof}
First note that if we denote
\begin{align*}
V   &= \{(u_0,\dotsc,u_n) \in (\ZZ/d\ZZ)^{n+1} : \sum_{j=0}^n u_j = -(n+1)\}, \\
W_j &= \{(u_0,\dotsc,u_n) \in (\ZZ/d\ZZ)^{n+1} : u_j = -1 \},
\end{align*}
then  $\cB$ is in one-to-one correspondence with the set $V-(W_0 \cup \dotsb \cup W_n)$. 
Now by the inclusion-exclusion principle, 
\begin{align*}
\abs{V \cap (W_0 \cup \dotsb \cup W_n)} 
& = \sum_{j=0}^n \abs{V \cap W_j} 
    - \sum_{0 \leq j < k \leq n} \abs{V \cap W_j \cap W_k} \\
& \quad + \dotsb + (-1)^{n} \abs{V \cap W_0 \cap \dotsb \cap W_n} \\
& = {n+1 \choose 1} d^{n-1} -{n+1 \choose 2} d^{n-2} 
    + \dotsb + (-1)^{n-1} {n+1 \choose n} + (-1)^{n} \\
& = \frac{1}{d} \bigl(d^{n+1}+(-1)^{n+1} - (d-1)^{n+1}\bigr)+(-1)^n,
\end{align*}
so that
\begin{align*}
\abs{V-(W_0 \cup \dotsb \cup W_n)}&=\abs{V}-\abs{V \cap (W_0 \cup \dotsb \cup W_n)} \\
&= d^n - \frac{1}{d} \bigl(d^{n+1}+(-1)^{n+1} - (d-1)^{n+1}+d (-1)^n \bigr) \\
&= \frac{1}{d} \bigl((d-1)^{n+1} + (-1)^{n+1}(d-1) \bigr),
\end{align*}
and the proof is complete.
\end{proof}

\begin{prop} \label{prop:rankcoho}
The rank of $\HdR^n(\mathfrak{U}/\mathfrak{S})$ is
\[
\frac{1}{d} \bigl((d-1)^{n+1} + (-1)^{n+1}(d-1) \bigr).
\]
\end{prop}

\begin{proof}
Let $\mathfrak{U}_m/\mathfrak{S}$ denote the complement of a family
$\mathfrak{X}_m/\mathfrak{S}$ of smooth projective hypersurfaces of 
degree~$d$ in $\mathbf{P}^m_{\mathfrak{S}}$ and let $b(d,m)$ denote 
the rank of $\HdR^m(\mathfrak{U}_m/\mathfrak{S})$. It is known that 
$b(d,m)$ only depends on $d$, $m$. Moreover, taking $x=y=z$ 
in~\citep[Corollaire~2.4~(i)]{sga7}, we find that
\[
\sum_{m=1}^{\infty} b(d,m) x^{m-1} 
  = \frac{(d-1)}{(1+x)(1-(d-1)x)} 
  = \frac{1}{xd} \Bigl( \frac{d-1}{1-(d-1)x} - \frac{d-1}{1+x} \Bigr)
\]
as formal power series. From this the result follows easily. 
\end{proof}

\begin{thm} \label{thm:Basis}
Suppose that the family of smooth projective hypersurfaces 
$\mathcal{X}/\mathcal{S}$ contains a diagonal fibre.  Then the 
set~$\cB$ from Definition~\ref{defn:MonBasis} is a basis for 
the $\QQ_q(t)$-vector space $\HdR^n\bigl(\mathfrak{U}_{\QQ_q(t)}\bigr)$.
\end{thm}

\begin{proof}
We already know that $\HdR^n\bigl(\mathfrak{U}_{\QQ_q(t)}\bigr)$ is 
spanned by the classes of the $n$-forms $Q \Omega / P^k$ with 
$Q \in \QQ_q(t)[x_0, \dotsc, x_n]$ homogeneous of degree 
$kd - (n+1)$ for $k \in \NN$. Applying Algorithm~\ref{alg:PoleRed}, we 
obtain an expression for the class of $Q \Omega / P^k$ as a 
$\QQ_q(t)$-linear combination of elements in $\cB$.  This shows that 
$\cB$ spans the vector space $\HdR^n\bigl(\mathfrak{U}_{\QQ_q(t)}\bigr)$. 
However, by the two propositions above,the dimension of 
$\HdR^n\bigl(\mathfrak{U}_{\QQ_q(t)}\bigr)$ is equal to the cardinality of
$\cB$, so that $\cB$ is linearly independent as well.
\end{proof}

\begin{rem} \label{rem:hnumbers}
Let $H_i$ denote the restriction of the Hodge filtration on 
$\HdR^{n-1}(\mathfrak{X}/\mathfrak{S})$ to $\HdR^{n}(\mathfrak{U}/\mathfrak{S})$
and write 
\[
h^{i,n-1-i}=\mbox{rank}(H^i/H^{i+1})
\] 
for the corresponding Hodge numbers.
Recall that $H_i=\mbox{Fil}^{n-i} \HdR^{n}(\mathfrak{U}/\mathfrak{S})$
for the filtration by the pole order defined in Section~\ref{sec:Connection}. Applying 
the loop in Algorithm~\ref{alg:PoleRed} just once, to lower the pole 
order from $k$ to $k-1$, we see that $\cB_k$ spans 
\[
\mbox{Fil}^{k} \HdR^{n}\bigl(\mathfrak{U}_{\QQ_q(t)}\bigr) / \mbox{Fil}^{k-1} \HdR^{n}\bigl(\mathfrak{U}_{\QQ_q(t)}\bigr)
\]
for all $1 \leq k \leq n$. Hence it follows that $\card{B_k} \geq h^{n-k,k-1}$ , but since
\[
\sum_{k=1}^n \card{B_k} = \dim \HdR^n\bigl(\mathfrak{U}_{\QQ_q(t)}\bigr) = \sum_{k=1}^n h^{n-k,k-1},
\] 
this implies that $\card{B_k} = h^{n-k,k-1}$ for all $1 \leq k \leq n$. So the Hodge numbers
can be read off from the basis $\cB$.
\end{rem}

We now describe the action of the Gauss--Manin connection~$\nabla$ on 
$\HdR^n\bigl(\mathfrak{U}_{\QQ_q(t)}\bigr)$.  Suppose that we are given a basis element 
$x^u \Omega / P^k \in \cB_k$.  Following the description in 
Section~\ref{sec:Background}, we compute
\begin{equation} \label{eqn:nabla}
\nabla \biggl(\frac{x^u \Omega}{P^k}\biggr) \equiv 
    dt \otimes \frac{- k x^u (\partial P / \partial t) \Omega}{P^{k+1}},
\end{equation}
where $\equiv$ denotes equality in 
$\Omega_{\QQ_q(t)} \otimes \HdR^n\bigl(\mathfrak{U}_{\QQ_q(t)}\bigr)$. 
We apply Algorithm~\ref{alg:PoleRed} in order to write
\begin{equation}
dt \otimes \frac{- k x^u (\partial P / \partial t) \Omega}{P^{k+1}} \equiv 
dt \otimes \left( \frac{\gamma_{1}}{P} + \dotsb + \frac{\gamma_n}{P^n} \right) \Omega,
\end{equation}
where $\gamma_i$ is an element in the $\QQ_q(t)$-span of~$B_i$ for $1 \leq i \leq n$. 

\begin{rem} \label{rem:precgm}
For the matrix $M$ of $\nabla$ to be correct to $p$-adic precision $N_M$, we have to 
carry out this computation to a somewhat higher working precision $N'_M$, because of 
precision loss in Algorithm~\ref{alg:PoleRed}. 
For all $k \in \NN$, by definition $\ord_p(\Delta_k) \geq 0$, and from the proof
of Theorem~\ref{thm:Isomorphism} we know that $\ord_p(\det(\Delta_k))=0$. 
Therefore, there is no precision loss 
in Algorithm~\ref{alg:Decompose}, and the only precision loss in Algorithm~\ref{alg:PoleRed} 
comes from dividing by $k-1$ for $k=2,\dotsc,n+1$. Hence it is sufficient to take 
\begin{equation*}
N_M'=N_M + \ord_p(n!).
\end{equation*}
\end{rem}
This is formalised in Algorithm~\ref{alg:Connection}.

\begin{algorithm}
\caption{Compute the Gauss--Manin connection matrix}
\label{alg:Connection}
\begin{algorithmic}
\Require $P$ in $\ZZ_q[t][x_0, \dotsc, x_n]$ homogeneous of degree~$d$, 
         defining a family $\mathfrak{X}/\mathfrak{S}$ of smooth projective 
         hypersurfaces that contains a diagonal fibre, $p$-adic precision $N_M$.
\Ensure  The matrix~$M$ of $\nabla$ with respect to $\cB$ to $p$-adic precision $N_M$.
\Procedure{GMConnection}{$P,N_M$}
\State $N'_M \gets N_M + \ord_p(n!)$
\State \textit{Execute the following steps with $p$-adic working precision $N'_M$}:
\For{$g \in B$} 
\State $k \gets  \bigl(\deg(g)+(n+1)\bigr)/d$
\State $Q \gets  - k g (\partial P / \partial t)$
\State $\gamma_{1}, \dotsc, \gamma_n \gets$ {\sc Reduce($P,Q$)} 
\For{$f \in B$}
\State $l \gets \bigl(\deg(f)+(n+1)\bigr)/d$
\State $M_{f,g} \gets$ coefficient of $f$ in $\gamma_l$
\EndFor
\EndFor
\Return $M$
\EndProcedure
\end{algorithmic}
\end{algorithm}

\begin{defn} \label{defn:resultant}
We define the polynomial $R \in \ZZ_q[t]$ by
\[
R = \prod_{k=2}^{n+1}  \det(\Delta_k).
\]
\end{defn}

\begin{prop} \label{thm:denom}
The matrix $M$ of $\nabla$ with respect to $\cB$ is of the form
$H/R$, with $H \in M_{b \times b}(\QQ_q[t])$.
\end{prop}

\begin{proof}
The only time in Algorithm~\ref{alg:Connection} that nonconstant denominators 
are introduced is when the subroutine {\sc{Reduce}} calls its subroutine 
{\sc{Decompose}} and a linear system $\Delta_k v = w$ is solved. Since this 
happens only for $k=2, \dotsc, n+1$ and at most once for each 
such~$k$, the result is clear.
\end{proof}

\begin{rem}
Recall that $\cB$ is a basis for $\HdR^n\bigl(\mathfrak{U}_{\QQ_q(t)}\bigr)$ 
but not necessarily for $\HdR^n(\mathfrak{U}/\mathfrak{S})$. However, if the 
zero locus of $R$ in $\mathbf{P}^1_{\ZZ_q}$ does not intersect $\mathcal{S}$, 
then $\cB$ is a basis for $\HdR^n(\mathfrak{U}/\mathfrak{S})$. That it 
spans the cohomology can be seen by applying Algorithm~\ref{alg:PoleRed}, and that it 
is linearly independent follows because it is so over $\QQ_q(t)$. Therefore, it will 
be convenient to choose $\mathcal{S}$ smaller, so that this condition is satisfied. 
\end{rem}

\begin{assump} \label{assump:R}
From now on we assume that the zero locus of~$R$ in~$\mathbf{P}^1_{\ZZ_q}$ 
does not intersect $\mathcal{S}$. In particular, this implies that $\cB$ is 
a basis for $\HdR^n(\mathfrak{U}/\mathfrak{S})$.
\end{assump}

\begin{rem}
Note that in Algorithm~\ref{alg:Connection}, we solve $\BigOh(\card{\cB})$~linear 
systems given by the matrices $\Delta_k$ for $k = 2, \dotsc, n+1$.  In practice, 
it is important to take advantage of this, e.g.\ by computing the decomposition 
$\Delta_k = LUP$, where $L$ is lower triangular, $U$ is upper triangular, and 
$P$ is a permutation matrix, which is guaranteed to exist for any square matrix. 
Since, moreover, the matrices~$\Delta_k$ contain many zeros, methods from sparse 
linear algebra can be used to compute such a decomposition.  Then, every call to 
{\sc{Decompose}} reduces to solving a lower and an upper diagonal linear system.
\end{rem}

In the rest of this section we will freely use the definitions and notation 
from Section~\ref{sec:Background}. We let all of our matrices be defined with 
respect to the basis $\cB$. We have seen that there exists a Frobenius 
structure $\Frob_p$ on $\HdR^n(\mathfrak{U}/\mathfrak{S})$ and let 
$\Phi \in M_{b \times b}(\QQ_q \langle t, 1/r \rangle^{\dag})$ 
denote the matrix of the action of $p^{-1}\Frob_p$, where $r$ is the 
denominator of the connection matrix $M$. Recall that 
$\FF_{\mathfrak{q}}/\FF_q$ 
denotes a finite field extension. 

We need a priori lower bounds on the $p$-adic valuations of the matrices 
$\Phi$ and $\Phi^{-1}$ to bound the loss of $p$-adic precision in our 
computations. We will now recall how such bounds can be obtained 
following~\citep{AbbottKedlayaRoe2006}.

\begin{thm} \label{thm:deltabound}
For any finite field extension $\FF_{\mathfrak{q}}/\FF_q$ and
all $\tau \in S(\FF_{\mathfrak{q}})$, 
there exists a matrix $W_{\tau} \in M_{b \times b}(\QQ_{\mathfrak{q}})$ such that 
\begin{align*}
\ord_p(W_{\tau}) &\geq -\sum_{i=1}^{n-1} \floor{\log_p(i)},
&\ord_p\bigl(W_{\tau} \Phi_{\tau} \sigma(W_{\tau}^{-1})\bigr) &\geq  0, \\
\ord_p(W_{\tau}^{-1}) &\geq -\ord_p \bigl((n-1)! \bigr),
&\ord_p\bigl(W_{\tau} \Phi_{\tau}^{-1} \sigma(W_{\tau}^{-1})\bigr) &\geq  -(n-1).
\end{align*}
\end{thm}

\begin{proof}
Let $\Lambda_{\tau, crys}$ be the image of the integral logarithmic de Rham 
cohomology space 
$\HdR^n\bigl(\mathbf{P}_{\mathbf{Z}_{\mathfrak{q}}}^n,\mathcal{X}_{\hat{\tau}}\bigr)$ 
in the rigid cohomology space $\Hrig^n(U_{\tau})$. It is known that
$\Lambda_{\tau, crys} \otimes \QQ_{\mathfrak{q}} \cong \Hrig^n(U_{\tau})$, 
so that $\Lambda_{\tau, crys}$ is a lattice in $\Hrig^n(U_{\tau})$.  Let 
$\Lambda_{\tau, mon}$ be the $\ZZ_{\mathfrak{q}}$-module in $\Hrig^n(U_{\tau})$ 
generated by the classes of $x^u \Omega/P^n$ with $u \in \NN_0^{n+1}$ such that 
$\sum_{i=0}^n u_i = nd-(n+1)$. Since these classes generate $\Hrig^n(U_{\tau})$, we have 
that $\Lambda_{\tau,mon}$ is a lattice in $\Hrig^n(U_{\tau})$ as well. We know 
that
\[
\Lambda_{\tau,crys} \subset \Lambda_{\tau,mon} \subset p^{-\sum_{i=1}^{n-1} \floor{\log_p(i)}} \Lambda_{\tau,crys}.
\]
The inclusion on the left is \citep[Lemma 3.4.3]{AbbottKedlayaRoe2006}, and the 
one on the right is \citep[Proposition 3.4.6]{AbbottKedlayaRoe2006}.

Let $\Lambda_{\tau,\cB}$ be the $\ZZ_{\mathfrak{q}}$-module in $\Hrig^n(U_{\tau})$ 
generated by the basis $\cB$. Since $\cB$ spans $\Hrig^n(U_{\tau})$, we have that 
$\Lambda_{\tau,\cB}$ is also a lattice in $\Hrig^n(U_{\tau})$. Now we know that
\[
(n-1)! \Lambda_{\tau,mon} \subset \Lambda_{\tau,\cB} \subset \Lambda_{\tau,mon}. 
\]
The inclusion on the left follows by explicitly reducing the generators of 
$\Lambda_{\tau,mon}$ to the basis $\cB$ with Algorithm~\ref{alg:PoleRed} using 
that $R(\hat{\tau}) \in \ZZ_{\mathfrak{q}}^{\times}$, and the inclusion on the right is clear.

Combining these inclusions of lattices, we find that
\begin{equation} \label{eq:lattices}
(n-1)! \Lambda_{\tau,crys} \subset \Lambda_{\tau,\cB} \subset p^{-\sum_{i=1}^{n-1} \floor{\log_p(i)}} \Lambda_{\tau,crys}.
\end{equation}
Now let $[d_1, \dotsc, d_b]$ be a $\ZZ_{\mathfrak{q}}$-basis for 
$\Lambda_{\tau,crys}$ and let $W_{\tau} \in M_{b \times b}(\QQ_q)$ be 
the matrix in which the $i$-th column 
consists of the coordinates of $d_i$ with respect to the basis $\cB$. From the 
inclusions~\eqref{eq:lattices} it is clear that 
\begin{eqnarray*}
\ord_p(W_{\tau}) &\geq& -\sum_{i=1}^{n-1} \floor{\log_p(i)}, \\
\ord_p(W_{\tau}^{-1}) &\geq& -\ord_p((n-1)!).
\end{eqnarray*}
Note that $W_{\tau} \Phi_{\tau} \sigma(W_{\tau}^{-1})$ is the matrix of 
$p^{-1}\Frob_{p}$ on $\Hrig^n(U_{\tau})$ with respect to the basis 
$[d_1,\dotsc,d_b]$. Now $\Lambda_{\tau,crys}$ is contained in the crystalline 
cohomology space $H^{n-1}_{crys}(X_{\tau})$, which maps to itself 
under~$\Frob_p$. So by the short exact sequence~\eqref{eqn:excision}, the 
lattice $\Lambda_{\tau,crys}$ maps to itself under $p^{-1}\Frob_{p}$, and
\[
\ord_p(W_{\tau} \Phi_{\tau} \sigma(W_{\tau}^{-1})) \geq 0.
\]
Similarly, note that $W_{\tau} \Phi_{\tau}^{-1} \sigma(W_{\tau}^{-1})$ is 
the matrix of $p\Frob_p^{-1}$ on $\Hrig^n(U_{\tau})$ with respect to the 
basis $[d_1,\dotsc,d_b]$. By Poincar\'e duality, the map $p^{n-1}\Frob_p^{-1}$ 
maps the crystalline cohomology space $H^{n-1}_{crys}(X_{\tau})$ to itself. 
So by the short exact sequence~\eqref{eqn:excision}, the lattice 
$\Lambda_{\tau,crys}$ maps to itself under $p^{n-1} (p\Frob_p^{-1})$, and 
\begin{equation*}
\ord_p(W_{\tau} \Phi_{\tau}^{-1} \sigma(W_{\tau}^{-1})) \geq -(n-1). \qedhere
\end{equation*}
\end{proof}

\begin{defn} \label{defn:delta}
Define the nonnegative integer
\[
\delta = \ord_p((n-1)!)+\sum_{i=1}^{n-1} \floor{\log_p(i)}.
\]
\end{defn}

\begin{cor} \label{cor:delta} We have 
\begin{align*}
\ord_p(\Phi) &\geq -\delta, \\
\ord_p(\Phi^{-1}) &\geq -\delta-(n-1).
\end{align*}
\end{cor}

\begin{proof}
For every $\tau \in \mathcal{S}(\bar{\FF}_{\mathfrak{q}})$, 
we can apply Theorem~\ref{thm:deltabound} 
to obtain
\begin{align*}
\ord_p(\Phi_{\tau}) &\geq \ord_p(W^{-1}_{\tau}) + 
                          \ord_p(W_{\tau} \Phi_{\tau} \sigma(W_{\tau}^{-1})) + 
                          \ord_p(\sigma(W_{\tau})) \\
                    &\geq -\delta,
\end{align*}
and also
\begin{align*}
\ord_p(\Phi^{-1}_{\tau}) &\geq \ord_p(W^{-1}_{\tau}) + 
                               \ord_p(W_{\tau} \Phi^{-1}_{\tau} \sigma(W_{\tau}^{-1})) + 
                               \ord_p(\sigma(W_{\tau})) \\
                         &\geq -\delta - (n-1).
\end{align*}
As these inequalities hold for infinitely many~$\tau \in S(\bar{\FF}_q)$ 
we obtain the required bounds.
\end{proof}

%%%%%%%%%%%%%%%%%%%%%%%%%%%%%%%%%%%%%%%%%%%%%%%%%%%%%%%%%%%%%%%%%%%%%%%%%%%%%%%

\section{Frobenius on diagonal hypersurfaces}
\label{sec:Diagonal}

\subsection{A formula of Dwork}

In this section we compute the action of Frobenius on the cohomology 
space $\Hrig^{n}(U_0) \cong \HdR^{n}(\mathfrak{U}_0)$ associated 
to the diagonal fibre of the family. Our method is based on an 
explicit formula of Dwork~\citep[\S 4]{Dwork1964}, which has already 
been used by Lauder \citep{Lauder2004b} and Gerkmann 
\citep{Gerkmann2007}. However, by rewriting this formula we obtain an 
algorithm that performs significantly better in practice than a direct 
implementation.

We consider a single smooth 
projective diagonal hypersurface~$\mathcal{X}_0$ over $\ZZ_p$ defined by 
a polynomial $P_0 \in \ZZ_p[x_0, \dotsc, x_n]$ of the form
\begin{equation*}
P_0(x_0, x_1, \dotsc, x_n) = 
    a_0 x_0^d + a_1 x_1^d + \dotsb + a_n x_n^d,
\end{equation*}
where $a_0, a_1, \dotsc, a_n \in \ZZ_p^{\times}$ and $p \nmid d$. 
Let $\mathfrak{X}_0 = \mathcal{X}_0 \otimes_{\ZZ_p} \QQ_p$ denote the generic 
fibre of $\mathcal{X}_0$ and let $\mathcal{U}_0$ and $\mathfrak{U}_0$ be the 
complements of $\mathcal{X}_0$ and $\mathfrak{X}_0$, respectively. 
We fix our choice of basis~$\cB$ for $\HdR^{n}(\mathfrak{U}_0)$ 
as in Definition~\ref{defn:MonBasis}. 

Our goal is to compute 
the matrix~$\Phi_0$ representing the action of $p^{-1} \Frob_p$ on 
$\Hrig^n(U_0) \cong \HdR^n(\mathfrak{U}_0)$, with respect to the basis~$\cB$, 
to $p$-adic precision~$N_{\Phi_0}$. It will turn out that this matrix has only one
nonzero element in every row and column.

We define the ramified extension~$\QQ_p(\pi)$ where $\pi^{p-1} = -p$, 
and extend the valuation such that \mbox{$\ord_p(\pi) = (p-1)^{-1}$}.

Let $u = (u_0, \dotsc, u_n)$ and $v = (v_0, \dotsc, v_n)$ be tuples 
of integers such that $x^u, x^v \in B$ and $p (u_i+1) \equiv v_i+1 \pmod{d}$,
for all $i$. Furthermore, let $k(u) \in \NN$ denote the positive integer such that 
\[
k(u) d -(n+1) = \sum_{i=0}^n u_i.
\] 
For $w \in \QQ$ 
and $r \geq 0$, we define the rising factorials $(w)_r = \prod_{j=0}^{r-1} (w + j)$, 
and for $m \geq 0$, we let $\lambda_m$ denote the coefficient of $z^m$ in the 
power series expansion of $\exp \bigl( \pi (z - z^p) \bigr)$. It is well known, see 
for example~\citep{Dwork1962}, that 
\begin{equation} \label{eqn:dworkbound}
\ord_p(\lambda_m) \geq \frac{p-1}{p^2} m. 
\end{equation}

\begin{defn} We define \label{defn:alpha}
\[
\alpha_{u,v} = \pi^{k(v) - k(u)} \prod_{i = 0}^n \biggl( \sum_{m, r} \lambda_m ((u_i + 1) / d)_r (-1)^r \pi^{-r} a_i^{m-r} \biggr),
\]
where the sum in the $i$-th factor of the product is over all $m, r \geq 0$  
that satisfy
\[
p(u_i+1)-(v_i+1)=d(m-pr).
\]
\end{defn}

\begin{thm} \label{thm:01-03-diagfrob}
Let $\omega_1$ denote an element of $\cB$ corresponding to a tuple 
$u \in \ZZ^{n+1}$ and let $\omega_2$ denote the unique element of~$\cB$ 
corresponding to a tuple $v \in \ZZ^{n+1}$ such that
$p (u_i + 1) \equiv v_i + 1 \pmod{d}$ for all $i$. Then
\begin{equation*}
p^{-1} \Frob_p (\omega_1) = 
    (-1)^{k(u) + k(v)} \frac{(k(v) - 1)!}{(k(u) - 1)!} p^n \alpha_{u,v}^{-1} \omega_2.
\end{equation*}
\end{thm}

\begin{proof}
The corresponding formula for the Dwork operator on Dwork cohomology can
be found in %\citep[\S 4]{Dwork1964}, or alternatively in 
\citep[\S 6.1]{Lauder2004b}. The more general case when $\mathcal{X}_0$ 
is defined over $\ZZ_q$ is also treated there. The formula for 
$p^{-1} \Frob_p$ on $\Hrig^n(U_0)$ follows from this by the comparison theorem 
between Dwork cohomology and rigid cohomology \citep[Theorem 1.12]{Katz}. 
For more details, see~\citep[Theorem 4.4]{Gerkmann2007}.
\end{proof}

\begin{rem}
We could eliminate $m$ from Definition~\ref{defn:alpha} by writing 
\[
m(r)=\frac{p(u_i+1) - (v_i+1)}{d}+pr.
\]
% Note that $p(u_i+1) - (v_i+1) \geq 0$, since it is divisible by $d$ 
% and greater than $-d$.  So after eliminating $m$, the sum would just 
% be over all $r \geq 0$.
However, to simplify notation, we will keep the index $m$.
\end{rem}

\subsection{Alternative formulas}

At first sight it appears that this computation genuinely has to 
take place in the extension field~$\QQ_p(\pi)$ even though the entries
of the matrix $\Phi_0$ are of course contained in~$\QQ_p$.  This is, however, 
not the case as we will show now.  The terms $\alpha_{u,v}$ 
will turn out to be elements of~$\ZZ_p$, and we provide expressions 
for them that are more suitable for computations.

First, it is straightforward to obtain a more explicit description 
of the coefficients~$\lambda_m$ via an elementary calculation:

\begin{lem} \label{lem:lambdam}
Let $\pi^{p-1} = -p$ and, for $m \geq 0$, let $\lambda_m$ 
be the coefficient of $z^m$ in the power series expansion 
of $\exp \bigl( \pi (z - z^p) \bigr)$ in $\QQ_p[[z]]$.  Then 
\begin{equation*}
\pi^{- (m \bmod{(p-1)})} \lambda_m = (-1)^{\floor{m/(p-1)}} \sum_{j=0}^{\floor{m/p}} p^{\floor{m/(p-1)} - j} \frac{1}{(m-pj)! j!}
\end{equation*}
where $m \bmod{(p-1)}$ denotes the remainder of $m$ upon Euclidean 
division by $p-1$. \hfill $\qedsymbol$
\end{lem}

\begin{thm} \label{thm:alpha}
Let $u, v \in \ZZ^{n+1}$ be such that 
$x^u, x^v \in B$ and 
$p (u_i + 1) \equiv v_i + 1 \pmod{d}$ for all~$i$. 
Then 
\begin{equation*}
\alpha_{u,v} = (-p)^{k(u)} \prod_{i=0}^n a_i^{(p (u_i + 1) - (v_i + 1))/d} 
    \biggl( \sum_{m,r} a_i^{(p-1)r} \Bigl( \frac{u_i+1}{d} \Bigr)_r 
        \sum_{j=0}^{\floor{m/p}} \frac{p^{r-j}}{(m-pj)!j!} \biggr),
\end{equation*}
where the sum in the $i$-th factor of the product is over all $m, r \geq 0$  
that satisfy
\[
p(u_i+1)-(v_i+1)=d(m-pr).
\]
\end{thm}

\begin{proof}
We start from Definition~\ref{defn:alpha} and use Lemma~\ref{lem:lambdam} to 
find that $\alpha_{u,v}$ is equal to 
\begin{gather*}
\pi^{k(v)-k(u)} \prod_{i=0}^n 
    \biggl( \sum_{m,r} (-1)^{\floor{m/(p-1)}} \pi^{m \bmod{(p-1)}} 
    \Bigl( \frac{u_i+1}{d} \Bigr)_r (-1)^r \pi^{-r} a_i^{m-r} 
    \sum_{j=0}^{\floor{m/p}} \frac{p^{\floor{m/(p-1)}-j}}{(m-pj)!j!} \biggr).
\intertext{Now we write $m = \floor{m/(p-1)} (p-1) + \bigl(m \bmod{(p-1)}\bigr)$, 
and simplify using $\pi^{p-1} = -p$, to obtain}
\alpha_{u,v} = \pi^{k(v)-k(u)} \prod_{i=0}^n \biggl( \sum_{m,r} \pi^{m - r} 
    \Bigl( \frac{u_i+1}{d} \Bigr)_r (-1)^r a_i^{m-r} 
    \sum_{j=0}^{\floor{m/p}} \frac{p^{-j}}{(m-pj)!j!} \biggr).
\end{gather*}
After substituting $m-r = (p-1)r + \bigl(p(u_i+1) - (v_i+1)\bigr)/d$, we obtain 
the required formula.
\end{proof}

Our aim is to show that $\alpha_{u,v}$ is $p$-adically integral.  First, 
we collect a few intermediate results.

\begin{prop} \label{prop:mpr}
Let $x^u, x^v \in B$, $0 \leq i \leq n$, and $m, r$, be such that 
\[
p(u_i + 1) - (v_i + 1) = d(m-pr).
\] 
Then we have $r = \floor{m/p}$.
\end{prop}

\begin{proof}
Since $0 \leq u_i, v_i \leq d-2$, we obtain
\[
p-(d-1) \leq p(u_i + 1) - (v_i + 1) \leq p(d-1)-1,
\]
and from this the result follows, also using that $m - pr \in \ZZ$.
\end{proof}

\begin{prop} \label{prop:rfac}
For all integers $u, d, r \geq 1$ with $p \nmid d$, 
\begin{equation*}
\ord_p\Bigl(\frac{u}{d}\Bigr)_r \geq \frac{r}{p-1} - \floor{\log_p(r) + 1}.
\end{equation*}
\end{prop}

\begin{proof}
Let $s_p(r)$ denote the sum of digits in the $p$-adic expansion of~$r$ 
and observe that $s_p(r) \leq (p-1)\floor{\log_p(r) + 1}$.  Using the 
standard fact that $\ord_p\bigl((u/d)_r\bigr) \geq \ord_p(r!)$, 
it follows that 
\begin{equation*}
\ord_p\Bigl(\frac{u}{d}\Bigr)_r \geq \ord_p(r!) 
    = \frac{r - s_p(r)}{p-1} \geq \frac{r}{p-1} - \floor{\log_p(r) + 1}
\end{equation*}
as required.
\end{proof}

\subsubsection{The case $p = 2$}

\begin{defn} \label{defn:mu2}
For $p = 2$ we define a sequence $\bigl(\mu_m^{(2)}\bigr)$ by 
\begin{equation*}
\mu_m^{(2)} = 
    \sum_{j=0}^{\floor{m/2}} \frac{2^{\floor{3m/4} - \nu_m - j}}{(m-2j)! j!}
\end{equation*}
where $\nu_m$ is equal to one whenever $m = 3, 7$ and zero otherwise, 
and we write $\mu_m =\mu_m^{(2)}$ whenever this does not cause confusion. 
\end{defn}

\begin{lem} \label{lem:mu2}
$\mu_m \in \ZZ_2$ for all $m \geq 0$.
\end{lem}

\begin{proof}
In the two cases $m = 3, 7$ we explicitly compute the values of 
$\mu_m$ as $4/3$ and $232/315$, respectively.  Now suppose that 
$m \neq 3, 7$.  From Lemma~\ref{lem:lambdam} we obtain 
\begin{equation*}
\ord_2 \bigl(\mu_m\bigr) 
    = \floorBig{\frac{3m}{4}} - m + \ord_2(\lambda_m).
\end{equation*}
Using the bound~\eqref{eqn:dworkbound}, we find that 
\begin{equation*}
\ord_2 \bigl(\mu_m\bigr) 
    \geq \floorBig{\frac{3m}{4}} - m + \ceil{\frac{m}{4}} = 0. \qedhere
\end{equation*}
\end{proof}

\begin{thm} \label{thm:alpha2}
Let $p = 2$ and suppose $u, v \in \ZZ^{n+1}$ are such that 
$x^u, x^v \in B$ and $p (u_i + 1) \equiv v_i + 1 \pmod{d}$ 
for all~$i$.  Then $\alpha_{u,v}$ is a $2$-adic integer and can be expressed as 
\begin{equation*}
\alpha_{u,v} = (-2)^{k(u)} \prod_{i=0}^n a_i^{(2 (u_i + 1) - (v_i + 1))/d} \biggl( \sum_{m,r} a_i^{r} \Bigl(\frac{u_i+1}{d}\Bigr)_r 2^{-\floor{(m+1)/4}+\nu_m} \mu_m \biggr), 
\end{equation*}
where the sum in the $i$-th factor of the product is over all $m, r \geq 0$  
that satisfy
\[
p(u_i+1)-(v_i+1)=d(m-pr).
\]
\end{thm}

\begin{proof}
The expression for $\alpha_{u,v}$ is an immediate consequence 
of Theorem~\ref{thm:alpha}, Proposition~\ref{prop:mpr}, and 
Definition~\ref{defn:mu2}. It remains to prove that 
$\alpha_{u,v}$ is a $2$-adic integer.  Following Lemma~\ref{lem:mu2}, 
it suffices to show that the valuation of the factor 
\begin{equation} \label{eq:fr}
f_r=\Bigl(\frac{u_i+1}{d}\Bigr)_r 2^{- \floor{(m+1)/4} + \nu_m}
\end{equation}
in each summand is nonnegative. Since
\[
\ord_{2} \Bigl(\frac{u_i+1}{d}\Bigr)_r \geq \ord_{2}(r!) = \ord_{2}(\floor{m/2}!),
\] 
we find that 
\begin{equation} \label{eq:alpha2.1}
\ord_{2}(f_r)
\geq \ord_{2} \Bigl(\floor{\frac{m}{2}}!\Bigr) - \floor{\frac{m+1}{4}} + \nu_m.
\end{equation}
Applying Proposition~\ref{prop:rfac}, we see that for $m \geq 2$ the 
right-hand side is bounded below by 
\begin{equation*}
\floor{\frac{m}{2}} - \floor{\frac{m+1}{4}} - \floor{\log_2(m)},
\end{equation*}
which is nonnegative whenever $m \geq 12$.  In the remaining 
cases $m = 0, \dotsc, 11$, we explicitly verify that the 
lower bound in~\eqref{eq:alpha2.1} is nonnegative.
\end{proof}

\begin{comment}
\begin{rem}
We observe that in Theorem~\ref{thm:alpha2} the exponent 
$-\floor{(m+1)/4}+\nu_m$ is nonpositive for each value $m \geq 0$, 
so at first sight it seems that the computation of the $f_r$ from
\eqref{eq:fr} will suffer from precision loss. 
However, we recall that $m = m(0) + 2r$, $m(0) \in \{0,1\}$, and noting that 
$d$ is odd let $g_r = d^r f_r$ and $z = u_i + 1$ to temporarily simplify 
our notation.  We now see that the sequence satisfies 
\begin{align*}
g_0 & = 1, \\
g_1 & = z, \\
g_5 & = 2^{-2-m(0)} \bigl( z (z + d) (z + 2d) (z + 3d) (z + 4d) \bigr) \\
g_r & = g_{r-2} \frac{(z + (r - 2)d)(z + (r - 1)d)}{2}
\end{align*}
for all remaining $r \geq 0$.  Since the numerator is an even integer, 
it follows that this computation can be carried out without precision loss.
\end{rem}
\end{comment}

\subsubsection{The case $p > 2$}

\begin{defn} \label{defn:mup}
Let $p >2$ be an odd prime and define a sequence 
$\bigl(\mu_m^{(p)}\bigr)$ by 
\begin{equation*}
\mu_m^{(p)} = \sum_{j=0}^{\floor{m/p}} \frac{p^{\floor{m/p} - j}}{(m-pj)! j!}, 
\end{equation*}
where we write $\mu_m = \mu_m^{(p)}$ when the prime can be identified 
from the context.
\end{defn}

\begin{lem} \label{lem:mup}
$\mu_m \in \ZZ_p$ for all $m \geq 0$.
\end{lem}

\begin{proof}
It is clear that $\mu_m \in \QQ$.  From Lemma~\ref{lem:lambdam} 
we observe that $\lambda_m = \pi^m p^{- \floor{m/p}} \mu_m$, and 
using the bound~\eqref{eqn:dworkbound}, it follows that 
\begin{equation*}
\ord_p (\mu_m) \geq \frac{p-1}{p^2} m + \floor{\frac{m}{p}} - \frac{m}{p-1}.
\end{equation*}
Let us write $m = q p + r$ with $0 \leq r \leq p-1$.  As the valuation 
of $\mu_m$ is an integer, it suffices to show that, for $q \geq 0$, 
\begin{equation*}
\frac{p-1}{p} q + q - \frac{q p + p - 1}{p - 1} > -1,
\end{equation*}
which is equivalent to $p^2 - 3p + 1 > 0$, and this holds true 
provided that $p > 2$.
\end{proof}

\begin{thm} \label{thm:alphap}
Let $p > 2$ and suppose that $u, v \in \ZZ^{n+1}$ are such 
that $x^u, x^v \in B$ and 
$p (u_i + 1) \equiv v_i + 1 \pmod{d}$ for all~$i$. Then 
\begin{equation*}
\alpha_{u,v} = (-p)^{k(u)} \prod_{i=0}^n 
    a_i^{(p (u_i + 1) - (v_i + 1))/d} \biggl( \sum_{m,r} a_i^{(p-1)r}
    \Bigl(\frac{u_i+1}{d}\Bigr)_r \mu_m \biggr)
\end{equation*}
where the sum in the $i$-th factor of the product is over all $m, r \geq 0$  
that satisfy
\[
p(u_i+1)-(v_i+1)=d(m-pr).
\]
In particular, we have that $\alpha_{u, v}$ is a $p$-adic integer. 
\end{thm}

\begin{proof}
This is an immediate consequence of Theorem~\ref{thm:alpha}, 
Proposition~\ref{prop:mpr}, Definition~\ref{defn:mup} and 
Lemma~\ref{lem:mup}.
\end{proof}

\subsection{Estimates}

If we want to use Theorem~\ref{thm:01-03-diagfrob} to compute the matrix 
$\Phi_0$ to $p$-adic precision~$N_{\Phi_0}$, we have to compute the elements
$(k(u)-1)!\alpha_{u,v}$ to a somewhat higher precision~$N'_{\Phi_0}$ than just
$N_{\Phi_0}-n$ because of the loss of precision in computing their inverses. Note 
that if a $p$-adic number $x$ is known to precision~$N$, then its inverse 
is in general only known to precision $N-2\ord_p(x)$. Therefore, we need an 
upper bound on the $p$-adic valuation of the elements $(k(u)-1)!\alpha_{u,v}$.

\begin{prop}
The valuation of $(k(u)-1)! \alpha_{u,v}$ satisfies
\begin{equation*}
\ord_p\bigl((k(u)-1)! \alpha_{u,v}\bigr) 
    \leq \ord_p\bigl((n-1)!\bigr) + n + \delta,
\end{equation*}
where $\delta$ is defined as in Definition~\ref{defn:delta}. 
\end{prop}

\begin{proof}
Recall from Corollary~\ref{cor:delta} that the valuations 
of the entries of the matrix~$\Phi_0$ are bounded from below by $-\delta$. 
Thus by Theorem~\ref{thm:01-03-diagfrob}, 
\begin{equation*}
-\delta \leq \ord_p\bigl((k(v)-1)!\bigr) + n 
           - \ord_p\bigl((k(u)-1)! \alpha_{u,v}\bigr),
\end{equation*}
and since $k(v) \leq n$, the result follows.
\end{proof}

\begin{cor} \label{cor:Ntilde}
In order to compute the matrix $\Phi_0$ to $p$-adic precision $N_{\Phi_0}$, 
it is sufficient to compute the elements $(k(u)-1)!\alpha_{u,v}$ to $p$-adic 
precision
\begin{equation*}
N'_{\Phi_0} = N_{\Phi_0} - n + 2 \bigl(\ord_p\bigl((n-1)!\bigr)+n+\delta\bigr).
\end{equation*}
\end{cor}

Up to this point, the sums over $m,r$ in our expressions 
for $\alpha_{u,v}$ have been infinite sums.  We now present 
a convergence result that will allow us to derive a finite 
expression for $(k(u)-1)!\alpha_{u,v}$ to precision $N'_{\Phi_0}$.
We start with an elementary lemma.

\begin{lem} \label{lem:log}
Given integers $\alpha, \beta \geq 2$ and defining 
$x = \beta + \log_{\alpha}(\beta) + 1$, 
for all real numbers $y \geq x$,
we have 
\begin{equation*}
y - \log_{\alpha}(y) \geq \beta.
\end{equation*}
\end{lem}

\begin{proof}
We first note that the function $y \mapsto y - \log_{\alpha}(y)$ is increasing 
for $y \geq 2$ because it has derivative $1 - \log_{\alpha}(e)/y > 0$.  Thus, 
it suffices to verify the result for $x$.  Indeed, as $\beta \geq 2$ we have 
that $\log_{\alpha}(\beta) + 1 \leq \beta$, hence 
$\beta + \log_{\alpha}(\beta) + 1 \leq 2 \beta \leq \alpha \beta$,
which upon taking logarithms and rearranging yields the result.
\end{proof}


\begin{prop} \label{prop:MR}
In order to compute $(k(u)-1)!\alpha_{u,v}$ to $p$-adic precision~$N'_{\Phi_0}$, 
it suffices to restrict the sums in Theorem~\ref{thm:alpha2} or 
Theorem~\ref{thm:alphap} to pairs $m,r \geq 0$ such that $m \leq \mathcal{M}$, 
or equivalently $r \leq \mathcal{R}$, where 
\begin{align*}
\mathcal{M} &= \ceil{\frac{p^2}{p-1}(N'_{\Phi_0}+\log_p(N'_{\Phi_0}+3)+4)} - 1,
&\mathcal{R} &= \floor{\mathcal{M}/p}.
\end{align*}
\end{prop}

\begin{proof}
We know from the proofs of Theorem~\ref{thm:alpha2} and 
Theorem~\ref{thm:alphap} that the terms in the sums over $m,r$ are 
$p$-adically integral. Moreover, the valuations of the terms with 
$r > 0$ satisfy
\begin{align*}
\ord_p \biggl( \Bigl( \frac{u_i+1}{d} \Bigr)_r 
               \sum_{j=0}^{\floor{m/p}} \frac{p^{r-j}}{(m-pj)!j!} \biggr) 
&\geq \frac{r}{p-1} - \floor{\log_p(r)+1} 
      + \frac{p-1}{p^2} m + \floorBig{\frac{m}{p}} - \frac{m}{p-1} \\
&\geq \Bigl( \frac{p-1}{p^2} \Bigr) m - \log_p(m) - 1,
\end{align*}
by the bound~\eqref{eqn:dworkbound}, Lemma~\ref{lem:lambdam}, 
Proposition~\ref{prop:mpr}, and Proposition~\ref{prop:rfac}. 
Therefore, it is sufficient to restrict the sums in Theorem~\ref{thm:alpha2} 
or Theorem~\ref{thm:alphap} to $m,r \geq 0$ for which 
\begin{equation*}
\frac{p-1}{p^2} m - \log_p(m) - 1 < N'_{\Phi_0}.
\end{equation*}
Note that this includes all terms with $r = 0$.
\end{proof}

Finally, we formalise the procedure for computing 
the entries of~$\Phi_0$ to $p$-adic precision~$N_{\Phi_0}$ in 
Algorithm~\ref{alg:Diagfrob}.

\begin{algorithm}
\caption{Compute the matrix $\Phi_0$.}
\label{alg:Diagfrob}
\begin{algorithmic}
\vspace{1mm}
\Require $P_0=a_0 x_0^d + \dotsb + a_n x_n^d$ 
         with $a_0,\dotsc,a_n \in \ZZ_p^{\times}$, 
         $p$-adic precision $N_{\Phi_0} \geq 0$.
\Ensure  The matrix $\Phi_0$ for the action of $p^{-1} \Frob_p$ 
         on $\Hrig^n(U_0)$ with respect to basis $\cB$ to $p$-adic 
         precision $N_{\Phi_0}$.
\Procedure{DiagFrob}{$P_0,N_{\Phi_0}$}
\State \begin{compactenum}[{\hspace{1em}} 1.] \vspace{-1.24em}
\item Determine $N'_{\Phi_0}$ from Corollary~\ref{cor:Ntilde}, and 
      $\mathcal{M}$,$\mathcal{R}$ from Proposition~\ref{prop:MR}. 
\item Compute the sequences $(d^{-r})_{r=0}^{\mathcal{R}}$, and 
      $(\mu_{m})_{m=0}^{\mathcal{M}}$ to $p$-adic precision~$N'_{\Phi_0}$.
\item Let $\Phi_0 \in M_{b \times b}(\QQ_p)$ be the zero matrix.
\item[] \textbf{for} $x^u \in B$ \textbf{do} 
\item[] \begin{compactenum}[{\hspace{1em}} 1.]
        \item Determine the unique $x^v \in B$ such that $v_i = p (u_i + 1) - 1 \bmod{d}$.
        \item Compute $x_1 \gets (-1)^{k(u)+k(v)} (k(v)-1)! p^n$ as an exact integer.
        \item Compute $x_2 \gets (k(u) - 1)! \alpha_{u,v}$ to $p$-adic 
              precision $N'_{\Phi_0}$ using Theorem~\ref{thm:alpha2} or 
              Theorem~\ref{thm:alphap}.
        \item Compute $x_2^{-1}$ to $p$-adic precision $N_{\Phi_0}-n$.
        \item Compute $(\Phi_0)_{u,v} \gets x_1 x_2^{-1}$ to $p$-adic precision $N_{\Phi_0}$.
      \end{compactenum}   
 \item \textbf{return} $\Phi_0$      
\end{compactenum}
\EndProcedure
\end{algorithmic}
\end{algorithm}

\begin{rem} \label{rem:mup}
The expressions for $\mu_m$ can be computed using integer arithmetic 
modulo~$p^{N'_{\Phi_0}}$ via 
\begin{equation*}
\mu_m^{(p)} = \begin{cases}
\frac{1}{m!} \sum_{j=0}^{\floor{m/2}} 2^{\floor{3m/4} - j} \frac{m!}{(m-2j)! j!}
    & \text{if $p = 2$, $m \neq 3, 7$,} \\
\frac{1}{m!} \sum_{j=0}^{\floor{m/p}} p^{\floor{m/p} - j} \frac{m!}{(m-pj)! j!}
    & \text{if $p$ is odd,}
\end{cases}
\end{equation*}
with only one $p$-adic inversion as all summands are integers. 
\end{rem}

\begin{assump} \label{assump:diag}
From now on we assume that our family of 
hypersurfaces~$\mathcal{X}/\mathcal{S}$ is defined by a 
polynomial $P \in \ZZ_q[t][x_0,\dotsc,x_n]$ for which 
$P(0) \in \ZZ_q[x_0,\dotsc,x_n]$ is of the form 
$P_0=a_0 x_0^d + \dotsb + a_n x_n^d$ with $a_0,\dotsc,a_n \in \ZZ_p^{\times}$, 
so that we can apply Algorithm~\ref{alg:Diagfrob}. 
\end{assump}

\begin{rem}
Note that Assumption~\ref{assump:diag} 
also implies that $\mathcal{S}$ can be chosen to satisfy Assumption~\ref{assump:S} 
and Assumption \ref{assump:R}, since $P_0$ defines a smooth hypersurface and 
we have that $R(0) \in \ZZ_p^{\times}$. 
\end{rem}

%%%%%%%%%%%%%%%%%%%%%%%%%%%%%%%%%%%%%%%%%%%%%%%%%%%%%%%%%%%%%%%%%%%%%%%%%%%%%%%

\section{Solving the differential equation}
\label{sec:DifferentialSystem}

In this section we explain how to solve the $p$-adic differential 
equation for the horizontal sections of the Gauss--Manin 
connection $\nabla$, in order to obtain a local expansion of the 
matrix for the action of $p^{-1} \Frob_p$ on $\Hrig^{n}(U/S)$.  

All our matrices will be defined with respect to the basis $\cB$. Recall 
that $M \in M_{b \times b}(\QQ_q(t))$ denotes 
the matrix for the Gauss--Manin connection $\nabla$ on 
$\HdR^n(\mathfrak{U}/\mathfrak{S})$ and 
$\Phi \in M_{b \times b} \bigl(\QQ_q \langle t,1/r \rangle^{\dag} \bigr)$ 
the matrix for the $\sigma$-semilinear action of~$p^{-1} \Frob_p$ 
on $\Hrig^{n}(U/S)$, where $\sigma$ is defined as 
in Definition~\ref{defn:sigma}.

As we saw in Section~\ref{sec:Background}, these matrices satisfy 
the differential equation
\begin{align} \label{eq:Phi}
\Bigl(\frac{d}{dt} + M\Bigr) \Phi &= p t^{p-1} \Phi \sigma(M), &\Phi(0)& = \Phi_0, 
\end{align}
where $\Phi_0 \in M_{b \times b}(\QQ_p)$ is the matrix for the action 
of $p^{-1} \Frob_p$ on $\Hrig^n(U_0)$. Our goal is the computation of 
the power series expansion of~$\Phi$ at $t=0$ to $t$-adic precision $K$ 
and $p$-adic precision $N_{\Phi}$, i.e.\ as an element of 
$M_{b \times b}(\QQ_q[[t]])$ modulo $t^K$ and $p^{N_{\Phi}}$.

We first observe that if 
$C \in M_{b \times b}(\QQ_q[[t]])$ denotes 
the unique solution to the differential equation
\begin{align} \label{eq:01-GMDE-Homogenous}
\Bigl(\frac{d}{dt} + M\Bigr) C &= 0, &C(0)&=I, 
\end{align}
where $I$ denotes the identity matrix, 
then the matrix $\Phi = C \Phi_0 \sigma(C)^{-1}$ satisfies 
Equation~\eqref{eq:Phi}. So it is sufficient to solve 
Equation~\eqref{eq:01-GMDE-Homogenous}. 
We now give a bound on the rate of 
convergence of $C=\sum_{i=0}^{\infty} C_i t^i$ that
follows from recent work of 
Kedlaya~\citep{Kedlaya2010}. 
We let $\delta$ be defined as in Definition~\ref{defn:delta}. 

\begin{thm} \label{thm:valC}
For all $i \geq 1$, we have
\begin{equation*}
\ord_p(C_i) \geq - \bigl(2 \delta + (n - 1)\bigr) \ceil{\log_p(i)}.
\end{equation*}
\end{thm}

\begin{proof}
It follows from~\citep[Theorem~{18.3.3}]{Kedlaya2010} that
\begin{equation*}
\ord_p(C_i) \geq \bigl( \ord_p(\Phi) + \ord_p(\Phi^{-1}) \bigr) \ceil{\log_p(i)},
\end{equation*}
but from Corollary~\ref{cor:delta} we know that $\ord_p(\Phi) \geq -\delta$ and 
$\ord_p(\Phi^{-1}) \geq -\delta-(n-1)$.
\end{proof}

\begin{rem}
In~\citep[Remark~18.3.4]{Kedlaya2010} Kedlaya also includes the bound
\begin{equation*}
\ord_p(C_i) \geq (b - 1) \ord_p(M) 
            + \bigl( \ord_p(\Phi) + \ord_p(\Phi^{-1}) \bigr) \floor{\log_p(i)},
\end{equation*}
which can sometimes be used to improve Theorem \ref{thm:valC} 
slightly, for example when $\ord_p(M)$ is nonnegative.
\end{rem}

\begin{rem} \label{rem:Cinv}
The bound from Theorem~\ref{thm:valC} 
also applies to the inverse matrix~$C^{-1}$, as this matrix satisfies 
the dual differential equation 
\begin{align} \label{eq:01-GMDE-Dual}
\Bigl(\frac{d}{dt} - M^t\Bigr) \bigl(C^{-1}\bigr)^t &= 0, &C^{-1}(0)& = I,
\end{align}
that carries a Frobenius structure 
given by the matrix~$(\Phi^{-1})^t$.
\end{rem}

We only know the matrix $M$ to some finite $p$-adic precision 
$N_M$, and we need to compute $C$ to some finite $t$-adic and 
$p$-adic precisions $K,N_C$, respectively. The following result gives an 
expression for $N_M$ in terms of $K,N_C$. For a matrix 
$A=\sum_{i=0}^{\infty} A_i t^i$ we write 
$\overline{A}=\sum_{i=0}^{K-1} A_i t^i$ 
in what follows. 

\begin{prop} \label{prop:N_M}
Let $K,N_C \in \NN$ and define
\[
N_M= N_C + (2 \delta + n) \ceil{\log_p(K-1)}+1.
\]
Let $\tilde{M} \in M_{b \times b}(\QQ_q(t))$ be
an approximation of $M$ to $p$-adic precision $N_M$, i.e.
such that $\ord_p(\tilde{M}-M) \geq N_M$, and suppose that 
$\tilde{C}=\sum_{i=0}^{\infty} \tilde{C}_i t^i$ satisfies the
differential equation
\begin{align*}
\Bigl(\frac{d}{dt}+\tilde{M} \Bigr) \tilde{C}&=0, &\tilde{C}(0)=I&.
\end{align*}
Then $\ord_p(\tilde{C}_i-C_i) \geq N_C$ for all $i < K$.
\end{prop}

\begin{proof}
From the expressions
\begin{align*}
C(t)&=\exp\Bigl(- \int_{0}^t M(s) ds\Bigr), &
\tilde{C}(t)&=\exp\Bigl(-\int_{0}^t \tilde{M}(s) ds\Bigr),
\end{align*}
it follows that
\begin{equation} \label{eq:CtildeminusC}
\tilde{C}(t) - C(t) = C(t) \Bigl( \exp \Bigl( \int_{0}^t \bigl( \tilde{M}(s)-M(s) \bigr) ds \Bigr)- I \Bigr).
\end{equation}
Since $\ord_p(\tilde{M}-M) \geq N_M$, we obtain
\begin{equation*}
\ord_p \Bigl( \overline{\frac{1}{i!} \Bigl(\int_{0}^t \bigl( \tilde{M}(s)-M(s) \bigr) ds \Bigr)^i } \Bigr) \geq 
N_C + (2 \delta + n-1) \ceil{\log_p(K-1)}
\end{equation*}
for all $1 \leq i \leq K-1$, where we have used that 
$\ord_p(i!) \leq i/(p-1) \leq 1$. Moreover, from 
Theorem~\ref{thm:valC}, we already know that 
\[
\ord_p(\overline{C}) \geq -(2 \delta + n-1) \ceil{\log_p(K-1)}.
\] 
From these two inequalities and Equation~\eqref{eq:CtildeminusC}, we 
deduce that $\ord_p(\overline{\tilde{C}-C}) \geq N_C$.
\end{proof}

Now we explain how to compute the solution $C$ to 
Equation~\eqref{eq:01-GMDE-Homogenous} to $p$-adic precision~$N_C$ and 
$t$-adic precision~$K$, assuming that the connection matrix~$M$ has been 
computed to $p$-adic precision~$N_M$ as defined in Proposition~\ref{prop:N_M}. 
We can write $M = G/r$, with 
$G = \sum_{i=0}^{\deg(G)} G_i t^i \in M_{b \times b}(\QQ_q[t])$ 
and $r = \sum_{i=0}^{\deg(r)} r_i t^i \in \ZZ_q[t]$ a divisor of the 
polynomial~$R$ defined in Definition~\ref{defn:resultant}. 
% TODO small problem, if M is computed to some p-adic precision, the poles 
% are not determined exactly, so R and r etc. should be computed to infinite 
% precision?? 
Note that the degree of~$r$ might be smaller than the degree 
of~$R$, which will speed up our computations. By Assumption~\ref{assump:diag}, 
we have $r(0) \neq 0 \pmod{p}$, so in particular $r(0) \neq 0$.  

We can obtain a power series solution $C = \sum_{i=0}^{\infty} C_i t^i$ to
the equation
\begin{align*}
r \frac{dC}{dt} + G C &= 0, &C(0)& = I,
\end{align*}
which is clearly equivalent to 
Equation~\eqref{eq:01-GMDE-Homogenous}, using the following recursion:  
\begin{align} \label{eq:recursiondifeq}
C_0 &= I, \nonumber \\
C_{i+1} &= \frac{-1}{r_0 (i+1)} \biggl(
    \sum_{j=\max{\{0,i-\deg(G)\}}}^i G_{i-j} C_j + 
    \sum_{j=\max{\{0,i-\deg(r)\}}+1}^i r_{i-j+1} (j C_j) \biggr).
\end{align}
Again we will only carry out this computation to some finite $p$-adic 
working precision~$N'_C$, and if we want $C$ to be correct to $p$-adic 
precision~$N_C$, then the precision $N'_C$ has to be somewhat higher 
because of error propagation. An expression for $N'_C$, in terms of 
$N_C$ and the desired $t$-adic precision $K$, was given by 
Lauder~\citep[Theorem~5.1]{Lauder2006}, but his result can be
significantly improved using Theorem~\ref{thm:valC}, as we will now show. 

Let $\tilde{C}=\sum_{i=0}^{\infty} \tilde{C}_i t^i$ denote an 
approximation to $C$ computed using the approximate recursion:  
\begin{align*}
\tilde{C}_0 &= I, \\
\tilde{C}_{i+1} &= \frac{-1}{r_0 (i+1)} \biggl(
    \sum_{j=\max{\{0,i-\deg({G})\}}}^i {G}_{i-j} \tilde{C}_j + 
    \sum_{j=\max{\{0,i-\deg({r})\}}+1}^i {r}_{i-j+1} (j \tilde{C}_j) \biggr) + {E}_{i+1},
\end{align*}
where $\ord_p({E}_i) \geq N'_{C}$ for all $i \geq 1$, 
so that the matrices~$\tilde{C}_i$ are computed with $p$-adic working 
precision~$N'_C$.

\begin{prop} \label{thm:errorprop}
Let $K,N_{C} \in \NN$, and suppose that
\begin{align*}
N'_C          &= N_{C} + \Bigl(2 \bigl(2 \delta + (n-1)\bigr) + 1\Bigr) \ceil{\log_p(K-1)}.
\end{align*} 
Then $\ord_p(\tilde{C}_i-C_i) \geq N_{C}$ for all $i < K$.
\end{prop}

\begin{proof}
The matrix $\tilde{C}$ satisfies the differential equation
\begin{align*}
\frac{d\tilde{C}}{dt}+M \tilde{C}&=E, &\tilde{C}(0)&=I,
\end{align*}
where we have denoted $E=\sum_{i=1}^{\infty} E_i t^i$.
One also checks that the matrix $C^{-1}\tilde{C}$ satisfies 
the differential equation
\begin{align*}
\frac{d(C^{-1}\tilde{C})}{dt} &=C^{-1} E, &(C^{-1}\tilde{C})(0)&=I,
\end{align*}
from which it follows that
\begin{equation} \label{eq:integral}
\tilde{C}(t)-C(t) = C(C^{-1} \tilde{C}-I) = C(t) \Bigl(\int_{0}^t C^{-1}(s) E(s) ds \Bigr).
\end{equation}
We know from Theorem~\ref{thm:valC} and Remark~\ref{rem:Cinv} that
\begin{equation} \label{eq:boundCCinv}
\ord_p(\overline{C}),\ord_p(\overline{C^{-1}}) \geq 
-(2 \delta + n-1) \ceil{\log_p(K-1)},
\end{equation}
and we hence obtain
\[
\ord_p \Bigl(\overline{\int_{0}^t C^{-1}(s) E(s) ds }\Bigr) \geq 
N_{C} + \Bigl( \bigl(2 \delta + (n-1)\bigr) \Bigr) \ceil{\log_p(K-1)}.
\]
From the bounds~\eqref{eq:boundCCinv} and Equation~\eqref{eq:integral}, 
we deduce that $\ord_p(\overline{\tilde{C}-C}) \geq N_C$.
\end{proof}

\begin{rem}
A result similar to Proposition~\ref{thm:errorprop} with a larger constant in front of 
the logarithm was obtained by Lauder in \citep[Theorem 5.1]{Lauder2006}. We have
not been able to find something similar to Proposition~\ref{prop:N_M} in Lauder's work.
\end{rem}

\begin{rem} \label{rem:sigmatrick}
In order to determine the power series expansion of the matrix~$\Phi$, 
we also need to compute the matrix $\sigma(C)^{-1}$. We could compute 
the matrix~$C^{-1}$ using matrix inversion over the ring $\mathbf{Q}_q[[t]]$. 
However, solving~\eqref{eq:01-GMDE-Dual} turns
out to be more efficient. 
\end{rem}

We finally give all the precisions necessary for computing the power series 
expansion of $\Phi$ at $t=0$ to $t$-adic precision $K$ and $p$-adic 
precision $N_{\Phi}$.

\begin{thm} \label{thm:Ni}
Let $K,N_{\Phi} \in \NN$ and define:
\begin{eqnarray*}
N_{\Phi_0}   		&=& N_{\Phi}+\bigl(2\delta+(n-1)\bigr) \bigl( \ceil{\log_p(K-1)} + \ceil{\log_p(\ceil{K/p}-1)}\bigr),\\
N_{C}				&=& N_{\Phi}+\bigl(2\delta+(n-1)\bigr) \ceil{ \log_p(\ceil{K/p}-1)} + \delta, \\
N_{C^{-1}}			&=& N_{\Phi}+\bigl(2\delta+(n-1)\bigr) \ceil{\log_p(K-1)} + \delta, \\
N_M                 &=& N_{\Phi}+\Bigl(2\bigl(2 \delta + (n-1)\bigr) + 1\Bigr) \ceil{\log_p(K-1)}+1, \\
N'_C			    &=& N_{\Phi}+\Bigl(2 \bigl(2 \delta + (n-1)\bigr) + 1\Bigr) \ceil{\log_p(K-1)}, \\
N'_{C^{-1}}	        &=& N_{\Phi}+\Bigl(2 \bigl(2 \delta + (n-1)\bigr) + 1\Bigr) \ceil{\log_p(\ceil{K/p}-1)}.
\end{eqnarray*}

In order to compute the power series expansion 
of the matrix $\Phi$ at $t=0$ with $t$-adic precision $K$ and $p$-adic precision $N_{\Phi}$,
it is sufficient to compute
the matrix $\Phi_0$ to $p$-adic precision $N_{\Phi_0}$,
the matrix $C$ to $t$-adic precision $K$ and $p$-adic precision $N_{C}$, 
the matrix $C^{-1}$ to $t$-adic precision $\ceil{K/p}$ and $p$-adic precision 
$N_{C^{-1}}$, and 
the matrix $M$ to $p$-adic precision $N_M$.

Therefore, while solving Equation~\eqref{eq:01-GMDE-Homogenous} for $C$ and 
Equation~\eqref{eq:01-GMDE-Dual} for $C^{-1}$, using a recursion like in 
Equation~\eqref{eq:recursiondifeq}, it is sufficient to use $p$-adic 
working precisions $N'_C$ and $N'_{C^{-1}}$, respectively.
\end{thm}

\begin{proof}
Recall that $\Phi = C \Phi_0 \sigma(C)^{-1}$. The sufficient $t$-adic 
precisions are clear. We can apply Proposition~\ref{prop:matrixproductval}, 
using Theorem~\ref{thm:valC} for both $C$ and $C^{-1}$ and the fact that 
$\ord_p(\Phi_0) \geq -\delta$ from Corollary~\ref{cor:delta}, to obtain 
the sufficient $p$-adic precisions for the matrices $\Phi_0$, $C$ and $C^{-1}$. 
The sufficient precision for the matrix $M$ follows from Proposition~\ref{prop:N_M},
and the sufficient working precisions $N'_C$ and $N'_{C^{-1}}$
follow from Proposition~\ref{thm:errorprop}.
\end{proof}

Now we have all the ingredients to compute the power series expansion of 
$\Phi$ at $t=0$ to any given $p$-adic and $t$-adic precisions, as formalised 
in Algorithm~\ref{alg:expansion}.

\begin{algorithm}
\caption{Compute the power series expansion of $\Phi$ at $t=0$.}
\label{alg:expansion}
\begin{algorithmic}
\vspace{1mm}
\Require $P \in \ZZ_q[t][x_0,\dotsc,x_n]$ satisfying Assumption~\ref{assump:diag}, $t$-adic precision~$K$, $p$-adic precision~$N_{\Phi}$.
\Ensure  The power series expansion of $\Phi$ at $t=0$ to $t$-adic precision $K$ and $p$-adic precision $N_{\Phi}$.
\Procedure{FrobSeriesExpansion}{$N_{\Phi},K$} 
\State \begin{compactenum}[{\hspace{6pt}} 1.] \vspace{-1.24em}
\item Determine $N_M,N_{\Phi_0},N_C,N_{C^{-1}},N'_C$, and $N'_{C^{-1}}$ from Theorem~\ref{thm:Ni}.
\item $M \gets \textsc{GMConnection}(P,N_M)$
\item $\Phi_0 \gets \textsc{DiagFrob}(P_0,N_{\Phi_0})$
\item Solve Equation~\eqref{eq:01-GMDE-Homogenous} for $C$ to $t$-adic precision $K$ and $p$-adic precision $N_{C}$:
\begin{compactenum}[a.] 
\item[] $C_0 \gets I$
\item[] \textbf{for} $i=0$ \textbf{to} $K-2$ \textbf{do} 
\item[] \hspace{0.6em} $C_{i+1} \gets \frac{-1}{r_0 (i+1)} \biggl(\sum_j G_{i-j} C_j + \sum_j r_{i-j+1} (j C_j) \biggr) \pmod{p^{N'_C}}$
\item[] $C \gets \sum_{i=0}^{K-1} C_i t^i \pmod{p^{N_C}}$
\end{compactenum}
\item Solve Equation~\eqref{eq:01-GMDE-Dual} for $C^{-1}$ to $t$-adic precision $\ceil{K/p}$ and $p$-adic precision $N_{C^{-1}}$:
\begin{compactenum}[a.]
\item[] $(C^{-1})_0 \gets I$
\item[] \textbf{for} $i=0$ \textbf{to} $\ceil{K/p}-2$ \textbf{do}
\item[] \hspace{0.6em} $(C^{-1})_{i+1} \gets  \frac{-1}{r_0(i+1)} \biggl(\sum_j -G_{i-j}^t (C^{-1})_j + \sum_j r_{i-j+1} (j (C^{-1})_j) \biggr) \pmod{p^{N'_{C^{-1}}}}$
\item[] $C^{-1} \gets \sum_{i=0}^{\ceil{K/p}-1} (C^{-1})_i t^{i} \pmod{p^{N_{C^{-1}}}}$
\end{compactenum}
\item $\Phi \gets C \Phi_0 \sigma(C^{-1})$
\item \Return $\Phi$
\end{compactenum}
\EndProcedure
\end{algorithmic}
\end{algorithm}

%%%%%%%%%%%%%%%%%%%%%%%%%%%%%%%%%%%%%%%%%%%%%%%%%%%%%%%%%%%%%%%%%%%%%%%%%%%%%%%

\section{The zeta function of a fibre}

\label{sec:ZetaFunctions}

\subsection{Evaluating $\Phi$ at a point}

In the previous section we described how to compute the power series 
expansion at $t=0$ of the matrix~$\Phi$ for the action of $p^{-1} \Frob_p$ on 
$\Hrig^{n}(U/S)$. We now want to evaluate $\Phi$ at the Teichm\"uller 
lift~$\hat{\tau}$ of some nonzero $\tau \in S(\FF_{\mathfrak{q}})$, 
where $\FF_{\mathfrak{q}}/\FF_q$ denotes a finite field extension, but 
the power series expansion of $\Phi$ 
at $t=0$ usually only converges on the open unit disc $\abs{t} < 1$, 
and so cannot be evaluated at $\hat{\tau}$. 

However, since $\Phi$ is a matrix of overconvergent functions, i.e.\ 
\[
\Phi \in M_{b \times b}\Bigl(\QQ_q \bigl\langle t,\frac{1}{r} \bigr\rangle^{\dag}\Bigr),
\]
it can be approximated to any given $p$-adic precision $N$ by a matrix 
of rational functions, which can then be evaluated at $\hat{\tau}$. To 
convert the power series expansion for $\Phi$ to such an approximation 
by rational functions, we need a bound on the pole order of these 
rational functions at their poles, as a function of~$N$.

The bounds used by Lauder~\citep[\S 8.1]{Lauder2004a} and 
Gerkmann~\citep[\S 6]{Gerkmann2007} were not very sharp, 
and this significantly slowed down their algorithms.
Recently, under some small additional assumptions, Kedlaya and the second 
author~\citep[Theorem~2.1]{KedlayaTuitman2012} obtained a sharper bound 
that we state below in a simplified form. For $z \in \mathbf{P}^1(\bar{\QQ}_q)$, we
denote the valuation on $\QQ_q(t)$ corresponding to $z$ by $\ord_z(-)$, and
we extend $\ord_z(-)$ to polynomials and matrices over $\QQ_q(t)$ by
taking the minimum over the coefficients and entries, respectively.
Recall that $M$ denotes the matrix of the Gauss--Manin connection $\nabla$
with respect to the basis $\cB$ from Definition~\ref{defn:MonBasis}.
Let $\delta$ be defined as in Definition~\ref{defn:delta}.

\begin{thm} \label{thm:KedlayaTuitman}
Let $D \subset \mathbf{P}^1(\bar{\QQ}_q)$ be a residue disk and
$z \in D$ a point such that $M$ has at most a simple pole at~$z$ 
and no other poles contained in $D$. Suppose that 
the exponents $\lambda_1, \dotsc, \lambda_{b}$ of $M$ at $z$, which are defined 
as the eigenvalues  of the matrix $(t - z) M \vert_{t=z}$ and 
known to be rational numbers, are contained in $\ZZ_p \cap \QQ$. 
For $i \in \NN$, put
\begin{align*}
f(i) = \max \Bigl\{ -\bigl(2 \delta + (n-1) \bigr) \lceil \log_p(i) \rceil, 
(b-1) \ord_p(M) -\bigl(2 \delta + (n-1) \bigr) \lfloor \log_p(i) \rfloor 
\Bigr\},
\end{align*}
and define 
\begin{align*}
c = \begin{cases}
0 & \mbox{if $\ord_p(M) \geq 0$}, \\
\min\{0, i + f(i): i \in \NN\} & \mbox{if $\ord_p(M) < 0$}.
\end{cases}
\end{align*}
For $N \in \NN$, put
\begin{align*}
g(N) &= \max \{i \in \NN : i + f(i) - \delta + c  < N \},
\end{align*}
and define
\begin{align*}
\alpha_1    &= \lfloor -p \min_i \{ \lambda_i \} + \max_{i} \{\lambda_i\} \rfloor, \\ 
\alpha_2    &=  \left \{ 
         \begin{array}{cl}
         0  & \mbox{if $M$ does not have a pole at $z$},  \\
         0  & \mbox{if $z \in \{0,\infty \}$}, \\
         g(N) & \mbox{otherwise}.
         \end{array}
         \right.
\end{align*}
Then the matrix $\Phi$ is congruent 
modulo $p^{N}$ to a matrix $\tilde{\Phi} \in M_{b \times b}(\QQ_q(t))$ for which
\begin{equation*}
\ord_z(\tilde{\Phi}) \geq -(\alpha_1+p \alpha_2)
\end{equation*}
and $\tilde{\Phi}$ has no other poles contained in $D$.
\end{thm}

\begin{proof}
Since the matrix $\Phi$ defines a Frobenius structure on the vector 
bundle $\Hrig^n(U/S)$ with connection $\nabla$, we can 
apply~\citep[Theorem~2.1]{KedlayaTuitman2012}. We have 
replaced $\ord_p(\Phi)$ and $\ord_p(\Phi)+\ord_p(\Phi^{-1})$ 
by their respective lower bounds $-\delta$ and $-2 \delta + (n-1)$, since 
we might not know them exactly a priori.
\end{proof}

\begin{rem}
In practice, the various constants $-(2 \delta + (n-1))$, $c$, $\alpha_1$, $-\delta$ 
are always very small in absolute value, so that $g(N)$ is about $N$, and 
the lower bound for the order of~$\Phi$ modulo $p^N$ at the point $z$ is 
roughly $-pN$.
\end{rem}

\begin{rem} \label{rem:basischange}
The condition that $M$ has a simple pole at $z$ is not a serious restriction. 
By \citep[Theorem 2.1]{Lauder2011}, one can always find a matrix $W \in GL_{b}(\QQ_q[t,1/(t-z)])$
such that the connection matrix has a simple pole at $z$ with respect to the basis 
$[w_1, \dotsc, w_b]$ defined by $w_j = \sum_{i=1}^b W_{ij} e_i$ for all 
$1 \leq j \leq b$, where $\cB=[e_1,\dotsc,e_b]$ denotes our basis from 
Definition~\ref{defn:MonBasis}. Now by \citep[Corollary 2.6]{KedlayaTuitman2012}, 
the matrix $\Phi$ is congruent modulo $p^{N+\ord_p(W)+\ord_p(W^{-1})}$ 
to a matrix $\tilde{\Phi} \in M_{b \times b}(\QQ_q(t))$ for which
\[
\ord_z(\tilde{\Phi}) \geq -(\alpha_1+p \alpha_2)+\ord_z(W)+p\ord_z(W^{-1})
\]
and $\tilde{\Phi}$ has no other poles contained in $D$.
Here, $\alpha_1$ and $\alpha_2$ are defined as in 
Theorem~\ref{thm:KedlayaTuitman}, but with $\delta$ replaced by
\begin{equation*}
\delta' = \delta-\bigl(\ord_p(W)+\ord_p(W^{-1})\bigr), 
\end{equation*}
since for the matrix $\Phi'$ of $p^{-1}\Frob_p$ with respect to the basis 
$[w_1,\dotsc,w_b]$, in general we only have that $\ord_p(\Phi') \geq -\delta'$. 
\end{rem}

When one of the other conditions in Theorem~\ref{thm:KedlayaTuitman} is not satisfied, 
we can use the alternative bound below. Recall that our family of 
smooth projective hypersurfaces $\mathcal{X}/\mathcal{S}$ is defined by the homogeneous 
polynomial $P \in \ZZ_q[t][x_0,\dotsc,x_n]$.

\begin{thm} \label{thm:Gerkmann}
Define the matrices $\Delta_k$ over $\QQ_q(t)$ as in 
Definition~\ref{defn:Deltak}.
Let $D \subset \mathbf{P}^1(\bar{\QQ}_q)$ be a residue disk
and for any point $z \in D$ put
\begin{eqnarray*}
\mu_z &=& \sum_{k=2}^n \ord_z(\Delta_k^{-1}), \\
\nu_z &=& \ord_z(\Delta_{n+1}^{-1}).
\end{eqnarray*}
For $N \in \NN$, define 
\begin{align*}
h(N) = \max \{ i \in \NN : i+(n-1)+\ord_p ((n-1)!)-n \floor{\log_p(p(n+i)-1)} < N \},
\end{align*}
and put
\[
\beta_z = \begin{cases}
-\mu_z -(p(n+h(N))-n) \nu_z & \mbox{if $z \neq \infty$}, \\
-\mu_z -(p(n+h(N))-n) \nu_z + p h(N) \deg_t(P) & \mbox{if $z = \infty$}.
\end{cases}
\]
Then $\Phi$ is congruent modulo $p^N$ to a matrix 
$\tilde{\Phi} \in M_{b \times b}(\QQ_q(t))$ for which
\begin{equation*}
\ord_z(\tilde{\Phi}) \geq -\beta_z,
\end{equation*}
for all points $z \in D$.
\end{thm}

\begin{proof}
First we extend $\sigma$ to the ring of overconvergent functions:
\begin{align*}
A^{\dag} =
&\Bigl\{ \sum_{i_0,\dotsc,i_{n+3}=0}^{\infty} a_{i_0, \dotsc, i_{n+3}} 
\frac{x_0^{i_0} \dotsm x_n^{i_{n}}t^{i_{n+1}}}{r^{i_{n+2}} P^{i_{n+3}}} \; : \;
a_{i_0, \dotsc i_{n+3}} \in \QQ_q, \; \\ 
&\exists c > 0 \text{ s.t.} \lim_{i_0+\dotsb+i_{n+3} \rightarrow \infty} \bigl(\ord_p(a_{i_0, \dotsc, i_{n+3}} ) - 
c(i_0+\dotsb+i_{n+3})\bigr) \geq 0 \Bigr\},
\end{align*}
by putting $\sigma(x_i) = x_i^p$ for $0 \leq i \leq n$. Note that
\begin{equation*}
\sigma\Bigl(\frac{1}{P}\Bigr) = 
    P^{-p} \biggl( 1-\frac{P^p-\sigma(P)}{P^p} \biggr)^{-1}
\end{equation*}
is an element of $A^{\dag}$ since 
\begin{equation*}
\ord_p(P^p-\sigma(P)) \geq 1.
\end{equation*}
We use the notation and terminology from Section~\ref{sec:Connection}.
It is known that the de Rham cohomology of 
$A^{\dag}/\QQ_q \langle t,1/r \rangle^{\dag}$ 
is isomorphic to the rigid cohomology of $U/S$, and that the action of 
$F_p$ on the rigid cohomology of $U/S$ is induced by $\sigma$. Recall
that our basis vectors for $\Hrig(U/S)$ 
%TODO, not really Hrig(U/S), only after shrinking S
are of the form $x^u \Omega / P^{\ell}$ with $x^u \in B_{\ell}$ and 
$\ell \leq n$.  We observe that
\begin{equation*}
p^{-1} \Frob_p \Bigl(\frac{x^u \Omega}{P^{\ell}} \Bigr) \equiv
\sum_{i=0}^{\infty} \eta_i p^{n-1+i} \Bigl(\frac{P^p-\sigma(P)}{p} \Bigr)^i 
(x_0 \dotsm x_n)^{p-1} \Bigl( \frac{x^{pu} \Omega}{P^{p(\ell+i)}} \Bigr)
\end{equation*}
in $\Hrig^n(U/S)$, where $\eta_i \in \NN$ is defined by the equality 
$(1-y)^{-\ell} = \sum_{i=0}^{\infty} \eta_i y^i$.
Using Algorithm~\ref{alg:PoleRed}, we can write
\begin{equation} \label{eqn:Froblift}
\eta_i p^{n-1+i} \Bigl(\frac{P^p-\sigma(P)}{p} \Bigr)^i 
(x_0 \dotsm x_n)^{p-1} \Bigl( \frac{x^{pu} \Omega}{P^{p(\ell+i)}} \Bigr) \equiv
\frac{\gamma_{i,1} \Omega}{P^1}+\dotsb+\frac{\gamma_{i,n} \Omega}{P^n},
\end{equation}
where $\gamma_{i,j}$ is contained in the $L$-span of $B_j$ for all $i \geq 0$ and 
$0 \leq j \leq n$. 

Actually, we can slightly modify 
Algorithm~\ref{alg:Decompose}, so that we only have to solve systems of the 
form $\Delta_{n+1} v = w$ as long as the pole order is at least~$n+1$.  
Indeed, we may assume, without loss of generality, that $Q=x^v$ is a monomial, 
and can find some other monomial $x^w$ such that $x^{v-w}$ is of degree 
$(n+1)d-(n+1)$. Now we apply Algorithm~\ref{alg:Decompose} to $x^{v-w}$, 
noting that $B_{n+1}=\emptyset$, and multiply the output by~$x^w$ to obtain 
the decomposition of~$x^v$.

Note that the left-hand side of Equation~\eqref{eqn:Froblift} does not have a 
pole at $z \neq \infty$, and has order at least $-pi\deg_t(P)$ at $z=\infty$. 
When applying Algorithm~\ref{alg:PoleRed} to this expression, as long as the 
pole order is at least~$n+1$, the order at~$z$ drops by at most $\nu_z$ in every 
reduction step, and it drops by at most $\mu_z$ in the remaining reduction steps. 
This implies that
\begin{align} \label{al:ordz}
\ord_z(\gamma_{i,j}) &\geq 
\begin{cases}
\mu_z + (p(n+i)-n) \nu_z                 &\mbox{if } z \neq \infty, \\
\mu_z + (p(n+i)-n) \nu_z - pi \deg_t(P)  &\mbox{if } z=\infty.
\end{cases}
\end{align}

For every $\tau \in S(\bar{\FF}_q)$, we
define lattices $\Lambda_{\tau,crys}$, $\Lambda_{\tau,\cB}$ in $\Hrig^n(U_{\tau})$
as in the proof of Theorem~\ref{thm:deltabound}. It follows from
\citep[Proposition 3.4.6]{AbbottKedlayaRoe2006}, Equation~\eqref{eqn:Froblift}, 
and the inclusion $(n-1)!\Lambda_{\tau,crys} \subset \Lambda_{\tau,\cB}$ that
\begin{equation*}
\frac{\gamma_{i,1} \Omega}{P^1}+\dotsb+\frac{\gamma_{i,n} \Omega}{P^n} \subset 
p^{i+(n-1)-\ord_p((n-1)!)-n \floor{\log_p(p(n+i)-1)}} \Lambda_{\tau,\cB},
\end{equation*}
which implies that
\begin{equation*}
\ord_p(\gamma_{i,j}) \geq i+(n-1) - \ord_p((n-1)!) - n \floor{\log_p(p(n+i)-1)}.
\end{equation*}
The entries of $\Phi$ are coefficients of sums of the form 
$\sum_{i=0}^{\infty} \gamma_{i,j}$. Now modulo $p^N$, we can 
restrict this sum to terms for which $\ord_p(\gamma_{i,j})<N$,
and we can compute the order at $z$ of these terms using
\eqref{al:ordz}. This completes the proof.
\end{proof}

\begin{rem}
A result similar to (but weaker than) Theorem \ref{thm:Gerkmann} 
was obtained by Gerkmann in \citep[Section 6]{Gerkmann2007}.
\end{rem}

\begin{thm} \label{thm:defD}
Let $N_{\Phi} \in \NN$. We can explicitly find $s \in \ZZ_q[t]$ and $K \in \NN$, 
where $s$ divides some power of the polynomial $R$ from Definition~\ref{defn:resultant}, such that 
$s \Phi$ is congruent modulo $p^{N_{\Phi}}$ to a matrix of polynomials of degree 
less than $K$. 
\end{thm}

\begin{proof}
On every residue disk $D \subset \mathbf{P}^1(\bar{\QQ}_q)$, we can 
apply either Theorem~\ref{thm:KedlayaTuitman} (when it applies) or Theorem~\ref{thm:Gerkmann}, to find
$\theta_z \in \NN_0$ for all $z \in D$, such that $\Phi$ is congruent modulo~$p^N$ to a matrix
$\tilde{\Phi}\in M_{b \times b}(\QQ_q(t))$ for which $\ord_z(\tilde{\Phi}) \geq -\theta_z$, for
all $z \in D$. By Theorem~\ref{thm:Gerkmann}, we can take $\theta_z=0$ for $z \neq \infty$ if 
$R(z) \neq 0$, so that $\theta_z$ vanishes for all but finitely many $z \in \mathbf{P}^1(\bar{\QQ}_q)$.
Moreover, by Theorem~\ref{thm:KedlayaTuitman}, we can also take $\theta_z=0$ if the matrix $M$ does
not have a pole in the residue disk at $z$. 
Finally, we may assume that $\theta_z=\theta_{z'}$ if $z,z'$ are conjugated over $\QQ_q$.
We now define
\begin{align*}
s &= \prod_{z \neq \infty, \ord_p(z) \geq 0} (t-z)^{\theta_z}
     \prod_{z \neq \infty, \ord_p(z) < 0}\Bigl(\frac{t}{z}-1\Bigr)^{\theta_z}, 
&K&= \Bigl(\sum_{z} \theta_z \Bigr) +1,
\end{align*}
which clearly satisfy all the required conditions.
\end{proof}

We compute the matrix $\Phi$ to $t$-adic precision $K$ and $p$-adic
precision $N_{\Phi}$ using Algorithm~\ref{alg:expansion}. For any 
$\tau \in S(\bar{\FF}_q)$, we can now compute
\begin{equation*}
\Phi_{\tau} = 
  s(\hat{\tau})^{-1} \bigl( s \Phi \bmod{t^{K}} \bigr)|_{t=\hat{\tau}} 
  \bmod{p^{N_{\Phi}}}.
\end{equation*}
Since $\ord_p(s(\hat{\tau}))=0$, $\ord_p(\hat{\tau}) \geq 0$ and
$\ord_p(\Phi) \geq -\delta$, 
the matrix $\Phi_{\tau}$ will also be correct to precision $N_{\Phi}$ 
provided that $\hat{\tau}$ is computed to $p$-adic precision $N_{\Phi}+\delta$,
by Proposition~\ref{prop:matrixproductval}.

\subsection{Computing the zeta function}

Now we want to compute the zeta function of the fibre $X_{\tau}$ of
our family $X/S$ lying over some $\tau \in S(\FF_{\mathfrak{q}})$, 
where $\FF_{\mathfrak{q}}/\FF_q$ denotes a finite field extension.  
Recall from Theorem~\ref{thm:hypersurface} 
that the zeta function of $X_{\tau}$ is of the form
\begin{equation*}
Z(X_{\tau},T) = \frac{\chi(T)^{(-1)^n}}{(1 - T) (1 - \mathfrak{q}T) \dotsm (1 - \mathfrak{q}^{n-1}T)},
\end{equation*}
where $\chi(T) = \det \bigl( 1 - T \mathfrak{q}^{-1} \Frob_{\mathfrak{q}} | \Hrig^n(U_{\tau}) \bigr) \in \ZZ[T]$ 
denotes the reverse characteristic polynomial of the action 
of $\mathfrak{q}^{-1} \Frob_{\mathfrak{q}}$ 
on $\Hrig^n(U_{\tau})$.

We start by computing the matrix of the action 
of $\mathfrak{q}^{-1} \Frob_{\mathfrak{q}}$ on $\Hrig^n(U_{\tau})$. 
Let us still denote $a=\log_p(\mathfrak{q})$. Recall that $\Phi_{\tau}$ 
is the matrix of the action of $p^{-1} \Frob_p$ on~$\Hrig^{n}(U_{\tau})$ 
with respect to the basis $\cB$. As this action is $\sigma$-semilinear, 
we have that 
\begin{equation*}
\Phi_{\tau}^{(a)} = 
    \Phi_{\tau} \sigma(\Phi_{\tau}) \dotsm \sigma^{a-1}(\Phi_{\tau})
\end{equation*}
is the matrix of the action of $\mathfrak{q}^{-1} \Frob_{\mathfrak{q}}$ 
on $\Hrig^n(U_{\tau})$. 

We now analyse the loss of $p$-adic precision when computing 
the reverse characteristic polynomial 
$\chi(T)=1+\sum_{i=1}^b \chi_i T^i=\det\bigl( 1 - T \Phi_{\tau}^{(a)}\bigr)$.

\begin{thm} \label{thm:preccharpoly}
Let $N_{\Phi} \in \NN$ be such that $N_{\Phi} \geq \delta$, 
with $\delta$ defined as in Definition~\ref{defn:delta}.
Suppose that $\tilde{\Phi}_{\tau}$ is an approximation to $\Phi_{\tau}$ satisfying
$\ord_p (\Phi_{\tau}-\tilde{\Phi}_{\tau}) \geq N_{\Phi}$.
Let us denote
\[
\tilde{\chi}(T) = 1 + \sum_{i=1}^b \tilde{\chi}_i T^i 
                = \det\bigl( 1 - T \tilde{\Phi}_{\tau}^{(a)}\bigr).
\]
Then for all $1 \leq i \leq b$, we have
\[
\ord_p (\chi_i - \tilde{\chi}_i) \geq N_{\Phi}-\delta.
\]
\end{thm}

\begin{proof} 
Recall from Theorem~\ref{thm:deltabound} that there exists a matrix 
$W_{\tau} \in M_{b \times b}(\QQ_q)$ satisfying 
$\ord_p(W_{\tau})+\ord_p(W_{\tau}^{-1}) \geq -\delta$, such that for 
$\Phi_{\tau}'=W_{\tau} \Phi_{\tau} \sigma(W)^{-1}$ 
we have \mbox{$\ord_p(\Phi_{\tau}') \geq 0$}.

Defining the matrix 
$\tilde{\Phi}_{\tau}'=W_{\tau} \tilde{\Phi}_{\tau} \sigma(W)^{-1}$, we find that 
$\ord_p(\Phi'_{\tau}-\tilde{\Phi}_{\tau}') \geq N-\delta$, and in particular
$\ord_p(\tilde{\Phi}_{\tau}') \geq 0$. So we obtain
\[
\ord_p \Bigl( \det\bigl(1 - T (\Phi'_{\tau})^{(a)}\bigr) 
            - \det\bigl(1 - T (\tilde{\Phi}'_{\tau})^{(a)}\bigr) \Bigr) \geq N-\delta.
\] 
Note that $(\Phi'_{\tau})^{(a)}= W_{\tau} \Phi_{\tau}^{(a)} W_{\tau}^{-1}$
and $(\tilde{\Phi}'_{\tau})^{(a)}= W_{\tau} \tilde{\Phi}_{\tau}^{(a)} W_{\tau}^{-1}$, so that
\begin{align*}
\chi(T) &= \det\bigl(1 - T (\Phi'_{\tau})^{(a)}\bigr),
&\tilde{\chi}(T) &= \det\bigl(1 - T (\tilde{\Phi}'_{\tau})^{(a)}\bigr).
\end{align*}
This completes the proof.
\end{proof}

\begin{rem} \label{rem:workprecchi}
If we know $\Phi_{\tau}$ to $p$-adic precision $N_{\Phi}$, then
$\chi(T)$ is determined to precision $N_{\Phi}-\delta$.
However, we cannot compute $\chi(T)$ as in the proof of 
Theorem~\ref{thm:preccharpoly}, since we do not know the matrix 
$W_{\tau}$ explicitly. When computing with respect to our basis $\cB$,
there will be loss of $p$-adic precision. The loss of precision 
in computing $\Phi_{\tau}^{(a)}$ from $\Phi_{\tau}$ is at most 
$(a-1)\delta$, and the loss of 
precision in computing $\chi(T)$ from $\Phi_{\tau}^{(a)}$ 
is at most $(b-1) \delta$, by Proposition~\ref{prop:productval} and 
Corollary~\ref{cor:delta}. Therefore, for $\chi(T)$ to be correct
to $p$-adic precision $N_{\Phi}-\delta$, it is sufficient to compute 
$\chi(T)$ from $\Phi_{\tau}$ using $p$-adic working precision 
\[
(N_{\Phi}-\delta)+\bigl((a-1)+(b-1)\bigr)\delta = N_{\Phi}+(a+b-3)\delta.
\] 
\end{rem}

When $p \geq n$, one can improve Theorem~\ref{thm:preccharpoly} by taking into 
account the Hodge numbers $h^{i,n-1-i}$ of $\HdR^{n}(\mathfrak{U}/\mathfrak{S})$.
Recall from Remark~\ref{rem:hnumbers} that $h^{i,n-1-i}=\card{B_i}$.

\begin{defn}
Define $\Gamma: [0,b] \rightarrow \RR$ to be the function whose graph is the convex 
polygon in the plane whose left-most point is the origin and which has slope~$i$ over
the interval of the horizontal axis 
\[
[h^{0,n-1} + \dotsb + h^{i-1,n-i}, h^{0,n-1} + \dotsb + h^{i,n-i-1}].
\]
Note that $\Gamma(b)=b(n-1)/2$ since $h^{i,n-1-i}=h^{n-1-i,i}$ for all $0 \leq i \leq n-1$.
\end{defn}

\begin{thm} \label{thm:pgeqn}
We continue to use the notation of Theorem~\ref{thm:preccharpoly}. 
Let $N_{\Phi} \in \NN$ and suppose that $p \geq n$. Let $\tilde{\Phi}_{\tau}$ 
be an approximation to $\Phi_{\tau}$ satisfying $\ord_p (\Phi_{\tau}-\tilde{\Phi}_{\tau}) \geq N_{\Phi}$. 
Then for all $1 \leq i \leq b$, we have
\[
\ord_p(\chi_i-\tilde{\chi}_i) \geq N_{\Phi} + a\Gamma(i-1).
\]
\end{thm}
 
\begin{proof} 
Note that in this case $\delta=0$, so that $\ord_p(\Phi_{\tau}) \geq 0$. 
By a theorem of Mazur~\citep{Mazur1972}, the map $p^{-1} \Frob_p$ 
sends $(H_i)_{\hat{\tau}} \cap \Lambda_{\tau,crys}$ into $p^i \Lambda_{\tau,crys}$, 
where $\Lambda_{\tau,crys}$ is defined as in the proof of 
Theorem~\ref{thm:deltabound}. This implies that the so called Hodge polygon 
of the $F$-crystal $(\Lambda_{\tau,crys},p^{-1} \Frob_p)$ lies above the graph of 
$\Gamma$. From this it follows that the Hodge polygon of the $F$-crystal 
$(\Lambda_{\tau,crys},\mathfrak{q}^{-1} \Frob_{\mathfrak{q}})$ lies
above the graph of $a\Gamma$. Hence we can find an invertible matrix 
$W_{\tau} \in M_{b \times b}(\ZZ_{\mathfrak{q}})$ such that for all 
$1 \leq i \leq b$, the sum of the valuations of any $i$ different columns
of $W_{\tau} \Phi_{\tau}^{(a)} W_{\tau}^{-1}$ is at least $a\Gamma(i)$.
Note that 
$\ord_p(W_{\tau }\Phi_{\tau}^{(a)} W_{\tau}^{-1}-W_{\tau } \tilde{\Phi}_{\tau}^{(a)} W_{\tau}^{-1}) \geq N_{\Phi}$.
The coefficients $\chi_i$ and $\tilde{\chi}_i$ are alternating sums 
of products of $i$ elements from different columns of 
$W_{\tau }\Phi_{\tau}^{(a)} W_{\tau}^{-1}$ 
and $W_{\tau } \tilde{\Phi}_{\tau}^{(a)} W_{\tau}^{-1}$, respectively.  
Therefore, the required bound follows from Theorem~\ref{prop:productval}. 
\end{proof}

\begin{rem}
For $a=1$, Theorem~\ref{thm:pgeqn} was obtained by Lauder in \citep[Proposition 9.4]{Lauder2006} and
the general idea of using the Hodge filtration to lower $p$-adic precision bounds had already been 
suggested before in \citep[Remark 1.6.4]{AbbottKedlayaRoe2006}. However, we have not been able
to find Theorem~\ref{thm:pgeqn} in the literature for $a \neq 1$. 
\end{rem}

Now we ask to what $p$-adic precision~$N_{\chi_i}$ we need to compute $\chi_i$ 
in order to recover the integer polynomial~$\chi(T)$ exactly.

\begin{thm} \label{thm:N0}
In order to recover the integer polynomial~$\chi(T)$ exactly, 
it suffices to compute $\chi_i$ to $p$-adic precision 
\begin{equation*}
N_{\chi_i} = \floorbig{\log_p \bigl( 2 \bigl( b/i \bigr) \mathfrak{q}^{i (n-1) / 2} \bigr)} + 1.
\end{equation*}
\end{thm}

\begin{proof}
By Theorem~\ref{thm:weildeligne} and the short exact 
sequence~\eqref{eqn:excision}, we have
\[
\chi(T)=\prod_{i=1}^b (1-\alpha_i T),
\]
where the $\alpha_i$ are algebraic integers of absolute 
value $\mathfrak{q}^{(n-1)/2}$ that are permuted under the 
map $\alpha \mapsto \mathfrak{q}^{n-1}/\alpha$. If we denote
$s_j = \sum_{i=1}^{b} \alpha_i^j$, then clearly
\[
|s_j| \leq b \mathfrak{q}^{j (n-1)/2}.
\]
for all $1 \leq j \leq b$. Moreover, if we denote 
$\chi(T) = 1+\sum_{i=1}^{b} \chi_i T^i$, then by the Newton--Girard 
identies, we have
\begin{equation} \label{eq:recursion}
s_j+j \chi_j = - \sum_{i=1}^{j-1} s_{j-i} \chi_i.
\end{equation}
This means that if we are given $\chi_1,\dotsc,\chi_{j-1}$, then 
we can limit $\chi_j$ to an explicit  disk in the complex plane of 
radius $(b/j) \mathfrak{q}^{j (n-1) / 2}$. Therefore, 
if we know each $\chi_i$ to $p$-adic precision $N_{\chi_i}$ satisfying
\[
p^{N_{\chi_i}} > 2 \bigl( b/i \bigr) \mathfrak{q}^{i (n-1) / 2},
\] 
then we can determine $\chi(T)$ exactly using the recursion~\eqref{eq:recursion}.
\end{proof}

\begin{rem}
This bound was first obtained by Kedlaya in~\citep[]{Kedlaya2007}, although it does not 
appear there in exactly this form.
\end{rem}

\begin{thm} \label{thm:precPhitau}
Let $N_{\chi_i}$ be defined as in Theorem~\ref{thm:N0}. To compute $\chi(T)$ exactly, 
it is sufficient to 
compute $\Phi_{\tau}$ with $p$-adic precision
\begin{align*}
N_{\Phi} &= \max_{1 \leq i \leq b} \{ N_{\chi_i} \} +\delta.
\intertext{If moreover $p \geq n$, then this can be improved to}
N_{\Phi} &= \max_{1 \leq i \leq b} \{ N_{\chi_i} -a\Gamma(i-1) \}.
\end{align*}
\end{thm}

\begin{proof}
This follows easily by combining Theorem~\ref{thm:N0} with 
Theorem~\ref{thm:preccharpoly} and Theorem~\ref{thm:pgeqn}, respectively.
\end{proof}

\begin{rem} \label{rem:epsilon}
If the sign $\epsilon = \pm 1$ is known for which 
$\det(\Phi_{\tau}^{(a)}) = \epsilon \mathfrak{q}^{b(n-1)/2}$, then this can 
be improved further. Since 
$\prod_{i=1}^b \alpha_i = \epsilon \mathfrak{q}^{b(n-1)/2}$, and 
the $\alpha_i$ are permuted under the map 
$\alpha \mapsto \mathfrak{q}^{n-1}/\alpha$, we have
\begin{equation*}
\chi_{b-i}=\epsilon (-1)^{b} \mathfrak{q}^{(n-1)(b/2-i)} \chi_b. 
\end{equation*}
So $\chi(T)$ is uniquely determined already by 
$\chi_1,\dotsc,\chi_{\lfloor b/2 \rfloor}$, and it is sufficient 
to take the maxima in Theorem~\ref{thm:precPhitau} running only over 
$1 \leq i \leq \floor{b/2}$.

It is well known that $\epsilon = 1$ when $n$ is even, but when $n$ 
is odd $\epsilon$ is usually not known. In practice it is then often 
still possible to use a smaller precision by computing $\epsilon$ first. 
Let $j$ be the smallest positive integer such that $\chi_{\ceil{b/2} - j} \neq 0$. 
To recover $\chi_0, \dotsc, \chi_{\floor{b/2}+j}$, it is sufficient to take 
the maxima in Theorem~\ref{thm:precPhitau} running only over 
$1 \leq i \leq \floor{b/2}+j$. This allows us to determine $\epsilon$ from 
the two coefficients $\chi_{\ceil{b/2}-j}$ and $\chi_{\floor{b/2}+j}$. 
\end{rem}

We now formalise the complete algorithm for computing $Z(X_{\tau},T)$ 
in Algorithm~\ref{alg:complete}.

\begin{algorithm} 
\caption{Compute $Z(X_{\tau},T)$.}
\label{alg:complete}
\begin{algorithmic}
\vspace{1mm}
\Require $P \in \ZZ_q[t][x_0,\dotsc,x_n]$ homogeneous of degree $d$ satisfying Assumption~\ref{assump:diag} and $\tau \in S(\mathbf{F}_{\mathfrak{q}})$.
\Ensure  The zeta function $Z(X_{\tau},T)$ of the fibre $X_{\tau}$ lying over $\tau$.
\Procedure{ZetaFunction}{$P,\tau$}
\State \begin{compactenum}[{\hspace{1em} } 1.] \vspace{-1.24em}
\item Determine $N_{\Phi}$ from Theorem~\ref{thm:precPhitau}.
\item $N_{\Phi}' \gets N_{\Phi}+(a+b-3) \delta$
\item Determine $K \in \NN$ and $s \in \ZZ_q[t]$ from Theorem \ref{thm:defD}.
\item Compute $\Phi \gets$ \textsc{FrobSeriesExpansion($N_{\Phi},K)$}.
\item Determine $\hat{\tau} \in \mathcal{S}(\ZZ_{\mathfrak{q}})$ to $p$-adic precision $N_{\Phi}+\delta$.
\item Compute $\Phi_{\tau} \gets s(\hat{\tau})^{-1} \bigl( a \Phi \bmod{t^{K}} \bigr)|_{t=\hat{\tau}}$ 
      to $p$-adic precision $N_{\Phi}$, using $p$-adic working precision $N_{\Phi}+\delta$.
\item Compute $\chi(T) \gets \det\bigl(1-T \bigl(\Phi_{\tau} \sigma(\Phi_{\tau}) \dotsm \sigma^{a-1}(\Phi_{\tau}) \bigr)  \bigr)$ to $p$-adic precision $N_{\Phi}-\delta$, 
      using $p$-adic working precision $N'_{\Phi}$.
\item Round $\chi(T)$ to $\ZZ[T]$ using the recursion~\eqref{eq:recursion}. 
\item Set $Z(X_{\tau},T) \gets \bigl( \chi(T)^{(-1)^n} \bigr)/\bigl((1 - T) (1 - \mathfrak{q}T) \dotsm (1 - \mathfrak{q}^{n-1}T)\bigr)$.
\item \Return $Z(X_{\tau},T)$
\end{compactenum}
\EndProcedure
\end{algorithmic}
\end{algorithm}

\begin{rem}
The output of Algorithm~\ref{alg:complete} only depends on $X_{\tau}$.  
In particular, we can also take a polynomial 
$\bar{P} \in \FF_q[t][x_0,\dotsc,x_n]$ as input and at the start of 
the algorithm take $P$ to be an arbitrary lift of $\bar{P}$. For the 
complexity analysis in the next section, we thus take the input size 
to be the size of $\bar{P}$.  However, from a practical point of view, 
it is convenient to keep the lift $P$ as the input, since:
\begin{enumerate} 
\item the matrices $M, \Phi_0, C, \Phi, \Phi_{\tau}$ do depend on $P$,
\item Assumption~\ref{assump:diag} needs to hold for $P$,
\item the runtime and memory usage of the algorithm depend on $P$.
\end{enumerate}
\end{rem}

%%%%%%%%%%%%%%%%%%%%%%%%%%%%%%%%%%%%%%%%%%%%%%%%%%%%%%%%%%%%%%%%%%%%%%%%%%%%%%%

\section{Complexity}

\label{sec:Complexity}

In this section we determine the complexity of Algorithm~\ref{alg:complete}.
We denote $a' = \log_p(q)$, noting that $a'$ divides $a$, 
and let $d_t$ denote the degree of $P$ in the variable~$t$. 
% does this add anything?
% We consider the input variables $n$, $d$, $p$, $a$, $a'$ and $d_t$. 

We use the $\SoftOh(-)$ notation that ignores logarithmic factors, 
i.e.\ $\SoftOh(f)$ denotes the class of functions 
that lie in $\BigOh(f \log^k(f))$ for some $k \in \NN$.

Let us first revisit some results concerning the complexity of the, 
sometimes basic, constituent operations.  We recall that two $k$-bit 
integers can be multiplied in $\SoftOh(k)$ bit operations, and that 
two degree-$k$ polynomials can be multiplied in $\SoftOh(k)$ ring 
operations.  We let $\omega$ denote the least exponent for matrix 
multiplication, so that two $k \times k$ matrices can be multiplied 
in $\BigOh(k^{\omega})$ ring operations. An invertible $k \times k$ matrix 
can then be inverted in $\BigOh(k^{\omega})$ ring operations as well. 
Moreover, it is known that $2 \leq \omega \leq 2.3727$.  Finally, we 
point out that the characteristic polynomial of a matrix can be 
computed in $\BigOh(b^{\omega} \log(b))$ field operations using 
an algorithm of Keller-Gehrig~\citep{KellerGehrig1985}.

Next we consider specific $p$-adic operations, referring the reader to 
Hubrechts~\citep{Hubrechts2010} for further details.  First, images of elements 
of $\QQ_{\mathfrak{q}}$ under~$\sigma^i$, for $0 < i < a$, can be computed to 
$p$-adic precision~$N$ in time $\SoftOh(a \log^2(p) + a N \log(p))$.  Second, 
the Teichm\"uller lift of an element of $\FF_{\mathfrak{q}}$ can be computed 
to precision~$N$ in time $\SoftOh(a N \log^2(p))$.

We start by estimating the degrees of the numerator and denominator of the
connection matrix~$M$.

\begin{prop}
The degrees of $H$, $R$ from Proposition~\ref{thm:denom}, and
$G$, $r$ from Section~\ref{sec:DifferentialSystem} are all
$\BigOh(n(de)^n d_t) \subset \SoftOh((de)^n d_t)$.
\end{prop}

\begin{proof}
Note that $\Delta_k$ in Definition~\ref{defn:resultant} is a square matrix 
of degree~$d_t$, with
\[
{kd-1 \choose n} \leq \biggl( \frac{e(kd-1)}{n} \biggr)^n < (de)^n
\]
columns, where $e$ denotes the base for the natural logarithm. The result 
follows easily from this. Note that the degrees of the numerators and 
denominators of all intermediate results in Algorithm~\ref{alg:Connection} 
are also $\SoftOh((de)^n d_t)$.
\end{proof}

Next we estimate the precisions that we require.

\begin{prop}
All $p$-adic precisions that we use lie in $\SoftOh(a d^n \log(d_t))$ 
and $\deg(s), K$ are both $\SoftOh(apd^n (de)^n d_t)$.
\end{prop}

\begin{proof}
Note that $b \in \BigOh(d^n)$ and $\delta \in \BigOh(n)$.  
In Theorem~\ref{thm:precPhitau}, we have
\begin{align*}
\max_{1 \leq i \leq b} \{N_{\chi_i}\} \in \BigOh\bigl(a n b + \log(b) \bigr) 
            &\subset \SoftOh\bigl(a d^n \bigr),
&N_{\Phi}   &\in \SoftOh(a d^n).
\end{align*}
The precisions $N'_{\Phi}$ and $N_{\Phi} \pm \delta$ in 
Algorithm~\ref{alg:complete} are then $\SoftOh(ad^n)$ as well.
It follows from Theorem~\ref{thm:Gerkmann} that in
Theorem~\ref{thm:defD} we can take
\begin{align*}
\deg{s} &\leq \deg  \Bigl( \prod_{k=2}^n \det(\Delta_k) \Bigr) + (p(n+h(N_{\Phi}))-n) \deg(\det(\Delta_{n+1})), \\
K &\leq \deg(s)+ 1+\deg \Bigl(\prod_{k=2}^n \Delta_k^{-1}\Bigr) + (p(n+h(N_{\Phi}))-n) \deg(\Delta_{n+1}^{-1}) + ph(N_{\Phi}) d_t. 
\end{align*} 
Consequently, we obtain
\[
\deg(s), K \in \SoftOh(apd^n (de)^n d_t).
\]

The $p$-adic precisions $N_{\Phi_0},N_C,N_M,N_{C^{-1}},N'_{C}$ and $N'_{C^{-1}}$ in
Theorem~\ref{thm:Ni} are $\SoftOh(ad^n \log(d_t))$, noting that the logarithms 
that appear there are to base~$p$. Finally, the remaining $p$-adic precisions 
$N'_M$ and $N'_{\Phi}$ are also $\SoftOh(a d^n \log(d_t))$ by Remark~\ref{rem:precgm} 
and Corollary~\ref{cor:Ntilde}, respectively.
\end{proof}

We now analyse the computation of the connection matrix~$M$.
\begin{prop}
The computation of the connection matrix~$M$ using 
Algorithm~\ref{alg:Connection} requires
\begin{align*}
\mbox{time: }  &\SoftOh(a a' \log(p) (d^{n(\omega+2)} e^{n(\omega+1)}+ d^{5n}e^{3n} ) d_t), \\
\mbox{space: } &\SoftOh(a a' \log(p) d^{4n}e^{3n} d_t).
\end{align*}
\end{prop}

\begin{proof}
We first need to construct and invert the matrices~$\Delta_k$.  This 
is dominated by the inversion, which requires $\BigOh((de)^{n \omega})$ 
operations in the ring $\QQ_q[t]$ to $p$-adic 
precision~$\SoftOh(a d^n \log(d_t))$. As there are $\BigOh(n)$ of 
these matrices, this takes time 
\[
\SoftOh((de)^{n \omega} (a' \log(p)) (a d^n \log(d_t)) ((de)^n d_t)) = 
    \SoftOh(a a' \log(p) d^{n(\omega+2)} e^{n(\omega+1)} d_t).
\]
Then we multiply each of the monomials in our basis with $-k \frac{\partial P}{\partial t}$ 
for some $1 \leq k \leq n$ and reduce the product to the basis by 
repeatedly using Algorithm~\ref{alg:Decompose}.  For each of these 
monomials this takes time 
\[
\SoftOh((de)^{2n} (a' \log(p)) (a d^n \log(d_t)) ((de)^n d_t)) = 
    \SoftOh(a a' \log(p) d^{4n}e^{3n} d_t),
\]
because of the quadratic complexity of the matrix-vector product.  
There are $b \in \BigOh(d^n)$ monomials in our basis and hence this 
takes time $\SoftOh(a a' \log(p) d^{5n}e^{3n} d_t)$.

During this computation we have to store $\BigOh(n)$ matrices 
and vectors of size 
\[
\SoftOh((de)^{2n} (a' \log(p)) (a d^n \log(d_t)) ((de)^n d_t)),
\] 
which completes the proof.
\end{proof}

Next we consider the computation of the matrix $\Phi_0$.

\begin{prop} \label{prop:complexityPhi0}
The computation of the matrix~$\Phi_0$ with Algorithm~\ref{alg:Diagfrob} requires
\begin{align*}
\mbox{time: }  &\SoftOh(a^3 p d^{3n} \log^3(d_t)), \\
\mbox{space: } &\SoftOh(a^2 \log(p) d^{2n} \log^2(d_t)).
\end{align*}
\end{prop}

\begin{proof}
We first consider the 
sequences $(d^{-r})_{r=0}^{\mathcal{R}}$ and $(\mu_m)_{m=0}^{\mathcal{M}}$. 
In Proposition~\ref{prop:MR}, we find that 
$\mathcal{M} \in \SoftOh(p N'_{\Phi_0})$ and 
$\mathcal{R} \in \SoftOh(N'_{\Phi_0})$.  Thus, the sequence 
$(d^{-r})_{r=0}^{\mathcal{R}}$ can be computed in time 
$\SoftOh((N'_{\Phi_0})^2 \log(p))$.  Note that we do not 
need all elements of the sequence $( \mu_{m} )_{m=0}^{\mathcal{M}}$, but only 
$\BigOh( \min \{ 1,(n/p) \} \mathcal{M} ) \subset \SoftOh(N'_{\Phi_0})$ 
of them.  Every $\mu_m$ is a sum of $\SoftOh(N'_{\Phi_0})$ terms, each 
of which can be computed from the previous one in $\BigOh(p)$ operations in 
$\ZZ/p^{N'_{\Phi_0}}\ZZ$.  Therefore, the required subsequence can be 
computed in time $\SoftOh((N'_{\Phi_0})^3 p)$.

Similarly, the computation of $\alpha_{u,v}$ for all of the 
$b \in \BigOh(d^n)$ basis vectors can be done in time 
$\SoftOh(d^n (N'_{\Phi_0})^2 \log(p))$.  Hence the whole computation 
takes time 
\[
\SoftOh((N'_{\Phi_0})^3 p) \subset \SoftOh(a^3 p d^{3n} \log^3(d_t)).
\]
The sequence $(d^{-r})_{r=0}^{\mathcal{R}}$ and the 
$\SoftOh(N'_{\Phi_0})$ elements that we need from the sequence 
$(\mu_{m})_{m=0}^{\mathcal{M}}$ can be stored in space 
$\SoftOh((N'_{\Phi_0})^2 \log (p)) \subset \SoftOh(a^2 \log(p) d^{2n} \log^2(d_t))$.
\end{proof}

\begin{rem}
Note that the time complexity in Proposition~\ref{prop:complexityPhi0} 
is quasilinear in~$p$, while in the work of Lauder~\citep{Lauder2004a} 
it is quasiquadratic.  The main reason for this is that Lauder computed 
in the unramified extension~$\QQ_p(\pi)$ of degree~$p$, where a single 
multiplication already takes time quasilinear in~$p$.  It will turn out 
that this crucial improvement also decreases the overall time complexity 
of the deformation method from being quasiquadratic to quasilinear in~$p$.

We should mention that there is a small downside to our approach.  The time 
complexity in Proposition~\ref{prop:complexityPhi0} is quasicubic in~$a$, 
while by using fast exponentials over $\QQ_p(\pi)[[z]]$, this can be 
decreased to being quasiquadratic in~$a$, which again has an effect on the 
entire deformation method.  We address this in the following proposition.
\end{rem}

\begin{prop} \label{prop:compPhi0}
The matrix~$\Phi_0$ can be computed to the required precision in 
\begin{align*}
\mbox{time: }  &\SoftOh\bigl(a^2 p d^{2n} (p + d^n) \log^2(d_t) \bigr), \\
\mbox{space: } &\SoftOh\bigl(a^2 p^2 d^{2n} \log^2(d_t)\bigr).
\end{align*}
\end{prop}

\begin{proof}
The key idea is to slightly modify Algorithm~\ref{alg:Diagfrob} 
and compute the $\BigOh(d^n)$ terms $\alpha_{u,v}$ directly from 
Definition~\ref{defn:alpha}.  We first observe that the valuation 
of the summands in the definition can be bounded by 
$\ord_p(\lambda_m ((u_i+1)/d)_r (-1)^r \pi^{-r} a_i^{m-r}) \geq -1/2$. 
As the only term with negative valuation is $\pi^{-r}$ and 
$\mathcal{R} \in \BigOh(N'_{\Phi_0})$, it suffices 
to compute the terms $\lambda_m$ and $d^{-r}$ to $p$-adic 
precision~$\BigOh(N'_{\Phi_0})$.  This is where the main 
difference occurs:  we compute the sequence 
$(\lambda_m)_{m=0}^{\mathcal{M}}$ via the series expansion of 
$\exp \pi (z - z^p)$ in $\QQ_p(\pi)[[z]]$ modulo $z^{\mathcal{M}+1}$, 
which can be done in $\SoftOh(\mathcal{M})$ operations in $\QQ_p(\pi)$ 
following Brent~\citep{Brent1976}.

The remaining modifications to Algorithm~\ref{alg:Diagfrob} are 
straightforward and the proof of the proposition can be completed 
as before.
\end{proof}

We now consider the computation of the power series expansion of the 
matrix~$\Phi$.

\begin{prop}
Assume that the matrices $M = G/r$ and $\Phi_0$ have been computed already.
The subsequent computation of the power series expansion of the matrix~$\Phi$ 
in Algorithm~\ref{alg:expansion} then requires
\begin{align*}
\mbox{time: }  &\SoftOh(a^2 a' p d^{n(\omega+4)}e^{2n} d_t^2), \\
\mbox{space: } &\SoftOh(a^2 a' p d^{5n} e^n d_t).
\end{align*}
\end{prop}

\begin{proof}
The computation of the power series expansion of~$\Phi$ comprises 
three steps, namely the computation of the matrices $C$, $\sigma(C)^{-1}$ 
and the matrix product $C \Phi_0 \sigma(C)^{-1}$.

As each of the $K$ steps in the computation of $C$ is dominated by the 
computation of $\SoftOh((de)^n d_t)$ matrix products, the matrix~$C$ can be 
computed in time 
\begin{equation*}
\SoftOh\bigl(K ((de)^n d_t) b^{\omega} (a' \log (p)) (a d^n \log (d_t))\bigr) 
    \subset \SoftOh\bigl(a^2 a' p d^{n(\omega + 4)} e^{2n} d_t^2 \bigr).
\end{equation*}
Similarly, the matrix $C^{-1}$ can be computed in time 
\begin{equation*}
\SoftOh\bigl( (K/p) ((de)^n d_t) b^{\omega} (a' \log (p)) (ad^n \log (d_t)) \bigr)
    \subset \SoftOh\bigl( a^2 a' \log(p) d^{n(\omega+4)} e^{2n} d_t^2 \bigr). 
\end{equation*}
Moreover, applying $\sigma$ to the matrix $C^{-1}$ takes time 
\begin{equation*}
\SoftOh\bigl( (K/p) b^2 \bigl(a' \log^2(p) + a' (a d^n \log(d_t)) \log(p) \bigl) \bigr)
\subset \SoftOh\bigl( a^2 a' \log^2(p) d^{5n} e^n d_t \bigr).
\end{equation*}
Finally, the matrix product 
$C \Phi_0 \sigma(C)^{-1}$ can be computed in time 
\begin{equation*}
\SoftOh\bigl( b^{\omega} K (a' \log(p)) (a d^n \log(d_t)) \bigr) 
    \subset \SoftOh\bigl( a^2 a' p d^{n(\omega + 3)} e^n d_t \bigr).
\end{equation*}
The result on the time complexity now follows.

The space requirement is dominated by the matrix~$C$, which has size
\begin{equation*}
\SoftOh(b^2 K (a' \log(p)) (a d^n \log(d_t))) \subset 
    \SoftOh(a^2 a' p d^{5n} e^n d_t). \qedhere
\end{equation*}
\end{proof}

We now move on to the computation of the matrix~$\Phi_{\tau}$.

\begin{prop}
The computation of the matrix~$\Phi_{\tau}$ from the matrix~$\Phi$ 
and $\tau \in S(\FF_{\mathfrak{q}})$ requires
\begin{align*}
\mbox{time: }  &\SoftOh\bigl(a^2 a' p d^{5n} e^n d_t \bigr), \\ 
\mbox{space: } &\SoftOh\bigl(a^2 a' p d^{5n} e^n d_t \bigr).
\end{align*}
\end{prop}

\begin{proof}
We first recall that the Teichm\"uller lift 
$\hat{\tau} \in \mathcal{S}(\ZZ_{\mathfrak{q}})$ 
can be computed to $p$-adic precision $N_{\Phi}+\delta \in \SoftOh(a d^n)$ 
in time $\SoftOh(a^2 d^n \log^2(p))$.  Next, we observe that the 
scalar-matrix product $s \Phi \bmod t^K$ over~$\QQ_q[t]$ 
requires time 
\begin{equation*}
\SoftOh\bigl( b^2 K a' (a d^n) \log(p) \bigr) 
    \subset \SoftOh\bigl( a^2 a' p d^{5n} e^n d_t \bigr).
\end{equation*}
Finally, we consider the substitution of $\hat{\tau}$ 
into the $b^2$ entries of the matrix $s \Phi \bmod t^K$. 
Each of these can be thought of as a modular composition of polynomials 
over~$\QQ_{q}$, where the modulus~$m(t)$ is an irreducible polynomial defining 
the extension $\QQ_{\mathfrak{q}} / \QQ_{q}$ as a quotient of $\QQ_q[t]$, 
which is of degree~$a/a'$. However, care has to be taken to include the 
additional reduction modulo~$m(t)$ of the polynomials of degree less than~$K$, 
so that polynomials involved in the modular composition have degree less 
than~$a/a'$. 

Thus, the substitutions require time %TODO include complexity of modular composition somewhere
\begin{equation*}
\SoftOh\Bigl( b^2 \bigl(K (a' \log(p)) (a d^n) + (a/a') (a' \log(p)) (a d^n) \bigr) \Bigr)
    \subset \SoftOh(a^2 a' p d^{5n} e^n d_t).
\end{equation*}
Clearly, evaluating and inverting $s(\hat{\tau})$ and performing the 
scalar multiplication can be ignored, and the result on the time complexity now follows.

The space requirement is dominated by the matrix $s \Phi \bmod t^K$, 
which has size 
\begin{equation*}
\SoftOh(b^2 K (a' \log(p)) (a d^n)) \subset \SoftOh(a^2 a' p d^{5n} e^n d_t). 
\qedhere
\end{equation*}
\end{proof}

Finally, we consider the computation of the polynomial $\chi(T)$.

\begin{prop}
The computation of $\chi(T)$ from $\Phi_{\tau}$ requires
\begin{align*}
\mbox{time: }  & \SoftOh(a^2  \log^2(p) d^{n(\omega+1)}), \\
\mbox{space: } & \SoftOh(a^2 \log(p) d^{3n}).
\end{align*}
\end{prop}

\begin{proof}
In order to compute $\Phi_{\tau}^{(a)}$ using fast exponentiation for 
semilinear maps, we first need to apply powers of $\sigma$ to $\BigOh(b^2 \log(a))$ 
elements of $\QQ_{\mathfrak{q}}$ and then multiply $\BigOh(\log (a))$ matrices 
of size $b \in \BigOh(d^n)$. This can be done in time 
\begin{gather*}
\SoftOh\Bigl( b^2 \log(a) \bigl( a \log^2(p) + (a \log(p))(a d^n)  \bigr) 
    + \log(a) b^{\omega} (a \log(p))(a d^n)  \Bigr) 
\subset \SoftOh(a^2 \log^2(p) d^{n(\omega+1)}).
\end{gather*}
Next, we compute the reverse characteristic polynomial of matrix 
$\Phi_{\tau}^{(a)} \in M_{b \times b}(\QQ_{\mathfrak{q}})$, which 
can be accomplished in~$\BigOh(b^{\omega} \log(b))$ field operations. 
This amounts to a time complexity of 
\begin{equation*}
\SoftOh\bigl(b^{\omega} (a \log(p))(a d^n) \bigr)
    \subset \SoftOh\bigl( a^2 \log(p) d^{n(\omega+1)} \bigr).
\end{equation*}
Rounding this polynomial to $\ZZ[T]$ can be ignored.

We need to store $\BigOh(\log(a))$ matrices of size~$b$ with entries 
in~$\QQ_{\mathfrak{q}}$.  This requires space 
$\SoftOh(b^2 (a \log(p)) (a d^n)) \subset \SoftOh(a^2 \log(p) d^{3n})$.
\end{proof}

We can now state the total time and space requirements of 
Algorithm~\ref{alg:complete}. Recall that $P \in \ZZ_q[t][x_0,\dotsc,x_n]$ 
denotes a homogeneous polynomial
of degree $d$ satisfying Assumption~\ref{assump:diag} 
and that $\tau \in S(\mathbf{F}_{\mathfrak{q}})$, where 
$\FF_{\mathfrak{q}}/\FF_q$ denotes a finite field extension
of characteristic $p$. 
Moreover, we write $a=\log_p(\mathfrak{q})$, $a'=\log_p(q)$ and let
$d_t$ be the degree of $P$ in the variable $t$. Finally,
we let $e$ denote the base of the natural
logarithm and $\omega$ the least exponent for matrix multiplication.

\begin{thm} \label{thm:totalcomplexity}
The computation of $Z(X_{\tau},T)$ using Algorithm~\ref{alg:complete} requires
\begin{align*}
\mbox{time: }  &\SoftOh\Bigl( 
                a^3 p d^{3n} + 
                a^2 a' p d^{n(\omega+4)} e^{2n} d_t^2+
                aa' \bigl(d^{n(\omega+2)} e^{n(\omega+1)}+ d^{5n} e^{3n}\bigr) d_t 
                \Bigr), \\
\mbox{space: } &\SoftOh\bigl( a^2 a' p d^{5n} d_t + a a' d^{4n} e^{3n} d_t \bigr).
\end{align*}
Alternatively, computing the matrix $\Phi_0$ as in the proof of Proposition~\ref{prop:compPhi0}, 
the computation of $Z(X_{\tau},T)$ requires
\begin{align*}
\mbox{time: }  &\SoftOh\Bigl( 
                a^2 p^2 d^{2n} + 
                a^2 a' p d^{n(\omega+4)} e^{2n} d_t^2+
                aa' \bigl(d^{n(\omega+2)} e^{n(\omega+1)}+ d^{5n} e^{3n}\bigr) d_t 
                \Bigr), \\
\mbox{space: } &\SoftOh\bigl( 
                a^2 p^2 d^{2n} + 
                a^2 a' p d^{5n} d_t + 
                a a' d^{4n} e^{3n} d_t \bigr).
\end{align*}
\end{thm}

\begin{proof}
This follows by adding all complexities from the previous propositions and 
leaving out terms that are dominated by other terms or powers of
logarithms of other terms.
\end{proof}

\begin{rem} In \citep{Lauder2004a}, Lauder took $d_t=1$ and showed that his
algorithm requires
\begin{align*}
\mbox{time: }  &\SoftOh\bigl(a^3 p^2 (d^{n(\omega+5)} e^{3n} + d^{6n} e^{5n}) \bigr), \\ 
\mbox{space: } &\SoftOh\bigl(a^3 p^2 d^{6n} e^{4n} \bigr).
\end{align*}
His main goal was to show that these complexities are $(pad^n)^{\BigOh(1)}$, 
which was not the case for previously known algorithms like 
\citep{AbbottKedlayaRoe2006, LauderWan2008}. Our Theorem~\ref{thm:totalcomplexity} 
improves Lauder's complexity bounds somewhat by lowering the constants hidden in the 
$\BigOh(1)$. As far as we know, the complexity bounds from 
Theorem~\ref{thm:totalcomplexity} are therefore the best ones known. Note that since
a natural measure for the input size is $\log(p) a d^n$, these bounds are only
polynomial in the input size for fixed $p$, which is something all $p$-adic algorithms 
tend to suffer from.
\end{rem}

%%%%%%%%%%%%%%%%%%%%%%%%%%%%%%%%%%%%%%%%%%%%%%%%%%%%%%%%%%%%%%%%%%%%%%%%%%%%%%%

\section{Examples}
\label{sec:Examples}

In this section we compute some examples using the (preliminary) implementation of our algorithm 
that is available at \url{https://github.com/SPancratz/deformation}. This implementation is restricted to
the case $q=p$, i.e. where the family of hypersurfaces is defined over a prime field. All computations were 
carried out on a laptop with two Intel Core i7-3540M processors running at 3GHz and with 8Gb of RAM, but only 
used a single processor and no more than about 1.5 Gb of RAM. Timings
are obtained using the C function clock() and are stated in minutes (m) or seconds (s). We are still
working on improving the implementation and hope to include updated examples and timings in the final 
version of this paper.

\subsection{Quintic curve}

We consider the family of genus six curves over $\ZZ$ given by the polynomial 
\begin{equation*}
P=x_0^5 + x_1^5 + x_2^5 + t x_0 x_1 x_2^3,
\end{equation*}
which Gerkmann~\citep[\S 7.4]{Gerkmann2007} considers as an element of 
$\ZZ_p[t][x_0,x_1,x_2]$ for $p=2,3,7$.  Note that these are, in fact, 
three examples.  The connection matrix $M \in M_{12 \times 12}(\QQ(t))$ 
only has to be computed once, and turns out to have denominator $r=27t^5+3125$.
The set of exponents at each of the zeros of $r$ is $\{-1,0\}$, and 
after changing basis by
\[ 
W=\mbox{diag}(t^{-1},t,t^{-2},t^{-1},1,1,1,1,1,t^2,t^2,1)
\] 
as in Remark~\ref{rem:basischange}, the set of exponents at~$\infty$ is 
$\{-2/3,2/3,1,4/3,7/3,8/3,13/3,14/3\}$.  Note that by
Remark~\ref{rem:epsilon}, we only need to determine the bottom 
half of the coefficients of the polynomial $\chi(T)$ directly.

\subsubsection{Precisions}

\noindent
\textit{Prime $p=2$.} 
We take $\log_p(\mathfrak{q})=50$ and let $\tau \in \FF_{\mathfrak{q}}$ be 
a zero of the Conway polynomial of $\FF_{\mathfrak{q}}/\FF_p$.  We first 
compute $\delta=0$ and $N_{\Phi}=N_{\Phi}'=153$.  As the roots of~$r$ are 
$2$-adic integers and distinct  modulo~$2$, we can apply 
Theorem~\ref{thm:KedlayaTuitman} everywhere. We find $\theta_z=320$ at all 
zeros~$z$ of $r$ and $\theta_{\infty}=12$, so that we can take $s=r^{320}$ 
and $K=1613$. We now compute the remaining $p$-adic precisions
$N_{\Phi_0}=174$, $N_C=163$, $N_{C^{-1}}=164$, $N_M=187$, 
$N_C'=184$, $N_{C^{-1}}'=183$, $N_{\Phi_0}'=176$, and $N_M'=188$. \\

\noindent 
\textit{Prime $p=3$.}
We take $\log_p(\mathfrak{q})=40$ and let $\tau \in \FF_{\mathfrak{q}}$ be 
a zero of the Conway polynomial of $\FF_{\mathfrak{q}}/\FF_p$.  We first 
compute $\delta=0$ and $N_{\Phi}=N_{\Phi}'=122$. As the roots of~$r$ are all 
contained in the residue disk at~$\infty$, we cannot apply 
Theorem~\ref{thm:KedlayaTuitman} and have to apply Theorem~\ref{thm:Gerkmann} 
instead. At each of the zeros $z$ of~$r$ we find $\mu_z=0$, $\nu_z=-1$, $\theta_z=397$ 
and at $\infty$ we find $\mu_{\infty}=-3$, $\nu_{\infty}=-2$, $\theta_{\infty}=793$, so that 
we can take $s=r^{397}$ and $K=2779$. We now compute the remaining $p$-adic 
precisions
$N_{\Phi_0}=137$, $N_C=129$, $N_{C^{-1}}=130$, $N_M=147$, 
$N_C'=146$, $N_{C^{-1}}'=143$, $N_{\Phi_0}'=139$, and $N_M'=147$. \\

\noindent
\textit{Prime $p=7$.}
We take $\log_p(\mathfrak{q})=10$, and let $\tau \in \FF_{\mathfrak{q}}$ be 
a zero of the Conway polynomial of $\FF_{\mathfrak{q}}/\FF_p$. We first 
compute $\delta=0$ and $N_{\Phi}=N_{\Phi}'=31$.  As the zeros of $r$ are 
$7$-adic integers and distinct modulo~$7$, we can apply 
Theorem~\ref{thm:KedlayaTuitman} everywhere.  We find $\theta_z=224$ at all 
zeros~$z$ of $r$ and $\theta_{\infty}=25$, so that we can take $s=r^{224}$ and
$K=1146$. We now compute the remaining $p$-adic precisions
$N_{\Phi_0}=38$, $N_C=34$, $N_{C^{-1}}=37$, $N_M=44$, 
$N_C'=43$, $N_{C^{-1}}'=40$, $N_{\Phi_0}'=40$, and $N_M'=44$.

\subsubsection{Timings}

We now compare the performance of our FLINT implementation with the 
timings reported by Gerkmann.  \\ 

\begin{center}
\begin{tabular}{l l l l l l l} \toprule
                & \multicolumn{2}{l}{$p = 2$} 
                & \multicolumn{2}{l}{$p = 3$} 
                & \multicolumn{2}{l}{$p = 7$} \\ \midrule
Computation     & P--T & G     & P--T  & G       & P--T & G     \\ \midrule
$M$             & 0.00s& ?     & 0.00s & ?       & 0.00s& ?     \\
$\Phi_0$        & 0.03s& 2.65m & 0.03s & 5.86m   & 0.01s& 1.31m \\
$\Phi$          & 0.67s& 1.33m & 1.28s & 1.38m   & 0.29s& 0.89m \\
$Z(X_{\tau},T)$ & 9.20s& 3.96m & 6.42s & 101.40m & 0.15s& 1.09m \\
Total           & 9.90s& 7.94m & 7.73s & 108.64m & 0.45s& 3.29m \\ \bottomrule
\end{tabular}
\end{center}

\begin{rem}
Our timings in these examples are $50-500$ lower than those of Gerkmann, 
taking into account that the 101.40m entry in the second column should probably
be 101.40s. 
\end{rem}

\subsection{Quartic surface}

We consider the family of quartic K3 surfaces over $\ZZ_3$ given by
\begin{equation*}
P=x_0^4 + x_1^4 + x_2^4 + x_3^4 + t x_0 x_1 x_2 x_3,
\end{equation*}
which Gerkmann considers in~\citep[\S 7.5]{Gerkmann2007}. The connection matrix 
$M \in M_{21 \times 21}(\QQ(t))$ turns out to have denominator 
$r=t^4-256$. The set of exponents at each of the zeros of $r$ is $\{-3/2,-1/2,0\}$, 
and after changing basis by
\[
W=\mbox{diag}(t^{-2},1,1,1,1,1,1,1,1,1,t^{-1},1,1,1,1,1,1,1,1,1,1)
\] 
as in Remark~\ref{rem:basischange}, the set of exponents  at $\infty$ is 
$\{1,2,3\}$. 

\subsubsection{Precisions}

We take $a=\log_3(\mathfrak{q})=20$, let $\alpha \in \FF_{\mathfrak{q}}$
be a zero of the Conway polynomial of $\FF_{\mathfrak{q}}/\FF_{3}$, and take
$\tau=\alpha^{2345}$.  We first compute $\delta=0$ and $N_{\Phi}=N_{\Phi}'=43$.  
Since the zeros of $r$ are $3$-adic integers and different modulo $3$, we can apply 
Theorem~\ref{thm:KedlayaTuitman} everywhere. We find $\theta_z=148$ at all zeros 
of $r$ and $\theta_{\infty}=6$, so that we can take $s=r^{148}$
and $K=599$. We now compute the remaining $p$-adic 
precisions $N_{\Phi_0}=65$, $N_C=53$, $N_{C^{-1}}=55$, $N_M=74$, 
$N_C'=73$, $N_{C^{-1}}'=68$, $N_{\Phi_0}'=68$ and $N_M'=75$.

\subsubsection{Timings}

\begin{center}
\begin{tabular}{l l l} \toprule
Computation     & P--T & G      \\ \midrule
$M$             & 0.00s& ?      \\
$\Phi_0$        & 0.01s& 45.26m \\
$\Phi$          & 0.22s& 19.90m \\
$Z(X_{\tau},T)$ & 0.76s& 18.66m \\
Total           & 0.99s& 83.82m \\ \bottomrule
\end{tabular}
\end{center}

As the final result of the computation we obtain
\begin{align*}
\mathfrak{q} \chi(T/\mathfrak{q})=&-3486784401T^{21} - 39675197243T^{20} - 191506614866T^{19} - 482588946510T^{18} \\
                                  &-552821487569T^{17} + 243001138765T^{16} + 1641410078472T^{15} + 1793016627512T^{14} \\
                                  &-410199003010T^{13} - 2617001208822T^{12} - 1586643774924T^{11} + 1586643774924T^{10} \\
                                  &+2617001208822T^9 + 410199003010T^8 - 1793016627512T^7 - 1641410078472T^6 \\
                                  &-243001138765T^5+ 552821487569T^4 + 482588946510T^3 + 191506614866T^2 \\
                                  &+39675197243T + 3486784401.
\end{align*}

\begin{rem}
Our timings in this example are about 5000 times lower than those of Gerkmann. 
\end{rem}

\begin{rem}
We should mention that Gerkmann does not include timings for part of his computations.
For example, he does not mention the time it takes to compute the $p$-adic precision 
parameter that he calls $\delta$  (which involves solving $b^2$ differential equations 
similar to the one for the matrix $C$!). Also, there seem to be some mistakes in 
his precision analysis. For example, for the example in this section, Gerkmann's final 
$p$-adic precision is not high enough to recover the exact zeta function with 
the precision bounds that he is using.
\end{rem}

Our remaining examples could not have been computed with previous implementations of the
deformation method and, as far as we know, not with any other method either. For example, Lauder noted 
that he could not compute the connection matrix for a family of quintic curves or quartic surfaces given 
by a polynomial with more than a \emph{few} nonzero terms. This was the main motivation for the work in 
the Phd-thesis of the first author and the present paper. We are now able to compute the zeta function of
for example quintic curves and quartic surfaces over small finite fields given by a polynomial with 
\emph{all} of its coefficients nonzero.

\subsection{Generic quintic curve}

We consider the family of generic quintic curves over $\ZZ_{11}$ given by 
\begin{equation*}
\begin{split}
%P:=X^5+Y^5+Z^5+t*(3*X^4*Y-X^4*Z+2*X*Y^4+Y^4*Z+4*X*Z^4+5*Y*Z^4
%   +X^3*Y^2+X^3*Z^2+ X^2*Y^3+Y^3*Z^2+X^2*Z^3+Y^2*Z^3
%   +X^3*Y*Z+X*Y^3*Z+X*Y*Z^3+X^2*Y^2*Z+X^2*Y*Z^2+X*Y^2*Z^2);
P=x_0^5 + x_1^5 + x_2^5
+ t \bigl( & 3x_0^4 x_1-x_0^4 x_2+2 x_0 x_1^4+x_1^4 x_2+4 x_0 x_2^4+5 x_1 x_2^4 + x_0^3 x_1^2+x_0^3 x_2^2 \\ 
           + & x_0^2 x_1^3+ x_1^3 x_2^2+x_0^2 x_2^3+x_1^2 x_2^3 + x_0^3 x_1 x_2 + x_0 x_1^3 x_2 + x_0 x_1 x_2^3 \\ 
           + & x_0^2 x_1^2 x_3 + x_0^2 x_1 x_2^2 + x_0 x_1^2 x_2^2 \bigr).
\end{split}
\end{equation*}
The connection matrix $M \in M_{12 \times 12}(\QQ(t))$ has denominator $r=r_1 r_2$, with $r_1,r_2 \in \ZZ[t]$ irreducible
of degree $30$ and $48$, respectively. The set of exponents is $\{ 0,1 \}$ at the zeros of $r_1$ and $\{ -1,0 \}$ at the
zeros of $r_2$. Moreover, the matrix $M$ has a simple pole at $\infty$ and the set of exponents is $\{1,2\}$ there.

\subsubsection{Precisions}

We take $a=\log_{11}(\mathfrak{q})=10$ and let $\tau$ be a zero of the Conway polynomial of 
$\FF_{\mathfrak{q}}/\FF_{11}$. We first compute $\delta=0$ and $N_{\Phi}=N_{\Phi}'=31$.
Since the zeros of $r$ are $11$-adic integers and different modulo $11$, we can apply 
Theorem~\ref{thm:KedlayaTuitman} everywhere. We find $\theta_z=352$ at the zeros of $r_2$
and $\theta_{\infty}=9$ at $\infty$. At the residue disk of a zero $z$ of $r_1$, noting that 
$R$ has no other zeros there, we get a better bound by applying 
Theorem~\ref{thm:Gerkmann} with $\mu_z=-1$, $\nu_z=0$ and find $\theta_z=1$, so that we can 
take $s=r_1 r_2^{352}$ and $K=16936$. We now compute the remaining $p$-adic precisions 
$N_{\Phi_0}=40$, $N_C=35$, $N_{C^{-1}}=36$, $N_M=47$, $N_C'=46$, $N_{C^{-1}}'=43$, 
$N_{\Phi_0}'=42$ and $N_M'=47$.

\subsubsection{Timings}

\begin{center}
\begin{tabular}{l l} \toprule
Computation     & P--T \\ \midrule
$M$             & 4.43s     \\
$\Phi_0$        & 0.04s     \\
$\Phi$          & 3.76m     \\
$Z(X_{\tau},T)$ & 9.48s     \\
Total           & 3.99m     \\ \bottomrule
\end{tabular}
\end{center}

As the final result of the computation we obtain
\begin{align*}
\chi(T) = &304481639541418099574449295360278774639038415066698088621947601T^{12} \\
          &+3777543732986291528931322507772938448980494046897871937792T^{11} \\
          &+23639674223084796290417361507756397439403378558130350T^{10} \\ 
          &+54141350391870148138663709375646947695242620108T^9 \\
          &-363231942297281636316475334779570949613459T^8 \\
          &-4138991673785569248268236480720472276T^7 \\
          &-28015243113507339254470240817992T^6 \\
          &-159576046483273177468242676T^5 \\ 
          &-539921137173243550659T^4 \\
          &+3102762464729708T^3 + 52231690350T^2 + 321792T + 1.
\end{align*}

\subsection{Generic quartic surface}

We consider the family of generic quartic K3 surfaces over $\ZZ_{7}$ given by 
\begin{equation*}
\begin{split}
P=x_0^4 + x_1^4 + x_2^4 + x_3^4 
+ t \bigl( & -3 x_0^3 x_1 + 2 x_0^3 x_2 - 2 x_0 x_1 x_2 x_3 + x_0^3 x_3 - x_0 x_1^3 - 3 x_1^3 x_2 + x_2^3 x_3 \\ &
          + 2 x_1^3 x_3 + x_0 x_2^3 - 2 x_1 x_2^3 - x_0 x_3^3 + x_1 x_3^3 + 3 x_2 x_3^3 + x_0^2 x_1^2 \\ & 
          + 3 x_0^2 x_2^2 + x_0^2 x_3^2 + 2 x_1^2 x_2^2 - 2 x_1^2 x_3^2 + x_2^2 x_3^2 
          + 2 x_0^2 x_1 x_2 \\ & + x_0^2 x_1 x_3 + 3 x_0^2 x_2 x_3 - x_0 x_1^2 x_2 + 2 x_0 x_1^2 x_3 
          + 3 x_1^2 x_2 x_3 - x_0 x_1 x_2^2 \\ & 
          + 3 x_0 x_2^2 x_3 + x_1 x_2^2 x_3 + 2 x_0 x_1 x_3^2 + 2 x_0 x_2 x_3^2 + 2 x_1 x_2 x_3^2 \bigr),
\end{split}
\end{equation*}
The connection matrix $M \in M_{21 \times 21}(\QQ(t))$ has denominator $r=r_1 r_2 r_3$, 
with $r_1,r_2,r_3 \in \ZZ[t]$ irreducible of degree $16$, $104$ and $108$, respectively. 
The matrix $M$ has a simple pole at $\infty$ and the set of exponents is $\{1,2,3\}$ there.

\subsubsection{Precisions}

We take $\tau$ to be  $1 \in \FF_7$. We first compute $\delta=0$ and $N_{\Phi}=N_{\Phi}'=4$. Since we
do not know the exponents of $M$ at its finite poles, we cannot use Theorem~\ref{thm:KedlayaTuitman} at poles
outside the residue disk at infinity. Applying Theorem~\ref{thm:Gerkmann} at all the zeros $z$ of $R$, using
the bound $\ord_z(\Delta_k^{-1}) \geq -\ord_z(\det(\Delta_k))$, we find that we can take 
$s=\det(\Delta_2)\det(\Delta_3)\det(\Delta_4)^{67}$, which has degree $25027$. However, most of the zeros of
$R$ do not lie in the residue disk of a pole of $M$, so the corresponding factors can be dropped from $s$, which
brings the degree of $s$ down to $8833$. We can apply Theorem~\ref{thm:KedlayaTuitman} at $\infty$ to find $\theta_{\infty}=-4$,
so that we can take $K=8830$. We now compute the remaining $p$-adic precisions 
$N_{\Phi_0}=22$, $N_C=12$, $N_{C^{-1}}=14$, $N_M=30$, $N_C'=29$, $N_{C^{-1}}'=24$, 
$N_{\Phi_0}'=25$ and $N_M'=30$.

\subsubsection{Timings}

\begin{center}
\begin{tabular}{l l} \toprule
Computation     & P--T \\ \midrule
$M$             & 3.55m     \\
$\Phi_0$        & 0.02s     \\
$\Phi$          & 17.53m     \\
$Z(X_{\tau},T)$ & 6.7s     \\
Total           & 21.19m     \\ \bottomrule
\end{tabular}
\end{center}

As the final result of the computation we obtain
\begin{align*}
7\chi(T/7) = &7T^{21} - 5T^{20} + 6T^{19} - 6T^{18} + 4T^{17} - 11T^{16} + 5T^{15} - 9T^{14} + 4T^{13} + 3T^{12} \\
             &+4T^{11} + 4T^{10} + 3T^9 + 4T^8 - 9T^7 + 5T^6 - 11T^5 + 4T^4 - 6T^3 + 6T^2 - 5T + 7.
\end{align*}

\begin{rem}
In all of our timings, we have not included the time used to compute 
the $p$-adic and $t$-adic precisions, since we used MAGMA for this. Note 
that we get $\det(\Delta_k)$ for free, since we already compute the $LUP$ 
decomposition of $\Delta_k$ anyway. Factorizing the polynomials $r,R$ over 
$\QQ$ and $\FF_p$ was instantaneous in all cases. The only thing that 
did take more time was the computation of the exponents in the 
generic examples. For the family of generic quintic curves this took 
$10s$,  $47s$ and $1.5s$ at the zeros of $r_1$, the zeros of $r_2$ and
at $\infty$, respectively. For the family of generic quartic surfaces,
we were not able to compute the exponents at the zeros of $r_2$ and $r_3$
within reasonable time. Note however that we did not use them in the
precision analysis either.
\end{rem}

\phantomsection

\bibliographystyle{plainnat}
\bibliography{deformation}

\end{document}

