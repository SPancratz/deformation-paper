\documentclass[a4paper,11pt]{article}

\author{Sebastian Pancratz, Jan Tuitman}
\title{Improvements to the deformation method for counting points on smooth projective hypersurfaces}

% Geometry and page layout %%%%%%%%%%%%%%%%%%%%%%%%%%%%%%%%%%%%%%%%%%%%%%%%%%%%

\usepackage[hmargin=3.2cm,vmargin=3.2cm,a4paper,centering,twoside]{geometry}

% Other packages %%%%%%%%%%%%%%%%%%%%%%%%%%%%%%%%%%%%%%%%%%%%%%%%%%%%%%%%%%%%%%

\usepackage[T1]{fontenc}
\usepackage{ae,aecompl}
\usepackage{verbatim}
\usepackage{ifpdf}

% hyperref %%%%%%%%%%%%%%%%%%%%%%%%%%%%%%%%%%%%%%%%%%%%%%%%%%%%%%%%%%%%%%%%%%%%

\usepackage{hyperref}
\hypersetup{
    colorlinks=false,   % false: boxed links; true: colored links
    citecolor=green,    % color of links to bibliography
    filecolor=magenta,  % color of file links
    linkcolor=red,      % color of internal links
    urlcolor=blue       % color of external links
}

\makeatletter
\newcommand\org@hypertarget{}
\let\org@hypertarget\hypertarget
\renewcommand\hypertarget[2]{%
    \Hy@raisedlink{\org@hypertarget{#1}{}}#2%
} 
\makeatother

\ifpdf
    \hypersetup{
        pdftitle={Deformation method},
        pdfauthor={Sebastian Pancratz, Jan Tuitman},
        pdfsubject={Computational Number Theory},
        bookmarks=true,
        bookmarksnumbered=true,
        unicode=true,
        pdfstartview={FitH},
        pdfpagemode={UseOutlines}
    }
\fi

% algorithmic %%%%%%%%%%%%%%%%%%%%%%%%%%%%%%%%%%%%%%%%%%%%%%%%%%%%%%%%%%%%%%%%%

\usepackage[section]{algorithm}
\usepackage[noend]{algpseudocode}

\renewcommand{\algorithmicrequire}{\textbf{Input:}}
\renewcommand{\algorithmicensure}{\textbf{Output:}}

% natbib %%%%%%%%%%%%%%%%%%%%%%%%%%%%%%%%%%%%%%%%%%%%%%%%%%%%%%%%%%%%%%%%%%%%%%

\usepackage{natbib}

\bibpunct{[}{]}{,}{n}{}{}

% url %%%%%%%%%%%%%%%%%%%%%%%%%%%%%%%%%%%%%%%%%%%%%%%%%%%%%%%%%%%%%%%%%%%%%%%%%

\usepackage{url}

\makeatletter
\def\url@leostyle{%
  \@ifundefined{selectfont}{\def\UrlFont{\sf}}{\def\UrlFont{\small\ttfamily}}}
\makeatother
\urlstyle{leostyle}

% Enumeration %%%%%%%%%%%%%%%%%%%%%%%%%%%%%%%%%%%%%%%%%%%%%%%%%%%%%%%%%%%%%%%%%

\usepackage{paralist}

%Why are these better?
%\setlength{\pltopsep}{0.24em}
%\setlength{\plpartopsep}{0em}
%\setlength{\plitemsep}{0.24em}

% This should do what we want
%   \setdefaultenum{(i)}{(a)}{1.}{A}
% but it does not work for references, dropping the parentheses.  The following
% hack does work.

\renewcommand{\theenumi}{(\roman{enumi})}
\renewcommand{\theenumii}{(\alph{enumii})}
\renewcommand{\theenumiii}{\arabic{enumiii}.}
\renewcommand{\theenumiv}{\Alph{enumiv}}

\renewcommand{\labelenumi}{\theenumi}
\renewcommand{\labelenumii}{\theenumii}
\renewcommand{\labelenumiii}{\theenumiii}
\renewcommand{\labelenumiv}{\theenumiv}

%%%%%%%%%%%%%%%%%%%%%%%%%%%%%%%%%%%%%%%%%%%%%%%%%%%%%%%%%%%%%%%%%%%%%%%%%%%%%%%
% Mathematics

% Packages %%%%%%%%%%%%%%%%%%%%%%%%%%%%%%%%%%%%%%%%%%%%%%%%%%%%%%%%%%%%%%%%%%%%

\usepackage{amsmath,amsthm,amscd,amsfonts,amssymb}
\usepackage{cases}
\usepackage[all]{xy}

\allowdisplaybreaks[4]
\numberwithin{equation}{section}

% Customised notation %%%%%%%%%%%%%%%%%%%%%%%%%%%%%%%%%%%%%%%%%%%%%%%%%%%%%%%%%

\providecommand{\abs}[1]{\lvert#1\rvert}                 % Absolute value
\providecommand{\absbig}[1]{\bigl\lvert#1\bigr\rvert}    % Absolute value
\providecommand{\absBig}[1]{\Bigl\lvert#1\Bigr\rvert}    % Absolute value
\providecommand{\absbigg}[1]{\biggl\lvert#1\biggr\rvert} % Absolute value

\providecommand{\norm}[1]{\lVert#1\rVert}              % Norm
\providecommand{\normbig}[1]{\bigl\lVert#1\bigr\rVert} % Norm
\providecommand{\normBig}[1]{\Bigl\lVert#1\Bigr\rVert} % Norm

\providecommand{\floor}[1]{\left\lfloor#1\right\rfloor}   % Floor
\providecommand{\floorts}[1]{\lfloor#1\rfloor}            % Floor
\providecommand{\floorbig}[1]{\bigl\lfloor#1\bigr\rfloor} % Floor
\providecommand{\floorBig}[1]{\Bigl\lfloor#1\Bigr\rfloor} % Floor

\providecommand{\ceil}[1]{\left\lceil#1\right\rceil}   % Ceiling
\providecommand{\ceilts}[1]{\lceil#1\rceil}            % Ceiling
\providecommand{\ceilbig}[1]{\bigl\lceil#1\bigr\rceil} % Ceiling
\providecommand{\ceilBig}[1]{\Bigl\lceil#1\Bigr\rceil} % Ceiling

\newcommand{\NN}{\mathbf{N}} % Natural numbers
\newcommand{\ZZ}{\mathbf{Z}} % Integers
\newcommand{\QQ}{\mathbf{Q}} % Rationals
\newcommand{\RR}{\mathbf{R}} % Real numbers
\newcommand{\CC}{\mathbf{C}} % Complex numbers
\newcommand{\FF}{\mathbf{F}} % Finite field

\renewcommand{\to}{\rightarrow}        % Right arrow
\newcommand{\into}{\hookrightarrow}    % Injection arrow
\newcommand{\onto}{\twoheadrightarrow} % Surjection arrow

\DeclareMathOperator{\fCoKer}{coker} % Cokernel
\DeclareMathOperator{\fKer}{ker}     % Kernel
\DeclareMathOperator{\fIm}{im}       % Image

\DeclareMathOperator{\Res}{Res}   % Resultant
\DeclareMathOperator{\Tr}{Tr}     % Trace
\DeclareMathOperator{\Trace}{Tr}  % Trace
\DeclareMathOperator{\Norm}{N}    % Norm
\DeclareMathOperator{\Disc}{Disc} % Discriminant

\DeclareMathOperator{\Gal}{Gal}          % Galois group
\DeclareMathOperator{\ord}{ord}          % Order
\DeclareMathOperator{\sgn}{sgn}          % Sign, signature
\DeclareMathOperator{\Frob}{\mathcal{F}} % Frobenius
\DeclareMathOperator{\Hom}{Hom}          % Space of homomorphisms
\DeclareMathOperator{\Spec}{Spec}        % Spectrum

\providecommand{\HdR}{H_{\text{dR}}}    % de Rham cohomology
\providecommand{\Het}{H_{\text{\'et}}}  % etale cohomology
\providecommand{\Hrig}{H_{\text{rig}}}  % rigid cohomology

\providecommand{\cB}{\mathcal{B}} % Basis
\providecommand{\cR}{\mathcal{R}} % Row index set
\providecommand{\cC}{\mathcal{C}} % Column index set
\providecommand{\cM}{\mathcal{M}} % Complexity of multiplication

\providecommand{\BigOh}{\mathcal{O}} % Big-oh notation

% Theorems etc %%%%%%%%%%%%%%%%%%%%%%%%%%%%%%%%%%%%%%%%%%%%%%%%%%%%%%%%%%%%%%%%

\theoremstyle{definition}

\newtheorem{thm}{Theorem}[section]
\newtheorem{lem}[thm]{Lemma}
\newtheorem{prop}[thm]{Proposition}
\newtheorem{cor}[thm]{Corollary}
\newtheorem{defn}[thm]{Definition}
\newtheorem{exmp}[thm]{Example}
\newtheorem{rem}[thm]{Remark}
\newtheorem{prob}[thm]{Problem}
\newtheorem{assump}[thm]{Assumption}

% Roman numerals %%%%%%%%%%%%%%%%%%%%%%%%%%%%%%%%%%%%%%%%%%%%%%%%%%%%%%%%%%%%%%

\makeatletter
\newcommand{\rmnum}[1]{\romannumeral #1}
\newcommand{\Rmnum}[1]{\expandafter\@slowromancap\romannumeral #1@}
\makeatother

%%%%%%%%%%%%%%%%%%%%%%%%%%%%%%%%%%%%%%%%%%%%%%%%%%%%%%%%%%%%%%%%%%%%%%%%%%%%%%%
% DOCUMENT                                                                    %
%%%%%%%%%%%%%%%%%%%%%%%%%%%%%%%%%%%%%%%%%%%%%%%%%%%%%%%%%%%%%%%%%%%%%%%%%%%%%%%

\begin{document}

\maketitle

\tableofcontents

%%%%%%%%%%%%%%%%%%%%%%%%%%%%%%%%%%%%%%%%%%%%%%%%%%%%%%%%%%%%%%%%%%%%%%%%%%%%%%%

\section{Introduction}
\label{sec:Introduction}

%%%%%%%%%%%%%%%%%%%%%%%%%%%%%%%%%%%%%%%%%%%%%%%%%%%%%%%%%%%%%%%%%%%%%%%%%%%%%%

\newpage

\section{Theoretical background}
\label{sec:Background}

Let $X$ denote an algebraic variety over $\FF_q$.

\begin{thm} If $X$ is a smooth proper of dimension $d$, 
then \label{thm:weildeligne}
\[
Z(X,T)=\frac{p_1 p_3 \ldots p_{2d-1}}{p_0 p_2 p_4 \ldots p_{2d}},
\]
where for all $i$:
\begin{enumerate}
\item $p_i = \prod_j (1-\alpha_{i,j}T) \in \ZZ[T]$,
\item the transformation $t \rightarrow q^d/t$ maps the $\alpha_{i,j}$ bijectively to 
      the $\alpha_{2d-i,k}$ (preserving multiplicities),
\item $|\alpha_{i,j}| = q^{i/2}$ for all $j$, and every embedding $\overline{\QQ} \hookrightarrow \CC$. 
\end{enumerate}
\end{thm}

\begin{proof}
\end{proof}

\begin{defn}
Let $\Hrig^{\bullet}(X)$ denote the rigid
cohomology spaces of $X$. These are finite dimensional vector spaces over $\QQ_q$ 
that are contravariantly functorial in $X$ and are equiped with an action of 
the $p$-th power Frobenius map on $X$ that we denote by $F_p$. For the 
construction and properties of the rigid cohomology spaces we refer to [[TODO Jan]].
\end{defn}

\begin{thm} If $X$ is a smooth proper algebraic variety over $\FF_q$ of dimension $d$, 
then
\[
Z(X,T) = \prod_{i=0}^{2d} \det(1- T F_p | \Hrig^i(X))^{(-1)^{(i+1)}}
\]
\end{thm}

\begin{proof}
\end{proof}

Let $\pi: \mathfrak{X} \rightarrow \mathfrak{S}$ be a smooth family of algebraic varieties defined over $\QQ_q$
and $\Omega^{\bullet}_{\mathfrak{X}/\mathfrak{S}}$ its complex of relative K\"ahler differentials.

\begin{defn}
The relative algebraic de Rham cohomology sheaf $\HdR^i(\mathfrak{X}/\mathfrak{S})$ on $\mathfrak{S}$ is 
defined as the hypercohomology sheaf $\mathbb{R}^i \pi_{*} \Omega^{\bullet}_{\mathfrak{X}/\mathfrak{S}}$. 
\end{defn}

\begin{rem}
If $\mathfrak{X}/\mathfrak{S}$ admits a relative normal crossing compactification, then 
$\HdR^i(\mathfrak{X}/\mathfrak{S})$ is a vector bundle.
\end{rem}

The $\HdR^i(\mathfrak{X}/\mathfrak{S})$ come equiped with an integrable connection, which is called
the Gauss-Manin connection. Let us first recall the notion of a connection on a vector bundle.

\begin{defn}
Let $V$ be a vector bundle on $\mathfrak{S}$. A connection on $V$ is a map of vector bundles
$\nabla: V \rightarrow V \otimes \Omega^1_{\mathfrak{S}}$
which satisfies the Leibniz rule
\begin{align*}
\nabla(f s)=f\nabla(s)+s \otimes df
\end{align*} 
for all local sections $f$ of $\mathcal{O}_{\mathfrak{S}}$ and $s$ of $V$. 
\end{defn}

The Gauss-Manin connection on $\HdR^i(\mathfrak{X}/\mathfrak{S})$ can be defined as follows.

\begin{defn}
The De Rham complex $\Omega^{\bullet}_{\mathfrak{X}}$ can be equiped with the decreasing
filtration
\[
F^i=im(\Omega^{\bullet-i}_{\mathfrak{X}} \otimes \pi^* \Omega^i_{\mathfrak{S}} \rightarrow \Omega^{\bullet}_{\mathfrak{X}}). 
\]

The spectral sequence associated to this filtration has as its first sheet
\[
E_1^{p,q}=\Omega^p_{\mathfrak{S}} \otimes \HdR^q(\mathfrak{X}/\mathfrak{S}).
\]

The Gauss-Manin connection $\nabla:H^i(\mathfrak{X}/\mathfrak{S}) \rightarrow H^i(\mathfrak{X}/\mathfrak{S}) \otimes 
\Omega^1_{\mathfrak{S}}$ is now defined as the differential $d_1: E_1^{0,i} \rightarrow E_1^{1,i}$ in 
this spectral sequence.
\end{defn}

\begin{rem}
We can give a more explicit description of $\nabla$ when $\mathfrak{X}/\mathfrak{S}$ is affine. If we lift a 
relative $i$-cocycle $\omega \in \Omega^i_{\mathfrak{X}/\mathfrak{S}}$ to an absolute $i$-form 
$\omega' \in \Omega^i_{\mathfrak{X}}$ and apply the absolute differential $d$, we get an element of 
$\Omega^1_{\mathfrak{S}} \wedge \Omega^i_{\mathfrak{X}/\mathfrak{S}}$. Projecting onto 
$\Omega^1_{\mathfrak{S}} \otimes \HdR^i(\mathfrak{X}/\mathfrak{S})$, we obtain $\nabla(\omega)$. 
\end{rem}


\begin{thm}
Existence of Frobenius structure on relative algebraic de Rham cohomology in liftable normal crossing case and base change.
\end{thm}

\begin{proof}
\end{proof}

\begin{thm}
Cohomology of a projective hypersurface
\end{thm}

\begin{proof}
\end{proof}

\begin{defn}
define $\delta$
\end{defn}

\begin{thm}
Lower bounds on valuation $\Phi$ and $\Phi^{-1}$ in terms of $\delta$.
\end{thm}

%%%%%%%%%%%%%%%%%%%%%%%%%%%%%%%%%%%%%%%%%%%%%%%%%%%%%%%%%%%%%%%%%%%%%%%%%%%%%%%
\section{Computing the connection matrix}
\label{sec:Connection}
\label{sec:deRham}

In this section we compute the action of the Gauss--Manin connection~$\nabla$ on 
the algebraic de~Rham cohomology space $\HdR^{n}(\mathfrak{U}/\mathfrak{S})$ of the 
complement $\mathfrak{U}/\mathfrak{S}$ of a family of smooth hypersurfaces 
$\mathfrak{X}/\mathfrak{S}$ contained in $\mathbf{P}^n_{\mathfrak{S}}$.
First we recall how to compute in $\HdR^{n}(\mathfrak{U}/\mathfrak{S})$ 
following the method of Griffiths and Dwork.  As we prefer to perform linear 
algebra operations over a field, we will mainly work with the de~Rham 
cohomology vector space $\HdR^{n}(\mathfrak{U}_L)$ of the generic fiber 
$\mathfrak{U}_L = \mathfrak{U}/\mathfrak{S} \times_{\mathfrak{S}} \Spec{L}$, 
where $L=\QQ_q(t)$ denotes the function field of $\mathfrak{S}$. 

\begin{thm}
Let $\Omega$ denote the $n$-form $\Omega$ in $\HdR^{n}(\mathfrak{U}_L)$ 
defined by 
\begin{equation}
\Omega = \sum_{i=0}^n (-1)^i x_i d x_0 \wedge \dotsb \wedge \widehat{d x_i} \wedge \dotsb \wedge d x_n.
\end{equation}
The algebraic de~Rham cohomology space $\HdR^{n}(\mathfrak{U}_L)$ is 
isomorphic as an $L$-vector space to the quotient of the space of closed $n$-forms 
$Q \Omega / P^k$ with $k \in \NN$ and $Q \in L[x_0, x_1, \dotsc, x_n]$ 
homogeneous of degree $k d - (n + 1)$, by the subspace of exact $n$-forms generated by
\begin{equation} \label{eq:deRhamRel}
\frac{(\partial_i Q) \Omega}{P^k} - k \frac{Q (\partial_i P) \Omega}{P^{k+1}},
\end{equation}
for all $0 \leq i \leq n$ and with $k \in \NN$ and $Q \in L[x_0, x_1, \dotsc, x_n]$ 
homogeneous of degree $kd-n$, where $\partial_i$ denotes the partial derivative operator 
with respect to~$x_i$.
\end{thm}

\begin{proof}
See for example Griffiths~\citep[\S 4]{Griffiths1969}.
\end{proof}

The cohomology space $\HdR^{n}(\mathfrak{U}_L)$ is equipped with 
an increasing filtration of $L$-vector spaces 
for which $\mbox{Fil}^k \HdR^{n}(\mathfrak{U}_L)$ consists of 
all elements that can be represented by $n$-forms $Q \Omega / P^k$ 
with $Q \in L[x_0, x_1, \dotsc, x_n]$ homogeneous of degree $kd - (n + 1)$. 
It follows from a theorem of Macaulay~\citep[\S 4, (4.11)]{Griffiths1969} 
that $\mbox{Fil}^n \HdR^{n}(\mathfrak{U}_L)= \HdR^{n}(\mathfrak{U}_L)$. 
Therefore, a basis for $\HdR^{n}(\mathfrak{U}_L)$ can in principle be 
computed using finite dimensional linear algebra. We now define an explicit 
basis of a simple form for $\HdR^{n}(\mathfrak{U}_L)$ for the families 
that we are interested in.

%TODO perhaps mention that all spaces of polynomials contain Q=0 (annoying)
%TODO mention that F_i is the (reverse) Hodge filtration on H^{n-1}(X) with the correct indices

\begin{defn} \label{defn:MonBasis}
For $k \in \NN$, we define the following sets of monomials: 
\begin{align*}
F_k & = \{ x^i : i \in \mathbf{N}_{0}^{n+1}, \abs{i} = k d - (n+1) \}, \\
B_k & = \{ x^i : i \in \mathbf{N}_{0}^{n+1}, \abs{i} = k d - (n+1) \text{ and $i_j < d-1$ for all $j$}\}, \\
R_k & = F_k - B_k,
\end{align*}
where $x^i = x_0^{i_0} \dotsm x_n^{i_n}$ and $\abs{i} = i_0 + \dotsb + i_n$. 
We also write $\cB_k = \{Q \Omega / P^k : Q \in B_k\}$ and denote
$B = B_1 \cup \dotsb \cup B_n$ and $\cB = \cB_1 \cup \dotsb \cup \cB_n$.
\end{defn}

We will show below that if the family $\mathfrak{X}/{\mathfrak{S}}$ contains 
a diagonal fiber then the set $\cB$ forms a basis for $\HdR^n(\mathfrak{U}_L)$.

\begin{defn} \label{defn:IndexSets}
For $k \in \NN$, we let $C_k^{(0)}$ be the set of monomials of total 
degree $(k-1)d - n$ and then inductively, for $j = 1, \dotsc, n$, define 
$C_k^{(j)}$ to be the set of monomials in $C_k^{(j-1)}$ except for those 
divisible by $x_{j-1}^{d-1}$.  Moreover, we define the multi-set $C_k$ as 
the disjoint union of $C_k^{(0)}, \dotsc, C_k^{(n)}$.  We shall write an 
element of this multi-set as $(j, g)$, when referring to a monomial~$g$ 
in~$C_k^{(j)}$.
\end{defn}

\begin{thm} \label{thm:Isomorphism}
Suppose that the family $\mathfrak{X}/\mathfrak{S}$ of smooth projective
hypersurfaces given by the polynomial~$P$ in $\QQ_q[t][x_0, \dotsc, x_n]$ contains 
a diagonal fibre.  For $k \in \NN$ and $0 \leq j \leq n$, let $U_k^{(j)}$ be 
the $L$-vector space of polynomials with basis $C_k^{(j)}$, and let $U_k$ 
denote the cartesian product $U_k = U_k^{(0)} \times \dotsb \times U_k^{(n)}$. 
Moreover, let $V_k$ and $W_k$ be the $L$-vector spaces of polynomials with 
bases $F_k$ and $R_k$, respectively, and let $\pi \colon V_k \rightarrow W_k$ 
denote the linear map that sends the elements of $B_k$ to zero and the 
elements of $R_k$ to themselves. %safer formulation, projection not unique
Then the map 
\begin{equation}
\phi_k \colon U_k \to W_k, \; \; \;
(Q_0, \dotsc, Q_n) \mapsto \pi \bigl( Q_0 \partial_0 P + \dotsb + Q_n \partial_n P \bigr)
\end{equation}
is an isomorphism of $L$-vector spaces.
\end{thm}

\begin{proof}
We first show that, for all $k \in \NN$, the multi-sets $R_k$ and $C_k$ 
have the same cardinality.

We construct a bijection $R_k \to C_k$, representing the 
monomials by their exponent tuples.  Let $i = (i_0, \dotsc, i_n)$ be an
element of $R_k$.  If $i_0 \geq d-1$, we define the image as
$(i_0-d-1, i_1, \dotsc, i_n) \in C_k^{(0)}$.  More generally, if 
$i_0 < d-1, \dotsc, i_{j-1} < d-1$ and $i_j \geq d-1$, we define the image as 
$(i_0, \dotsc, i_{j-1}, i_j-(d-1), i_{j+1}, \dotsc, i_n) \in C_k^{(j)}$.  
It is easy to verify that this map is indeed a bijection.

If $R_k$ and $C_k$ are empty, then $U_k$ and $W_k$ are the zero vector spaces, and
the theorem holds trivially. So suppose that $R_k$ and $C_k$ are nonempty, that is to
say, $k \geq n/d + 1$.

In order to establish that the map $\phi_k \colon U_k \to W_k$ is an 
isomorphism of $L$-vector spaces, we look at its matrix with respect to 
the given bases.   We define the auxiliary matrix $\Delta_k$ with 
row and column index sets $R_k$ and $C_k$, respectively, as follows.  
Given $f \in R_k$ and $(j,g) \in C_k$, we set the corresponding entry in 
$\Delta_k$ to be the coefficient of the monomial $f/g$ in $\partial_j P$ if 
$g$ divides $f$ and $0$ otherwise. It is immediate that $\Delta_k$ is the 
matrix representing $\phi_k$ with respect to the bases $C_k$ and $R_k$ of 
$U_k$ and $W_k$, respectively.

The assumption that the family~$\mathfrak{X}/\mathfrak{S}$ contains a diagonal 
hypersurface means that for some~$t_0$ the fibre $\mathfrak{X}_{t_0}$ is 
defined by a polynomial of the form 
\begin{equation}
P_{t_0}(x_0, \dotsc, x_n) = a_0 x_0^d + \dotsb + a_n x_n^d
\end{equation}
with $a_0, \dotsc, a_n \in \QQ_q^{\times}$.

We can now show that the determinant of $\Delta_k$ is nonzero in~$L$.  Since 
specialisation to the diagonal fibre, that is, evaluation of the matrix at 
$t = t_0$, commutes with computing the determinant, it suffices to show that 
the determinant of $(\Delta_k) \big |_{t=t_0}$ is nonzero.  Since, for 
$0 \leq j \leq n$, we have $\partial_j P_{t_0} (x_0, \dotsc, x_n) = d a_j x_j^{d-1}$, 
there is precisely one nonzero entry in each column and each row of $\Delta_k$.  
Namely, in column $(j, g) \in C_k$ and row $g x_j^{d-1} \in R_k$ there is the 
nonzero entry $d a_j$, concluding the proof.
\end{proof}

We can use Theorem~\ref{thm:Isomorphism} to give a routine {\sc Decompose}, 
formalised in Algorithm~\ref{alg:Decompose}, which given 
$Q \in L[x_0, \dotsc, x_n]$ homogeneous of degree $kd - (n+1)$, 
returns an expression 
\begin{equation}
Q = Q_0 \partial_0 P + \dotsb + Q_n \partial_n P + \gamma_k
\end{equation} 
with $Q_0, \dotsc, Q_n \in L[x_0, \dotsc, x_n]$ homogeneous of 
degree $kd-n$ and $\gamma_k$ in the $L$-span of $B_k$. We can in turn 
use {\sc Decompose} to furnish another routine {\sc Reduce}, formalised 
in Algorithm~\ref{alg:PoleRed}, which given a closed $n$-form $Q\Omega/P^k$ 
with $Q \in L[x_0, \dotsc, x_n]$ homogeneous of degree $kd - (n+1)$ returns 
an expression
\begin{equation}
Q \Omega / P^k \equiv \gamma_{1} \Omega / P^{1} + \dotsb + \gamma_n \Omega / P^n,
\end{equation}
with $\gamma_i$ in the $L$-span of $B_i$ for $1 \leq i \leq n$ and 
where $\equiv$ denotes equality in cohomology.


\begin{algorithm}
\caption{Obtain coordinates in the Jacobian ideal modulo basis elements}
\label{alg:Decompose}
\begin{algorithmic}
\Require  $P$ in $\QQ_q[t][x_0, \dotsc, x_n]$ homogeneous of degree~$d$, 
         defining a family $\mathfrak{X}/\mathfrak{S}$ of smooth projective 
         hypersurfaces that contains a diagonal fibre, $Q \in L[x_0, \dotsc, x_n]$ homogeneous of degree $kd - (n+1)$.
\Ensure  $Q_0, \dotsc, Q_n \in L[x_0, \dotsc, x_n]$ homogeneous of degree 
         $k(d-1)-n$, and $\gamma_k$ in the $L$-span of $B_k$, such that 
         $Q = Q_0 \partial P_0 + \dotsb + Q_n \partial_n P +\gamma_k$.
\Procedure{Decompose}{$P,Q$}
\State \begin{compactenum}[{\hspace{1em}} 1.] \vspace{-1.24em}
\item Let $b$ be the vector of length $\abs{R_k}$ such that the entry 
      corresponding to $x^i \in R_k$ is the coefficient of 
      $x^i$ in $Q$.
\item Solve for the unique vector $v$ of length $\abs{C_k}$ satisfying 
      $\Delta_k v = b$.  Note that we can write $v$ 
      as $\bigl(v^{(0)}, \dotsc, v^{(n)}\bigr)$ 
      where $v^{(j)}$ is a vector of length $\abs{C_k^{(j)}}$ 
      for $0 \leq j \leq n$.
\item \textbf{for} $j=0$ \textbf{to} $n$ \textbf{do} 
\item[] \hspace{1em} $Q_j \gets \sum_{g \in C_k^{(j)}} v_g^{(j)} g$ ($v_g^{(j)}$ is the entry in $v^{(j)}$ corresponding to $g \in C_k^{(j)}$)
\item[] $\gamma_k \gets Q-(Q_0 \partial P_0 + \dotsb + Q_n \partial_n P)$
\item \textbf{return} $Q_0, \dotsc, Q_n,\gamma_k$.
\EndProcedure
\end{compactenum}
\end{algorithmic}
\end{algorithm}


\begin{algorithm}
\caption{Reduce $Q \Omega / P^k$ in $\HdR^n(\mathfrak{U}_L)$}
\label{alg:PoleRed}
\begin{algorithmic}
\vspace{1mm}
\Require  $P$ in $\QQ_q[t][x_0, \dotsc, x_n]$ homogeneous of degree~$d$, 
         defining a family $\mathfrak{X}/\mathfrak{S}$ of smooth projective 
         hypersurfaces that contains a diagonal fibre, $Q \in L[x_0, \dotsc, x_n]$ 
         homogeneous of degree $kd - (n+1)$.
\Ensure  $\gamma_i$ in the $L$-span of $B_i$ for $1 \leq i \leq n$, with  
         $Q \Omega / P^k \equiv \gamma_{1} \Omega / P^{1} + \dotsb + \gamma_n \Omega / P^n$.
\Procedure{Reduce}{$P,Q$}
\While{$k \geq n+1$}
\State $Q_0, \dotsc, Q_n, \bullet \gets \Call{Decompose}{Q}$
\State $k \gets k-1$
\State $Q \gets k^{-1} \sum_{i=0}^n \partial_i Q_i$
\EndWhile
\While{$Q \not \in B_k$}
\State $Q_0, \dotsc, Q_n, \gamma_k \gets \Call{Decompose}{Q}$
\State $k \gets k-1$
\State $Q \gets k^{-1} \sum_{i=0}^n \partial_i Q_i$
\EndWhile
\If{$Q \neq 0$}
\State $\gamma_{k} \gets Q$
\State $k \gets k-1$
\EndIf
\State $\gamma_{1}, \dotsc, \gamma_{k} \gets 0$
\State \textbf{return} $\gamma_{1}, \dotsc, \gamma_n$
\EndProcedure
\end{algorithmic}
\end{algorithm}

We now establish that the set~$\cB$ indeed forms a basis for 
$\HdR^n(\mathfrak{U}_L)$, as announced before.  We begin with an 
auxiliary result describing the cardinality of the set~$\cB$.

\begin{prop} \label{prop:BasisSize}
The set $\cB$ has cardinality
\begin{equation}
\bigl((d-1)/d\bigr) \bigl((d-1)^{n} - (-1)^{n}\bigr).
\end{equation}
\end{prop}

\begin{proof}
First note that if we denote
\begin{align*}
V &= \{(i_0,\cdots,i_n) \in (\ZZ/d\ZZ)^{n+1} : \sum_{j=0}^n i_j = n+1\}, \\
H_j &= \{(i_0,\cdots,i_n) \in (\ZZ/d\ZZ)^{n+1} : i_j = -1 \},
\end{align*}
then  $\cB$ is in one-to-one correspondence with the set $V-(H_0 \cup \cdots \cup H_n)$. Now by the inclusion-exclusion
principle, 
\begin{align*}
\abs{V \cap (H_0 \cup \cdots \cup H_n)} &= \sum_{j=0}^n \abs{V \cap H_j} - \sum_{0 \leq j < k \leq n} \abs{V \cap H_j \cap H_k}
+ \cdots + (-1)^{n} \abs{V \cap H_0 \cap \cdots \cap H_n} \\
&= {n+1 \choose 1} d^{n-1} -{n+1 \choose 2} d^{n-2} + \cdots + (-1)^{n-1} {n+1 \choose n} + (-1)^{n} \\
&= (1/d)\bigl(d^{n+1}+(-1)^{n+1} - (d-1)^{n+1}\bigr)+(-1)^n,
\end{align*}
so that
\begin{align*}
\abs{V-(H_0 \cup \cdots \cup H_n)}&=\abs{V}-\abs{V \cap (H_0 \cup \cdots \cup H_n)} \\
&= d^n - (1/d)\bigl(d^{n+1}+(-1)^{n+1} - (d-1)^{n+1}+d (-1)^n \bigr) \\
&= \bigl((d-1)/d\bigr) \bigl((d-1)^{n} - (-1)^{n}\bigr),
\end{align*}
and the proof is complete.
\end{proof}

\begin{thm} \label{thm:Basis}
Suppose that the family of smooth projective hypersurfaces $\mathfrak{X}/\mathfrak{S}$ 
contains a diagonal fibre.  Then the set~$\cB$ from Definition~\ref{defn:MonBasis} 
is a basis for the $L$-vector space $\HdR^n(\mathfrak{U}_L)$.
\end{thm}

\begin{proof}
We already know that $\HdR^n(\mathfrak{U}_L)$ is spanned by the classes of the 
$n$-forms $Q \Omega / P^k$ with $Q \in L[x_0, \dotsc, x_n]$ homogeneous of degree 
$kd - (n+1)$ for $k \in \NN$. Applying Algorithm ~\ref{alg:PoleRed}, we obtain an 
expression for the class of $Q \Omega / P^k$ as an $L$-linear combination of 
elements in $\cB$.  This shows that $\cB$ spans the vector space 
$\HdR^n(\mathfrak{U}_L)$. However, we know from [[TODO Jan]] that 
$\dim \HdR^n(\mathfrak{U}_L) = \bigl((d-1)/d\bigr) \bigl( (d-1)^n - (-1)^n \bigr)$.  
Therefore, it follows from Proposition~\ref{prop:BasisSize} that $\cB$ is linearly 
independent as well.
\end{proof}

We now finally describe the action of the Gauss--Manin connection~$\nabla$ on 
$\HdR^n(\mathfrak{U}_L)$.  Suppose that we are given a basis element 
$x^i \Omega / P^k \in \cB_k$.  Following the description in Section [[TODO Jan]], 
we compute
\begin{equation} \label{eqn:nabla}
\nabla \biggl(\frac{x^i \Omega}{P^k}\biggr) \equiv 
dt \otimes \frac{- k x^i P_t \Omega}{P^{k+1}},
\end{equation}
where $P_t = dP/dt$ and $\equiv$ denotes equality in 
$\Omega_{L} \otimes \HdR^n(\mathfrak{U}_L)$. We apply 
Algorithm~\ref{alg:PoleRed} in order to write
\begin{equation}
dt \otimes \frac{- k x^i P_t \Omega}{P^{k+1}} \equiv 
dt \otimes \left( \frac{\gamma_{1}}{P} + \dotsb + \frac{\gamma_n}{P^n} \right) \Omega
\end{equation}
where $\gamma_i$ is an element in the $L$-span of~$B_i$ for $1 \leq i \leq n$. This is
formalised in Algorithm~\ref{alg:Connection} below.

\begin{algorithm}
\caption{Compute the Gauss--Manin connection matrix}
\label{alg:Connection}
\begin{algorithmic}
\Require $P$ in $\QQ_q[t][x_0, \dotsc, x_n]$ homogeneous of degree~$d$, 
         defining a family $\mathfrak{X}/\mathfrak{S}$ of smooth projective 
         hypersurfaces that contains a diagonal fibre.
\Ensure  The matrix~$M$ of $\nabla$ with respect to $\cB$.
\Procedure{GMConnection}{$P$}
\For{$g \in B$} 
\State $k \gets  (\deg(g)+(n+1))/d$
\State $Q \gets  - k g P_t$
\State $\gamma_{1}, \dotsc, \gamma_n \gets$
      {\sc Reduce($P,Q$)} 
\For{$f \in B$}
\State $l \gets (\deg(f)+(n+1))/d$
\State $M_{f,g} \gets$ the coefficient of $f$ in $\gamma_l$
\EndFor
\EndFor
\textbf{return} $M$
\EndProcedure
\end{algorithmic}
\end{algorithm}

\begin{defn} \label{defn:resultant}
We define the resultant $\Delta \in \ZZ_q[t]$ of the polynomial $P$ with respect to the
basis $\cB$ by
\[
\Delta = \prod_{k=\lfloor n/d \rfloor+1}^{n+1}  \det(\Delta_k),
\]
with the matrices $\Delta_k$ defined as in the proof of Theorem~\ref{thm:Isomorphism}.
\end{defn}

\begin{thm} \label{thm:denom}
The matrix $M$ of $\nabla$ with respect to $\cB$ is of the form
$H/\Delta$, with $H \in M_{b \times b}(\QQ_q[t])$.
\end{thm}

\begin{proof}
The only time in Algorithm \ref{alg:Connection} that nonconstant denominators are introduced is when the subroutine {\sc{Reduce}} calls its subroutine {\sc{Decompose}} and a linear system $\Delta_k v = b$ is solved. Since this happens only for $k=\lfloor n/d \rfloor+1, \dotsc, n+1$ and at most once for each such $k$, the result is clear.
\end{proof}

\begin{assump} \label{assump:r0}
We will always assume that $\Delta(0) \neq 0 \bmod p$ so that the matrix $M$ does not have any pole in the residue disk around $0$.
\end{assump}

\begin{rem} \mbox{ }
[[TODO some remarks on computing $\Delta_k^{-1}$ only once and sparse linear algebra]]
\end{rem}

%%%%%%%%%%%%%%%%%%%%%%%%%%%%%%%%%%%%%%%%%%%%%%%%%%%%%%%%%%%%%%%%%%%%%%%%%%%%%%%

\section{Frobenius on diagonal hypersurfaces}
\label{sec:Diagonal}

\subsection{A formula of Dwork}

In this section we compute the action of Frobenius on the cohomology 
space $\Hrig^{n}(U_0) \cong \HdR^{n}(\mathfrak{U}_0)$ associated 
to the diagonal fibre of the family. Our method is based on an 
explicit formula of Dwork~\citep[\S 4]{Dwork1964}, which has already 
appeared in the work of Lauder~\citep[\S 6]{Lauder2004b} and 
Gerkmann~\citep[\S 4.4]{Gerkmann2007}. However, by rewriting this formula 
we obtain an algorithm that performs significantly better in practice 
than direct implementation.

We consider a single smooth 
projective diagonal hypersurface~$\mathcal{X}_0$ over $\ZZ_p$ defined by 
a polynomial $P_0 \in \ZZ_p[x_0, \dotsc, x_n]$ of the form
\begin{equation*}
P_0(x_0, x_1, \dotsc, x_n) = 
    a_0 x_0^d + a_1 x_1^d + \dotsb + a_n x_n^d,
\end{equation*}
where $a_0, a_1, \dotsc, a_n \in \ZZ_p^{\times}$ and $p \nmid d$. 
Let $\mathfrak{X}_0 = \mathcal{X}_0 \otimes_{\ZZ_p} \QQ_p$ denote the generic 
fibre of $\mathcal{X}_0$ and let $\mathcal{U}_0$ and $\mathfrak{U}_0$ be the complements
of $\mathcal{X}_0$ and $\mathfrak{X}_0$, respectively. 
We fix our choice of basis~$\cB$ for $\HdR^{n}(\mathfrak{U}_0)$ 
as in Definition~\ref{defn:MonBasis}. 

Our goal is to compute 
the matrix~$\Phi_0$ representing the action of $p^{-1} \Frob_p$ on 
$\Hrig^n(U_0) \cong \HdR^n(\mathfrak{U}_0)$, with respect to the basis~$\cB$, 
to $p$-adic precision~$N$. It will turn out that this matrix has only one
nonzero element in every row and column.

We work over the ramified extension~$\QQ_p(\pi)$ where $\pi^{p-1} = -p$, 
and normalise the valuation such that \mbox{$\ord_p(\pi) = (p-1)^{-1}$}.

Let $u = (u_0, \dotsc, u_n)$ and $v = (v_0, \dotsc, v_n)$ be tuples 
of integers such that $x^u, x^v \in B$ and $p (u_i+1) \equiv v_i+1 \pmod{d}$
for all $i$. Furthermore, let $k_u$ and $k_v$ denote integers such that 
$d k_u = \sum_{i=0}^n (u_i + 1)$ and similarly for $k_v$.  For $m \geq 0$, 
let $\lambda_m$ denote the coefficient of $z^m$ in the power series expansion 
of $\exp \pi (z - z^p)$ and define products $(w)_r = \prod_{j=0}^{r-1} (w + j)$ 
for $w \in \QQ$ and $r \geq 0$. We introduce terms
\begin{equation} \label{eq:alpha}
\alpha_{u,v} = \pi^{k_v - k_u} \prod_{i = 0}^n \sum_{m, r} \lambda_m ((u_i + 1) / d)_r (-1)^r \pi^{-r} {\hat{a}_i}^{m-r},
\end{equation}
where $\hat{a}_i \in \ZZ_p$ denotes the Teichm\"uller lift of 
$a_i \in \mathbf{F}_p$ and the summation indices $m, r \geq 0$ 
satisfy $p (u_i+1) - (v_i+1) = d (m - pr)$.

\begin{rem}
We could eliminate $m$ from Equation \eqref{eq:alpha} by writing 
\[
m(r)=\frac{p(u_i+1) - (v_i+1)}{d}+pr.
\]
Note that $p(u_i+1) - (v_i+1) \geq 0$, since it is divisible by $d$ and greater than $-d$.
So after eliminating $m$, the sum would just be over all $r \geq 0$. However, to simplify
notation, we will keep the index $m$.
\end{rem}

\begin{thm} \label{thm:01-03-diagfrob}
Let $\omega_1$ denote an element of $\cB$ corresponding to a tuple 
$u \in \ZZ^{n+1}$ and let $\omega_2$ denote the unique element of~$\cB$ 
corresponding to a tuple $v \in \ZZ^{n+1}$ such that
we have $p (u_i + 1) \equiv v_i + 1 \pmod{d}$ for all $i$. Then
\begin{equation*}
p^{-1} \Frob_p (\omega_1) = 
    (-1)^{k_u + k_v} \frac{(k_v - 1)!}{(k_u - 1)!} p^n \alpha_{u,v}^{-1} \omega_2.
\end{equation*}
\end{thm}

\begin{proof}
See Dwork~\citep[\S 4]{Dwork1964} or Lauder~\citep[\S 6.1]{Lauder2004b}.  
Both references also treat the more general case when $\mathcal{X}$ is 
a smooth projective diagonal hypersurface over $\ZZ_q$, for $q$ a prime 
power.
\end{proof}

\subsection{Improved formulas}

At first sight it appears that this computation genuinely has to 
take place in the extension field~$\QQ_p(\pi)$.  This is, however, 
not the case as we will show now.  The terms $\alpha_{u,v}$ 
will turn out to be elements of~$\ZZ_p$. We provide expressions 
for them that are more suitable for computations.

First, it is straightforward to obtain a more explicit description 
of the coefficients~$\lambda_m$ via an elementary calculation:

\begin{lem} \label{lem:lambdam}
Let $\pi^{p-1} = -p$ and, for $m \geq 0$, let $\lambda_m$ 
be the coefficient of $z^m$ in the power series expansion 
of $\exp \pi (z - z^p)$ in $\QQ_p[[z]]$.  Then 
\begin{equation*}
\pi^{- (m \bmod{(p-1)})} \lambda_m = (-1)^{\floor{m/(p-1)}} \sum_{k=0}^{\floor{m/p}} p^{\floor{m/(p-1)} - k} \frac{1}{(m-pk)! k!}
\end{equation*}
where $m \bmod{(p-1)}$ denotes the remainder of $m$ upon Euclidean 
division by $p-1$. \hfill $\qedsymbol$
\end{lem}

\begin{thm} \label{thm:alpha}
Let $u, v \in \ZZ^{n+1}$ be such that 
$x^u, x^v \in B$ and 
$p (u_i + 1) \equiv v_i + 1 \pmod{d}$ for all~$i$. 
Then 
\begin{equation*}
\alpha_{u,v} = (-p)^{k_u} \prod_{i=0}^n 
    \hat{a}_i^{(p (u_i + 1) - (v_i + 1))/d} \sum_{m,r} 
    \Bigl(\frac{u_i+1}{d}\Bigr)_r 
    \sum_{k=0}^{\floor{m/p}} \frac{p^{r-k}}{(m-pk)! k!}.
\end{equation*}
where $m, r \geq 0$ satisfy $p (u_i + 1) - (v_i + 1) = d (m - pr)$.
\end{thm}

\begin{proof}
We start from Equation \eqref{eq:alpha} and use Lemma~\ref{lem:lambdam} to find that $\alpha_{u,v}$ 
is equal to 
\begin{gather*}
\pi^{k_v-k_u} \prod_{i=0}^n \sum_{m,r} (-1)^{\floor{m/(p-1)}} \pi^{m \bmod{(p-1)}} \biggl( \sum_{k=0}^{\floor{m/p}} \frac{p^{\floor{m/(p-1)}-k}}{(m-pk)!k!} \biggr) \Bigl( \frac{u_i+1}{d} \Bigr)_r (-1)^r \pi^{-r} \hat{a}_i^{m-r}.
\intertext{Now we write $m = \floor{m/(p-1)} (p-1) + \bigl(m \bmod{(p-1)}\bigr)$, and simplify using $\pi^{p-1} = -p$, to obtain}
\alpha_{u,v}=\pi^{k_v-k_u} \prod_{i=0}^n \sum_{m,r} \pi^{m - r} \biggl( \sum_{k=0}^{\floor{m/p}} \frac{p^{-k}}{(m-pk)!k!} \biggr) \Bigl( \frac{u_i+1}{d} \Bigr)_r (-1)^r \hat{a}_i^{m-r}. \\
\intertext{Using that $m-r = (p-1)r + \bigl(p(u_i+1) - (v_i+1)\bigr)/d$, we finally arrive at}
\alpha_{u,v}=(-p)^{k_u} \prod_{i=0}^n \hat{a}_i^{(p (u_i + 1) - (v_i + 1))/d} \sum_{m,r} \Bigl( \frac{u_i+1}{d} \Bigr)_r \sum_{k=0}^{\floor{m/p}} \frac{p^{r-k}}{(m-pk)!k!},
\end{gather*}
as required.
\end{proof}

In particular, Theorem~\ref{thm:alpha} implies that 
$\alpha_{u, v} \in \QQ_p$.  Our next aim is to show
that $\alpha_{u,v}$ is $p$-adically integral.  First, we collect a few 
intermediate results.

\begin{prop} \label{prop:mpr1}
Let $x^u, x^v \in B$ and 
$m, r \geq 0$ such that $d(m-pr) = p(u_i + 1) - (v_i + 1)$ for all $i$.  Then 
\begin{equation*}
0 \leq m - p r \leq \frac{p(d-1)-1}{d}.
\end{equation*}
\end{prop}

\begin{proof}
This can be easily verified using that $0 \leq u_i, v_i \leq d - 2$ 
and $m - pr \in \ZZ$.
\end{proof}

\begin{prop} \label{prop:mpr2}
Let $x^u, x^v \in B$ and $m, r \geq 0$ such that 
$d(m-pr) = p(u_i + 1) - (v_i + 1)$ for all~$i$.  Then 
\begin{equation*}
r - \floor{\frac{m}{p}} \geq 0.
\end{equation*}
\end{prop}

\begin{proof}
Using the previous proposition,
\begin{equation*}
r - \floor{\frac{m}{p}} 
= - \floor{\frac{m-pr}{p}} 
\geq - \floor{\frac{p(d-1)-1}{pd}} 
= -1 + \ceil{\frac{p + 1}{pd}} 
= 0 
\end{equation*}
as $p \geq 2$ and $d \geq 2$.
\end{proof}

\begin{prop} \label{prop:rfac}
For all integers $u, d \geq 1$ and $r \geq 0$ with $p \nmid d$, 
\begin{equation*}
\ord_p\Bigl(\frac{u}{d}\Bigr)_r \geq \frac{r}{p-1} - \floor{\log_p(r) + 1}.
\end{equation*}
\end{prop}

\begin{proof}
Let $s_p(r)$ denote the sum of digits in the $p$-adic expansion of~$r$ 
and observe that $s_p(r) \leq (p-1)\floor{\log_p(r) + 1}$.  Using the 
standard fact that $\ord_p\bigl((u/d)_r\bigr) \geq \ord_p(r!)$, 
it follows that 
\begin{equation*}
\ord_p\Bigl(\frac{u}{d}\Bigr)_r \geq \ord_p(r!) = \frac{r - s_p(r)}{p-1} \geq \frac{r}{p-1} - \floor{\log_p(r) + 1}
\end{equation*}
as required.
\end{proof}

\subsubsection{The case $p = 2$}

We first note that in the case when $p = 2$, the expression for 
$\lambda_m$ in Lemma~\ref{lem:lambdam} can be simplified:

\begin{lem} \label{lem:mu2}
For $p = 2$ we define a sequence $\bigl(\mu_m^{(2)}\bigr)$ by 
\begin{equation*}
\mu_m^{(2)} = 
    \sum_{k=0}^{\floor{m/2}} \frac{2^{\floor{3m/4} - \nu_m - k}}{(m-2k)! k!}
\end{equation*}
where $\nu_m$ is equal to one whenever $m = 3, 7$ and zero otherwise, 
and we write $\mu_m =\mu_m^{(2)}$ whenever this does not cause confusion. 
Then $\mu_m \in \ZZ_2$ for all $m \geq 0$.
\end{lem}

\begin{proof}
In the two cases $m = 3, 7$ we explicitly compute the values of 
$\mu_m$ as $4/3$ and $232/315$.  Now suppose that $m \neq 3, 7$. 
From Lemma~\ref{lem:lambdam} we obtain that 
\begin{equation*}
\ord_2 \bigl(\mu_m\bigr) 
    = \floor{3m/4} - m + \ord_2(\lambda_m).
\end{equation*}
Using the bound $\ord_p(\lambda_m) \geq \bigl((p-1)/p^2\bigr) m$ from 
Dwork~\citep[pp.~55--57]{Dwork1962}, we obtain that 
\begin{equation*}
\ord_2 \bigl(\mu_m\bigr) 
    \geq \floor{3m/4} - m + \ceil{\frac{m}{4}} = 0. \qedhere
\end{equation*}
\end{proof}

\begin{thm} \label{thm:alpha2}
Let $p = 2$ and suppose $u, v \in \ZZ^{n+1}$ are such that 
$x^u, x^v \in B$ and $p (u_i + 1) \equiv v_i + 1 \pmod{d}$ 
for all~$i$.  Then $\alpha_{u,v}$ can be expressed as 
\begin{equation*}
\alpha_{u,v} = (-2)^{k_u} \prod_{i=0}^n \sum_{m,r} 
    \Bigl(\frac{u_i+1}{d}\Bigr)_r 2^{-\floor{(m+1)/4}+\nu_m} \mu_m, 
\end{equation*}
and $\alpha_{u,v}$ is a $2$-adic integer.
\end{thm}

\begin{proof}
The expression for $\alpha_{u,v}$ is an immediate consequence 
of Theorem~\ref{thm:alpha} and Lemma~\ref{lem:mu2}, together with 
Proposition~\ref{prop:mpr1} implying $0 \leq m - 2r \leq 1$ and 
hence $r = \floor{m/2}$.  It remains to prove that 
$\alpha_{u,v}$ is a $2$-adic integer.  Following Lemma~\ref{lem:mu2}, 
it suffices to show that the valuation of the factor 
\begin{equation} \label{eq:fr}
f_r=\Bigl(\frac{u_i+1}{d}\Bigr)_r 2^{- \floor{(m+1)/4} + \nu_m}
\end{equation}
in each summand is nonnegative.  From the proof of 
Proposition~\ref{prop:rfac} we get that 
\begin{equation} \label{eq:alpha2.1}
\ord_p(f_r)
\geq \ord_p\Bigl(\floor{\frac{m}{2}}!\Bigr) - \floor{\frac{m+1}{4}} + \nu_m.
\end{equation}
Applying Proposition~\ref{prop:rfac}, we see that the right-hand side 
is bounded below by 
\begin{equation*}
\floor{\frac{m}{2}} - \floor{\frac{m+1}{4}} - \floor{\log_2 m}
\end{equation*}
which is nonnegative whenever $m \geq 12$.  In the remaining 
cases $m = 0, \dotsc, 11$, we explicitly verify that the 
lower bound in~\eqref{eq:alpha2.1} is nonnegative.
\end{proof}

\begin{rem}
We observe that in Theorem \ref{thm:alpha2} the exponent 
$-\floor{(m+1)/4}+\nu_m$ is nonpositive for each value $m \geq 0$, 
so at first sight it seems that the computation of the $f_r$ from
\eqref{eq:fr} will suffer from precision loss. However, the $f_r$
can be computed recursively using:
\begin{align*}
f_0 &=1, \; \; \; f_1=\frac{u_i+1}{d} \\
f_r & = f_{r-2} \frac{(u + (r - 2)d)(u + (r - 1)d)}{2d^2} ,
\end{align*}
(Strictly speaking this formula only holds for $r \neq 5$, but one can write down a similar formula for $r=5$).
Since $d$ is odd when $p=2$, the numerator is always even and the denominator is odd,
so this computation can be carried out without precision loss.
\end{rem}

\subsubsection{The case $p > 2$}

\begin{lem} \label{lem:mup}
Let $p \geq 3$ be an odd prime and define a sequence 
$\bigl(\mu_m^{(p)}\bigr)$ by 
\begin{equation*}
\mu_m^{(p)} = \sum_{k=0}^{\floor{m/p}} \frac{p^{\floor{m/p} - k}}{(m-pk)! k!}, 
\end{equation*}
where we write $\mu_m = \mu_m^{(p)}$ when the prime can be identified 
from the context.  Then $\mu_m \in \ZZ_p$ for all $m \geq 0$.
\end{lem}

\begin{proof}
It is clear that $\mu_m \in \QQ$.  From Lemma~\ref{lem:lambdam} 
we observe that $\lambda_m = \pi^m p^{- \floor{m/p}} \mu_m$.  Using the 
bound $\ord_p(\lambda_m) \geq \bigl((p-1)/p^2\bigr) m$ 
from Dwork~\citep[pp.~55--57]{Dwork1962}, it follows that 
\begin{equation*}
\ord_p (\mu_m) \geq \frac{p-1}{p^2} m + \floor{\frac{m}{p}} - \frac{m}{p-1}.
\end{equation*}
Let us write $m = q p + r$ with $0 \leq r \leq p-1$.  As the valuation 
of $\mu_m$ is an integer, it suffices to show that, for $q \geq 0$, 
\begin{equation*}
\frac{p-1}{p} q + q - \frac{q p + p - 1}{p - 1} > -1,
\end{equation*}
which is equivalent to $p^2 - 3p + 1 > 0$, and this holds true 
provided that $p \geq 3$.
\end{proof}

\begin{thm} \label{thm:alphap}
Let $p \geq 3$ and suppose that $u, v \in \ZZ^{n+1}$ are such 
that $x^u, x^v \in B$ and 
$p (u_i + 1) \equiv v_i + 1 \pmod{d}$ for all~$i$. Then 
\begin{equation*}
\alpha_{u,v} = (-p)^{k_u} \prod_{i=0}^n 
    \hat{a}_i^{(p (u_i + 1) - (v_i + 1))/d} \sum_{m,r} 
    \Bigl(\frac{u_i+1}{d}\Bigr)_r p^{r - \floor{m/p}} \mu_m
\end{equation*}
where $m, r \geq 0$ satisfy $p (u_i + 1) - (v_i + 1) = d (m - pr)$. 
In particular, $\alpha_{u, v} \in \ZZ_p$. 
\end{thm}

\begin{proof}
This follows from Theorem~\ref{thm:alpha}, Proposition~\ref{prop:mpr2} 
and Lemma~\ref{lem:mup}.
\end{proof}

\subsection{Estimates}

If we want to use Theorem \ref{thm:01-03-diagfrob} to compute the matrix 
$\Phi$ to $p$-adic precision $N$, then we have to compute the elements
$(k_u-1)!\alpha_{u,v}$ to a somewhat higher precision $\tilde{N}$ than just
$N-n$, because 
of the loss of precision in computing their inverses. Note that if a $p$-adic 
number $x$ is known with (absolute) precision $k$, then its inverse is in general 
only known with precision $k-2\ord_p(x)$. Therefore, we need an upper bound
on the $p$-adic valuation of the elements $(k_u-1)!\alpha_{u,v}$.

\begin{prop}
The valuation of $(k_u-1)! \alpha_{u,v}$ satisfies
\begin{equation*}
\ord_p\bigl((k_u-1)! \alpha_{u,v}\bigr) 
    \leq \ord_p\bigl((n-1)!\bigr) + n + \delta,
\end{equation*}
where $\delta$ is defined as in [[TODO Jan]]. 
\end{prop}

\begin{proof}
Recall from section [[TODO Jan]] that the valuations 
of the entries of the matrix~$\Phi$ are bounded from below by $-\delta$. 
Thus, by Theorem~\ref{thm:01-03-diagfrob}, 
\begin{equation*}
-\delta \leq \ord_p\bigl((k_v-1)!\bigr) + n 
           - \ord_p\bigl((k_u-1)! \alpha_{u,v}\bigr)
\end{equation*}
Noting that $d k_v = \sum_{i=0}^n (v_i + 1) \leq n d$ and 
hence $k_v \leq n$, the result follows.
\end{proof}

\begin{cor} \label{cor:Ntilde}
To compute the matrix $\Phi$ with $p$-adic precision $N$, it is sufficient to compute the elements 
$(k_u-1)!\alpha_{u,v}$ with $p$-adic precision
\begin{equation*}
\tilde{N}=N-n+2(\ord_p\bigl((n-1)!\bigr)+n+\delta).
\end{equation*}
\end{cor}

Up to this point, the sum over $m,r$ in our expressions 
for $\alpha_{u,v}$ has been an infinite sum.  We now present 
a convergence result, that will allow us to derive a finite 
expression for $(k_u-1)!\alpha_{u,v}$ modulo $p^{\tilde{N}}$. We
start with an elementary lemma.

\begin{lem} \label{lem:log}
Given integers $b,c \geq 2$ and defining $x = c + \log_b c + 1$ 
we have that, for all real numbers $y \geq x$, 
\begin{equation*}
y - \log_b y \geq c.
\end{equation*}
\end{lem}

\begin{proof}
We first note that the function $y \mapsto y - \log_b y$ is increasing 
for $y \geq 2$ because it has derivative $1 - \log_b(e)/y > 0$.  Thus, it 
suffices to verify the result for $x$.  Indeed, as $c \geq 2$ we have 
that $\log_b c + 1 \leq c$, hence $c + \log_b c + 1 \leq b^{\log_b c + 1}$,
which upon taking logarithms and rearranging yields the result.
\end{proof}

\begin{prop} \label{prop:MR}
In order to compute $(k_u-1)!\alpha_{u,v} \bmod p^{\tilde{N}}$, it 
suffices to restrict the inner sum in Theorem \ref{thm:alpha2} or 
Theorem \ref{thm:alphap} to pairs $m,r \geq 0$ such that $m \leq \mathcal{M}$, or 
equivalently $r \leq \mathcal{R}$, where 
\begin{equation*}
\mathcal{M} = \floor{ p^2 \biggl( \frac{\tilde{N}}{p-1} +2 
            + \log_p\Bigl(\frac{\tilde{N}}{p-1} + 2\Bigr) + 1 \biggr)}, \; \; \; \mathcal{R}=\floor{\mathcal{M}/p}.
\end{equation*}
\end{prop}

\begin{proof}
This follows from~\citep[\S 6.2]{Lauder2004b} and Lemma~\ref{lem:log}.
\end{proof}

Finally, we describe how to compute an approximation to the matrix~$\Phi_0$ 
representing the action of $p^{-1} \Frob_p$ on $\Hrig^{n}(U_0)$ using our 
previous results. This allows us to formalise the procedure for computing 
the entries of~$\Phi_0$ modulo~$p^N$ in Algorithm~\ref{alg:Diabfrob} 
below.

\begin{algorithm}
\caption{Compute the matrix $\Phi_0$.}
\label{alg:Diabfrob}
\begin{algorithmic}
\vspace{1mm}
\Require $P_0=a_0 x_0^d + \dotsb + a_n x_n^d$ 
         with $a_0,\dotsc,a_n \in \ZZ_p^{\times}$, 
         $p$-adic precision~$N \geq 0$.
\Ensure  Matrix $\Phi$ for action of $p^{-1} \Frob_p$ on $\Hrig^n(U_0)$ with respect to basis $\cB$ modulo $p^N$.
\Procedure{DiagFrob}{$P_0,N$}
\State \begin{compactenum}[{\hspace{1em}} 1.] \vspace{-1.24em}
\item Determine $\tilde{N}$ from Corollary \ref{cor:Ntilde}, and $\mathcal{M}$,$\mathcal{R}$ from Proposition \ref{prop:MR}. 
\item Compute the Teichm\"uller lifts $\hat{a}_0, \dotsc, \hat{a}_n$, 
      and the sequences $(d^{-r})_{r=0}^{\mathcal{R}}$, $(\mu_m)_{m=0}^{\mathcal{M}}$ 
      to $p$-adic precision~$\tilde{N}$.
\item Let $\Phi_0 \in M_{b \times b}(\QQ_p)$ be the zero matrix.
\item[] \textbf{for} $x^u \in B$ \textbf{do} 
\item[] \begin{compactenum}[{\hspace{1em}} 1.]
        \item Determine the unique $x^v \in B$ such that $v_i = p (u_i + 1) - 1 \bmod{d}$.
        \item $x_1 \gets (-1)^{k_u+k_v} (k_v-1)! p^n$ as an exact integer %TODO? a sufficient $p$-adic precision here is N+ord_p((n-1)!)+n+\delta
        \item $x_2 \gets (k_u - 1)! \alpha_{u,v} \pmod{p^{\tilde{N}}}$ using Theorem \ref{thm:alpha2} or 
               Theorem \ref{thm:alphap}.
        \item Compute $x_2^{-1} \pmod{p^{N-n}}$.
        \item $(\Phi_0)_{u,v} \gets x_1 x_2^{-1} \pmod{p^N}$
      \end{compactenum}   
 \item \textbf{return} $\Phi_0$      
\end{compactenum}
\EndProcedure
\end{algorithmic}
\end{algorithm}

\begin{rem} \label{rem:mup}
The expressions for $\mu_m$ can be computed using integer arithmetic 
modulo~$p^{\tilde{N}}$ via 
\begin{equation*}
\mu_m^{(p)} = \begin{cases}
\frac{1}{m!} \sum_{k=0}^{\floor{m/2}} 2^{\floor{3m/4} - k} \frac{m!}{(m-2k)! k!}
    & \text{if $p = 2$, $m \neq 3, 7$} \\
\frac{1}{m!} \sum_{k=0}^{\floor{m/p}} p^{\floor{m/p} - k} \frac{m!}{(m-pk)! k!}
    & \text{if $p$ is odd},
\end{cases}
\end{equation*}
with only one $p$-adic inversion as all summands are integers. Moreover, 
these summands are of size $\BigOh(m \log m)$ bits, which also allows us 
to avoid performing intermediate reductions modulo~$p^{\tilde{N}}$.
\end{rem}

%TODO put all the complexity stuff together at the end, and max a couple
%of pages total
\begin{comment}
\begin{thm} \label{thm:DiagfrobComplexity1}
The time complexity of Algorithm~\ref{alg:Diabfrob} is given by 
\begin{equation*}
p \tilde{N}^2 \cM\bigl(p \tilde{N} \log (p \tilde{N})\bigr)
    + d^n n \bigl( \cM(\log d) + (\tilde{N} + \log p) \cM(\tilde{N} \log p) \bigr)
\end{equation*}
where $\cM(-)$ denotes the complexity of integer multiplication and 
$\tilde{N}$ is $\BigOh(N + n \log_p n)$.
\end{thm}

\begin{proof}
We consider each of the steps in Algorithm~\ref{alg:Diabfrob}. 
We can ignore the computational cost of {Step~(i)}, but observe 
that $\tilde{N} \in \BigOh(N + n \log_p n)$, 
$M \in \BigOh(p \tilde{N})$, and $R \in \BigOh(\tilde{N})$.  

In {Step~(ii)}, we compute $n+1$ Teichm\"uller lifts with an 
overall complexity of $\BigOh(n (\log p) \cM(\tilde{N} \log p))$. 
The computation of the sequence $(d^{-r})_{r=0}^{R}$ requires 
one reduction of $d$ modulo $p^{\tilde{N}}$, one $p$-adic inversion 
and $R-1$~products to precision~$\tilde{N}$, yielding 
$\BigOh\bigl(\cM(\log d) + (\log \log p) \cM(\log p) + \tilde{N} \cM(\tilde{N} \log p)\bigr)$. 
Finally, we carry out the computation of $(\mu_m)_{m=0}^{M}$ following 
Remark~\ref{rem:mup}, which requires time 
$\BigOh\bigl(p \tilde{N}^2 \cM(p \tilde{N} \log(p \tilde{N}))\bigr)$.

The following {Steps~(iii) through (vii)} are executed 
$\dim \Hrig^{n}(U)$ times, where this dimension 
is $\BigOh(d^n)$.
The time complexity of {Step~(iii)} is 
$\BigOh\bigl(n \cM(\log \max\{p,d\})\bigr)$.
We observe that we can ignore {Step~(iv)}.
Step~(v) involves an $(n+1)$-fold product of series 
with~$\BigOh(R)$ terms modulo~$p^{\tilde{N}}$, where each summand 
requires an absolutely bounded number of products, as well as 
$n+1$~exponentiations of Teichm\"uller lifts with exponents given 
by $d^{-1} \bigl(p (u_i+1) - (v_i+1)\bigr) < p$.  Computing the 
exponents has complexity $\BigOh(\cM(\log \max\{p,d\}))$ and 
we note that we may ignore the update of the term 
\mbox{$\bigl((u_i+1)/d\bigr)_r$} throughout the summation 
assuming we have computed the reduction of $d \bmod{p^{\tilde{N}}}$ 
once and for all earlier.  Therefore, the complexity of each 
invocation of {Step~(v)} is 
$\BigOh\bigl( n \cM(\log d) + n (R + \log p) \cM(\tilde{N} \log p)\bigr)$.  
The $p$-adic inverse in {Step~(vi)} requires time 
$\BigOh\bigl((\log \log p) \cM(\log p) + \cM(\tilde{N} \log p)\bigr)$.
Finally, we can ignore the product in {Step~(vii)}.  
Thus, the aggregate time complexity of {Steps~(iii)} through {(vii)} 
is given by 
$\BigOh\bigl(d^n n \bigl( \cM(\log d) + (\tilde{N} + \log p) \cM(\tilde{N} \log p) \bigr)\bigr)$.
\end{proof}

While the current implementation of Algorithm~\ref{alg:Diabfrob} 
performs very well in practice, its time complexity is quasi-cubic in 
the $p$-adic precision~$N$, which is \emph{not} optimal.  Both of 
these aspects are due to the use of the sequence $(\mu_m)_{m=0}^{M}$ 
defined over $\ZZ_p$ instead of the coefficients $(\lambda_m)_{m=0}^{M}$ 
defined over $\QQ_p(\pi)$.  We can achieve a better time complexity by 
utilising fast exponentials of power series:

\begin{thm} \label{thm:DiagfrobComplexity2}
There exists an algorithm for computing the matrix for the 
action of $p^{-1} \Frob_p$ on $\Hrig^{n}(U)$ in time complexity 
\begin{equation*}
(M \log M \log \log M) (p \log p) \cM(\tilde{N} \log p) 
+ d^n n \bigl( \cM(\log d) 
              + \tilde{N} (p \log p) \cM(\tilde{N} \log p) \bigr)
\end{equation*}
where $\cM(-)$ denotes the complexity of integer multiplication 
and $\tilde{N} \in \BigOh(N + n \log_p n)$, $M \in \BigOh(p \tilde{N})$.
\end{thm}

\begin{proof}
The key idea is to slightly modify Algorithm~\ref{alg:Diabfrob} 
and use Equation~\eqref{eq:alpha}, working directly with the sequence 
$(\lambda_m)_{m=0}^{M}$.  As we will be using operations in the 
totally ramified extension~$\QQ_p(\pi)$ to 
$p$-adic precision~$\tilde{N}$, we remark that the cost of 
an arithmetic operation in its ring of integers is 
$\BigOh\bigl((p \log p) \cM(\tilde{N} \log p)\bigr)$, 
achieved by polynomial multiplication based on the fast Fourier 
transform.

We first observe that the valuation of the summands in 
Equation~\eqref{eq:alpha} can be bounded by 
$\ord_p \bigl(\lambda_m (u_i / d)_r (-1)^r \pi^{-r} {\hat{a}_i}^{m-r} \bigr) \geq - 1/2$.
As the only term with negative valuation is $\pi^{-r}$ and 
$R \in \BigOh(\tilde{N})$, it suffices to precompute 
the sequences $(\lambda_m)_{m=0}^{M}$ and $(d^{-r})_{r=0}^{R}$ 
to $p$-adic precision $\tilde{N}$.  While the computation 
of the latter remains unchanged, the computation of the former 
can be improved significantly using fast exponentials of 
power series in $\QQ_p(\pi)[[z]]$ as described by 
Bernstein~\citep[\S 9.3]{Bernstein2008}.  This allows for 
computing the sequence $(\lambda_m)_{m=0}^{M}$ in 
$\BigOh\bigl( (M \log M \log \log M) \bigr)$ operations 
in~$\QQ_p(\pi)$ to precision~$\tilde{N}$.  The only 
remaining change to our analysis of Algorithm~\ref{alg:Diabfrob} 
occurs in {Step~(v)}, where for each matrix entry we have to consider 
$\BigOh(n R)$ multiplications in~$\QQ_p(\pi)$ instead 
of~$\ZZ_p$.
\end{proof}
\end{comment}

%%%%%%%%%%%%%%%%%%%%%%%%%%%%%%%%%%%%%%%%%%%%%%%%%%%%%%%%%%%%%%%%%%%%%%%%%%%%%%%

\section{Solving the differential equation}
\label{sec:DifferentialSystem}

In this section we explain how to solve the $p$-adic differential 
equation describing the horizontal sections of the Gauss--Manin 
connection $\nabla$, in order to obtain a local expansion of the 
matrix for the action of $p^{-1} \Frob_p$ on $\Hrig^{n}(U/S)$.  
Our discussion largely follows the work of Lauder~\citep{Lauder2006}, 
but incorporates improved convergence bounds by Kedlaya~\citep{Kedlaya2010}.

Recall from [[TODO Jan]] that $M \in M_{b \times b}(\QQ_q(t))$ denotes the matrix for the Gauss--Manin connection 
$\nabla$ on $\HdR^n(\mathfrak{U}/\mathfrak{S})$ with respect to the basis $\cB$. We let $r \in \ZZ_q[t]$
denote the denominator of $M$, so that $M = G/r$ with $G \in M_{b \times b}(\QQ_q[t])$.
Note that $r$ is a divisor of the resultant $\Delta$ defined in Definition \ref{defn:resultant}.
Also recall that $\Phi \in M_{b \times b} \bigl(\QQ_q \langle t,\frac{1}{r(t)} \rangle^{\dag} \bigr)$ 
denotes the matrix for the $\sigma$-semilinear action of~$p^{-1} \Frob_p$ on $\Hrig^{n}(U/S)$ with 
respect to the same basis $\cB$, where $\sigma$ is the standard lift of the $p$th-power Frobenius 
sending $t \mapsto t^p$. 

As we saw in section [[TODO Jan]], these matrices satisfy the 
differential equation

\begin{equation} \label{eq:Phi}
\Bigl(\frac{d}{dt} + M\Bigr) \Phi = p t^{p-1} \Phi \sigma(M), \; \; \; \Phi(0)=\Phi_0,
\end{equation}
where $\Phi_0 \in M_{b \times b}(\QQ_p)$ is the matrix for the action of $p^{-1} \Frob_p$ 
on $\Hrig^n(U_0)$, again with respect to the basis $\cB$. Our goal is 
the computation of a power series expansion of~$\Phi$ at $t=0$.

We first observe that if $C \in M_{b \times b}(\QQ_q[[t]])$ denotes the unique solution to the 
homogeneous system
\begin{equation} \label{eq:01-GMDE-Homogenous}
\Bigl(\frac{d}{dt} + M\Bigr) C = 0, \; \; \; C(0)=I,
\end{equation}
where $I$ denotes the identity matrix, then the matrix
\begin{equation*}
\Phi = C \Phi_0 \sigma(C)^{-1}
\end{equation*}
satisfies Equation \eqref{eq:Phi}. So it is sufficient to solve Equation 
\eqref{eq:01-GMDE-Homogenous}.

Let us write $G = \sum_{i=0}^{\deg(G)} G_i t^i$ and $r= \sum_{i=0}^{\deg(r)} r_i t^i$,
with $G_i \in M_{b \times b}(\QQ_q)$ and $r_i \in \ZZ_q$, for all $i$. By our assumption 
\ref{assump:r0}, we may suppose that $r_0 \neq 0 \pmod{p}$, so in particular $r_0 \neq 0$.

We can now obtain a power series solution $C = \sum_{i \geq 0} C_i t^i$ at $t=0$ for 
the equation
\begin{equation*}
r \frac{dC}{dt} + G C = 0, \; \; \; C(0)=I,
\end{equation*}
which is clearly equivalent to Equation \eqref{eq:01-GMDE-Homogenous}, using the recursion 
\begin{align}
C_0 &= I, \notag \\ 
C_{i+1} &= \frac{-1}{r_0 (i+1)} \biggl(
    \sum_{j=\max{\{0,i-\deg(B)\}}}^i G_{i-j} C_j + 
    \sum_{j=\max{\{0,i-\deg(r)\}}+1}^i r_{i-j+1} (j C_j) \biggr). \label{eq:recursion}
\end{align}

However, we will carry out this computation to some finite $p$-adic working precision $\tilde{N}$. If
we want $C$ to be correct to $p$-adic precision $N$, then $\tilde{N}$ has to be somewhat greater than $N$, 
because of error propagation. A bound for $\tilde{N}$, in terms of $N$ and the desired $t$-adic precision, 
was given by Lauder~\citep[Theorem~5.1]{Lauder2006}, but his result can be improved significantly by including 
more recent bounds on the valuation of $C_0, C_1, \dotsc$ by Kedlaya~\citep{Kedlaya2010}, as we will now show.  

\begin{thm} \label{thm:valC}
For all $i \geq 1$, the $p$-adic valuation of the matrix~$C_i$ 
satisfies 
\begin{equation*}
\ord_p(C_i) \geq - \bigl(2 \delta + (n - 1)\bigr) \ceil{\log_p i}.
\end{equation*}
\end{thm}

\begin{proof}
It follows from Theorem~{18.3.3} in Kedlaya~\citep{Kedlaya2010} that
\begin{equation*}
\ord_p(C_i) \geq \bigl( \ord_p(\Phi) + \ord_p(\Phi^{-1}) \bigr) \ceil{\log_p i}.
\end{equation*}
However, by [[TODO Jan]], we know that $\ord_p(\Phi) \geq -\delta$ and 
$\ord_p(p^{n-1}\Phi^{-1}) \geq -\delta$, so
\begin{equation*}
\ord_p(\Phi) + \ord_p(\Phi^{-1}) \geq - 2 \delta - (n-1),
\end{equation*}
and the result now follows.
\end{proof}

Now let $D_0, D_1, \dotsc$ denote an approximation to $C_0, C_1, \dotsc$ 
defined by the recursion
\begin{align*}
D_0 &= I \\
D_{i+1} &= \frac{-1}{r_0 (i+1)} \biggl(
    \sum_{j=\max{\{0,i-\deg(B)\}}}^i G_{i-j} D_j + 
    \sum_{j=\max{\{0,i-\deg(r)\}}+1}^i r_{i-j+1} (j D_j) \biggr) + 
    p^{\tilde{N}} E_{i+1},
\end{align*}
where $(E_i)_{i \geq 1}$ is a sequence of $p$-adically integral matrices i.e., the matrices $D_0, D_1, \dotsc$
are computed with working precision $\tilde{N}$.

\begin{thm} \label{thm:errorprop}
For all $i \geq 1$, 
\begin{equation*}
\ord_p(C_i - D_i) \geq 
    \tilde{N} - \Bigl(2 \bigl(2 \delta + (n-1)\bigr) + 1\Bigr) \ceil{\log_p i}.
\end{equation*}
\end{thm}

\begin{proof}
This follows analogously to
Lauder~\citep[Theorem~5.1]{Lauder2006}.  
Indeed, the proof there shows that 
\begin{equation*}
\ord_p(C_i - D_i) \geq 
    \tilde{N} + \min_{k+\ell=i} \Bigl(\ord_p(C_k) + 
                                      \ord_p(\ell^{-1} C_{\ell-1}^{-1})\Bigr).
\end{equation*}
We use the bound from Theorem~\ref{thm:valC} and observe that it 
also applies to the inverse matrix~$C^{-1}$, as this matrix satisfies 
the dual differential equation 
\begin{equation} \label{eq:01-GMDE-Dual}
\Bigl(\frac{d}{dt} - M^t\Bigr) \bigl(C^{-1}\bigr)^t = 0, \; \; \; C^{-1}(0)=I,
\end{equation}
that carries a Frobenius structure given by the matrix $\bigl(\Phi^{-1}\bigr)^t$. 
The result now follows.
\end{proof}

\begin{rem}
Kedlaya~\citep[Remark~18.3.4]{Kedlaya2010} also includes the bound
\begin{equation*}
\ord_p(C_i) \geq (b - 1) \ord_p(M) 
            + \bigl( \ord_p(\Phi) + \ord_p(\Phi^{-1}) \bigr) \floor{\log_p i},
\end{equation*}
which can sometimes be used to improve the bounds from the above theorems slightly, 
for example when $\ord_p(M)$ is positive.
\end{rem}

\begin{rem}
In order to determine the power series expansion of the matrix~$\Phi$, 
we also need to compute the matrix $\sigma(C)^{-1}$. The matrix~$C^{-1}$ 
could be computed using matrix inversion over the ring $\mathbf{Q}_q[[t]]$.  
However, solving Equation~\eqref{eq:01-GMDE-Dual} for $C^{-1}$ is typically 
more efficient in practice.
\end{rem}

We need to recall some bounds on the loss of $p$-adic precision when
multiplying $p$-adic numbers and matrices. [[TODO perhaps move the 
following propositions somewhere else]]

\begin{prop} \label{prop:productval}
Let $N \in \mathbf{Z}$ and $x_1, \dotsc, x_{\ell} \in \mathbf{Q}_q$  
be such that $N \geq \sum_{j=1}^{\ell} \ord_p(x_j)$, and let 
$\tilde{x}_1, \dotsc, \tilde{x}_{\ell}$ denote $p$-adic approximations to 
$x_1, \dotsc, x_{\ell}$ satisfying $\ord_p(x_i - \tilde{x}_i) \geq N - \sum_{j \neq i} v_j$ 
for all $i$.  Then 
\begin{equation*}
\ord_p(x_1 \dotsm x_{\ell} - \tilde{x}_1 \dotsm \tilde{x}_{\ell}) \geq N.
\end{equation*}
[[TODO Seb]] I weakened the conditions, check!
\end{prop}

\begin{proof}
Note that, for all $i$,
\begin{align*}
\ord_p(\tilde{x}_i) &\geq \min \{ \ord_p(x_i-\tilde{x}_i), \ord_p(x_i) \} \\
                    &\geq \min \{ N- \sum_{j \neq i} \ord_p(x_j), \ord_p(x_i)\} = \ord_p(x_i).
\end{align*}
Therefore, we also have
\begin{align*}
\ord_p \bigl( (x_{i}-\tilde{x}_{i})(\tilde{x}_1 \dotsc \tilde{x}_{i-1} x_{i+1} \dotsc x_{\ell} \bigr) \geq N,
\end{align*}
for all $i$. The result now follows by combining these inequalities.
\end{proof}

\begin{prop} \label{prop:matrixproductval}
Let $N \in \mathbf{Z}$ and $A_1, \dotsc, A_{\ell} \in M_{b \times b}(\QQ_q)$
be such that $N \geq \sum_{j=1}^{\ell} v_j$, and let
$\tilde{A}_1, \dotsc, \tilde{A}_{\ell}$ denote $p$-adic approximations 
to $A_1, \dotsc A_{\ell}$ satisfying $\ord_p(A_i - \tilde{A}_i) \geq N - \sum_{j \neq i} v_j$ 
for all $i$.  Then 
\begin{equation}
\ord_p(A_1 \dotsm A_{\ell} - \tilde{A}_1 \dotsm \tilde{A}_{\ell}) \geq N.
\end{equation}
\end{prop}

\begin{proof}
We can follow the proof of Proposition~\ref{prop:productval}, 
observing that for matrices $A$, $B$ we have 
$\ord_p(A + B) \geq \min \{\ord_p(A), \ord_p(B)\}$.
\end{proof}

We can now finally give all the necessary precisions for computing the power series expansion for $\Phi$ at $t=0$ to
any given precision.

\begin{thm} \label{thm:Ni}
Let $N,K \in \NN$ and suppose that $N \geq \ord_p(\Phi_0)$ [[TODO Jan, try to get rid of this]]. Define
\begin{eqnarray*}
N_{\Phi_0}   		&=& N+\bigl(2\delta+(n-1)\bigr) \bigl(\ceil{\log_p K} + \ceil{\log_p \floor{K/p}}\bigr) \\
N_{C}				&=& N+\bigl(2\delta+(n-1)\bigr) \ceil{\log_p \floor{K/p}} + \delta \\
N_{C^{-1}}			&=& N+\bigl(2\delta+(n-1)\bigr) \ceil{\log_p K} + \delta \\
\tilde{N}_C			&=& N+\Bigl(2 \bigl(2 \delta + (n-1)\bigr) + 1\Bigr) \ceil{\log_p K} \\
\tilde{N}_{C^{-1}}	&=& N+\Bigl(2 \bigl(2 \delta + (n-1)\bigr) + 1\Bigr) \ceil{\log_p \floor{K/p}}
\end{eqnarray*}

In order to compute the power series expansion 
of the matrix $\Phi$ at $t=0$ with $t$-adic precision $K$ and $p$-adic precision $N$, 
%(i.e., to compute $\Phi$ as an element of $M_{b \times b}(\QQ_q[[t]])$ modulo $t^K$ and $p^N$), 
it is sufficient to compute
the matrix $\Phi_0$ to $p$-adic precision $N_{\Phi_0}$,
the matrix $C$ to $t$-adic precision $K$ and $p$-adic precision $N_{C}$, and
the matrix $C^{-1}$ to $t$-adic precision $\floor{K/p}$ and $p$-adic precision 
$N_{C^{-1}}$.

Therefore, while solving Equation \eqref{eq:01-GMDE-Homogenous} for $C$ and Equation \eqref{eq:01-GMDE-Dual} for $C^{-1}$, using a recursion like in Equation \eqref{eq:recursion}, it is sufficient to use $p$-adic working precisions $\tilde{N}_C$ and $\tilde{N}_{C^{-1}}$, respectively.
\end{thm}

\begin{proof}
Recall that $\Phi = C \Phi_0 \sigma(C)^{-1}$. The sufficient $t$-adic precisions are clear. 
Since $\ord_p(C) \leq 0$ and $\ord_p(\sigma(C)^{-1}) \leq 0$, we have that 
\[
N \geq \ord_p(C) + \ord_p(\Phi_0) + \ord_p(\sigma(C)^{-1}).
\]
So we can apply Proposition \ref{prop:matrixproductval}, using Theorem \ref{thm:valC} (for both $C$ and $C^{-1}$) 
and the fact that $\ord_p(\Phi_0) \geq -\delta$ [[TODO Jan]], to obtain the sufficient $p$-adic precisions for the matrices 
$\Phi_0$, $C$ and $C^{-1}$. The sufficient working precisions $\tilde{N}_C$ and $\tilde{N}_{C^{-1}}$
now follow from Theorem \ref{thm:errorprop}.
\end{proof}

Now we have all the ingredients to compute the power series expansion of $\Phi$ at $t=0$ to any 
given $p$-adic and $t$-adic precisions, as formalised in Algorithm \ref{alg:expansion} below.

\begin{algorithm}
\caption{Compute the power series expansion of $\Phi$ at $t=0$.}
\label{alg:expansion}
\begin{algorithmic}
\vspace{1mm}
\Require $P \in \ZZ_q[t][x_0,\ldots,x_n]$ satisfying ..., $t$-adic precision $K$, $p$-adic precision~$N$.
\Ensure  The power series expansion of $\Phi$ at $t=0$ to $t$-adic precision $K$ and $p$-adic precision $N$.
\Procedure{FrobSeriesExpansion}{$N,K$} 
\State \begin{compactenum}[{\hspace{1em}} 1.] \vspace{-1.24em}
\item Determine $N_{\Phi_0},N_C,N_{C^{-1}},\tilde{N}_C$, and $\tilde{N}_{C^{-1}}$ from Theorem \ref{thm:Ni}
\item $M \gets \textsc{GMConnection}(P)$
\item $\Phi_0 \gets \textsc{DiagFrob}(P_0,N_{\Phi_0})$
\item Solve Equation \eqref{eq:01-GMDE-Homogenous} for $C$ to $t$-adic precision $K$ and $p$-adic precision $N_{C}$:
\begin{compactenum}[{\hspace{1em}} a.] 
\item[] $C_0 \gets I$
\item[] \textbf{for} $i=0$ \textbf{to} $K-1$ \textbf{do} 
\item[] \hspace{1em} $C_{i+1} \gets \frac{-1}{r_0 (i+1)} \biggl(\sum_j G_{i-j} C_j + \sum_j r_{i-j+1} (j C_j) \biggr) \pmod{p^{\tilde{N}_C}}$
\item[] $C \gets \sum_{i=0}^K C_i t^i \pmod{p^{N_C}}$
\end{compactenum}
\item Solve Equation \eqref{eq:01-GMDE-Dual} for $C^{-1}$ to $t$-adic precision $\floor{K/p}$ and $p$-adic precision $N_{C^{-1}}$:
\begin{compactenum}[{\hspace{1em}} a.]
\item[] $(C^{-1})^t_0 \gets I$
\item[] \textbf{for} $i=0$ \textbf{to} $\floor{K/p}-1$ \textbf{do}
\item[] \hspace{1em} $(C^{-1})^t_{i+1} \gets  \frac{-1}{r_0 (i+1)} \biggl(\sum_j (-G_{i-j})^t C_j^t + \sum_j r_{i-j+1} (j C_j^t) \biggr) \pmod{p^{\tilde{N}_{C^{-1}}}}$
\item[] $C^{-1} \gets \sum_{i=0}^{\floor{K/p}} (C^{-1})_i t^i \pmod{p^{N_{C^{-1}}}}$
\end{compactenum}
\item $\Phi \gets C \Phi_0 \sigma(C^{-1})$
\item \textbf{return} $\Phi$
\end{compactenum}
\EndProcedure
\end{algorithmic}
\end{algorithm}

%%%%%%%%%%%%%%%%%%%%%%%%%%%%%%%%%%%%%%%%%%%%%%%%%%%%%%%%%%%%%%%%%%%%%%%%%%%%%%%

\section{Recovering the zeta function of a fiber}

\subsection{Evaluation at a point}
\label{sec:Evaluation}

In the previous section we described how to compute the power series expansion at $t=0$ of
the matrix~$\Phi$ for the action of $p^{-1} \Frob_p$ on $\Hrig^{n}(U/S)$. We now want to
evaluate $\Phi$ at the Teichm\"uller lift $\hat{\tau}$ of some $\tau \neq 0 \in S(\overline{\FF}_q)$,
but the power series expansion of $\Phi$ at $t=0$ usually only converges on the open 
unit disc $|t|<1$, and so cannot be evaluated at $\hat{\tau}$. 

However, since $\Phi$ is a matrix of overconvergent functions i.e., 
\[
\Phi \in M_{b \times b}(\QQ_q \langle t,\frac{1}{r(t)} \rangle^{\dag}),
\]
it can be approximated to any given $p$-adic precision $N$ by a matrix of rational 
functions, that can be evaluated at $\hat{\tau}$. To convert the power series expansion
for $\Phi$ to such an approximation by rational functions, we need a bound on the pole order
of these rational functions at their poles, as a function of $N$. 

The bounds that
Lauder ~\citep[\S 8.1]{Lauder2004a} and Gerkmann~\citep[\S 6]{Gerkmann2007} used for this
were very far from being sharp, which significantly slowed down their algorithms.
Recently, under some small additional assumptions, Kedlaya and the second 
author~\citep[Theorem~2.1]{KedlayaTuitman2012} obtained a much sharper bound,
that we state here in a slightly simplified form.

\begin{thm} \label{thm:KedlayaTuitman}
Let $\mathfrak{Z}$ denote the complement of $\mathfrak{S}$ in $\mathbf{P}^{1}_{\mathbf{Q}_q}$, 
and let $z$ be an unramified geometric point of $\mathfrak{Z}$ such that $\mathfrak{Z}$ does not 
contain any other points with the same reduction modulo~$p$.  Suppose that $[v_1,\dotsc,v_b]$ is 
a basis for $\Hrig^n(U/S)$ with respect to which the matrix $M$ for the connection~$\nabla$ 
has at most a simple pole at~$z$. Moreover, assume that the exponents $\{ \lambda_1, \dotsc, \lambda_{b} \}$ 
of $M$ at $z$, which are defined as the eigenvalues of $(t - z) M \vert_{t=z}$, and known to be rational 
numbers, have nonnegative $p$-adic valuations. Let $\Phi$ denote the matrix of the action of $p^{-1} \Frob_p$ 
on $\Hrig^{n}(U/S)$, with respect to the basis $[v_1,\dotsc,v_b]$, and let $\delta$ be defined as in [[TODO Jan]]. 
For $i \in \NN$, put
\begin{align*}
f(i)=\max \{ \bigl( -2 \delta + (n-1) \bigr) \lceil \log_p(i) \rceil, 
(b-1) \ord_p(M) + \bigl( -2 \delta + (n-1) \bigr) \lfloor \log_p(i) \rfloor \},
\end{align*}
and define 
\begin{align*}
c = \begin{cases}
0 & \mbox{if $\ord_p(M) \geq 0$} \\
\min\{0, i + f(i): i \in \NN\} & \mbox{if $\ord_p(M) < 0$}.
\end{cases}
\end{align*}
For $N \in \NN$, put
\begin{align*}
g(N) &= \max \{i \in \NN \; | \; i + f(i) - \delta + c  < N \},
\end{align*}
and define
\begin{align*}
\alpha_1 &= \lfloor -p \min_i \{ \lambda_i \} + \max_{i} \{\lambda_i\} \rfloor, \\ 
\alpha_2 &=  \left \{ 
         \begin{array}{cl}
         0  & \mbox{if $M$ does not have a pole at $z$},  \\
         0  & \mbox{if $z \in \{0,\infty \}$}, \\
         g(N) & \mbox{otherwise}.
         \end{array}
         \right. 
\end{align*}
Then $\Phi$ is congruent modulo $p^{N}$ to a matrix of rational functions of order greater 
than or equal to $-(\alpha_1+p \alpha_2)$ at $z$. 
\end{thm}

\begin{proof}
Since the matrix $\Phi$ defines a Frobenius structure on the vector bundle $\Hrig^n(U/S)$ with
connection $\nabla$, we can apply \citep[Theorem~2.1]{KedlayaTuitman2012}. We have 
replaced $\ord_p(\Phi)$ and $ \bigl( \ord_p(\Phi)+\ord_p(\Phi^{-1}) \bigr)$ by their respective lower bounds 
$-\delta$ and $\bigl( -2 \delta + (n-1) \bigr)$, since we might not know them exactly a priori.
\end{proof}

\begin{rem}
In practice the various constants $\bigl( -2 \delta + (n-1) \bigr), c, \alpha_1, -\delta$ are always very close
to zero, so that $g(N)$ is about $N$, and the lower bound for the order of $\Phi$ modulo $p^N$ at $z$ is roughly $-pN$.
%This is what one should remember from this theorem.
\end{rem}

\begin{rem}
The assumptions in the theorem are a lot less restrictive than one might think
at first.

That $M$ has a simple pole at $z$ is not a serious restriction, since
one can always change the basis $[v_1,\dotsc,v_b]$ such that this condition is
satisfied, because the Gauss--Manin connection is known to be regular singular. 
Applying the theorem to this new basis and transforming back to the old one, one
obtains more or less the same bound as before \citep[Corollary~2.6]{KedlayaTuitman2012}.

That the exponents are $p$-adically integral turns out to be an unnecessary 
condition [[TODO Jan, this seems to contradict one of Seb's examples, check]], since it is 
implied already by the existence of a Frobenius structure [[TODO Jan]].
At least one of the authors of \cite{KedlayaTuitman2012} did not realize this
at the time it was written.

The only real restriction is therefore that $\mathfrak{Z}$ does not contain any points
with the same reduction modulo $p$ as $z$. However, this condition is almost always
satisfied, and even when it is not, Theorem \ref{thm:KedlayaTuitman} still seems
to hold experimentally.
\end{rem}

Now if we want to compute the matrix $\Phi_{\tau}$ of the action of $p^{-1} \Frob_p$ on 
$\Hrig^{n}(U_{\tau})$ to $p$-adic precision $N$, then we can use Theorem \ref{thm:KedlayaTuitman} 
to find $m,K \in \NN$ such that $r(t)^m \Phi(t)$ is congruent modulo $p^N$ to a matrix of polynomials 
of degree at most $K$ that can be evaluated at $\hat{\tau}$. We then compute
\begin{equation}
\Phi_{\tau} = r(\hat{\tau})^{-m} \bigl( r(t)^m \Phi(t) \bmod{t^{K}} \bigr)|_{t=\hat{\tau}} \bmod{p^{N}}.
\end{equation}

\subsection{Computing the zeta function}
\label{sec:ZetaFunctions}

Now we want to compute the zeta function of the fiber $X_{\tau}$ of
our family $X/S$ lying over some $\tau \in S(\FF_{\mathfrak{q}})$ with $\mathfrak{q}=q^f$ 
for some $f \in \NN$. Recall that the 
zeta function of $X_{\tau}$ is of the form
\begin{equation}
Z(X_{\tau},T) = \frac{\chi(T)^{(-1)^n}}{(1 - T) (1 - \mathfrak{q}T) \dotsm (1 - \mathfrak{q}^{n-1}T)},
\end{equation}
where $\chi(T) = \det \bigl( 1 - T \mathfrak{q}^{-1} \Frob_{\mathfrak{q}} | \Hrig^n(U_{\tau}) \bigr) \in \ZZ[T]$ 
denotes the reverse characteristic polynomial of the action of $\mathfrak{q}^{-1} \Frob_{\mathfrak{q}}$ 
on $\Hrig^n(U_{\tau})$.

So to compute the zeta function of $X_{\tau}$ we first need to determine the matrix of the action of 
$\mathfrak{q}^{-1} \Frob_{\mathfrak{q}}$ on $\Hrig^n(U_{\tau})$ from that of $p^{-1} \Frob_{p}$,
and then compute its reverse characteristic polynomial. All of this has to be
carried out to high enough $p$-adic precision in order to recover the exact zeta function.

We start by computing the matrix of the action of 
$\mathfrak{q}^{-1} \Frob_{\mathfrak{q}}$ on $\Hrig^n(U_{\tau})$.
Let us denote $a=\log_p(\mathfrak{q})$. Recall that $\Phi_{\tau}$ denotes the matrix of the action 
of $p^{-1} \Frob_p$ on~$\Hrig^{n}(U_{\tau})$ with respect to the basis $\cB$. As this action is 
$\sigma$-semilinear, we have that 
\begin{equation}
\Phi_{\tau}^{(a)} = 
    \Phi_{\tau} \sigma(\Phi_{\tau}) \dotsm \sigma^{a-1}(\Phi_{\tau})
\end{equation}
is the matrix of the action of $\mathfrak{q}^{-1} \Frob_{\mathfrak{q}}$ on $\Hrig^n(U_{\tau})$. 

\begin{comment}
\begin{thm}
In order to determine $\Phi_{\tau}^{(a)}$ to $p$-adic precision~$N \geq 0$, it suffices to 
compute $\Phi_{\tau}$ to precision $N + (a-1) \delta$.
\end{thm}

\begin{proof}
Since we have that $\ord_p(\Phi_{\tau}) \geq -\delta$ by [[TODO Jan]], the valuations of the matrices 
$\Phi_{\tau}, \sigma(\Phi_{\tau}), \dotsc, \sigma^{a-1}(\Phi_{\tau})$ 
are all at least $-\delta$. The theorem now follows from Proposition 
\ref{prop:matrixproductval}. 
\end{proof}
\end{comment}

We now address the loss of $p$-adic precision when computing the reverse
characteristic polynomial $\det( 1 - T \Phi_{\tau}^{(a)})$
staring from the matrix $\Phi_{\tau}$.  

\begin{thm}
Suppose that $\tilde{\Phi}_{\tau}$ is an approximation to 
$\Phi_{\tau}$ with 
\[
\ord_p (\Phi_{\tau}-\tilde{\Phi}_{\tau}) \geq N + \delta,
\]
with $N \geq 0$ and $\delta$ defined as in [[TODO Jan]]. Then
\begin{equation}
\ord_p \Bigl( \det\bigl(1 - T \Phi_{\tau}^{(a)}\bigr) 
            - \det\bigl(1 - T \tilde{\Phi}_{\tau}^{(a)}\bigr) \Bigr) \geq N.
\end{equation}
\end{thm}

\begin{proof} 
Recall from [[TODO Jan]] that there exists a matrix $W \in M_{b \times b}(\QQ_q)$
which satisfies $\ord_p(W)+\ord_p(W^{-1}) \geq -\delta$, 
such that for $\Phi_{\tau}'=W \Phi_{\tau} \sigma(W)^{-1}$ we have that $\ord_p(\Phi_{\tau}') \geq 0$. 
Defining the matrix $\tilde{\Phi}_{\tau}'=W \tilde{\Phi}_{\tau} \sigma(W)^{-1}$,
we find that $\ord_p(\Phi'_{\tau}-\tilde{\Phi}_{\tau}') \geq N$, and in particular
also $\ord_p(\tilde{\Phi}_{\tau}') \geq 0$. So we get that
\[
\ord_p \Bigl( \det\bigl(1 - T (\Phi'_{\tau})^{(a)}\bigr) 
            - \det\bigl(1 - T (\tilde{\Phi}'_{\tau})^{(a)}\bigr) \Bigr) \geq N.
\] 
Note that $(\Phi'_{\tau})^{(a)}= W \Phi_{\tau}^{(a)} W^{-1}$
and $(\tilde{\Phi}'_{\tau})^{(a)}= W \tilde{\Phi}_{\tau}^{(a)} W^{-1}$, so that
$\det\bigl(1 - T (\Phi_{\tau})^{(a)}\bigr)=\det\bigl(1 - T (\Phi'_{\tau})^{(a)}\bigr)$
and 
$\det\bigl(1 - T (\tilde{\Phi}_{\tau})^{(a)}\bigr)=\det\bigl(1 - T (\tilde{\Phi}'_{\tau})^{(a)}\bigr)$.
This finishes the proof.
\end{proof}
Now we ask to what $p$-adic precision~$N$ we need to compute
$\det\bigl(1 - T \Phi_{\tau}^{(a)} \bigr)$ in order to recover 
$\chi(T)$.

\begin{thm} \label{thm:N0}
In order to recover $\chi(T)$, it suffices to compute 
$\det\bigl(1 - T \Phi_{\tau}^{(a)} \bigr)$ to $p$-adic precision
$N$ satisfying
\begin{equation}
p^N > 2 \max_{0 \leq i \leq b} \{ \binom{b}{i} \mathfrak{q}^{i (n-1) / 2} \}.
\end{equation}
Moreover, if  the sign $\epsilon = \pm 1$ is known for which $\det(\Phi^{(a)}) = \epsilon \mathfrak{q}^{b(n-1)/2}$, then this can be improved to
\begin{equation}
p^N > 2 \binom{b}{\floor{b/2}} \mathfrak{q}^{\floor{b/2} (n-1) / 2}.
\end{equation}
\end{thm}

\begin{proof}
Recall that [[TODO Jan]]
\[
\chi(T)=\prod_{i=1}^b (1-\alpha_i T),
\]
where the $\alpha_i$ are algebraic integers of absolute value $\mathfrak{q}^{(n-1)/2}$ that are
permuted under the map $\alpha \mapsto \mathfrak{q}^{n-1}/\alpha$. If we denote
$\chi(T) = \sum_{i=0}^{b} \chi_i T^i$, then this implies that $|\chi_i| \leq \binom{b}{i} \mathfrak{q}^{i (n-1)/2}$.
Therefore, if we compute $\chi(t)$ with $p$-adic precision $N$ satisfying
\[
p^N > 2 \max_{0 \leq i \leq b} \{ \binom{b}{i} \mathfrak{q}^{i (n-1) / 2} \},
\] 
then we know it exactly.

Since $\prod_{i=1}^b \alpha_i = \epsilon \mathfrak{q}^{b(n-1)/2}$, and the $\alpha_i$ are permuted
under the map $\alpha \mapsto \mathfrak{q}^{n-1}/\alpha$ [[TODO Jan]], we have that
\begin{equation*}
\chi_{b-i}=\epsilon (-1)^{b} \mathfrak{q}^{(n-1)(b/2-i)} \chi_b. 
\end{equation*}
So if $\epsilon$ is known, then
$\chi(T)$ is uniquely determined already by $\chi_0,\dotsc,\chi_{\lfloor b/2 \rfloor}$,
and it is sufficient to compute this polynomial with $p$-adic precision $N$ satisfying
\[
p^N > 2 \binom{b}{\floor{b/2}} \mathfrak{q}^{\floor{b/2} (n-1) / 2}.
\]
\end{proof}
\begin{algorithm}
\caption{Compute $Z(X_{\tau},T)$.}
\label{alg:complete}
\begin{algorithmic}
\vspace{1mm}
\Require $P \in \ZZ_q[t][x_0,\ldots,x_n]$ satisfying .., $\tau \in S(\mathbf{F}_{\mathfrak{q}})$.
\Ensure  The zeta function $Z(X_{\tau},T)$ of the fiber $X_{\tau}$ lying over $\tau$.
\Procedure{ZetaFunction}{$P,\tau$}
\State \begin{compactenum}[{\hspace{1em} } 1.] \vspace{-1.24em}
\item   \textbf{if} $\epsilon$ is known 
\item[] \hspace{1em} $N \gets \ceil{\log_p ( 2 \binom{b}{\floor{b/2}} \mathfrak{q}^{\floor{b/2} (n-1) / 2})}$ 
\item[] \textbf{else}
\item[] \hspace{1em} $N \gets \ceil{\log_p (2 \max_{0 \leq i \leq b} \{ \binom{b}{i} \mathfrak{q}^{i (n-1) / 2} \})}$ 
\item[] $\tilde{N} \gets N + \delta$
\item Compute $\hat{\tau} \in \mathcal{S}(\ZZ_{\mathfrak{q}})$ with $p$-adic precision $\tilde{N}$.
\item Determine $m,K$ such that $r^m \Phi$ is a matrix of polynomials of degree at most $K$ to $p$-adic 
      precision $\tilde{N}$ using Theorem \ref{thm:KedlayaTuitman}.
\item $\Phi \gets$ \textsc{FrobSeriesExpansion($\tilde{N},K)$}
\item $\Phi_{\tau} \gets r(\hat{\tau})^{-m} \bigl( r(t)^m \Phi(t) \bmod{t^{K}} \bigr)|_{t=\hat{\tau}} \pmod{p^{\tilde{N}}}$
\item $\chi(T) \gets \det\bigl(1-T \bigl(\Phi_{\tau} \sigma(\Phi_{\tau}) \ldots \sigma^{a-1}(\Phi_{\tau}) \bigr)  \bigr) \pmod{p^{N}}$ as an element of $\ZZ[T]$ 
\item $Z(X_{\tau},T) \gets \frac{\chi(T)^{(-1)^n}}{(1 - T) (1 - \mathfrak{q}T) \dotsm (1 - \mathfrak{q}^{n-1}T)}$
\item \Return $Z(X_{\tau},T)$
\end{compactenum}
\EndProcedure
\end{algorithmic}
\end{algorithm}

\begin{rem}
It is well known that $\epsilon = 1$ when $n$ is even, but when $n$ is odd
$\epsilon$ is usually not known. In practice it is often still possible to avoid 
having to use the greater of the two precisions in Theorem~\ref{thm:N0} by
computing $\epsilon$ first.  Again denote $\chi(T) = \sum_{i=0}^{b} \chi_i T^i$ and 
let $k$ be the smallest positive integer such that $\chi_{\ceil{b/2} - k} \neq 0$.
We know that we can recover $\chi_0, \dotsc, \chi_{\floor{b/2}+k}$ provided that
\begin{equation*}
p^N > 2 \max_{0 \leq i \leq (\floor{b/2}+k)} \{ \binom{b}{i} \mathfrak{q}^{i (n-1) / 2} \}.
\end{equation*}
This allows us to determine $\epsilon$ from the two coefficients $\chi_{\ceil{b/2}-k}$ and 
$\chi_{\floor{b/2}+k}$. We can then recover the remaining coefficients using the functional 
equation. 
%The best strategy for odd $n$ is to first choose $N$ corresponding to $k=1$ and compute 
%$\chi(T)$ with precision $N$. In the (unlikely) case that $\chi_{\ceil{b/2} - 1} = 0$, we raise
%$N$ to the value corresponding to the smallest $k$ such that $\chi_{\ceil{b/2} - k} \neq 0$
%(which is now known) and compute $\chi(T)$ again. 
\end{rem}

\begin{comment}
[[TODO It does not make sense to state this only for surfaces, either state it
more generally (possibly sketchy), or take out completely]]
\begin{rem} \label{rem:N0Surfaces}
In the case of smooth projective surfaces, when $p > 2$ and subject to 
certain technical conditions, we can often exploit the growing divisibility 
of the coefficients of $p(T)$ ensured by the Hodge polygon.  For such a surface 
of degree~$d$, we know that the Hodge numbers $h_{0,2}$, $h_{1,1}$ and $h_{2,0}$ 
satisfy $h_{0,2} = h_{2,0} = \binom{d-1}{3}$ and $2 h_{0,2} + h_{1,1} = b$. 
It is now easier to determine the integer coefficients of the 
polynomial~$q^{h_{0,2}} p(T/q)$ as the roots of the polynomial $p(T/q)$ lie 
on the unit circle.  This allows us to take 
\begin{align*}
N_0 & = a h_{0,2} + \floor{\log_p \biggl( 2 \binom{b}{\floor{b/2}}\biggr)} + 1,\\
N_1 & = N_0 + a,
\end{align*}
where $N_0$ here refers to the required precision for $q^{h_{0,2}} p(T/q)$ 
and $N_1$ as before refers to the precision required for the matrix 
representing $q^{-1} \Frob_q$ on $\Hrig^{n}(U_{t_1})$.
For further details, we refer the reader to 
Lauder~\citep[\S 9.3.2, Proposition~9.6]{Lauder2006}.
\end{rem}
\end{comment}

%%%%%%%%%%%%%%%%%%%%%%%%%%%%%%%%%%%%%%%%%%%%%%%%%%%%%%%%%%%%%%%%%%%%%%%%%%%%%%%

\section{Complexity and Implementation}

\begin{comment}

We now provide an outline of the deformation method.  The list below 
includes nearly full details of the computation, omitting only the 
specific choices of precisions $N_1, \dotsc, N_7$
and $K_1, K_2$. These will be provided and justified later.

\begin{enumerate}
\item[Step~$\Rmnum{1}$.]
Compute the matrix $\Phi_0 \in M_{b \times b}(\mathbf{Q}_q)$ for the action 
of $p^{-1} \Frob_p$ on $\Hrig^{n}(U_0)$ with respect to the standard basis 
to $p$-adic precision~$N_1$.  
\item[Step~$\Rmnum{2}$.]
Compute the matrix $M \in M_{b \times b}(\mathbf{Q}_q(t))$ of the 
Gauss--Manin connection $\nabla$ on $\HdR^{n}(\mathfrak{U}_L)$ with 
respect to the standard basis to $p$-adic precision~$N_2$.
\item[Step~$\Rmnum{3}$.]
Let $C \in M_{b \times b} (\mathbf{Q}_q[[t]])$ denote the matrix of 
local solutions of the connection~$\nabla$ around zero, i.e., 
the matrix that satisfies the $p$-adic differential equation 
$\bigl(\tfrac{d}{dt} + M\bigr) C = 0$ with initial condition $C(0)=I$.  
We compute an approximation for the matrices~$C$ and 
$C^{\sigma}(t^p)^{-1}$ modulo~$t^{K_1}$ to $p$-adic precision~$N_3$.
\item[Step~$\Rmnum{4}$.]
Let $\Phi(t) \in M_{b \times b}(\mathbf{Q}_p\langle t,r(t)^{-1}\rangle^{\dagger})$ 
denote the matrix for the action of $p^{-1} \Frob_p$ on $\Hrig^{n}(U/S)$ with 
respect to the standard basis (which is only a basis generically).  One can 
show that this matrix satisfies $\Phi(t) = C(t) \Phi_0 C^{\sigma}(t^p)^{-1}$.
Using this formula we compute a power series approximation of $\Phi(t)$ 
modulo~$t^{K_1}$ to $p$-adic precision~$N_4$.
\item[Step~$\Rmnum{5}$.]
Let $\Phi_{t_1} \in M_{b \times b}(\mathbf{Q}_q)$ denote the matrix for 
the action of $p^{-1} \Frob_p$ on $\Hrig^{n}(U_{t_1})$. First we compute 
$\Psi(t) = r(t)^{K_2} \Phi(t)$ modulo~$t^{K_1}$ to $p$-adic precision~$N_5$. 
Then we compute an approximation $r(\hat{t}_1)^{-K_1} \Psi(\hat{t}_1)$ for 
$\Phi_{t_1}$ to $p$-adic precision~$N_5$.
\item[Step~$\Rmnum{6}$.]
Let $\Phi_{t_1}^{(q^m)}$ denote the matrix for the action of 
$q^{-m} \Frob_{q^m}$ on $\Hrig^{n}(U_{t_1})$, which satisfies 
$\Phi_{t_1}^{(q^m)} = \Phi_{t_1} \sigma(\Phi_{t_1}) \dotsm \sigma^{a-1}(\Phi_{t_1})$ 
and has entries in $\mathbf{Q}_{q^m}$.  We compute an approximation to 
this matrix with $p$-adic precision~$N_6$.
%TODO change notation for q^m power Frobenius matrix, looks like power of matrix
\item[Step~$\Rmnum{7}$.]
Let $p(T) = \det\bigl(1 - T (q^m)^{-1} \Frob_{q^m} | \Hrig^n(U_{t_1})\bigr)$, 
which is an integer polynomial.  We recover this exactly by computing an 
approximation to the reverse characteristic polynomial of $\Phi_{t_1}^{(q^m)}$
to $p$-adic precision~$N_7$.
\end{enumerate}
\end{comment}

\subsection{Complexity}

\subsection{Implementation}

We observe that most of the steps in the algorithm can be
carried out in parallel:
\begin{itemize}
\item In Algorithm \ref{alg:Connection}, the images of the different 
      basis vectors under $\nabla$ can be computed independently.
\item In Algorithm \ref{alg:Diabfrob}, the elements of the sequence 
      $(\mu_m)_{m=0}^{\mathcal{M}}$ can be computed independently.
\item At first sight Algorithm \ref{alg:expansion} cannot be parallelised so easily 
      as every $C_{i}$ depends on (some of) the $C_{j}$ with $j < i$. However, in
      the computation of every single $C_i$ the different matrix products can
      still be computed independently.
\item The computation of the $\mathfrak{q}$th-power Frobenius 
      $\Phi_{\tau} \sigma(\Phi_{\tau}) \dotsm \sigma^{a-1}(\Phi_{\tau})$, can be carried 
      out in parallel.  This can be achieved if, instead of computing 
      a running product from left to right, we express the matrix in a 
      product tree.  We formalise this in Algorithm~\ref{alg:ParallelProduct}, 
      assuming for simplicity that $a = 2^k$.  In this notation, the execution 
      of the loops indexed by~$i$ can be parallelised.
      \begin{algorithm}
      \caption{Parallel computation of $q^{-1} \Frob_{q} | \Hrig^{n}(U_{\tau})$}
      \label{alg:ParallelProduct}
      \begin{algorithmic}
      \vspace{1mm}
      \For{$i \gets 0$ \textbf{to} $a-1$}
          \State $G_i \gets \sigma^{i}(F)$
      \EndFor
      \For{$j=k-1$ \textbf{to} $0$}
          \State $h \gets 2^{k-j}$
          \For{$i \gets 0$ \textbf{to} $2^j-1$}
          \State $G_{2^j-1 + ih} \gets G_{2^j-1 + i h} G_{2^j-1 + ih + h/2}$
          \EndFor
      \EndFor
      \Return $G_0$
      \end{algorithmic}
      \end{algorithm}
\end{itemize}

[[TODO Seb, should it not be clearer that $\sigma^i$ is not computed by composing $\sigma$ with itself $i$ times? ]]

%%%%%%%%%%%%%%%%%%%%%%%%%%%%%%%%%%%%%%%%%%%%%%%%%%%%%%%%%%%%%%%%%%%%%%%%%%%%%%%

\section{Examples}
\label{sec:Examples}

\phantomsection

\bibliographystyle{plainnat}
\bibliography{deformation}

\end{document}
