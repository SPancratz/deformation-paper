\documentclass[a4paper,11pt]{article}

\author{Sebastian Pancratz, Jan Tuitman}
\title{Improvements to the deformation method}

% Geometry and page layout %%%%%%%%%%%%%%%%%%%%%%%%%%%%%%%%%%%%%%%%%%%%%%%%%%%%

\usepackage[hmargin=3.2cm,vmargin=3.2cm,a4paper,centering,twoside]{geometry}

% Other packages %%%%%%%%%%%%%%%%%%%%%%%%%%%%%%%%%%%%%%%%%%%%%%%%%%%%%%%%%%%%%%

\usepackage[T1]{fontenc}
\usepackage{ae,aecompl}

\usepackage{ifpdf}

% hyperref %%%%%%%%%%%%%%%%%%%%%%%%%%%%%%%%%%%%%%%%%%%%%%%%%%%%%%%%%%%%%%%%%%%%

\usepackage{hyperref}
\hypersetup{
    colorlinks=false,   % false: boxed links; true: colored links
    citecolor=green,    % color of links to bibliography
    filecolor=magenta,  % color of file links
    linkcolor=red,      % color of internal links
    urlcolor=blue       % color of external links
}

\makeatletter
\newcommand\org@hypertarget{}
\let\org@hypertarget\hypertarget
\renewcommand\hypertarget[2]{%
    \Hy@raisedlink{\org@hypertarget{#1}{}}#2%
} 
\makeatother

\ifpdf
    \hypersetup{
        pdftitle={Deformation method},
        pdfauthor={Sebastian Pancratz, Jan Tuitman},
        pdfsubject={Computational Number Theory},
        bookmarks=true,
        bookmarksnumbered=true,
        unicode=true,
        pdfstartview={FitH},
        pdfpagemode={UseOutlines}
    }
\fi

% algorithmic %%%%%%%%%%%%%%%%%%%%%%%%%%%%%%%%%%%%%%%%%%%%%%%%%%%%%%%%%%%%%%%%%

\usepackage[section]{algorithm}
\usepackage[noend]{algpseudocode}

\renewcommand{\algorithmicrequire}{\textbf{Input:}}
\renewcommand{\algorithmicensure}{\textbf{Output:}}

% natbib %%%%%%%%%%%%%%%%%%%%%%%%%%%%%%%%%%%%%%%%%%%%%%%%%%%%%%%%%%%%%%%%%%%%%%

\usepackage{natbib}

\bibpunct{[}{]}{,}{n}{}{}

% url %%%%%%%%%%%%%%%%%%%%%%%%%%%%%%%%%%%%%%%%%%%%%%%%%%%%%%%%%%%%%%%%%%%%%%%%%

\usepackage{url}

\makeatletter
\def\url@leostyle{%
  \@ifundefined{selectfont}{\def\UrlFont{\sf}}{\def\UrlFont{\small\ttfamily}}}
\makeatother
\urlstyle{leostyle}

% Enumeration %%%%%%%%%%%%%%%%%%%%%%%%%%%%%%%%%%%%%%%%%%%%%%%%%%%%%%%%%%%%%%%%%

\usepackage{paralist}

\setlength{\pltopsep}{0.24em}
\setlength{\plpartopsep}{0em}
\setlength{\plitemsep}{0.24em}

% This should do what we want
%   \setdefaultenum{(i)}{(a)}{1.}{A}
% but it does not work for references, dropping the parentheses.  The following
% hack does work.

\renewcommand{\theenumi}{(\roman{enumi})}
\renewcommand{\theenumii}{(\alph{enumii})}
\renewcommand{\theenumiii}{\arabic{enumiii}.}
\renewcommand{\theenumiv}{\Alph{enumiv}}

\renewcommand{\labelenumi}{\theenumi}
\renewcommand{\labelenumii}{\theenumii}
\renewcommand{\labelenumiii}{\theenumiii}
\renewcommand{\labelenumiv}{\theenumiv}

%%%%%%%%%%%%%%%%%%%%%%%%%%%%%%%%%%%%%%%%%%%%%%%%%%%%%%%%%%%%%%%%%%%%%%%%%%%%%%%
% Mathematics

% Packages %%%%%%%%%%%%%%%%%%%%%%%%%%%%%%%%%%%%%%%%%%%%%%%%%%%%%%%%%%%%%%%%%%%%

\usepackage{amsmath,amsthm,amscd,amsfonts,amssymb}
\usepackage{cases}
\usepackage[all]{xy}

\allowdisplaybreaks[4]
\numberwithin{equation}{section}

% Customised notation %%%%%%%%%%%%%%%%%%%%%%%%%%%%%%%%%%%%%%%%%%%%%%%%%%%%%%%%%

\providecommand{\abs}[1]{\lvert#1\rvert}                 % Absolute value
\providecommand{\absbig}[1]{\bigl\lvert#1\bigr\rvert}    % Absolute value
\providecommand{\absBig}[1]{\Bigl\lvert#1\Bigr\rvert}    % Absolute value
\providecommand{\absbigg}[1]{\biggl\lvert#1\biggr\rvert} % Absolute value

\providecommand{\norm}[1]{\lVert#1\rVert}              % Norm
\providecommand{\normbig}[1]{\bigl\lVert#1\bigr\rVert} % Norm
\providecommand{\normBig}[1]{\Bigl\lVert#1\Bigr\rVert} % Norm

\providecommand{\floor}[1]{\left\lfloor#1\right\rfloor}   % Floor
\providecommand{\floorts}[1]{\lfloor#1\rfloor}            % Floor
\providecommand{\floorbig}[1]{\bigl\lfloor#1\bigr\rfloor} % Floor
\providecommand{\floorBig}[1]{\Bigl\lfloor#1\Bigr\rfloor} % Floor

\providecommand{\ceil}[1]{\left\lceil#1\right\rceil}   % Ceiling
\providecommand{\ceilts}[1]{\lceil#1\rceil}            % Ceiling
\providecommand{\ceilbig}[1]{\bigl\lceil#1\bigr\rceil} % Ceiling
\providecommand{\ceilBig}[1]{\Bigl\lceil#1\Bigr\rceil} % Ceiling

\newcommand{\NN}{\mathbf{N}} % Natural numbers
\newcommand{\ZZ}{\mathbf{Z}} % Integers
\newcommand{\QQ}{\mathbf{Q}} % Rationals
\newcommand{\RR}{\mathbf{R}} % Real numbers
\newcommand{\CC}{\mathbf{C}} % Complex numbers
\newcommand{\FF}{\mathbf{F}} % Finite field

\renewcommand{\to}{\rightarrow}        % Right arrow
\newcommand{\into}{\hookrightarrow}    % Injection arrow
\newcommand{\onto}{\twoheadrightarrow} % Surjection arrow

\DeclareMathOperator{\fCoKer}{coker} % Cokernel
\DeclareMathOperator{\fKer}{ker}     % Kernel
\DeclareMathOperator{\fIm}{im}       % Image

\DeclareMathOperator{\Res}{Res}   % Resultant
\DeclareMathOperator{\Tr}{Tr}     % Trace
\DeclareMathOperator{\Trace}{Tr}  % Trace
\DeclareMathOperator{\Norm}{N}    % Norm
\DeclareMathOperator{\Disc}{Disc} % Discriminant

\DeclareMathOperator{\Gal}{Gal}          % Galois group
\DeclareMathOperator{\ord}{ord}          % Order
\DeclareMathOperator{\sgn}{sgn}          % Sign, signature
\DeclareMathOperator{\Frob}{\mathcal{F}} % Frobenius
\DeclareMathOperator{\Hom}{Hom}          % Space of homomorphisms
\DeclareMathOperator{\Spec}{Spec}        % Spectrum

\providecommand{\HdR}{H_{\text{dR}}}    % de Rham cohomology
\providecommand{\Het}{H_{\text{\'et}}}  % etale cohomology
\providecommand{\Hrig}{H_{\text{rig}}}  % rigid cohomology

\providecommand{\cB}{\mathcal{B}} % Basis
\providecommand{\cR}{\mathcal{R}} % Row index set
\providecommand{\cC}{\mathcal{C}} % Column index set
\providecommand{\cM}{\mathcal{M}} % Complexity of multiplication

\providecommand{\BigOh}{\mathcal{O}} % Big-oh notation

% Theorems etc %%%%%%%%%%%%%%%%%%%%%%%%%%%%%%%%%%%%%%%%%%%%%%%%%%%%%%%%%%%%%%%%

\theoremstyle{definition}

\newtheorem{thm}{Theorem}[section]
\newtheorem{lem}[thm]{Lemma}
\newtheorem{prop}[thm]{Proposition}
\newtheorem{cor}[thm]{Corollary}
\newtheorem{defn}[thm]{Definition}
\newtheorem{exmp}[thm]{Example}
\newtheorem{rem}[thm]{Remark}
\newtheorem{prob}[thm]{Problem}

%%%%%%%%%%%%%%%%%%%%%%%%%%%%%%%%%%%%%%%%%%%%%%%%%%%%%%%%%%%%%%%%%%%%%%%%%%%%%%%
% DOCUMENT                                                                    %
%%%%%%%%%%%%%%%%%%%%%%%%%%%%%%%%%%%%%%%%%%%%%%%%%%%%%%%%%%%%%%%%%%%%%%%%%%%%%%%

\begin{document}

\maketitle

\tableofcontents

%%%%%%%%%%%%%%%%%%%%%%%%%%%%%%%%%%%%%%%%%%%%%%%%%%%%%%%%%%%%%%%%%%%%%%%%%%%%%%%

\section{Introduction}
\label{sec:Introduction}

\subsection{Overview of the algorithm}

We now provide an outline of the deformation method.  The list below 
includes nearly full details of the computation, omitting only the 
specific choice of precisions $N_0, \dotsc, N_5$, and $K_1, K_2$.  
These are provided and justified in subsequent sections.

\begin{enumerate}
\item[Step~$I$.]
Let $\Phi_0$ denote the matrix for $p^{-1} \Frob_p$ on 
$\Hrig^{n}(U_0)$ with entries in~$\mathbf{Q}_p$.  
We require an approximation to $p$-adic precision~$N_5$.
\item[Step~$II$.]
Let $M$ denote the connection matrix on $\HdR^{n}(\mathfrak{U}/\mathfrak{S})$ 
with entries in~$\mathbf{Q}(t)$.  We require an approximation 
to precision~$N_4$.
\item[Step~$III$.]
Let $C$ denote the matrix over $\mathbf{Q}[[t]]$ for the local 
expansion of the solution to the $p$-adic differential equation 
$\bigl(\tfrac{d}{dt} + M\bigr) C = 0$.  We compute an approximation 
modulo~$t^{K_2}$ to $p$-adic precision~$N_3$.  We also compute an 
approximation for the matrix~$C(t^p)^{-1}$ modulo~$t^{K_2}$ to $p$-adic 
precision~$N_3'$.
\item[Step~$IV$.]
Let $\Phi(t)$ denote the matrix for the action of $p^{-1} \Frob_p$ on 
$\Hrig^{n}(U/S)$, which satisfies the equation 
$\Phi(t) = C(t) \Phi_0 C(t^p)^{-1}$ and has entries in 
$\mathbf{Q}_p\langle t,r(t)^{-1}\rangle^{\dagger}$.  We compute 
a power series approximation modulo~$t^{K_2}$ to $p$-adic precision~$N_2$.
\item[Step~$V$.]
Let $\Phi_{t_1}$ denote the matrix for the action of $p^{-1} \Frob_p$ 
on $\Hrig^{n}(U_{t_1})$, which has entries in $\mathbf{Q}_q$. 
Firstly, we compute an approximation to $\Psi(t) = r(t)^{K_1} \Phi(t)$ 
modulo~$t^{K_2}$ to $p$-adic precision~$N_2$.  We can then compute 
an approximation for $\Phi_{t_1}$ to $p$-adic precision~$N_2$ 
as $r(\hat{t}_1)^{-K_1} \Psi(\hat{t}_1)$.
\item[Step~$VI$.]
Let $\Phi_{t_1}^{(q)}$ denote the matrix for the action of $q^{-1} \Frob_q$ on 
$\Hrig^{n}(U_{t_1})$, which satisfies 
$\Phi_{t_1}^{(q)} = \Phi_{t_1} \sigma(\Phi_{t_1}) \dotsm \sigma^{a-1}(\Phi_{t_1})$ 
and has entries in $\mathbf{Q}_q$.  We compute an approximation to this 
matrix with $p$-adic precision~$N_1$.
\item[Step~$VII$.]
Let $p(T) = \det\bigl(1 - T q^{-1} \Frob_q | \Hrig^n(U_{t_1})\bigr)$, 
which is an integer polynomial.  We recover this exactly 
by computing an approximation to the reverse characteristic polynomial 
of $\Phi_{t_1}^{(q)}$ to $p$-adic precision~$N_0$.
\end{enumerate}

%%%%%%%%%%%%%%%%%%%%%%%%%%%%%%%%%%%%%%%%%%%%%%%%%%%%%%%%%%%%%%%%%%%%%%%%%%%%%%%

\section{Theoretical background}
\label{sec:Background}

%%%%%%%%%%%%%%%%%%%%%%%%%%%%%%%%%%%%%%%%%%%%%%%%%%%%%%%%%%%%%%%%%%%%%%%%%%%%%%%

\section{Computing in de~Rham cohomology}
\label{sec:deRham}

In this section we present a concrete representation of the algebraic 
de~Rham cohomology spaces $\HdR^{n}(\mathfrak{U}/\mathfrak{S})$, 
restricted to the case of the complement $\mathfrak{U}/\mathfrak{S}$ 
of a smooth projective family $\mathfrak{X} / \mathfrak{S}$.  We first 
follow Abbott, Kedlaya and Roe~\citep[Remark~3.2.5]{AbbottKedlayaRoe2006} 
in the description of the so-called \emph{reduction of poles} procedure for 
the de~Rham cohomology $L$-vector space $\HdR^{n}(\mathfrak{U} / L)$, 
considering $\mathfrak{U}$ as a scheme over $L = \QQ_{p}(t)$.

\begin{thm}
Let $\Omega$ denote the $n$-form $\Omega$ in $\HdR^{n}(\mathfrak{U}/L)$ 
defined by 
\begin{equation}
\Omega = \sum_{i=0}^n (-1)^i x_i d x_0 \wedge \dotsb \wedge \widehat{d x_i} \wedge \dotsb \wedge d x_n.
\end{equation}
The algebraic de~Rham cohomology space $\HdR^{n}(\mathfrak{U}/L)$ is 
isomorphic as a $L$-vector space to the quotient of the group of $n$-forms 
$Q \Omega / P^k$ with $k \in \NN$ and $Q \in L[x_0, x_1, \dotsc, x_n]$ 
homogeneous of degree $k d - (n + 1)$ by the subgroup generated by the 
relation 
\begin{equation} \label{eq:deRhamRel}
\frac{(\partial_i Q) \Omega}{P^k} - k \frac{Q (\partial_i P) \Omega}{P^{k+1}}
\end{equation}
for all $0 \leq i \leq n$, where here and in the following $\partial_i$ 
denotes the partial derivative operator with respect to~$x_i$.
\end{thm}

\begin{proof}
This result follows from a calculation as 
in Griffiths~\citep[\S 4]{Griffiths1969}.
\end{proof}

It is clear that $\HdR^{n}(\mathfrak{U}/L)$ can be equipped with a 
filtration whose $i$th parts consists of all elements which can be 
represented by $n$-forms with degree $\deg Q = kd - (n + 1)$ for 
$1 \leq k \leq i + 1$. We can obtain a basis for $\HdR^{n}(\mathfrak{U}/L)$ 
respecting this filtration as follows.  For every $k \in \NN$, we find 
an independent set $B_k$ of polynomials of degree $kd-(n+1)$ generating 
the quotient of the space of all such polynomials by the Jacobian ideal 
$(\partial_0 P, \dotsc, \partial_n P)$.  This yields a generating set 
$\bigcup_{k \in \NN} \cB_k$ for $\HdR^n(\mathfrak{U}/L)$ where 
$\cB_k = \{Q \Omega / P^k : Q \in B_k\}$.  It follows from a theorem of 
Macaulay~\citep[\S 4, (4.11)]{Griffiths1969} that in fact the set 
$\cB = \cB_1 \cup \dotsb \cup \cB_n$ already forms a generating set.  
Later we shall exhibit an explicit basis of monomials in the case where 
the family of projective hypersurfaces contains a diagonal fibre.

Now, to obtain a unique representative for the class of $Q \Omega / P^k$ 
in terms of the basis elements in $\cB$, we express $Q$ in terms of 
$\partial_0 P, \dotsc, \partial_n P$ as well as elements of $B_k$, and 
then iteratively reduce the pole order of the first part according to 
the relations given by the expressions from equation~\eqref{eq:deRhamRel}. 
Assume that we have at our disposal a routine {\sc Decompose} which, 
given a polynomial $Q$ of degree $kd - (n+1)$ in the ideal generated by 
$(\partial_0 P, \dotsc, \partial_n P)$ and $B_k$, returns an expression 
$Q = Q_0 \partial_0 P + \dotsb + Q_n \partial_n P + \gamma_k$ with 
$Q_0, \dotsc, Q_n$ homogeneous polynomials in $L[x_0, \dotsc, x_n]$ and 
$\gamma_k$ in the $L$-span of $B_k$.  The reduction of poles procedure can 
then be formalised as in Algorithm~\ref{alg:PoleRed} below, which we will 
refer to as {\sc Reduce}.  In this generality, the correctness of the 
algorithm depends on a theorem of Macaulay~\citep[\S 4, (4.11)]{Griffiths1969}.

\begin{algorithm}
\caption{Reduce $Q \Omega / P^k$ in $\HdR^n(\mathfrak{U}/L)$}
\label{alg:PoleRed}
\begin{algorithmic}
\vspace{1mm}
\Require $Q \in L[x_0, \dotsc, x_n]$ homogeneous of degree $kd - (n+1)$.
\Ensure  $\gamma_i$ in the $L$-span of $B_i$, for $1 \leq i \leq n$, with  
         $Q \Omega / P^k \equiv \gamma_{1} \Omega / P^{1} + \dotsb + \gamma_n \Omega / P^n$.
\Procedure{Reduce}{$P, B_1, \dotsc, B_n, Q$}
\While{$k \geq n+1$}
\State $Q_0, \dotsc, Q_n, \bullet \gets \Call{Decompose}{Q, \partial_0 P, \dotsc, \partial_n P, B_k}$
\State $k \gets k-1$
\State $Q \gets k^{-1} \sum_{i=0}^n \partial_i Q_i$
\EndWhile
\State $\gamma_{k+1}, \dotsc, \gamma_{n} \gets 0$
\While{$Q \not \in B_k$}
\State $Q_0, \dotsc, Q_n, \gamma_k \gets \Call{Decompose}{Q, \partial_0 P, \dotsc, \partial_n P, B_k}$
\State $k \gets k-1$
\State $Q \gets k^{-1} \sum_{i=0}^n \partial_i Q_i$
\EndWhile
\If{$Q \neq 0$}
\State $\gamma_{k} \gets Q$
\State $k \gets k-1$
\EndIf
\State $\gamma_{1}, \dotsc, \gamma_{k} \gets 0$
\State \textbf{return} $\gamma_{1}, \dotsc, \gamma_n$
\EndProcedure
\end{algorithmic}
\end{algorithm}

We now specialise to the case of a smooth family of projective 
hypersurfaces containing a diagonal fibre.  We exhibit a basis 
of monomials~$B = B_1 \cup \dotsb \cup B_n$ such that the 
corresponding set~$\cB$ forms a basis for $\HdR^n(\mathfrak{U}/L)$ 
and re-express the problem of decomposing a homogeneous polynomial 
in the language of linear algebra.  From this description, we furnish 
an explicit reduction procedure in terms of matrices.  The approach 
is based on a generalisation of Sylvester matrices from two 
polynomials to $n+1$~polynomials, following Macaulay~\citep{Macaulay1994}.

\begin{defn} \label{defn:MonBasis}
For $k \in \NN$, we define the following sets of monomials, 
\begin{align*}
F_k & = \{ x^i : i \in \mathbf{N}_{0}^{n+1}, \abs{i} = k d - (n+1) \}, \\
B_k & = \{ x^i : i \in \mathbf{N}_{0}^{n+1}, \abs{i} = k d - (n+1) \text{ and $i_j < d-1$ for $0 \leq j \leq n$}\},
\end{align*}
where $x^i = x_0^{i_0} \dotsm x_n^{i_n}$ and 
$\abs{i} = i_0 + \dotsb + i_n$.
\end{defn}

In order to ensure that our computations are not vacuous, we assume 
that $\HdR^n(\mathfrak{U}/L)$ is non-zero.  After setting 
\begin{equation}
\ell = \ceil{\frac{n+1}{d}}, \quad u = \floor{\frac{(n+1)(d-1)}{d}} = n+1 - \ell,
\end{equation}
we note that $B_k$ is non-empty if and only if $\ell \leq k \leq u$.  The 
assumption is thus that $\ell \leq u$, which is equivalent to the statement 
that $d \geq 2$ whenever $n$ is odd and $d \geq 3$ whenever $n$ is even.

In the case when the family $\mathfrak{X}/\mathfrak{S}$ contains a 
diagonal fibre, it turns out that we can add additional constraints 
to our decomposition problem:

\begin{prob} \label{prob:Decomposition}
Given a homogeneous multivariate polynomial $Q \in L[x_0, \dotsc, x_n]$ 
of degree \mbox{$k d - (n + 1)$}, for some $k \in \NN$, we try to find 
homogeneous polynomials $Q_0, \dotsc, Q_n$ in $L[x_0, \dotsc, x_n]$ such 
that 
\begin{equation} \label{eq:Decomposition}
Q \equiv Q_0 \partial_0 P + \dotsb + Q_n \partial_n P
\end{equation}
modulo the $L$-span of $B_k$.  Moreover, for each $1 \leq j \leq n$,  the 
polynomial $Q_j$ may only contain non-zero coefficients for monomials of 
degree $(k-1)d-n$ that are not divisible by any of the monomials 
$x_0^{d-1}, \dotsc, x_{j-1}^{d-1}$.
\end{prob}

\begin{rem}
Immediately, we see that, for each $0 \leq i \leq n$, either $Q_i$ is 
identically zero or has degree $(k - 1) d - n$ since $\partial_i P$ is 
homogeneous of degree $d - 1$ and the elements of $B_k$ have degree 
$kd - (n+1)$.
\end{rem}

\begin{defn} \label{defn:IndexSets}
For $k \in \NN$, we define the following sets of monomials in 
$L[x_0, \dotsc, x_n]$.  Let $R_k = F_k - B_k$ be the set of monomials of 
total degree $kd-(n+1)$ and partial degree at least $d-1$ with respect to some 
of the $n+1$ variables.  Let $C_k^{(0)}$ be the set of monomials of total 
degree $(k-1)d - n$, and then inductively, for $j = 1, \dotsc, n$, define 
$C_k^{(j)}$ to be the set of monomials in $C_k^{(j-1)}$ except for those 
divisible by $x_{j-1}^{d-1}$.  Moreover, we define the multi-set $C_k$ as 
the disjoint union of $C_k^{(0)}, \dotsc, C_k^{(n)}$.  We shall write an 
element of this multi-set as $(j, g)$, referring to a monomial~$g$ 
in~$C_k^{(j)}$.
\end{defn}

With this set-up, the following theorem provides a solution to 
Problem~\ref{prob:Decomposition} in the cases that we are interested in:

\begin{thm} \label{thm:Isomorphism}
Suppose that the family of projective hypersurfaces given by the 
polynomial~$P$ in $L[x_0, \dotsc, x_n]$ contains a diagonal fibre.  
Let $k \in \NN$ and suppose that $R_k$ and $C_k$ are non-empty.  For 
$0 \leq j \leq n$, let $V_k^{(j)}$ be the $L$-vector space of 
polynomials with basis $C_k^{(j)}$, and let $V_k$ denote their cartesian 
product $V_k = V_k^{(0)} \times \dotsb \times V_k^{(n)}$.  Let $W_k$ be 
the $L$-vector space of polynomials with basis~$R_k$.  Then the $L$-linear 
map 
\begin{equation}
\phi_k \colon V_k \to W_k, 
(Q_0, \dotsc, Q_n) \mapsto Q_0 \partial_0 P + \dotsb + Q_n \partial_n P
\end{equation}
is an isomorphism of $L$-vector spaces.
\end{thm}

\begin{proof}
We first show that, for all $k \in \NN$, the multi-sets $R_k$ and $C_k$ 
have the same cardinality:

We construct the following bijection $R_k \to C_k$, representing the 
monomials by their exponent tuple.  Let $i = (i_0, \dotsc, i_n)$ be in 
$R_k$.  If $i_0 \geq d-1$, we define the image as
 $(i_0-d-1, i_1, \dotsc, i_n) \in C_k^{(0)}$.  More generally, if 
$i_0 < d-1, \dotsc, i_{j-1} < d-1$ and $i_j \geq d-1$, the image is 
$(i_0, \dotsc, i_{j-1}, i_j-d-1, i_{j+1}, \dotsc, i_n) \in C_k^{(j)}$.  
It is easy to verify that this map is indeed a bijection.

In order to establish that the map $\phi_k \colon V_k \to W_k$ is an 
isomorphism of $L$-vector spaces, we now exhibit its matrix with respect to 
the given basis:

Let $k \in \NN$ and suppose that $R_k$ and $C_k$ are non-empty, that is 
to say, $k \geq n/d + 1$.  We define the auxiliary matrix $\Delta_k$ with 
row and column index sets $R_k$ and $C_k$, respectively, as follows.  
Given $f \in R_k$ and $(j,g) \in C_k$, we set the corresponding entry in 
$\Delta_k$ to be the monomial coefficient of $f/g$ in $\partial_j P$ if 
$g$ divides $f$ and $0$ otherwise.  It is immediate that $\Delta_k$ is the 
matrix representing $\phi_k$ with respect to the bases $C_k$ and $R_k$ of 
$V_k$ and $W_k$, respectively.

The assumption that the family~$\mathfrak{X}$ of projective hypersurfaces 
given by~$P$ contains a diagonal hypersurface means that for some~$t_0$ 
the fibre $\mathfrak{X}_{t_0}$ is given by an equation of the form 
\begin{equation}
P_{t_0}(x_0, \dotsc, x_n) = a_0 x_0^d + \dotsb a_n x_n^d = 0
\end{equation}
with $a_0, \dotsc, a_n \in L^{\times}$.

We show that the determinant of $\Delta_k$ is non-zero in~$L$.  Since 
the specialisation to the diagonal fibre viz.\ evaluation of the matrix at 
$t = t_0$ commutes with computing the determinant, it suffices to show that 
the determinant of $(\Delta_k) \big |_{t=t_0}$ is non-zero in~$L$.
Since, for $0 \leq j \leq n$, 
$\partial_j P_{t_0} (x_0, \dotsc, x_n) = d a_j x_j^{d-1}$, there is 
precisely one non-zero entry in each column of $\Delta_k$.  Namely, in column 
$(j, g) \in C_k$ and row $g x_j^{d-1} \in R_k$ there is the non-zero entry 
$d \alpha_j$, concluding the proof.
\end{proof}

In principle, by including any of the methods available for solving linear 
systems, we are in a position to furnish a routine {\sc Decompose}, which 
we formalise in Algorithm~\ref{alg:Decompose}.

\begin{algorithm}[ht]
\caption{Obtain co-ordinates for $Q$ in the Jacobian ideal modulo basis elements}
\label{alg:Decompose}
\begin{algorithmic}
\Require $Q$ is a homogeneous polynomial of degree $kd - (n+1)$.
\Ensure  Homogeneous polynomials $Q_0, \dotsc, Q_n$ such that 
         $Q \equiv Q_0 \partial P_0 + \dotsb + Q_n \partial_n P$ modulo the 
         $L$-span of $B_k$.
\Procedure{Decompose}{$Q, \partial_0 P, \dotsc, \partial_n P, B_k$}
\State \begin{compactenum}[\it {Step} I.] \vspace{-1.24em}
\item Let $b$ be the vector of length $\abs{R_k}$ such that the entry 
      corresponding to the monomial $x^i \in R_k$ is the coefficient of 
      $x^i$ in $Q$.
\item Let $v$ be the unique vector of length $\abs{C_k}$ satisfying 
      $\Delta_k v = b$.  From the description of $C_k$ as a disjoint union, 
      we can write $v$ accordingly as $\bigl(v^{(0)}, \dotsc, v^{(n)}\bigr)$ 
      where, for $0 \leq j \leq n$, $v^{(j)}$ is a vector of length 
      $\abs{C_k^{(j)}}$.
\item For $j = 0, \dotsc, n$, set $Q_j = \sum_{g \in C_k^{(j)}} v_g^{(j)} g$,
      where $v_g^{(j)}$ is the entry in $v^{(j)}$ corresponding to the 
      monomial $g \in C_k^{(j)}$.
\item \textbf{return} $Q_0, \dotsc, Q_n$
\end{compactenum}
\EndProcedure
\end{algorithmic}
\end{algorithm}

We now establish that the set $\cB$ is indeed a basis for 
$\HdR^n(\mathfrak{U}/L)$, as claimed earlier, using the 
reduction of poles procedure.

\begin{thm} \label{thm:Basis}
Suppose that the family of projective hypersurfaces $\mathfrak{X}$ contains 
a diagonal fibre.  Then the set $\cB$ as in Definition~\ref{defn:MonBasis} 
is a basis for the $L$-vector space $\HdR^n(\mathfrak{U}/L)$.
\end{thm}

\begin{proof}
We know that $\HdR^n(\mathfrak{U}/L)$ is spanned by the classes of the 
$n$-forms $Q \Omega / P^k$ for all homogeneous polynomials~$Q$ of degree 
$kd-(n+1)$ and $k \in \NN$.  
By a theorem of Macaulay~\citep[\S 4, (4.11)]{Griffiths1969}, 
we may assume that $1 \leq k \leq n$, that is to say, any class in 
$\HdR^n(\mathfrak{U}/L)$ can be represented by an $n$-form with a pole 
of order at most~$n$.

Without loss of generality, we may thus start the reduction of poles 
procedure with a homogeneous polynomial~$Q$ of degree $(n+1)d-(n+1)$.  Then, 
since $B_{n+1} = \emptyset$ and $R_{n+1} = F_{n+1}$, 
Theorem~\ref{thm:Isomorphism} shows that there exist homogeneous polynomials 
$Q_0, \dotsc, Q_n$ either zero or homogeneous of degree $nd-(n+1)$ such that 
$Q = Q_0 \partial_0 P + \dotsb + Q_n \partial_n P$.  Continuing with the 
reduction of poles procedure as described in Algorithm~\ref{alg:PoleRed}, 
we obtain an expression for $Q \Omega / P^{n+1}$ as an $L$-linear combination 
of elements in $\cB$.  This shows that this set spans the vector space 
$\HdR^n(\mathfrak{U}/L)$.

To see that this set is linearly independent, note that it contains only 
monomials whose partial degrees are strictly less than $d-1$.  However, since 
$P$ is a homogeneous polynomial of degree $d$ and the family of hypersurfaces 
contains a diagonal fibre, it follows that, for each $0 \leq i \leq n$, the 
partial derivative $\partial_i P$ is a homogeneous polynomial of degree $d-1$ 
and contains precisely one monomial term with partial degree equal to $d-1$, 
namely that of the monomial $x_i^{d-1}$.  It follows that the elements of~$B$ 
cannot be reduced further modulo the Jacobian ideal and hence that the 
set~$\cB$ is linearly independent.
\end{proof}

%%%%%%%%%%%%%%%%%%%%%%%%%%%%%%%%%%%%%%%%%%%%%%%%%%%%%%%%%%%%%%%%%%%%%%%%%%%%%%%

\section{Constructing the connection matrix}
\label{sec:Connection}

Following the description of the Gauss--Manin connection 
in~[[TODO:  Reference]]
and the practical description by Kedlaya~\citep[\S 3.2]{Kedlaya2008a}, 
we now provide an explicit algorithm for computing the action of the 
Gauss--Manin connection.  Once again, we consider $\mathfrak{X}$ and 
$\mathfrak{U}$ as schemes over $L = \mathbf{Q}_p(t)$ and let $K = \mathbf{Q}_p$.
As $\Spec(L)$ is affine, we may apply the global section functor $\Gamma(L, -)$ 
throughout in Equation~[[TODO: Reference]] and consider
\begin{equation}
\nabla \colon \HdR^j(\mathfrak{X}/L) \to 
    \Omega_{L/K}^1 \otimes_{L} \HdR^j(\mathfrak{X}/L).
\end{equation}
The action of $\nabla$ can be computed as follows.  Given 
$\omega \in \HdR^j(\mathfrak{X}/L)$, we arbitrarily lift this to 
$\tilde{\omega} \in \Omega_{\mathfrak{X}/K}^j$.  After computing the 
exterior derivative, which decomposes as 
\mbox{$d_{\mathfrak{X}/K} = d_{L} + d_{\mathfrak{X}/L}$}, 
the image of $\omega$ under the Gauss--Manin connection is given as 
the projection of $d_{\mathfrak{X}/K}(\tilde{\omega})$ onto 
$\Omega_{L/K}^1 \otimes_{L} \HdR^j(\mathfrak{X}/L)$.

Since the fibres of the relative de~Rham cohomology 
$\HdR^j(\mathfrak{X}/L)$ can be identified with the de~Rham 
cohomology $\HdR^j(\mathfrak{X}_t/K)$ of the fibres, it 
follows that we can compute the action of the Gauss--Manin connection 
$\nabla \colon \HdR^{n-1}(\mathfrak{X}/L) \to 
\Omega_{L/K}^1 \otimes_{L} \HdR^{n-1}(\mathfrak{X}/L)$
via the induced map 
$\HdR^n(\mathfrak{U}/L) \to 
\Omega_{L/K}^1 \otimes_{L} \HdR^n(\mathfrak{U}/L)$, 
which by abuse of notation we shall refer to as $\nabla$, too.  
Let $\omega \in \HdR^{n}(\mathfrak{U}/L)$, where we may 
assume it is of the form $Q \Omega / P^k$ as described in the 
previous section.  Since $\Spec(L)$ is an affine curve, we have 
that $\Omega_{L/K}^1$ is free of rank one, generated by the symbol~$dt$. 
The action of $\nabla$ is then given by 
\begin{equation}
\nabla \colon \omega \mapsto d_{\mathfrak{U}}(\omega) = 
    d_{L}(\omega) + d_{\mathfrak{U}/L}(\omega) = 
    \frac{\partial}{\partial t} \omega \wedge dt
\end{equation}
where the term $d_{\mathfrak{U}/L}(\omega)$ vanishes by 
definition of $\Omega$.  A unique representative for the 
right-hand side can then be obtained using the reduction 
of poles procedure.  We formalise this in Algorithm~\ref{alg:Connection}.

\begin{algorithm}
\caption{Computing the Gauss--Manin connection matrix}
\label{alg:Connection}
\begin{algorithmic}
\Require Homogeneous polynomial $P$ in $K[t][x_0, \dotsc, x_n]$ of degree $d$, 
         defining a family of smooth projective hypersurfaces containing 
         a diagonal fibre.
\Ensure  Basis $B_1 \cup \dotsb \cup B_n$ for 
         $\HdR^n(\mathfrak{U}/L)$;  connection matrix~$M$ with 
         respect to this basis.
\Procedure{GMConnection}{$P$}
\State \begin{compactenum}[\it {Step} I.] \vspace{-1.24em}
\item Compute the partial and exterior derivatives 
      $\partial_0 P, \dotsc, \partial_n P$ and $dP/dt$.
\item Compute the basis $B = B_1 \cup \dotsb \cup B_n$.  In the following, 
      we let $(i,f)$ denote the polynomial $f \in B_i$.
\item Compute the auxiliary matrices $\Delta_k$ for 
      $k = \floor{n/d}+1, \dotsc, n+1$ as described 
      in Section~\ref{sec:deRham}, including pre-processing.
\item For all $(j, g) \in B$, let $Q = -j g P_t$ and set 
      $\gamma_{1}, \dotsc, \gamma_n$ to the output of 
      {\sc Reduce($P, B_1, \dotsc, B_n, Q$)}.  
      Then, for each $(i, f) \in B$, let $M_{(i,f),(j,g)}$ 
      be the coefficient of $f$ in $\gamma_i$.
\item \textbf{return} $B$, $M$
\end{compactenum}
\EndProcedure
\end{algorithmic}
\end{algorithm}

%%%%%%%%%%%%%%%%%%%%%%%%%%%%%%%%%%%%%%%%%%%%%%%%%%%%%%%%%%%%%%%%%%%%%%%%%%%%%%%

\section{Frobenius on diagonal hypersurfaces}
\label{sec:Diagonal}

\subsection{Introduction}

In this section we discuss the action of Frobenius on the vector 
space $\Hrig^{n}(U_0) \cong \HdR^{n}(\mathfrak{U}_0)$ associated 
to the diagonal fibre.  We describe a method based on an explicit 
formula due to Dwork~\citep[\S 4]{Dwork1964}, following the expositions 
Lauder~\citep[\S 6]{Lauder2004b} and Gerkmann~\citep[\S 4.4]{Gerkmann2007}. 
We present an algorithm which is suboptimal asymptotically but 
performs very well in practice.

In order to simplify our notation, we temporarily suppress the earlier 
setup of a family of hypersurfaces. Thus, we consider a smooth, 
projective, diagonal hypersurface~$\mathcal{X}$ over $\ZZ_p$ given by 
a polynomial $P \in \ZZ[x_0, \dotsc, x_n]$ of homogeneous degree~$d$ 
and its reduction~$X$ over $\mathbf{F}_p$ given by 
\begin{equation}
\tilde{P}(x_0, x_1, \dotsc, x_n) = 
    a_0 x_0^d + a_1 x_1^d + \dotsb + a_n x_n^d = 0
\end{equation}
where $a_0, a_1, \dotsc, a_n \in \FF_p^{\times}$ and $p \nmid d$.  
We also define the generic fibre and the various complements as 
before.  We fix our choice of basis for $\HdR^{n}(\mathfrak{U})$ 
as in Definition~\ref{defn:MonBasis}, that is, 
\begin{align}
B_k & = \{ x^i : \text{$\abs{i} = kd - (n+1)$ and $0 \leq i_j < d-1$, for all $j$} \}, \\
\mathcal{B}_k & = \{x^i \Omega / P^k : x^i \in B_k \}.
\end{align}
Finally, we recall the condition that $d \geq 2$ whenever $n$ is 
odd and $d \geq 3$ whenever $n$ is even.  Our goal is to compute 
the matrix~$\Phi$ representing the action of $p^{-1} \Frob_p$ on 
$\Hrig^n(U) \cong \HdR^n(\mathfrak{U})$ to $p$-adic precision~$N$.

We first describe an explicit formula for the $\dim \HdR^n(\mathfrak{U})$ 
non-zero coefficients of~$\Phi$ due to Dwork~\citep{Dwork1964}, see 
e.g.\ Lauder~\citep[\S 6.1]{Lauder2004b}.   We work over the ramified 
extension~$\QQ_p(\pi)$ where $\pi^{p-1} = -p$, and we normalise 
the valuation such that \mbox{$\ord_p(\pi) = (p-1)^{-1}$}.

Let $u = (u_0, \dotsc, u_n)$ and $v = (v_0, \dotsc, v_n)$ be tuples such 
that $x^u, x^v \in B_1 \cup \dotsb \cup B_n$,  and let $u'$ and $v'$ 
denote integers such that $d u' = \sum_{i=0}^n (u_i + 1)$ and similarly 
for $v'$.  Finally, for $m \geq 0$, let $\lambda_m$ denote the coefficient 
of $z^m$ in the expansion of $\exp \pi (z - z^p)$ and define products 
$(w)_r = \prod_{j=0}^{r-1} (w + j)$ for $w \in \QQ$ and $r \geq 0$. 
We introduce terms~$\alpha_{u,v}$, 
\begin{equation} \label{eq:alpha}
\alpha_{u,v} = \pi^{v' - u'} \prod_{i = 0}^n \sum_{m, r} \lambda_m (u_i / d)_r (-1)^r \pi^{-r} {\hat{a}_i}^{m-r}
\end{equation}
where $\hat{a}_i$ is the Teichm\"uller lift of $a_i \in \mathbf{F}_p$ 
to $\QQ_p$ and the summation indices $m, r \geq 0$ satisfy 
$p u_i - v_i = d (m - pr)$.

\begin{thm} \label{thm:01-03-diagfrob}
Let $\omega_1$ and $\omega_2$ denote the forms in 
$\cB_1 \cup \dotsb \cup \cB_n$ corresponding to $u$ and $v$, 
respectively.  Then $p^{-1} \Frob_p (\omega_1) = 0$ unless, 
for all $i = 0, \dotsc, n$, $p (u_i + 1) \equiv v_i + 1 \pmod{d}$.  
In this case, 
\begin{equation}
p^{-1} \Frob_p (\omega_1) = 
    (-1)^{u' + v'} \frac{(v' - 1)!}{(u' - 1)!} p^n \alpha_{u+1,v+1}^{-1} \omega_2
\end{equation}
where $u + 1 = (u_0 + 1, \dotsc, u_n + 1)$ and similarly for $v + 1$.
\end{thm}

\begin{proof}
See Dwork~\citep[\S 4]{Dwork1964} for Dwork's original work, or 
Lauder~\citep[\S 6.1]{Lauder2004b}.  Both references also treat the 
more general case when $\mathcal{X}$ is a smooth, projective, 
diagonal hypersurface over $\ZZ_q$, for $q$ a prime power.
\end{proof}

\subsection{Improvements}

It appears that this computation genuinely has to take place over 
the extension field $\QQ_p(\pi)$.  This is, however, not the case 
as we will show now.  Indeed, the terms $\alpha_{u+1,v+1}$ are $p$-adic 
integers and we provide expressions for these that are more suitable 
computationally.

We first obtain a more explicit description for the coefficients~$\lambda_m$ 
via an elementary calculation:

\begin{lem} \label{lem:lambdam}
Let $\pi^{p-1} = -p$ and, for $m \geq 0$, let $\lambda_m$ 
be the coefficient of $z^m$ in the power series expansion 
of $\exp \pi (z - z^p)$ in $\QQ_p[[z]]$.  Then 
\begin{equation}
\pi^{- (m \bmod{(p-1)})} \lambda_m = (-1)^{\floor{m/(p-1)}} \sum_{k=0}^{\floor{m/p}} p^{\floor{m/(p-1)} - k} \frac{1}{(m-pk)! k!}
\end{equation}
where $m \bmod{(p-1)}$ denotes the remainder of $m$ upon Euclidean 
division by $p-1$. \hfill $\qedsymbol$
\end{lem}

We can apply this lemma in conjunction with the definition 
of the quantities $\alpha_{u+1, v+1}$.  Upon observing that 
$m - r \bmod{(p-1)}$ is invariant under changes in the summation 
indices $m, r$, we obtain the following theorem:

\begin{thm} \label{thm:alpha}
Let $u, v \in \ZZ^{n+1}$ be such that 
$x^u, x^v \in B_1 \cup \dotsb \cup B_n$ and satisfy, 
for all~$i$, $p (u_i + 1) \equiv v_i + 1 \pmod{d}$. 
Then 
\begin{equation}
\alpha_{u+1,v+1} = (-p)^{u'} \prod_{i=0}^n 
    \hat{a}_i^{(p (u_i + 1) - (v_i + 1))/d} \sum_{m,r} 
    \Bigl(\frac{u_i+1}{d}\Bigr)_r 
    \sum_{k=0}^{\floor{m/p}} \frac{p^{r-k}}{(m-pk)! k!}.
\end{equation}
where $m, r \geq 0$ satisfy $p (u_i + 1) - (v_i + 1) = d (m - pr)$.
\hfill \qedsymbol
\end{thm}

In particular, Theorem~\ref{thm:alpha} implies that 
$\alpha_{u+1, v+1} \in \QQ_p$.  Our next aim is to obtain 
expressions for $\alpha_{u+1,v+1}$ which are defined over 
$\ZZ_p$.  But before progressing, we collect a few 
intermediate results.

\begin{prop} \label{prop:mpr1}
Let $u, v \in B_1 \cup \dotsb \cup B_n$, $i \in \{0,\dotsc,n\}$ and 
let $m, r \geq 0$ satisfy $d(m-pr) = p(u_i + 1) - (v_i + 1)$.  Then 
\begin{equation*}
0 \leq m - p r \leq \frac{p(d-1)-1}{d}.
\end{equation*}
In particular, $m = m(r) = pr + d^{-1}\bigl(p(u_i+1)-(v_i+1)\bigr) \geq 0$.
\end{prop}

\begin{proof}
This can be easily verified using that $0 \leq u_i, v_i \leq d - 2$ 
and $m - pr \in \ZZ$.
\end{proof}

\begin{prop} \label{prop:mpr2}
Let $u, v \in B_1 \cup \dotsb \cup B_n$, $i \in \{0,\dotsc,n\}$ and 
let $m, r \geq 0$ satisfy $d(m-pr) = p(u_i + 1) - (v_i + 1)$.  Then 
\begin{equation*}
r - \floor{\frac{m}{p}} \geq 0.
\end{equation*}
\end{prop}

\begin{proof}
Using the previous proposition,
\begin{equation*}
r - \floor{\frac{m}{p}} 
= - \floor{\frac{m-pr}{p}} 
\geq - \floor{\frac{p(d-1)-1}{pd}} 
= -1 + \ceil{\frac{p + 1}{pd}} 
= 0 
\end{equation*}
as $p \geq 2$, $d \geq 2$.
\end{proof}

\begin{prop} \label{prop:rfac}
For all integers $u, d \geq 1$ and $r \geq 0$ with $p \nmid d$, 
\begin{equation*}
\ord_p\Bigl(\frac{u}{d}\Bigr)_r \geq \frac{r}{p-1} - \floor{\log_p(r) + 1}.
\end{equation*}
\end{prop}

\begin{proof}
Let $s_p(r)$ denote the sum of digits in the $p$-adic expansion of~$r$ 
and observe that $s_p(r) \leq \floor{\log_p(r) + 1}$.  Using the fact that 
$\ord_p\bigl((u/d)_r\bigr) \geq \ord_p(r!)$ from 
Clark~\citep[p.~265]{Clark1966} it follows that 
\begin{equation*}
\ord_p\Bigl(\frac{u}{d}\Bigr)_r \geq \ord_p(r!) = \frac{r - s_p(r)}{p-1} \geq \frac{r}{p-1} - \floor{\log_p(r) + 1}
\end{equation*}
as required.
\end{proof}

\subsubsection{The case $p = 2$}

We first note that in the case when $p = 2$, the expression for 
$\lambda_m$ in Lemma~\ref{lem:lambdam} can be simplified.

\begin{lem} \label{lem:mu2}
For $p = 2$ we define a sequence $\bigl(\mu_m^{(2)}\bigr)$ by 
\begin{equation}
\mu_m^{(2)} = 
    \sum_{k=0}^{\floor{m/2}} \frac{2^{\floor{3m/4} - \nu_m - k}}{(m-2k)! k!}
\end{equation}
where $\nu_m$ is equal to one whenever $m = 3, 7$ and zero otherwise, 
and we write $\mu_m =\mu_m^{(2)}$ whenever this does not cause confusion. 
Then $\mu_m \in \ZZ_p$ for all $m \geq 0$.
\end{lem}

\begin{proof}
In the two cases $m = 3, 7$ we explicitly compute the values of 
$\mu_m$ as $4/3$ and $232/315$.  Now suppose that $m \neq 3, 7$. 
From Lemma~\ref{lem:lambdam} we obtain that 
\begin{equation*}
\ord_2 \bigl(\mu_m\bigr) 
    = \floor{3m/4} - m + \ord_2(\lambda_m).
\end{equation*}
Using the bound $\ord_p(\lambda_m) \geq \bigl((p-1)/p^2\bigr) m$ from 
Dwork~\citep[pp.~55--57]{Dwork1962}, we obtain the lower bound 
\begin{equation*}
\ord_2 \bigl(\mu_m\bigr) 
    \geq \floor{3m/4} - m + \ceil{\frac{m}{4}} = 0. \qedhere
\end{equation*}
\end{proof}

\begin{thm} \label{thm:alpha2}
Let $p = 2$ and suppose $u, v \in \ZZ^{n+1}$ are such that 
$x^u, x^v \in B_1 \cup \dotsb \cup B_n$ and satisfy, for all~$i$, 
$p (u_i + 1) \equiv v_i + 1 \pmod{d}$.  Then $\alpha_{u+1,v+1}$ 
can be expressed as 
\begin{equation} \label{eq:alpha2.0}
\alpha_{u+1,v+1} = (-2)^{u'} \prod_{i=0}^n \sum_{m,r} 
    \Bigl(\frac{u_i+1}{d}\Bigr)_r 2^{-\floor{(m+1)/4}+\nu_m} \mu_m, 
\end{equation}
and $\alpha_{u+1,v+1}$ is a $p$-adic integer.
\end{thm}

\begin{proof}
The expression for $\alpha_{u+1,v+1}$ is an immediate consequence 
of Theorem~\ref{thm:alpha} and Lemma~\ref{lem:mu2}, together with 
Proposition~\ref{prop:mpr1} implying $0 \leq m - 2r \leq 1$ and 
hence $r = \floor{m/2}$.  It remains to prove that 
$\alpha_{u+1,v+1}$ is a $p$-adic integer.  Following Lemma~\ref{lem:mu2}, 
it suffices to show that the valuation of the factor 
\begin{equation*}
\Bigl(\frac{u_i+1}{d}\Bigr)_r 2^{- \floor{(m+1)/4} + \nu_m}
\end{equation*}
in each summand is non-negative.  From the proof of 
Proposition~\ref{prop:rfac} we obtain that 
\begin{equation} \label{eq:alpha2.1}
\ord_p \biggl( \Bigl(\frac{u_i+1}{d}\Bigr)_r 2^{- \floor{(m+1)/4} + \nu_m} \biggr)
\geq \ord_p\Bigl(\floor{\frac{m}{2}}!\Bigr) - \floor{\frac{m+1}{4}} + \nu_m.
\end{equation}
Applying Proposition~\ref{prop:rfac}, we see that the right-hand side 
is bounded below by 
\begin{equation}
\floor{\frac{m}{2}} - \floor{\frac{m+1}{4}} - \floor{\log_2 m}
\end{equation}
which is non-negative whenever $m \geq 12$.  In the remaining 
cases $m = 0, \dotsc, 11$, we explicitly verify that the 
lower bound in~\eqref{eq:alpha2.1} is non-negative.
\end{proof}

\begin{rem}
We observe that in equation~\eqref{eq:alpha2.0} the exponent 
$-\floor{(m+1)/4}+\nu_m$ is non-positive for each value $m \geq 0$, 
which suggests the computational question of how to determine the 
sequence of factors 
\begin{equation*}
\Bigl(\frac{u_i+1}{d}\Bigr)_r 2^{- \floor{(m+1)/4} + \nu_m}
\end{equation*}
to $p$-adic precision $\tilde{N}$ without intermediate precision loss.
Changing our notation slightly, suppose we wish to compute the sequence
\begin{equation*}
f_r = 2^{- \floor{(m+1)/4} + \nu_m} \prod_{j=0}^{r-1} (u + j d)
\end{equation*}
for all $r \geq 0$, where $m(r) = m(0) + 2r$, $m(0) \in \{0,1\}$, 
$u$ is a positive integer and $d$ is a positive odd integer.
Explicitly, we find that this sequence satisfies 
\begin{align*}
f_0 & = 1, \\
f_1 & = u, \\
f_5 & = \begin{cases}
        \displaystyle \tfrac{1}{4} u (u + d) (u + 2d) (u + 3d) (u + 4d)
            & \text{if $m(0)=0$,} \\
        \displaystyle \tfrac{1}{8} u (u + d) (u + 2d) (u + 3d) (u + 4d)
            & \text{otherwise,}
        \end{cases} \\
f_r & = f_{r-2} \frac{(u + (r - 2)d)(u + (r - 1)d)}{2}
\end{align*}
for all remaining $r \geq 0$.  Since the product in the numerator is 
an even integer, it is clear that this computation can be carried 
out without precision loss despite reducing previous values of~$f_r$ 
modulo~$p^{\tilde{N}}$.
\end{rem}

\subsubsection{The case of odd primes}

\begin{lem} \label{lem:mup}
Let $p \geq 3$ be an odd prime and define a sequence 
$\bigl(\mu_m^{(p)}\bigr)$ by 
\begin{equation}
\mu_m^{(p)} = \sum_{k=0}^{\floor{m/p}} \frac{p^{\floor{m/p} - k}}{(m-pk)! k!}, 
\end{equation}
where we write $\mu_m = \mu_m^{(p)}$ when the prime can be identified 
from the context.  Then $\mu_m \in \ZZ_p$ for all $m \geq 0$.
\end{lem}

\begin{proof}
It is clear that $\mu_m \in \QQ$.  From Lemma~\ref{lem:lambdam} 
we observe that $\lambda_m = \pi^m p^{- \floor{m/p}} \mu_m$.  Using the 
bound $\ord_p(\lambda_m) \geq \bigl((p-1)/p^2\bigr) m$ 
from Dwork~\citep[pp.~55--57]{Dwork1962}, it follows that 
\begin{equation}
\ord_p (\mu_m) \geq \frac{p-1}{p^2} m + \floor{\frac{m}{p}} - \frac{m}{p-1}.
\end{equation}
Let us write $m = q p + r$ with $0 \leq r \leq p-1$.  As the valuation 
of $\mu_m$ is an integer, it suffices to show that, for $q \geq 0$, 
\begin{equation}
\frac{p-1}{p} q + q - \frac{q p + p - 1}{p - 1} > -1,
\end{equation}
which is equivalent to $p^2 - 3p + 1 > 0$, and this holds true 
provided that $p \geq 3$.
\end{proof}

\begin{thm} \label{thm:alphap}
Let $p \geq 3$ and suppose that $u, v \in \ZZ^{n+1}$ are such 
that $x^u, x^v \in B_1 \cup \dotsb \cup B_n$ and satisfy, for all~$i$, 
$p (u_i + 1) \equiv v_i + 1 \pmod{d}$. Then 
\begin{equation} \label{eq:alphap.0}
\alpha_{u+1,v+1} = (-p)^{u'} \prod_{i=0}^n 
    \hat{a}_i^{(p (u_i + 1) - (v_i + 1))/d} \sum_{m,r} 
    \Bigl(\frac{u_i+1}{d}\Bigr)_r p^{r - \floor{m/p}} \mu_m
\end{equation}
where $m, r \geq 0$ satisfy $p (u_i + 1) - (v_i + 1) = d (m - pr)$. 
In particular, $\alpha_{u+1, v+1} \in \ZZ_p$. 
\end{thm}

\begin{proof}
This follows from Theorem~\ref{thm:alpha}, Proposition~\ref{prop:mpr2} 
and Lemma~\ref{lem:mup}.
\end{proof}

\subsection{Description of the algorithm}

Up to this point, the double sum over $m,r$ in our expressions 
for $\alpha_{u+1,v+1}$ has been an infinite sum.  We now present 
a convergence result, which will allow us to derive a finite 
expression for $\alpha_{u+1,v+1}$ modulo $p^{\tilde{N}}$.

\begin{lem} \label{lem:log}
Given integers $b,c \geq 2$ and defining $x = c + \log_b c + 1$ 
we have that, for all real numbers $y \geq x$, 
\begin{equation}
y - \log_b y \geq c.
\end{equation}
\end{lem}

\begin{proof}
We first note that the function $y \mapsto y - \log_b y$ is increasing 
for $y \geq 2$ because it has derivative $1 - \log_b(e)/y > 0$.  Thus, it 
suffices to verify the result for $x$.  Indeed, as $c \geq 2$ we have 
that 
\begin{gather}
\log_b c + 1 \leq c \\
\intertext{and hence}
c + \log_b c + 1 \leq b^{\log_b c + 1}
\end{gather}
which upon taking logarithms and rearranging yields the result.
\end{proof}

\begin{prop}
In order to compute $\alpha_{u+1,v+1} \bmod p^{\tilde{N}}$ it 
suffices to restrict the inner sum to pairs $m,r \geq 0$ such 
that $m \leq M = M(\tilde{N}, p)$ where 
\begin{equation}
M = \floor{ p^2 \biggl( \frac{\tilde{N}}{p-1} 
            + \log_p\Bigl(\frac{\tilde{N}}{p-1} + 2\Bigr) + 3 \biggr) }.
\end{equation}
\end{prop}

\begin{proof}
This follows from~\citep[\S 6.2]{Lauder2004b} and Lemma~\ref{lem:log}.
\end{proof}

Finally, we describe how to compute an approximation to the matrix~$\Phi$ 
representing the action of $p^{-1} \Frob_p$ on $\Hrig^{n}(U)$ using our 
previous results.

\begin{prop}
The valuation of $(u'-1)! \alpha_{u+1,v+1}$ is bounded from above by 
\begin{equation}
\ord_p\bigl((u'-1)! \alpha_{u+1,v+1}\bigr) 
    \leq \ord_p\bigl((n-1)!\bigr) + n + \delta
\end{equation}
where $\delta = \ord_p\bigl((n-1)!\bigr) + (n+1) \floor{\log_p(n-1)}$. 
\end{prop}

\begin{proof}
Recall from Gerkmann~\citep[Lemma~3.3]{Gerkmann2007} that the valuations 
of the entries of the matrix~$\Phi$ are bounded from below by $-\delta$. 
Thus, by Theorem~\ref{thm:01-03-diagfrob}, 
\begin{equation}
-\delta \leq \ord_p\bigl((v'-1)!\bigr) + n 
           - \ord_p\bigl((u'-1)! \alpha_{u+1,v+1}\bigr)
\end{equation}
Noting that $d v' = \sum_{i=0}^n (v_i + 1) \leq n d$ and 
hence $v' \leq n$, the result follows.
\end{proof}

This allows us to formalise the procedure for computing the 
entries of~$\Phi$ modulo~$p^N$ in Algorithm~\ref{alg:Diabfrob} 
below.

\begin{algorithm}
\caption{Compute the matrix for $p^{-1} \Frob_p$ on $\HdR^n(\mathfrak{U})$}
\label{alg:Diabfrob}
\begin{algorithmic}
\vspace{1mm}
\Require Polynomial $a_0 x_0^d + \dotsb + a_n x_n^d$, 
         prime~$p$, precision~$N \geq 0$.
\Ensure  Matrix for $p^{-1} \Frob_p$ on $\HdR^n(\mathfrak{U})$ modulo $p^N$.
\Procedure{DiagFrob}{$n, d, p, N, a_0, \dotsc, a_n$}
\begin{enumerate}
\item Determine the numbers 
      $C = C(n,p) = n + 2 \ord_p((n-1)!) + (n+1) \floor{\log_p(n-1)}$, 
      $\tilde{N} = N - n + 2 C$, $M = M(\tilde{N}, p)$, and 
      $R = R(\tilde{N}, p) = \floor{M/p}$.
\item Precompute the Teichm\"uller lifts $\hat{a}_0, \dotsc, \hat{a}_n$, 
      and the sequences $(d^{-r})_{r=0}^R$, and $(\mu_m)_{m=0}^{M}$ 
      to precision~$\tilde{N}$.
\item For each monomial $u = (u_0, \dotsc, u_n) \in B_1 \cup \dotsb \cup B_n$, 
      determine the unique monomial $v = (v_0, \dotsc, v_n)$ such that 
      $v_i = p (u_i + 1) - 1 \bmod{d}$.  Use the following steps to 
      compute the entry at position $(u,v)$ in the matrix.
\item Compute the expression $x_1 = (-1)^{u'+v'} (v'-1)! p^n$ as an 
      exact integer and observe that \mbox{$\ord_p(x_1) \geq n$}.
\item Compute the expression $x_2 = (u' - 1)! \alpha_{u+1,v+1}$ to 
      precision~$\tilde{N}$ using equation~\eqref{eq:alpha2.0} or 
      \eqref{eq:alphap.0} depending on whether $p$ is even or odd. 
\item Compute the inverse $x_2^{-1}$ to precision $N - n$.
\item Finally, compute the product $x_1 x_2^{-1}$ to precision~$N$.
\end{enumerate}
\EndProcedure
\end{algorithmic}
\end{algorithm}

\begin{rem} \label{rem:mup}
The expressions for $\mu_m$ can be computed efficiently 
modulo~$p^{\tilde{N}}$ via 
\begin{equation}
\mu_m^{(p)} = \begin{cases}
\frac{2^{\floor{3m/4}}}{m!} 
    \sum_{k=0}^{\floor{m/2}} \frac{m!}{2^k (m-2k)! k!}
    & \text{if $p = 2$, $m \neq 3, 7$} \\
\frac{p^{\floor{m/p}}}{m!} 
    \sum_{k=0}^{\floor{m/p}} \frac{m!}{p^k (m-pk)! k!}
    & \text{if $p$ is odd},
\end{cases}
\end{equation}
using only one $p$-adic inversion.  We observe that the summands are 
integers of size $\BigOh(m \log m)$ bits, which also allows us to 
avoid performing intermediate reductions modulo~$p^{\tilde{N}}$.
\end{rem}

\begin{thm} \label{thm:DiagfrobComplexity1}
The time complexity of Algorithm~\ref{alg:Diabfrob} is given by 
\begin{equation*}
p \tilde{N}^2 \cM\bigl(p \tilde{N} \log (p \tilde{N})\bigr)
    + d^n n \bigl( \cM(\log d) + (\tilde{N} + \log p) \cM(\tilde{N} \log p) \bigr)
\end{equation*}
where $\cM(-)$ denotes the complexity of integer multiplication and 
$\tilde{N}$ is $\BigOh(N + n \log_p n)$.
\end{thm}

\begin{proof}
We consider each of the steps in Algorithm~\ref{alg:Diabfrob}. 
We can ignore the computational cost of {Step~(i)}, but observe 
that $\tilde{N} \in \BigOh(N + n \log_p n)$, 
$M \in \BigOh(p \tilde{N})$, and $R \in \BigOh(\tilde{N})$.  

In {Step~(ii)}, we compute $n+1$ Teichm\"uller lifts with an 
overall complexity of $\BigOh(n (\log p) \cM(\tilde{N} \log p))$. 
The computation of the sequence $(d^{-r})_{r=0}^{R}$ requires 
one reduction of $d$ modulo $p^{\tilde{N}}$, one $p$-adic inversion 
and $R-1$~products to precision~$\tilde{N}$, yielding 
$\BigOh\bigl(\cM(\log d) + (\log \log p) \cM(\log p) + \tilde{N} \cM(\tilde{N} \log p)\bigr)$. 
Finally, we carry out the computation of $(\mu_m)_{m=0}^{M}$ following 
Remark~\ref{rem:mup}, which requires time 
$\BigOh\bigl(p \tilde{N}^2 \cM(p \tilde{N} \log(p \tilde{N}))\bigr)$.

The following {Steps~(iii) through (vii)} are executed 
$\dim \Hrig^{n}(U)$ times, where this dimension 
is $\BigOh(d^n)$.
The time complexity of {Step~(iii)} is 
$\BigOh\bigl(n \cM(\log \max\{p,d\})\bigr)$.
We observe that we can ignore {Step~(iv)}.
Step~(v) involves an $(n+1)$-fold product of series 
with~$\BigOh(R)$ terms modulo~$p^{\tilde{N}}$, where each summand 
requires an absolutely bounded number of products, as well as 
$n+1$~exponentiations of Teichm\"uller lifts with exponents given 
by $d^{-1} \bigl(p (u_i+1) - (v_i+1)\bigr) < p$.  Computing the 
exponents has complexity $\BigOh(\cM(\log \max\{p,d\}))$ and 
we note that we may ignore the update of the term 
\mbox{$\bigl((u_i+1)/d\bigr)_r$} throughout the summation 
assuming we have computed the reduction of $d \bmod{p^{\tilde{N}}}$ 
once and for all earlier.  Therefore, the complexity of each 
invocation of {Step~(v)} is 
$\BigOh\bigl( n \cM(\log d) + n (R + \log p) \cM(\tilde{N} \log p)\bigr)$.  
The $p$-adic inverse in {Step~(vi)} requires time 
$\BigOh\bigl((\log \log p) \cM(\log p) + \cM(\tilde{N} \log p)\bigr)$.
Finally, we can ignore the product in {Step~(vii)}.  
Thus, the aggregate time complexity of {Steps~(iii)} through {(vii)} 
is given by 
$\BigOh\bigl(d^n n \bigl( \cM(\log d) + (\tilde{N} + \log p) \cM(\tilde{N} \log p) \bigr)\bigr)$.
\end{proof}

While the current implementation of Algorithm~\ref{alg:Diabfrob} 
performs very well in practice, its time complexity is quasi-cubic in 
the $p$-adic precision~$N$, which is \emph{not} optimal.  Both of 
these aspects are due to the use of the sequence $(\mu_m)_{m=0}^{M}$ 
defined over $\ZZ_p$ instead of the coefficients $(\lambda_m)_{m=0}^{M}$ 
defined over $\QQ_p(\pi)$.  We can achieve a better time complexity by 
utilising fast exponentials of power series:

\begin{thm} \label{thm:DiagfrobComplexity2}
There exists an algorithm for computing the matrix for the 
action of $p^{-1} \Frob_p$ on $\Hrig^{n}(U)$ in time complexity 
\begin{equation*}
(M \log M \log \log M) (p \log p) \cM(\tilde{N} \log p) 
+ d^n n \bigl( \cM(\log d) 
              + \tilde{N} (p \log p) \cM(\tilde{N} \log p) \bigr)
\end{equation*}
where $\cM(-)$ denotes the complexity of integer multiplication 
and $\tilde{N} \in \BigOh(N + n \log_p n)$, $M \in \BigOh(p \tilde{N})$.
\end{thm}

\begin{proof}
The key idea is to slightly modify Algorithm~\ref{alg:Diabfrob} 
and use equation~\eqref{eq:alpha}, working directly with the sequence 
$(\lambda_m)_{m=0}^{M}$.  As we will be using operations in the 
totally ramified extension~$\QQ_p(\pi)$ to 
$p$-adic precision~$\tilde{N}$, we remark that the cost of 
an arithmetic operation in its ring of integers is 
$\BigOh\bigl((p \log p) \cM(\tilde{N} \log p)\bigr)$, 
achieved by polynomial multiplication based on the fast Fourier 
transform.

We first observe that the valuation of the summands in 
equation~\eqref{eq:alpha} can be bounded by 
$\ord_p \bigl(\lambda_m (u_i / d)_r (-1)^r \pi^{-r} {\hat{a}_i}^{m-r} \bigr) \geq - 1/2$.
As the only term with negative valuation is $\pi^{-r}$ and 
$R \in \BigOh(\tilde{N})$, it suffices to precompute 
the sequences $(\lambda_m)_{m=0}^{M}$ and $(d^{-r})_{r=0}^{R}$ 
to $p$-adic precision $\tilde{N}$.  While the computation 
of the latter remains unchanged, the computation of the former 
can be improved significantly using fast exponentials of 
power series in $\QQ_p(\pi)[[z]]$ as described by 
Bernstein~\citep[\S 9.3]{Bernstein2008}.  This allows for 
computing the sequence $(\lambda_m)_{m=0}^{M}$ in 
$\BigOh\bigl( (M \log M \log \log M) \bigr)$ operations 
in~$\QQ_p(\pi)$ to precision~$\tilde{N}$.  The only 
remaining change to our analysis of Algorithm~\ref{alg:Diabfrob} 
occurs in {Step~(v)}, where for each matrix entry we have to consider 
$\BigOh(n R)$ multiplications in~$\QQ_p(\pi)$ instead 
of~$\ZZ_p$.
\end{proof}

%%%%%%%%%%%%%%%%%%%%%%%%%%%%%%%%%%%%%%%%%%%%%%%%%%%%%%%%%%%%%%%%%%%%%%%%%%%%%%%

\section{Solving Picard--Fuchs differential systems}
\label{sec:DifferentialSystem}

%%%%%%%%%%%%%%%%%%%%%%%%%%%%%%%%%%%%%%%%%%%%%%%%%%%%%%%%%%%%%%%%%%%%%%%%%%%%%%%

\section{Analytic continuation and evaluation}
\label{sec:Evaluation}

In the previous section we described how to compute the matrix~$\Phi$ 
for the action of $p^{-1} \Frob_p$ on $\Hrig^{n}(U/S)$ 
as a local power series expansion around $t = 0$.  In order to evaluate it 
at a second point $t = t_1$, we wish to $p$-adically approximate it 
using rational functions with denominators a power of $r(t)$, which 
is possible as the entries of the matrix~$\Phi$ are overconvergent 
functions.

The analytic continuation of the local expansion is described by Lauder 
in~\citep[\S 5.2]{Lauder2006} and the computational details are explained 
in~\citep[\S 8.1]{Lauder2004a}.  Combining this step with the subsequent 
evaluation, our specific problem is the following.  Given a desired 
$p$-adic precision~$N_2$, we would like to determine integers~$K_1$ 
and~$K_2$ such that, modulo~$p^{N_2}$, the entries of the matrix 
$\Psi(t) = r(t)^{K_1} \Phi(t)$ truncated modulo~$t^{K_2}$ allow us to 
compute the matrix $\Phi_{t_1}$ for the action of $p^{-1} \Frob_p$ 
on $\Hrig^{n}(U_{t_1})$ as $r(\hat{t_1})^{-1} \Psi(\hat{t_1})$.

Besides the work of Lauder~\citep[\S 8.1]{Lauder2004a}, there are also 
suitable results by Gerkmann~\citep[\S 6]{Gerkmann2007}, however the 
estimates are not sharp in practice.  There has been recent progress by 
Kedlaya and Tuitman~\citep[Theorem~2.1]{KedlayaTuitman2012}, which we 
present in a slightly simplified form:

\begin{thm} \label{thm:KedlayaTuitman}
Let $\mathfrak{Z}$ denote the complement of the open dense 
subscheme $\mathfrak{S}$ of $\mathbf{P}^{1}(\mathbf{Q}_q)$ 
and let $z$ be an unramified geometric point of $\mathfrak{Z}$ 
such that $\mathfrak{Z}$ does not contain any other points 
with the same reduction modulo~$p$.  For a fixed a basis for 
$\Hrig^n(U/S)$, suppose that the matrix 
for the connection~$\nabla$ has at most a simple pole at~$z$ 
and that the exponents $\lambda_1, \dotsc, \lambda_{b}$, 
which are defined as the eigenvalues of $(t - z) M \vert_{t=z}$ 
and known to be rational numbers, have non-negative $p$-adic 
valuation.  Then the matrix~$\Phi$ for the action of $p^{-1} \Frob_p$ 
has a pole at $z$ of order at most 
\begin{equation} \label{eq:KedlayaTuitman}
\max_{i} \lambda_i - p \min_{i} \lambda_i + p g(N)
\end{equation}
where $g(N)$ is defined by 
\begin{equation}
g(N) = \max\Bigl\{ i \in \mathbf{N} : i - \delta - \bigl(2\delta + (n-1)\bigr) \ceil{\log_p i} < N \Bigr\}.
\end{equation}
\end{thm}

\begin{proof}
See \citep[Theorem~2.1]{KedlayaTuitman2012}.
\end{proof}

[[TODO:  Please rewrite the above theorem and include whichever 
form and notation you see fit.]]

\begin{rem} \label{rem:KedlayaTuitman}
In practice, it might be convenient to avoid computing the exponents and 
verifying the hypotheses of the previous theorem.  This is particularly 
relevant since even when the hypotheses are not satisfied, the required 
bounds are often closer to the result of Kedlaya and Tuitman than to the 
estimates provided by Gerkmann.

The contribution of the terms $\max_i \lambda_i - p \min_i \lambda_i$ 
in equation~\eqref{eq:KedlayaTuitman} typically is small compared to the 
term~$p g(N_2)$. Therefore, in a heuristic implementation, one 
could choose e.g.\ $K_1 = 1.1 \times p N_2$ and $K_2 = (\deg(r) + 1) K_1$.
\end{rem}

Having obtained an expression for the action of~$p^{-1} \Frob_p$ 
on~$\Hrig^{n}(U_{t_1})$ in terms of rational functions, we can 
compute the matrix $\Phi_{t_1}$ representing $p^{-1} \Frob_p$ on 
$\Hrig^{n}(U_{t_1})$ modulo~$p^{N_2}$ as 
\begin{equation}
\Phi_{t_1} = 
    r(\hat{t}_1)^{-K_1} 
    \Bigl( r(t)^{K_1} \Phi(t) \bmod{t^{K_2}} \Bigr) \Big\vert_{t=\hat{t}_1} \pmod{p^{N_2}}.
\end{equation}
Since, by assumption, $r(t_1)$ is a $p$-adic unit and hence so is 
$r(\hat{t}_1)^{-K_1}$ we observe that it suffices to compute the 
Teichm\"uller lift~$\hat{t}_1$ and the matrix~$r(t)^{K_1} \Phi(t)$ 
over~$\mathbf{Q}_p[t]$ to $p$-adic precision~$N_2$.

%%%%%%%%%%%%%%%%%%%%%%%%%%%%%%%%%%%%%%%%%%%%%%%%%%%%%%%%%%%%%%%%%%%%%%%%%%%%%%%

\section{Recovering zeta functions}
\label{sec:ZetaFunctions}

Recall that the zeta function of the hypersurface~$X_{t_1}$ is of the form,
\begin{equation}
Z(X_{t_1},T) = \frac{p(T)^{(-1)^n}}{(1 - T) (1 - qT) \dotsm (1 - q^{n-1}T)}
\end{equation}
where $p(T) = \det \bigl( 1 - T q^{-1} \Frob_q | \Hrig^n(U_{t_1}) \bigr)$ 
is a polynomial defined over the integers.
Thus, the remaining two steps in the deformation method are now clear.  
Firstly, we determine the action of the $q$th-power Frobenius 
from that of the $p$th-power Frobenius.  Secondly, we compute 
its reverse characteristic polynomial to suitable \mbox{$p$-adic} 
precision in order to recover the zeta function.

\subsection{Computing the action of $q^{-1} \Frob_q$ on $\Hrig^{n}(U_{t_1})$}

As before, let $\Phi_{t_1}$ and $\Phi_{t_1}^{(q)}$ denote the matrices 
representing the actions of $p^{-1} \Frob_p$ and $q^{-1} \Frob_q$ 
on~$\Hrig^{n}(U_{t_1})$, respectively.  As the action of $p^{-1} \Frob_p$ 
is $\sigma$-semilinear, we have that 
\begin{equation}
\Phi_{t_1}^{(q)} = 
    \Phi_{t_1} \sigma(\Phi_{t_1}) \dotsm \sigma^{a-1}(\Phi_{t_1}),
\end{equation}
where $a = \log_p q$.  Note that the lift of Frobenius~$\sigma$ 
is valuation preserving and hence the valuations of the matrices 
$\Phi_{t_1}, \sigma(\Phi_{t_1}), \dotsc, \sigma^{a-1}(\Phi_{t_1})$ 
are at least $-\delta$ by [[TODO: Reference]].
In particular, it follows that in order to determine $\Phi_{t_1}^{(q)}$ 
to $p$-adic precision~$N_1$ it suffices to provide $\Phi_{t_1}$ to 
precision $N_2 \geq N_1 + (a-1) \delta$.

\subsection{Computing Weil polynomials}

Before considering Weil polynomials, we briefly address the loss 
of precision when computing the reverse characteristic polynomial 
$\det(1 - t A)$ of a $b \times b$ matrix~$A$ given to finite precision 
over $\mathbf{Q}_p$.  In general, from the elementary definition of the 
determinant function, it appears that the precision loss could be as 
great as $(b-1) \ord_p(A)$.  However, in the case of matrices representing 
the action of Frobenius on $\Hrig^n(U_{t_1})$, much better bounds are 
available:

\begin{lem}
Define $\delta = \delta(n,p)$ as in Lemma~[[TODO: Reference]] and 
suppose that $\tilde{\Phi}_{t_1}^{(q)}$ is an approximation to 
$\Phi_{t_1}^{(q)}$ with 
$\ord_p\bigl(\Phi_{t_1}^{(q)}-\tilde{\Phi}_{t_1}^{(q)}\bigr) \geq N + \delta$.
Then 
\begin{equation}
\ord_p \Bigl( \det\bigl(1 - t \Phi_{t_1}^{(q)}\bigr) 
            - \det\bigl(1 - t \tilde{\Phi}_{t_1}^{(q)}\bigr) \Bigr) \geq N.
\end{equation}
\end{lem}

\begin{proof} 
See Gerkmann~\citep[Lemma~3.3, Lemma~3.4]{Gerkmann2007}.
\end{proof}

The previous lemma allows us to take $N_1 = N_0 + \delta$ in our 
description of the deformation algorithm.

Finally, we compute the 
reverse characteristic polynomial $p(T)$ of the matrix $\Phi_{t_1}^{(q)}$, 
which represents $q^{-1} \Frob_q$ on $\Hrig^n(U_{t_1})$, to some finite 
$p$-adic precision.  Which precision~$N_0$ is necessary in order to 
correctly recover the polynomial~$p(T)$ over the integers?

\begin{thm} \label{thm:N0}
In order to recover $p(T)$ over $\mathbf{Z}$ it suffices to compute 
an approximation modulo~$p^N$ where 
\begin{equation}
p^N > 2 \max_{0 \leq i \leq b} \binom{b}{i} q^{i (n-1) / 2}.
\end{equation}
Moreover, this can be improved to 
\begin{equation}
p^N > 2 \binom{b}{\floor{b/2}} q^{\floor{b/2} (n-1) / 2}
\end{equation}
provided that the sign $\epsilon = \sgn(\det(F_q))$ of the 
functional equation for the zeta function is known.
\end{thm}

\begin{proof}
This is a slight reformulation of {Theorem~3.2} in~\citep{Gerkmann2007}, 
which follows readily from its proof.
\end{proof}

\begin{rem}
We observe that the sign $\epsilon = \sgn(\det(q^{-1} \Frob_q)) = 1$ 
whenever $n$ is even.  [[TODO: Give justification.  If nothing else, 
include this in the statement of the Weil conjectures.]]
\end{rem}

\begin{rem}
In the case that $n$ is odd, the sign~$\epsilon$ of the functional 
equation is unknown a priori.  However, in practice it is often still 
possible to avoid having to use the greater of the two precisions in 
Theorem~\ref{thm:N0} by just computing one additional coefficient 
exactly.  Writing $p(T) = \sum_{i=0}^{b} a_i T^i$, we can recover 
$a_0, \dotsc, a_{\floor{b/2}+1}$ exactly provided 
\begin{equation}
p^N > 2 \max\biggl\{\binom{b}{\floor{b/2}} q^{\floor{b/2} (n-1) / 2}, 
                   \binom{b}{\floor{b/2}+1} q^{(\floor{b/2}+1) (n-1)/2} \biggr\}.
\end{equation}
This allows us to determine the sign~$\epsilon$ from the two 
coefficients $a_{\ceil{b/2}-1}$ and $a_{\floor{b/2}+1}$, provided 
that they are non-zero, and we can then recover the remaining 
coefficients using the functional equation.
\end{rem}

\begin{rem} \label{rem:N0Surfaces}
In the case of smooth projective surfaces, when $p > 2$ and subject to 
certain technical conditions, we can often exploit the growing divisibility 
of the coefficients of $p(T)$ ensured by the Hodge polygon.  For such a surface 
of degree~$d$, we know that the Hodge numbers $h_{0,2}$, $h_{1,1}$ and $h_{2,0}$ 
satisfy $h_{0,2} = h_{2,0} = \binom{d-1}{3}$ and $2 h_{0,2} + h_{1,1} = b$. 
It is now easier to determine the integer coefficients of the 
polynomial~$q^{h_{0,2}} p(T/q)$ as the roots of the polynomial $p(T/q)$ lie 
on the unit circle.  This allows us to take 
\begin{align*}
N_0 & = a h_{0,2} + \floor{\log_p \biggl( 2 \binom{b}{\floor{b/2}}\biggr)} + 1,\\
N_1 & = N_0 + a,
\end{align*}
where $N_0$ here refers to the required precision for $q^{h_{0,2}} p(T/q)$ 
and $N_1$ as before refers to the precision required for the matrix 
representing $q^{-1} \Frob_q$ on $\Hrig^{n}(U_{t_1})$.
For further details, we refer the reader to 
Lauder~\citep[\S 9.3.2, Proposition~9.6]{Lauder2006}.
\end{rem}

%%%%%%%%%%%%%%%%%%%%%%%%%%%%%%%%%%%%%%%%%%%%%%%%%%%%%%%%%%%%%%%%%%%%%%%%%%%%%%%

\section{Complexity analysis}
\label{sec:Complexity}

%%%%%%%%%%%%%%%%%%%%%%%%%%%%%%%%%%%%%%%%%%%%%%%%%%%%%%%%%%%%%%%%%%%%%%%%%%%%%%%

\section{Examples}
\label{sec:Examples}

\subsection{Parallelisation}

From a practical point of view, we observe that the individual steps 
of the algorithm which contribute significantly to the runtime can be 
carried out in parallel:
\begin{itemize}
\item In the implementation of the computation of $p^{-1} \Frob_p$ 
      on $\Hrig^{n}(U_0)$ which is quasi-cubic in the precision, 
      the elements of the sequence $(\mu_m)_{m=0}^{M}$ can be computed 
      independently.
\item The computation of the connection matrix can be parallelised 
      as the computations of the images of the basis vectors for 
      $\HdR^{n}(\mathfrak{U}/\mathfrak{S})$ are independent.
\item The major contributor to the runtime in many examples is the 
      computation of the matrix~$C(t)$ which is the local solution 
      of the differential equation.  While the outer loop in 
      Equation~[[TODO: Reference]] cannot be parallelised easily 
      as $C_{i}$ depends on certain $C_{j}$ with $j < i$, the 
      matrix products in the inner loop are independent.
\item The computation of the $q$th-power Frobenius 
      $\Phi^{(q)} \sigma(\Phi^{(q)}) \dotsm \sigma^{a-1}(\Phi^{(q)})$, 
      where $a = \log_{p} q$, can be carried out in parallel.  This can 
      be achieved if, instead of computing a running product from left to 
      right, we express the matrix in a product tree.  We formalise this in 
      Algorithm~\ref{alg:ParallelProduct}, assuming for simplicity 
      that $a = 2^k$.  In this notation, the execution of the 
      loops indexed by~$i$ can be parallelised.
      \begin{algorithm}
      \caption{Parallel computation of $q^{-1} \Frob_{q} | \Hrig^{n}(U_{t_1})$}
      \label{alg:ParallelProduct}
      \begin{algorithmic}
      \vspace{1mm}
      \For{$i \gets 0$ \textbf{to} $a-1$}
          \State $G_i \gets \sigma^{i}(F)$
      \EndFor
      \For{$j=k-1$ \textbf{to} $0$}
          \State $h \gets 2^{k-j}$
          \For{$i \gets 0$ \textbf{to} $2^j-1$}
          \State $G_{2^j-1 + ih} \gets G_{2^j-1 + i h} G_{2^j-1 + ih + h/2}$
          \EndFor
      \EndFor
      \Return $G_0$
      \end{algorithmic}
      \end{algorithm}
\end{itemize}

\phantomsection

\bibliographystyle{plainnat}
\bibliography{deformation}

\end{document}

